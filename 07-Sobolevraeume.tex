\chapter{Sobolevräume}

\section{Das Lebesgue-Integral}

\begin{defn}
\label{defn:7.1}
$\Sigma\in\PP(\R^n)$ heißt $\sigma$-Algebra, falls
\begin{defnenum}
\item $\varnothing\in\Sigma$,
\item $A\in\Sigma\Rightarrow A^c\in\Sigma$,
\item $A_j\in \Sigma \Rightarrow \bigcup_{j\in\N} A_j \in \Sigma$.\fishhere
\end{defnenum}
\end{defn}

Es gibt zahllose $\sigma$-Algebren auf dem $\R^n$. Die $\sigma$-Algebra, die
die offenen Mengen enthält, heißt \emph{Borel-$\sigma$-Algebra}.

\begin{cor}
\label{prop:7.2}
\begin{propenum}
\item $\R^n\in\Sigma$.
\item $A_j\in\Sigma \Rightarrow \bigcap_{j\in\N} A_j \in \Sigma$.
\item $A,B\in\Sigma \Rightarrow A\setminus B\in\Sigma$.\fishhere
\end{propenum}
\end{cor}

\begin{defn}
\label{defn:7.3}
Eine Abbildung $\mu: \Sigma \to [0,\infty]$ heißt \emph{Maß}, falls
\begin{defnenum}
\item $\mu(\varnothing) = 0$.
\item $\mu$ \emph{$\sigma$-additiv}, d.h. für $A_j\in\Sigma$ mit $A_j\cap A_k =
\varnothing$ falls $j\neq k$, gilt
\begin{align*}
\mu\left(\dot{\bigcup}_{j\in\N} A_j\right) = \sum_{j=1}^\infty
\mu(A_j).\fishhere
\end{align*}
\end{defnenum}
\end{defn}

Mit $A \dcup B$ bzw. $\dot{\bigcup}_j A_j$ fordern wir implizit, dass $A$ und
$B$ bzw. die $A_j$ disjunkt sind.

\begin{cor}
\label{prop:7.4}
\begin{propenum}
\item Wenn $A,B\in\Sigma$ und $A\subseteq B$, dann folgt $\mu(A)\le\mu(B)$.
\item $A_j\in\Sigma$, $A_1\subseteq A_2\subseteq A_3 \subseteq \ldots$,
\begin{align*}
\Rightarrow \mu\left(\bigcup_{j\in\N} A_j\right) = \lim\limits_{j\to\infty}
\mu(A_j).\fishhere
\end{align*}
\end{propenum}
\end{cor}

\begin{prop}
\label{prop:7.5}
Es existiert eine $\sigma$-Algebra $\Sigma\subseteq \PP(\R^n)$ und ein Maß
$\mu$ auf $\Sigma$ mit den Eigenschaften:
\begin{propenum}
\item\label{prop:7.5:1} $O\subseteq\R^n$ offen $\Rightarrow O\in\Sigma$.
\item\label{prop:7.5:2} $\mu\left([a_1,b_1]\times \ldots \times [a_n,b_n]\right)
=\prod_{j=1}^n (b_j-a_j)$.
\item\label{prop:7.5:3} $N\in\Sigma$ mit $\mu(N) = 0$ und $M\subseteq N$, so ist
$M\in\Sigma$ und $\mu(M) = 0$.
\item\label{prop:7.5:4} $\mu$ ist translationsinvariant, d.h.
\begin{align*}
A\in\Sigma,\; x\in\R^n \Rightarrow
\begin{cases}
x+A\in\Sigma,\\
\mu(x+A) =\mu(A).\fishhere
\end{cases}
\end{align*}
\end{propenum}
\end{prop}

\begin{defn}
\label{defn:7.6}
Für das Paar $(\Sigma,\mu)$, das von allen $(\Sigma,\mu)$, die
\ref{prop:7.5:1}-\ref{prop:7.5:4} aus \ref{prop:7.5} erfüllen, die kleinste
$\sigma$-Algebra besitzt, heißt $\mu$ \emph{Lebesgue-Maß} und $\Sigma$ die
$\sigma$-Algebra der \emph{Lebesgue-messbaren} Mengen.\fishhere
\end{defn}

\begin{bem}
\label{bem:7.7}
$A\subseteq\R^n$ abzählbar $\Rightarrow$ $A\in\Sigma$ und $\mu(A) = 0$.\maphere
\end{bem}

\begin{defn}
\label{defn:7.8}
Sei $B\subseteq A$ mit $\mu(A\setminus B) = 0$. Gilt eine Bedingung auf $B$, so
sagt man die Bedingung gilt auf $A$ \emph{fast überall}.\fishhere
\end{defn}

\begin{defn}
\label{defn:7.9}
\begin{defnenum}
\item $f: D\to \RA$, $D\subseteq \R^n$, $\RA:=\R\cup\setd{-\infty,\infty}$
heißt \emph{messbar}, falls
\begin{align*}
\forall a\in\RA : f^{-1}([a,\infty]) \in \Sigma.
\end{align*}
\item $f: D\to \C$ heißt \emph{messbar}, falls $\Re f$ messbar und $\Im f$
messbar.\fishhere
\end{defnenum}
\end{defn}

\begin{prop}
\label{prop:7.10}
\begin{propenum}
\item Sei $D\in\Sigma$ und $f$ stetig, so ist $f$ messbar.
\item Falls $f,g$ messbar, dann auch
\begin{defnenum}
\item $\min\setd{f,g}$ und $\max\setd{f,g}$.
\item $f\cdot g$ mit der Konvention $0\cdot \infty = 0$.
\item $f+g$, falls $f(x) = \pm\infty \Rightarrow g(x)\neq \mp \infty \mufu$.\\
Insbesondere sind daher $c\cdot f$, $f_+:=\max\setd{f,0}$, $f_-:=
-\min\setd{f,0}$ und $\abs{f}:= f_++f_-$.
\end{defnenum}
\item Sei $f_n$ messbar, dann sind es auch
\begin{defnenum}
\item $\sup_n f_n$, $\inf_n f_n$,
\item $\limsup_n f_n$, $\liminf_n f_n$.
\end{defnenum}
\item Sei $f:\R\to\R$ stetig und $g: D\to\RA$ messbar. Setze
\begin{align*}
f\circ g(x) :=
\begin{cases}
\pm \infty, & g(x) = \pm\infty,\\
f\circ g(x),& \text{sonst}.
\end{cases}
\end{align*}
So ist auch $f\circ g$ messbar.\fishhere
\end{propenum}
\end{prop}

\begin{defn}
\label{defn:7.11}
\begin{defnenum}
\item Sei $A\subseteq \R^n$. Dann heißt
\begin{align*}
\chi_A : x\mapsto
\begin{cases}
1, & x\in A,\\
0, & \text{sonst},
\end{cases}
\end{align*}
\emph{charakteristische Funktion} von $A$.
\item $A_j\in\Sigma$, $\alpha_j\in\R $, $N\in\N$ dann heißt,
\begin{align*}
f = \sum_{j=1}^N \alpha_j \chi_{A_j}
\end{align*} 
\emph{einfache Funktion}.\fishhere
\end{defnenum}
\end{defn}

\begin{prop}
\label{prop:7.12}
Sei $A\in\Sigma$, $f: A\to[0,\infty]$ messbar. Dann existiert eine Folge
einfacher Funktionen $(s_n)$ mit
\begin{align*}
\forall x\in A : s_1(x)\le s_2(x)\le \ldots
\end{align*}
und $f(x) = \lim\limits_{n\to\infty} s_n(x)$. Falls $f$ beschränkt kann $(s_n)$
so gewählt werden, dass $s_n\unito g$ auf $A$.\fishhere
\end{prop}

\begin{prop}
\label{prop:7.13}
Sei $A\in\Sigma$.
\begin{propenum}
\item Für $A_j\in\Sigma$, $s=\sum_{j=1}^N \alpha_j \chi_{A_j}$ mit $\alpha_j\ge
0$ setze
\begin{align*}
\int_A s \dmu := \sum_{j=1}^N \alpha_j \mu(A_j).
\end{align*}
\item Sei $f: A\to [0,\infty]$ messbar. Dann
\begin{align*}
\int_A f \dmu := \sup
\setdef{\int_A s \dmu}{s\text{ einfach, positiv und }s\le f}.
\end{align*}
\item Sei $f: A\to\RA$ messbar.
\begin{align*}
\int_A f\dmu := \int_A f_+ \dmu - \int_A f_-\dmu,
\end{align*}
falls nicht beide Integrale $\infty$. Andernfalls ist das Integral nicht
definiert.
\item $f: A\to\RA$ heißt \emph{Lebesgue-integrierbar}, falls
\begin{align*}
\int_A f_+\dmu,\quad \int_A f_-\dmu < \infty.
\end{align*}
\item $f: A\to \C$ heißt \emph{Lebesgue-integrierbar}, falls $\Re f$ und $\Im
f$ Lebesgue-integrierbar. Schreibe dann $f\in L^1(A)$ und setze
\begin{align*}
\int_A f\dmu = \int_A \Re f \dmu + i\int_A \Im f\dmu.\fishhere
\end{align*}
\end{propenum}
\end{prop}

\begin{prop}[Eigenschaften]
\label{prop:7.14}
Seien $A,B\in\Sigma$, $f,g:A\to\RA$ oder $\C$ messbar.
\begin{propenum}
\item $f\in L^1(A)$ genau dann, wenn $\abs{f}\in L^1(A)$. Dann gilt
\begin{align*}
\abs{\int_A f\dmu} \le \int_A \abs{f}\dmu.
\end{align*}
\item Sind $f\in L^1(A)$ und $\abs{g}\le f$, so ist $g\in L^1(A)$ und
\begin{align*}
\int_A \abs{g}\dmu \le \int_A f\dmu.
\end{align*}
\item Die Abbildung
\begin{align*}
L^1(A) \to \R,\quad f\mapsto \int_A f\dmu 
\end{align*}
ist linear.
\item Falls $\mu(A)<\infty$ und $f$ $\mufu$ beschränkt, so ist $f\in L^1(A)$ und
\begin{align*}
\int_A \abs{f}\dmu \le \norm{f}_\infty \mu(A).
\end{align*}
\item Seien $a,b\in\R$ mit $a\le f(x)\le b\mufu$ auf $A$ und $\mu(A)<\infty$.
Dann ist $f\in L^1(A)$ und
\begin{align*}
a\mu(A) \le \int_A f\dmu \le b\mu(A).
\end{align*}
\item Sei $f\le g\mufu$ auf $A$, so ist
\begin{align*}
\int_A f\dmu \le \int_A g\dmu,
\end{align*}
falls beide Integrale existieren.
\item Sei $f\in L^1(A)$ und $B\subseteq A$, so ist $f\in L^1(B)$ und
\begin{align*}
\int_B \abs{f}\dmu \le \int_A \abs{f}\dmu.
\end{align*}
\item Sei $\mu(A) = 0$, so ist $\int_A f\dmu = 0$.
\item Sei $f\in L^1(A)$ und für jedes $B\in \Sigma$ gelte
\begin{align*}
B\subseteq A \Rightarrow \int_B f\dmu = 0,
\end{align*}
so ist $f=0\mufu$ auf $A$.
\item Sei $A\cap B = \varnothing$, $f\in L^1(A)$ und $f\in L^1(B)$, so ist
$f\in L^1(A\cap B)$ und
\begin{align*}
\int_{A\cup B} f \dmu = \int_A f\dmu + \int_B f\dmu.\fishhere
\end{align*} 
\end{propenum}
\end{prop}

\subsection{Konvergenzsätze}

\begin{prop}[Satz von der monotonen Konvergenz]
\label{prop:7.15}
Seien $A\in\Sigma$, $f_n : A\to [0,\infty]$ messbar und $0\le f_1(x)\le f_2(x)
\le \ldots$ auf $A$. Dann gilt
\begin{align*}
\lim\limits_{n\to\infty}\int_A f_n\dmu = 
\int_A \lim\limits_{n\to\infty }f_n \dmu.\fishhere
\end{align*}
\end{prop}

\begin{prop}[Lemma von Fatou]
\label{prop:7.16}
Seien $A\in\Sigma$, $f_n: A\to [0,\infty]$ messbar. Dann gilt
\begin{align*}
\int_A \liminf_n f_n \dmu \le \liminf_n \int_A f_n\dmu.\fishhere
\end{align*}
\end{prop}

\begin{prop}[Satz von der majorisierten Konvergenz]
\label{prop:7.17}
Seien $A\in\Sigma$, $f_n,f: A\to \RA$ messbar, $f_n\to f$. Existiert ein $g:
A\to \RA$ messbar mit $\int_A g\dmu < \infty$ und $\abs{f_n}\le g$, so gilt
\begin{align*}
\lim\limits_{n\to\infty} \int_A \abs{f_n-f}\dmu = 0.\fishhere 
\end{align*}
\end{prop}

\section{$L^p$-Räume}

Für alles Weitere sei $\Omega\subseteq \R^n$ ein Gebiet.

\begin{defn}
\label{defn:7.18}
\begin{defnenum}
\item Für $1\le p < \infty$ sei
\begin{align*}
L^p(\Omega) := \setdef{u:\Omega\to\C \text{ messbar}}{\norm{u}_p := \left(
\int_\Omega \abs{u}^p \dmu
\right)^{1/p} < \infty}.
\end{align*}
Identifizieren wir Funktionen, die sich nur auf Nullmengen unterscheiden, wird
$\norm{\cdot}_p$ zur Norm (Minkowski Ungleichung).
\item $u: \Omega\to \C$ heißt \emph{wesentlich beschränkt}, falls
\begin{align*}
\exists K > 0 : \abs{u}\le K\mufu
\end{align*} 
In diesem Fall heißt
\begin{align*}
\esssup(\abs{u}) := \inf\setdef{K>0}{\abs{u}\le K \mufu}
\end{align*}
\emph{wesentliches Supremum} von $u$.
\begin{align*}
L^\infty(\Omega) := \setdef{u:\Omega \to \C\text{ messbar}}{\norm{u}_\infty :=
\esssup\abs{u} < \infty}.
\end{align*}
Durch die obige Identifikation wird $\norm{\cdot}_\infty$ zur Norm.\fishhere
\end{defnenum}
\end{defn}

\begin{prop}
\label{prop:7.19}
\begin{propenum}
\item Für $1\le p\le \infty$ ist $\norm{\cdot}_p$ eine Norm.
\item Für $1\le p \le \infty$ ist $L^p(\Omega)$ ein Banachraum. $L^2(\Omega)$
ist mit
\begin{align*}
\lin{u,v} = \int_\Omega u\overline{v}\dmu
\end{align*}
ein Hilbertraum.\fishhere
\end{propenum}
\end{prop}

\begin{prop}
\label{prop:7.20}
\begin{propenum}
\item Die Menge
\begin{align*}
C_0^\infty (\Omega \to \C) :=
\setdef{\ph\in C^\infty(\Omega\to \C)}{\supp \ph := \overline{\setdef{x\in
\Omega}{\ph(x)\neq 0}}\text{ ist kompakt}}
\end{align*}
ist dicht in $L^p(\Omega)$ im Fall $1\le p < \infty$.
\item $L^p(\Omega)$ ist separabel.\fishhere
\end{propenum}
\end{prop}

Damit erhalten wir auch einen alternativen Zugang zum $L^p(\Omega)$ als
Abschluss von $C_0^\infty(\Omega\to\C)$ bezüglich der $\norm{\cdot}_p$-Norm.

\begin{proof}
Den Beweis verschieben wir auf später.\qedhere
\end{proof}

\begin{prop}
Sei $1\le p < \infty$. Die Abbildung
\begin{align*}
\Phi : L^q(\Omega)\to L^p(\Omega)'
\end{align*}
mit
\begin{align*}
\Phi(u)(f) = F_u(f) = \int_\Omega f\overline{u}\dmu
\end{align*}
ist eine isometrische antilineare Bijektion.\fishhere
\end{prop}

\begin{proof}
\begin{proofenum}
\item Seien $f\in L^p(\Omega)$, $u\in L^q(\Omega)$, so folgt mit der
Hölder-Ungleichung, dass $f\cdot \overline{u}\in L^1(\Omega)$, sowie
\begin{align*}
\int_\Omega \abs{f\overline{u}}\dmu \le \norm{f}_p \norm{u}_q.
\end{align*}
Somit ist $\Phi(u)$ wohldefiniert und linear, da das Integral linear ist. Es ist
sofort ersichtlich, dass
\begin{align*}
\norm{\Phi(u)} \le \norm{u}_q,
\end{align*}
insbesondere ist $\Phi(u)\in L^q(\Omega)$.
\item Setzen wir
\begin{align*}
f(x) = 
\begin{cases}
\abs{u(x)}^{\frac{q}{p}-1}u(x), & u(x)\neq 0,\\
0, & \text{sonst},
\end{cases}
\end{align*}
so ist 
\begin{align*}
\norm{f}_p^p = \int_\Omega \abs{f}^p \dmu = \int_\Omega \abs{u}^q \dmu =
\norm{u}_q^p.
\end{align*}
Folglich ist auch
\begin{align*}
\abs{\Phi(u)(f)} &= \abs{\int_\Omega f\overline{u}\dmu}
=
\abs{\int_\Omega \abs{u}^{\frac{q}{p}-1} u\overline{u}\dmu}
= 
\abs{\int_\Omega \abs{u}^{\frac{q}{p}+1}\dmu}\\
&=
\int_\Omega \abs{u}^{q}\dmu
= \norm{u}_q^q \overset{!}{=}\norm{f}_p\norm{u}_q,
\end{align*}
denn $\norm{f}_p\norm{u}_q = \norm{u}_q^{\frac{q}{p}+1} = \norm{u}_q^q$, also
existiert ein $f\in L^p(\Omega)$ mit
\begin{align*}
\abs{\Phi(u)(f)} = \norm{f}_p=\norm{u}_q
\end{align*}
Somit ist $\Phi$ normerhaltend.
\item $\Phi(\alpha u + \beta v) = \overline{\alpha}\Phi(u) +
\overline{\beta}\Phi(v)$ ist klar.
\item $\Phi$ ist injektiv, da isometrisch. Zur Surjektivität siehe \cite{Alt99}
Kapitel 4.3. Der Beweis basiert auf dem Satz von Radon-Nikodym.\qedhere
\end{proofenum}
\end{proof}

\begin{cor}
\label{prop:7.22}
Für $1<p<\infty$ ist $L^p(\Omega)$ reflexiv.\fishhere
\end{cor}
\begin{proof}
Analoger Beweis zu \ref{prop:5.18}.\qedhere
\end{proof}

\begin{prop}
\label{prop:7.23}
$L^p(\Omega)' = (C_0^\infty(\Omega\to\C),\norm{\cdot}_p)'$.\fishhere
\end{prop}

Insbesondere ist $F_u$ (mit $u\in L^q(\Omega)$) und damit auch $u\in
L^q(\Omega)$ eindeutig bestimmt durch die Werte $F_u(\ph)$ für $\ph\in
C_0^\infty(\Omega\to\C)$.

\begin{proof}
Zu $F\in L^p(\Omega)'$ ist
\begin{align*}
F\Big|_{C_0^\infty(\Omega\to\C)} \in 
(C_0^\infty(\Omega\to\C),\norm{\cdot}_p)'.
\end{align*}
Umkehert existiert zu $f\in (C_0^\infty(\Omega\to\C),\norm{\cdot}_p)'$ nach dem
Fortsetzungssatz \ref{prop:2.5} eine eindeutige Fortsetzung $F\in
L^p(\Omega)'$, da $C_0^\infty(\Omega\to\C)$ dicht in $L^p(\Omega)$.\qedhere 
\end{proof}

\section{Verallgemeinerung der Ableitung I}

Sei wieder $\Omega\subset\R^n$ ein Gebiet.

\begin{bsp}
\label{bsp:7.24}
Seien $\Omega=\R^n$, $f\in C^1(\R^n\to \C)$ und $f,\partial_{x_i} f \in
L^p(\Omega)$. Identifiziere $L^p(\Omega)$ mit $L^q(\Omega)'$, also $F_f$ mit
$f$. Dann sind $\partial_{x_i} f$ bzw $F_{\partial_{x_i} f}$ eindeutig bestimmt
durch die Werte auf $C_0^\infty(\Omega\to\C)$.
\begin{align*}
F_{\partial_{x_i}f}(\ph) &=
\int_{\R^n} \ph\overline{\partial_{x_i}f}\dmu = 
\int_{\supp \ph} \ph\overline{\partial_{x_i}f}\dmu
\overset{\text{part.int.}}{=}
-\int_{\supp \ph} \partial_{x_i}\ph\overline{f}\dmu\\
&= -F_{f}(\partial_{x_i}\ph).\bsphere
\end{align*}
\end{bsp}

\begin{defn}
\label{defn:7.25}
Seien $1\le p < \infty$, $f\in L^p(\Omega)$, $\alpha\in \N_0^n$ und $u\in
L^p(\Omega)$ mit
\begin{align*}
\forall \ph \in C_0^\infty(\Omega\to \C) : F_u(\ph) =
(-1)^{\abs{\alpha}}F_f(\nabla^\alpha \ph) 
\end{align*}
bzw.
\begin{align*}
\int_\Omega \ph\overline{u}\dmu =
(-1)^{\abs{\alpha}}\int_\Omega \nabla^\alpha \ph \overline{f}\dmu.
\end{align*}
Dann heißt $u$ \emph{schwache Ableitung}\index{schwache!Ableitung} von $f$ der
Ordnung $\alpha$. Schreibweise
\begin{align*}
u:= \D^\alpha f.\fishhere
\end{align*}
\end{defn}
Für $\ph\in C_0^\infty(\Omega\to\C)$ gilt also
\begin{align*}
\lin{\D^\alpha f,\ph} = 
\int_\Omega (\D^\alpha f)\overline{\ph} \dmu = 
(-1)^{\abs{\alpha}}\int_\Omega f \nabla^\alpha \overline{\ph}\dmu = 
(-1)^{\abs{\alpha}}\lin{f, \nabla^\alpha\ph}.
\end{align*}

Für alles Folgende seien - sofern nicht anders angegeben -
$\alpha,\beta\in\N_0^n$.

\begin{cor}
\label{prop:7.26}
Sei $f\in C^k(\Omega\to\C)$ mit $\nabla^\alpha f \in L^p(\Omega)$ für
$\abs{\alpha}\le k$. Dann gilt
\begin{align*}
\nabla^\alpha f = \D^\alpha f.\fishhere
\end{align*}
\end{cor}
\begin{proof}
Sei $\ph\in C_0^\infty(\Omega\to\C)$,
\begin{align*}
\int_\Omega \ph \overline{\nabla^\alpha f}\dmu
\overset{\text{part.int.}}{=}
 (-1)^{\abs{\alpha}}\int_\Omega \nabla^\alpha \ph \overline{f}\dmu.\qedhere
\end{align*}
\end{proof}

\begin{bsp}
\label{bsp:7.27}
Sei $\Omega=(a,b)$ mit $0\in (a,b)$ und $f(x) = \abs{x}$. Bestimme $\D^1 f$.
\begin{align*}
\int_a^b \ph(x) \overline{D^1 f(x)}\dx
&= - \int_a^b \ph'(x)\overline{f(x)}\dx \\ &= 
\int_a^0 x\ph'(x)\dx - \int_0^b x\ph'(x)\dx\\
&\overset{\text{part.int.}}{=}
-\int_a^0 \ph(x)\dx + \int_0^b \ph(x)\dx\\
&= \int_a^b \sign(x)\ph(x)\dx,
\end{align*}
wobei
\begin{align*}
\sign(x) = 
\begin{cases}
-1, & x < 0,\\
1, &  x > 0.
\end{cases}
\end{align*}
Die Definition von $\sign$ im Nullpunkt ist irrelevant.

Somit ist für jedes $\ph\in C_0^\infty(\Omega\to \C)$
\begin{align*}
\int_a^b \ph \overline{\D^1 f}\ph\dmu = 
\int_a^b \ph \overline{\sign(x)}\ph\dmu 
\end{align*}
also $\D^1 f = \sign(x)$ in $L^p(\Omega)$.

$\sign(x)$ ist jedoch in $L^p(\Omega)$ nicht schwach differenzierbar (vgl.
Distributionentheorie).\bsphere
\end{bsp}


\begin{defn}
\label{defn:7.28}
Sei $1\le p < \infty$ und $m\in\N$. Dann heißt
\begin{align*}
W^{m,p}(\Omega) := \setdef{u\in L^p(\Omega)}{\D^\alpha u \in L^p(\Omega),\quad
\forall \abs{\alpha}\le m}
\end{align*}
versehen mit der Norm
\begin{align*}
\norm{u}_{m,p} = \left(\sum_{\abs{\alpha}\le m} \norm{\D^\alpha
u}_p^p\right)^{1/p}
\end{align*}
\emph{Sobolevraum}\index{Sobolevraum} der Ordnung $m$.\fishhere
\end{defn}

\begin{prop}
\label{prop:7.29}
$W^{m,p}(\Omega)$ ist ein Banachraum. $W^{m,2}(\Omega)$ ein
Hilbertraum.\fishhere
\end{prop}
\begin{proof}
\begin{proofenum}
\item Sei $(u_n)$ Cauchyfolge in $W^{m,p}(\Omega)$, so ist für $\abs{\alpha}\le
m$ auch $(\D^\alpha u)$ Cauchyfolge in $L^p(\Omega)$ und damit konvergent,
$\D^\alpha u_n \to g_\alpha$.

Wir zeigen nun $\D^\alpha g_0 = g_\alpha$ und $\norm{u_n-g_0}_{m,p} \to 0$.

Für $\ph\in C_0^\infty(\Omega)$ gilt
\begin{align*}
F_{g_\alpha}(\ph) &= \lim\limits_{n\to\infty} F_{\D^\alpha u_n}(\ph)
= \lim\limits_{n\to\infty}(-1)^{\abs{\alpha}}F_{u_n}(\nabla^\alpha \ph)\\
&= (-1)^{\abs{\alpha}} F_{g_0}(\ph)(\nabla^\alpha \ph) = F_{\D^\alpha g_0}(\ph).
\end{align*}
Also ist $g_\alpha = \D^\alpha g_0$ und daher
\begin{align*}
\norm{u_n-g_0}_{m,p}^p
= \sum_{\abs{\alpha}\le m}\norm{\D^\alpha u_n - \D^\alpha g_0}_{p}^p
=
\sum_{\abs{\alpha}\le m}\norm{\D^\alpha u_n - g_\alpha}_{p}^p
\to 0.
\end{align*}
\item $W^{2,m}$ ist Hilbertraum mit dem Skalarprodukt
\begin{align*}
\lin{u,v} = \sum_{\abs{\alpha}\le m} \lin{\D^\alpha u,\D^\alpha v}.\qedhere
\end{align*}
\end{proofenum}

\end{proof}

\begin{prop}
\label{prop:7.30}
$W^{m,p}(\Omega)$ ist separabel und reflexiv.\fishhere
\end{prop}
\begin{proof}
Sei $\alpha_1,\ldots,\alpha_k$ eine Nummerierung aller $\alpha$ mit
$\abs{\alpha}\le m$. Setze
\begin{align*}
P : W^{m,p}(\Omega)\to L^p(\Omega)^k,\quad
u\mapsto (\D^{\alpha_1}u,\ldots,\D^{\alpha_k}u),  
\end{align*}
so ist $P$ linear und isometrisch, denn
\begin{align*}
\norm{u} = 
\left(\sum_{j=1}^k \norm{\D^{\alpha_j}u}_p^p \right)^{1/p}
 = \norm{(\D^{\alpha_1}u,\ldots,\D^{\alpha_k}u)}_{L^p(\Omega)^k}.
\end{align*}
Diese Norm ist äquivalent zur Standard-Norm des $L^p(\Omega)^k$ und daher ist
$L^p(\Omega)^k$ auch bezüglich dieser Norm separabel und reflexiv.

$P$ ist Isometrie, also ist $W^{m,p}(\Omega)\cong \im P \subseteq L^p(\Omega)^k$
ein Banachraum und damit abgeschlossen. Da $L^p(\Omega)^k$ separabel, ist
somit auch $W^{m,p}(\Omega)$ separabel. Mit \ref{prop:4.23} folgt außerdem, dass
$W^{m,p}(\Omega)$ reflexiv.\qedhere
\end{proof}

Wir wollen nun die Produktregel auf die schwache Ableitung verallgemeinern. Für
$u,v\in L^p(\Omega)$ ist jedoch nicht zwingend $u\cdot v\in L^p(\Omega)$. Somit
ist klar, dass die Produktregel nur unter zusätzlichen Voraussetzungen an $u$
und $v$ gelten kann.

\begin{prop}[Produktregel]
\index{schwache!Produktregel}
\label{prop:7.31}
Sei $m\in\N$, $1\le p < \infty$, $u\in W^{m,p}(\Omega)$ und $f\in
C_0^\infty(\Omega\to \C)$. Dann gelten
\begin{propenum}
\item $f\cdot u\in W^{m,p}$.
\item $\forall \abs{\alpha}\le m : \D^\alpha (f\cdot u) = \sum_{\beta\le
\alpha} \binom{\alpha}{\beta} (\nabla^\beta f) (\D^{\alpha-\beta}u)$.\fishhere
\end{propenum}
\end{prop}

\begin{proof}
Sei $m=1$ und $\alpha=e_j$, so gilt für $\ph\in C_0^\infty(\Omega\to\C)$,
\begin{align*}
-\lin{f\cdot u,\nabla^{e_j}\ph} &= - \int_\Omega f\cdot u
\overline{\partial_{x_j}\ph}\dmu\\
&= - \int_\Omega u
\partial_{x_j}(f\overline{\ph})\dmu
+ \int_\Omega u
(\partial_{x_j}f)\overline{\ph}\dmu\\
&=
\int_\Omega \D^{e_j} u\,
f\,\overline{\ph}\dmu
+ \int_\Omega u\,
\partial_{x_j}f\,\overline{\ph}\dmu\\
&= \lin{f D^{e_j}u + \partial_{x_j}f u,\ph}
\end{align*}
Somit folgt, $\D^{e_j} (f\cdot u) = f\D^{e_j} u + \partial_{x_j}f\, u$ und
insbesondere, $\D^{e_j}(f\cdot u)\in L^p(\Omega)$, also $f\cdot u\in W^{1,p}$.

Der Rest folgt mit Induktion über $m$.\qedhere
\end{proof}

\begin{defn}
\label{defn:7.32}
Sei die partielle Differentialgleichung
\begin{align*}
\sum_{\abs{\alpha}\le m} a_\alpha\, \nabla^\alpha u = f,\qquad
a_\alpha\in C_0^\infty(\Omega\to\C),\quad f\in L^p(\Omega)
\end{align*}
gegeben. Falls $u\in L^p(\Omega)$ und für alle $\ph\in C_0^\infty(\Omega\to\C)$
gilt,
\begin{align*}
\sum_{\abs{\alpha}\le m} (-1)^{\abs{\alpha}}
\lin{u,\nabla^\alpha(\overline{a_\alpha}\ph)} = \lin{f,\ph},
\end{align*}
so heißt $u$ \emph{schwache Lösung}\index{schwache!Lösung} der
Differentialgleichung.\fishhere
\end{defn}

In obiger Definition wird nicht $u\in W^{m,p}(\Omega)$ vorausgesetzt, denn
falls zusätzlich $u\in W^{m,p}(\Omega)$ und $u$ schwache Lösung, gilt sogar
\begin{align*}
\sum_{\abs{\alpha}\le m} a_\alpha\, \D^\alpha u = f.
\end{align*}

\newcommand{\WO}{W_0}
\begin{defn}
\label{defn:7.33}
$\WO^{m,p}(\Omega) :=
\overline{C_0^\infty(\Omega\to\C)}^{W^{m,p}(\Omega)}$.\fishhere
\end{defn}

$\WO^{m,p}(\Omega)$ ist als abgeschlossene Teilmenge des Banachraums
$W^{m,p}(\Omega)$ selbst ein Banachraum und $\WO^{m,2}(\Omega)$ ein Hilbertraum.


\begin{prop}[Veranschaulichung]
\label{prop:7.34}
Sei $1<p<\infty$ und $u\in W_0^{1,p}((0,1))\cap C([0,1])$. Dann gilt $u(0) =
u(1)=0$. Allgemeiner: Ist $u\in W_0^{m,p}((0,1))\cap C^{m-1}([0,1])$, so gilt
\begin{align*}
\forall j=0,\ldots,m-1 : u^{(j)}(0) = 0 = u^{(j)}(1).\fishhere
\end{align*}
\end{prop}
\begin{proof}
$u$ ist stetig auf $[0,1]$, d.h. insbesondere ist
\begin{align*}
u(0) =\lim\limits_{h\to0}\frac{1}{h} \int_0^h u(x)\dx
\end{align*}
und da $u\in W_0^{1,p}((0,1))$, existiert eine Folge $(u_n)$ in
$C_0^\infty((0,1)\to\C)$ mit $\norm{u_n-u}_{1,p}\to 0$, wobei
\begin{align*}
u_n(x) = \int_0^x u_n'(s)\ds \le
\left(\int_0^x 1\ds\right)^{1/q}
\left(\int_0^x \abs{u_n'(s)}^p\ds\right)^{1/p}
\le C\cdot x^{1/q},
\end{align*}
mit $C=\sup\limits_{n\ge 1} \norm{u_n}_p < \infty$. Sei $A\subset[0,1]$
messbar, so gilt $\norm{(u_n-u)\chi_A}_1 \le
\underbrace{\norm{\chi_A}_q}_{\const}\underbrace{\norm{u_n-u}_p}_{\to 0}$. Somit
gilt
\begin{align*}
\abs{\int_0^h u(x)\dx} = \lim\limits_{n\to\infty}
\abs{\int_0^h u_n(x)\dx} \le C \int_0^h x^{1/q}\dx = C\frac{h^{1+1/q}}{1+1/q}.
\end{align*}
also
\begin{align*}
\abs{u(0)} = \lim\limits_{h\to 0}\frac{1}{h}
\abs{\int_0^h u(x)\dx} \le
\lim\limits_{h\to 0}\frac{1}{h}
C\frac{h^{1+1/q}}{1+1/q} = 0.
\end{align*}
$u(1)=0$ folgt analog.\qedhere
\end{proof}

\begin{prop}
\label{prop:7.35}
Das Dirichlet Problem
\begin{align*}
-\Delta u + u = f\text{ in }\Omega,\qquad u\big|_{\delta \Omega} = 0
\end{align*}
besitzt für $f\in L^2(\Omega)$ in $L^2(\Omega)$ eine eindeutig bestimmte
schwache Lösung $u$ mit $u\in W_0^{1,2}(\Omega)$, so dass für alle $\ph\in
C_0^\infty(\Omega\to\C)$ gilt
\begin{align*}
-\lin{u,\Delta \ph} + \lin{u,\ph} = \lin{f,\ph}.\fishhere
\end{align*}
\end{prop}
\begin{proof}
\textit{Eindeutigkeit}.
Seien $u,v$ Lösungen der inhomogenen Gleichung, so ist $u-v$ Lösung der
homogenen Gleichung. Es genügt also sich auf den homogenen Fall zu beschränken.

Sei $u\in W_0^{1,2}(\Omega)$ Lösung von $-\Delta u + u = 0$ und $(u_n)$
Folge in $C_0^\infty(\Omega\to\C)$ mit $\norm{u_n-u}_{1,2}\to 0$, so gilt
\begin{align*}
0 &= -\lin{u,\Delta u_n} + \lin{u,u_n} = \sum_{j=1}^n
\lin{\D^{e_j}u,\partial_{x_j}u_n} + \lin{u,u_n}\\
&\to \lin{\D^{e_j}u,\D^{e_j}u} + \lin{u,u} = \norm{u}_{1,2}^2.
\end{align*}
Folglich ist $u=0$ und damit die Lösung eindeutig.

\textit{Existenz}. Sei $F: W_0^{1,2}(\Omega)\to \C$ mit
\begin{align*}
F(v) = \lin{v,f},
\end{align*}
so ist $F\in  W_0^{1,2}(\Omega)'$, denn
\begin{align*}
\abs{F(v)} \le \norm{v}_{0,2}\norm{f}_2 \le \norm{f}_2 \norm{v}_{1,2}.
\end{align*}
Das Lemma von Riesz besagt nun, dass es genau ein $u\in W_0^{1,2}(\Omega)$
existiert so dass für alle $v\in W_0^{1,2}(\Omega)$ gilt
\begin{align*}
\lin{v,f}_2 = F(v) = \lin{v,u}_{1,2}
= \sum_{j=1}^n \lin{\D^{e_j}v,\D^{e_j}u} + \lin{v,u}.
\end{align*}
Für $v=\ph\in C_0^\infty(\Omega\to\C)$ folgt
\begin{align*}
\lin{f,\ph} = \sum_{j=1}^n \lin{\D^{e_j}u,\nabla^{e_j}\ph} + \lin{u,\ph}
= -\sum_{j=1}^n \lin{u,\Delta\ph} + \lin{u,\ph}.\qedhere
\end{align*}
\end{proof}

\begin{bem}
\label{bem:7.36}
Offensichtlich gilt
\begin{align*}
\sum_{j=1}^n \D^{2e_j}u = f-u \in L^2(\Omega).
\end{align*}
Die elliptische Regularitätstheorie zeigt, dass sogar $u\in W^{2,2}(\Omega)$,
falls $\Omega$ ``genügend gut''.\maphere
\end{bem}

\begin{prop}
\label{prop:7.37}
Sei $1<p<\infty$, $u\in L^p(\Omega)$ mit $\D^\alpha u=0$ für $\abs{\alpha}=1$.
Dann folgt $u=c$. Falls $\mu(\Omega)=\infty$ oder $u\in W_0^{1p}(\Omega)$ folgt
sogar $u=0$.\fishhere
\end{prop}
\begin{proof}
\begin{proofenum}
\item Sei $n=1$, $\Omega=(a,b)$ mit $-\infty\le a < b \le \infty$. Sei $\psi\in
C_0^\infty((a,b)\to \C)$ mit
\begin{align*}
\int_a^b \psi(x)\dx = 1,
\end{align*}
so lässt sich jedes $\ph\in C_0^\infty((a,b)\to\C)$ schreiben als
\begin{align*}
\ph = \tilde{\ph} + \left(\int_a^b \ph(t)\dt\right)\psi.
\end{align*}
Insbesondere ist $\int_a^b \tilde{\ph}\dmu = 0$ und $\tilde{\ph}\in C_0^\infty
((a,b)\to\C)$. Setze nun
\begin{align*}
\Phi(x) := \int_a^x \tilde{\ph}(t)\dt,
\end{align*}
so ist $\Phi\in C_0^\infty((a,b)\to\C)$, denn
\begin{align*}
\Phi(x) = 0
\begin{cases}
\text{für } a<x \le \min\supp\tilde{\ph},\\
\text{für } \max\supp\tilde{\ph}\le x < b.
\end{cases}
\end{align*}
Damit können wir $\ph$ darstellen durch,
\begin{align*}
\ph = \Phi' + \left(\int_a^b \ph(t)\dt\right)\psi. 
\end{align*}

Sei nun $u$ wie vorausgesetzt, so gilt
\begin{align*}
F_u(\ph) &= \lin{u,\ph} = \lin{u,\Phi'}
+ \int_a^b \overline{\ph}(t)\dt \lin{u,\psi}\\
&= -\underbrace{\lin{D^1u,\Phi}}_{=0} + \int_a^b \overline{\ph}(t)\dt
\underbrace{\lin{u,\psi}}_{=c} = \lin{c,\ph}.
\end{align*}
Und da $\ph$ beliebig war, ist $u=c$.
\item Wir behandlen exemplarisch den Fall $n=2$, die übrigen Fälle ergeben sich
dann automatisch.
\begin{enumerate}[label=\alph{*}),leftmargin=0pt]
  \item Sei $(x_1,x_2)\in\Omega$ und $W_\ep :=
  (x_1-\ep,x_1+\ep)\times(x_2-\ep,x_2+\ep)\subseteq \Omega$. Wähle $\psi_j\in
  C_0^\infty((x_j-\ep,x_j+\ep)\to\C)$ mit
\begin{align*}
\int_{x_j-\ep}^{x_j+\ep} \psi_j(x) = 1,\qquad j=1,2.
\end{align*}
Wir zeigen nun, dass $u=0$ auf $W_\ep$.

Für festes $y_2\in (x_2-\ep,x_2+\ep)$ besitzt $\ph\in C_0^\infty(W_\ep\to\C)$
analog zum Vorangegangenen eine Darstellung,
\begin{align*}
\ph(y_1,y_2) = F_{y_2}'(y_1) + \underbrace{\int_{x_1-\ep}^{x_1+\ep}
\ph(t,y_2)\dt}_{:=g(y_2)} \psi_1(y_1)
\end{align*}
mit $F_{y_2}\in C_0^\infty((x_1-\ep,x_1+\ep)\to\C)$. Aus dem ersten Teil folgt
ebenfalls, dass ein $G\in C_0^\infty((x_2-\ep,x_2+\ep)\to\C)$ existiert, so dass
\begin{align*}
g(y_2) = G'(y_2) + \int_{x_2-\ep}^{x_2+\ep} g(s)\ds \psi_2(y_2).
\end{align*}
Wir können somit $\ph$ schreiben als
\begin{align*}
\ph(y_1,y_2) =
\underbrace{F_{y_2}'(y_1)}_{\partial_1 \Phi_1(y_1,y_2)}
% \partial_1 \Phi_1(y_1,y_2)
+ \underbrace{G'(y_2)\psi_1(y_1)}_{\partial_2 \Phi_2(y_1,y_2)}
+ \int_{W_\ep} \ph\dmu\, \psi_1(y_1)\psi_2(y_2).
\end{align*}
Somit gilt (wobei $\lin{\cdot,\cdot}_\ep$ das Integral über $W_\ep$ beschreibt),
\begin{align*}
\lin{u,\ph}_\ep &= \lin{u,\partial_1 \Phi_1}_\ep
+ \lin{u,\partial_2 \Phi_2}_\ep +
\left(\int_{W_\ep}
\overline{\ph}\dmu\right)\underbrace{\lin{u,\psi_1\psi_2}_\ep}_{:=c}\\
&=\lin{u,\partial_1 \Phi_1}_\ep
+ \lin{u,\partial_2 \Phi_2}_\ep + \lin{c,\ph}_\ep
\end{align*}
Setzen wir $\Phi_1,\Phi_2$ durch Null auf ganz  $\Omega$ fort, so gilt
\begin{align*}
\lin{u,\ph}_\ep &= \lin{u,\partial_1 \Phi_1}
+ \lin{u,\partial_2 \Phi_2} +  \lin{c,\ph}_\ep
\\ &=
\underbrace{\lin{\D^{e_1}u, \Phi_1}
+ \lin{\D^{e_2}u,\Phi_2}}_{=0} + 
  \lin{c,\ph}_\ep.
\end{align*}
Also ist $u=c$ auf $W_\ep$.
\item Seien $x,y\in\Omega$. Da $\Omega$ wegzusammenhängend, existiert ein
Polygonzug $\Gamma$ von $x$ nach $y$, der ganz in $\Omega$ verläuft. Nun ist
$\Gamma$ kompakt, also kann ganz $\Gamma$ mit endlich vielen $W_\ep^{(j)}$
überdeckt werden, $j=1,\ldots,J$. Auf jedem $W_\ep^{(j)}$ ist $u$ konstant.

Da $\Gamma$ alle $W_\ep^{(j)}$ durchläuft und aneinandergrenzende $W_\ep^{(j)}$
nichtleeren Schnitt haben, ist $u=c$ mit einer einzigen Konstanten für alle
$W_\ep^{(j)}$. Somit ist $u=c$ auf $\Omega$.
\end{enumerate}
\item Sei $u=c$ und $\mu(\Omega)=\infty$, so gilt
\begin{align*}
\int_\Omega \abs{u}^p \dmu = \abs{c}^p\mu(\Omega)< \infty,
\end{align*}
also ist $c=0$.
\item Sei $u\in W_0^{1,p}(\Omega)$ und $\mu(\Omega) < \infty$. Eine leichte
Übung zeigt
\begin{align*}
\lin{\D^{e_j}v,u} = -\lin{v,\D^{e_j}u},
\end{align*}
für $v\in W^{1,p}(\Omega)$.  Setzen wir $v(x) = \arctan(x_1)$, so ist
\begin{align*}
\partial_1 v(x) = \frac{1}{1+x_1^2},\qquad \partial_j v(x) = 0,\quad
j=2,\ldots,n.
\end{align*}
Somit ist $v\in C^1(\Omega\to\R)$ und $v,\nabla v$ sind beschränkt, also ist
$v\in W^{1,p}(\Omega)$, da $\mu(\Omega)<\infty$.

Nach Voraussetzung ist $\lin{v,\D^{e_j}u} = 0$, also gilt
\begin{align*}
0 = -\lin{v,\D^{e_1}u} = \lin{D^{e_1}v,u} = \lin{\partial_1 v, c}
= \overline{c} \underbrace{\int_\Omega \frac{1}{1+x_1^2}\dx}_{>\mu(\Omega)\neq
0}
\end{align*}
und folglich ist $c=0$.\qedhere
\end{proofenum}
\end{proof}

\section{Verallgemeinerung der Ableitung II}

Für alles Weitere sei wieder $\Omega\subseteq\R^n$ ein Gebiet.

\begin{defn}
\label{defn:7.38}
\begin{defnenum}
\item Seien $1\le p<\infty$, $u\in L^p(\Omega)$ und
$\alpha\in\N_0^n$. Falls eine Folge $(u_n)$ in $C^\infty(\Omega\to\C)$
existiert mit $u_n\to u$ und $\nabla^\alpha u_n \to v$ in $L^p(\Omega)$,
so heißt $v$ \emph{starke Ableitung}\index{starke Ableitung} von $u$ der
Ordnung $\alpha$. Schreibe $\D^\alpha_s u := v$.
\item $H^{m,p}(\Omega) := \overline{\setdef{\ph\in
C^\infty(\Omega\to\C)}{\norm{\ph}_{m,p}<\infty}}$ heißt \emph{$m$-ter
Sobolevraum}\index{Sobolevraum}.

$H_0^{m,p}(\Omega) := W_0^{m,p}(\Omega)$.\fishhere
\end{defnenum}
\end{defn}

\begin{prop}
\label{prop:7.39}
\begin{propenum}
\item Sei $u\in L^p(\Omega)$ mit $\D_s^\alpha u\in L^p(\Omega)$ für ein
$\alpha\in\N_0^n$, so stimmen schwache und starke Ableitung überein,
\begin{align*}
\D_s^\alpha u = \D^\alpha u.
\end{align*}
\item $H^{m,p}(\Omega) \subseteq W^{m,p}(\Omega)$ und $H^{m,p}(\Omega)$ ist ein
Banachraum.\fishhere
\end{propenum}
\end{prop}
\begin{proof}
\begin{proofenum}
\item Sei $\ph\in C_0^\infty(\Omega\to\C)$ und $u$ wie vorausgesetz, dann
\begin{align*}
\lin{\D^\alpha u,\ph} &= (-1)^{\abs{\alpha}}\lin{u,\nabla^\alpha \ph}
= \lim\limits_{n\to\infty}
(-1)^{\abs{\alpha}}\lin{u_n,\nabla^\alpha \ph}\\
&= \lim\limits_{n\to\infty}
(-1)^{\abs{\alpha}}\int_\Omega u_n \overline{\nabla^\alpha \ph}\dmu
= \lim\limits_{n\to\infty}
\int_\Omega \nabla^\alpha u_n \overline{\ph}\dmu\\
&=\lin{\D_s^\alpha u,\ph}.
\end{align*}
Da die rechte Seite existiert, existiert auch die linke und somit ist
$\D^\alpha u = \D^\alpha_s u$.
\item Sei $\ph\in C^\infty(\Omega\to\C)$ und $\norm{\ph}_{m,p}< \infty$, so
ist $\ph\in W^{m,p}(\Omega)$. Also ist $H^{m,p}(\Omega)$ sinnvoll definiert und
$\subseteq W^{m,p}(\Omega)$. Da $H^{m,p}(\Omega)$ abgeschlossener Teilraum von
$W^{m,p}(\Omega)$ ist $H^{m,p}(\Omega)$ Banachraum.\qedhere
\end{proofenum}
\end{proof}

\begin{prop}[Satz von Meyers und Serrin (1964)]
\index{Satz!Meyers und Serrin}
Für $1\le p < \infty$ und $m\in\N$ ist
\begin{align*}
W^{m,p}(\Omega)=H^{m,p}(\Omega).\fishhere
\end{align*}
\end{prop}
\begin{proof}
Zum Beweis siehe \cite{Adams75}.\qedhere
\end{proof}

\begin{defn}
\label{defn:7.41}
Gilt $u\in H^{m,p}(\Omega)$ und
\begin{align*}
\sum_{\abs{\alpha}\le m} a_\alpha \D^\alpha u = f,\qquad
\qquad
a_\alpha\in C_b(\Omega\to\C),\quad f\in L^p(\Omega).
\end{align*}
Dann heißt $u$ \emph{starke Lösung}\index{starke Lösung} der
Differentialgleichung.

Gilt sogar $u\in C^m(\Omega\to\C)$, so heißt $u$ \emph{klassische
Lösung}\index{klassische Lösung}.\fishhere
\end{defn}

\section{Approximation}

Für alles Weitere sei $\Omega\subset O$ offen.

\begin{defn}
\label{defn:7.42}
\begin{defnenum}
\item Seien $M$ und $K$ Mengen mit $K\subset M$ und $K$ kompakt, so schreiben
wir $K\Subset M$.\index{kompakt enthalten}
\item Sei $A\subset\R^n$ messbar, dann ist
\begin{align*}
L^1_\loc(A) := \setdef{u: A\to\C}{\forall K\Subset A : u\big|_K \in L^1(K)}.
\end{align*}
\item Für $u: O\to\C$ sei die Nullfortsetzung\index{Nullfortsetzung} von
$u$ bezeichnet mit
\begin{align*}
\tilde{u}(x) := \begin{cases}
                u(x), & x\in \Omega,\\
                0, & \text{sonst}.
                \end{cases}
\end{align*}  
\item Sei $j\in C_0^\infty(\R^n\to\R)$ mit
\begin{equivenum}
\item $j(x)\ge 0$ auf $\R^n$,
\item $j(x) = 0$ für $\abs{x}\ge 1$,
\item $\int_{\R^n} j\dmu = 1$.
\end{equivenum}
Zu $\ep > 0$ sei
\begin{align*}
j_\ep(x) := \frac{1}{\ep^n}j\left(\frac{x}{\ep}\right),
\end{align*}
so ist
\begin{equivenum}
\item $j_\ep(x) = 0$ für $\abs{x}\ge \ep$,
\item $\int_{\R^n} j_\ep \dmu = 1$.
\end{equivenum}
Für $u\in L^1_\loc(\overline{O})$ sei
\begin{align*}
J_\ep u(x) := j_\ep * u(x) := \int_{\R^n} j_\ep(x-y)u(y)\dy. 
\end{align*}
$J_\ep$ heißt \emph{Glättungsoperator
(Mollifier)}\index{Glättungsoperator}\index{Mollifier}.\fishhere
\end{defnenum}
\end{defn}

Man rechnet leicht nach, dass eine Funktion $j$ mit den obigen Eigenschaften
gegeben ist durch
\begin{align*}
j(x) = \frac{1}{\int_{\R^n} f(x)\dx} f(x),\qquad f(x) = e^{-(1+\abs{x}^2)^{-1}}.
\end{align*}

\begin{prop}
\label{prop:7.43}
Sei $u\in L^1_\loc(\overline{O})$, so ist $J_\ep u \in
C^\infty(\R^n\to\C)$ und
\begin{align*}
\nabla^\alpha J_\ep u(x) = \int_{\R^n} \nabla^\alpha
j_\ep(x-y)\tilde{u}(y)\dy.\fishhere
\end{align*}
\end{prop}
\begin{proof}
\begin{proofenum}
\item $j_\ep$ ist stetig differenzierbar und hat kompakten Träger, ist also
global Lipschitz-stetig mit Konstante $L$.

Sei nun $x\in\R^n$ und $K_\ep(x)$ Umgebung von $x$.
Da $u\in L^1_\loc(\overline{O})$, wird das Integral über
$\overline{K_{2\ep}(x)\cap\Omega}$ durch eine
Konstante $C$ majorisiert. Also gilt für $x'\in K_\ep(x)$,
\begin{align*}
\abs{J_\ep u(x)-J_\ep u(x')} &\le 
\int_{\R^n} \abs{j_\ep(x-y)-j_\ep(x'-y)}\abs{\tilde{u}(y)}\dy\\
&=\int_{K_{2\ep}(x)} \abs{j_\ep(x-y)-j_\ep(x'-y)}\abs{\tilde{u}(y)}\dy\\
&\le  L\abs{x-x'} \int_{\overline{K_{2\ep}(x)\cap \Omega}}
\abs{\tilde{u}(y)}\dy
\le L\, C\, \abs{x-x'}.
\end{align*}
Somit ist $J_\ep u$ sogar lokal lipschitz stetig.
\item Mit dem verallgemeinerten Mittelwertsatz und dem Satz von der
majorisierten Konvergenz zeigt man analog,
\begin{align*}
\nabla^\alpha J_\ep u(x) = \int_{\R^n} \nabla^\alpha
j_\ep(x-y)\tilde{u}(y)\dy.\qedhere
\end{align*}
\end{proofenum}
\end{proof}

\begin{prop}[Abschneidefunktion]
\index{Abschneidefunktion}
\label{prop:7.44}
Sei $M\subset\R^n$ und für $\ep > 0$
\begin{align*}
M_\ep := \setdef{x\in\R^n}{d(x,M) < \ep} = \bigcup_{x\in M} K_\ep(x),
\end{align*}
so ist $M_\ep$ offen und damit messbar. Setze
\begin{align*}
\psi = J_\ep \chi_{M_\ep} = \int_{M_\ep} j_\ep(\cdot-y)\dy,
\end{align*}
so gelten
\begin{equivenum}
\item $0\le \psi(x)\le 1$ auf $\R^n$,
\item $\psi\in C^\infty(\R^n\to\R)$,
\item $\psi(x) = 0$,\quad $d(x,M_\ep) > \ep$,
\item $\psi(x) = 1$ auf $M$.\fishhere
\end{equivenum}
\end{prop}

\begin{figure}[!htpb]
\centering
\begin{pspicture}(-0.1,-1.2770555)(6.2682223,1.2511667)
\psline{<-}(0.38,1.2370555)(0.38,-1.0970556)
\psline{->}(0,-0.83705556)(6.2541113,-0.83705556)
\psline(1.7543283,-0.7)(1.7543283,-1)
\psline(4.3656735,-0.7)(4.3656735,-1)
\psline[linecolor=darkblue](1.374111,0.58294445)(4.7741113,0.58294445)
\psline[linecolor=purple](1.8141111,0.5629445)(4.374111,0.5629445)
\psline[linecolor=purple](5.334111,-0.80705553)(6.034111,-0.80705553)
\psline[linecolor=purple](0.8741111,-0.80705553)(0.37411112,-0.80705553)
\psbezier[linecolor=purple](4.374111,0.5629445)(5.054111,0.5629445)(4.7741113,-0.80705553)(5.354111,-0.80705553)
\psbezier[linecolor=purple](1.8141111,0.5629445)(1.1341112,0.5629445)(1.4741111,-0.80705553)(0.8741111,-0.80705553)

\rput(3.254111,0.7479445){\color{darkblue}$\chi_{M_\ep}$}
\rput(5.1941113,0.067944445){\color{purple}$\psi$}
\rput(3.054111,-1.0920556){\color{darkblue}$M$}
\rput(0.01,0.58205557){\color{gdarkgray}1}

\psline(0.2,0.59705555)(0.52,0.59705555)
\psline[linestyle=dotted,linecolor=darkblue](1.4,0.58)(1.4,-0.83705556)
\psline[linestyle=dotted,linecolor=darkblue](4.75,0.58)(4.75,-0.83705556)

\end{pspicture} 
\caption{Zur Abschneidefunktion.}
\end{figure}

\begin{prop}
\label{prop:7.45}
\begin{propenum}
\item\label{prop:7.45:1} Sei $u\in L^1_\loc(O)$ und $\supp u \Subset
O$, dann
\begin{align*}
J_\ep u \in C_0^\infty(O\to\C),\qquad \text{für } \ep < d(\supp u,\partial
O).
\end{align*}
\item\label{prop:7.45:2} Für $1\le p <\infty$ und $u\in L^p(O)$ gelten,
\begin{defnenum}
\item $J_\ep u \in L^p(O)$,
\item $\norm{J_\ep u}_p \le \norm{u}_p$,
\item $\norm{J_\ep u - u}_p \to 0$,\quad für $\ep \downarrow 0$.
\end{defnenum}
\item\label{prop:7.45:3} Ist $u\in L^\infty(O)$, so gilt $\abs{J_\ep
u(x)}\le \norm{u}_\infty$.
\item\label{prop:7.45:4} Ist $u\in C(O\to\C)$ und $K\Subset O$, dann $J_\ep u
\unito u$ auf $K$ für $\ep\downarrow 0$.\fishhere
\end{propenum}
\end{prop}
\begin{proof}
\begin{proofenum}
``\ref{prop:7.45:1}'': Siehe Skizze, die Details sind eine leichte Übung. 

\begin{figure}[!htpb]
\centering
\begin{pspicture}(0,-1.8093171)(6.1396832,1.7894094)
\psbezier(0.31968328,-0.5693171)(0.0,0.6307752)(1.7110358,0.120429)(2.4996834,0.21068287)(3.2883308,0.30093673)(4.669233,1.7694094)(4.819683,0.09068287)(4.9701333,-1.5880437)(0.63936657,-1.7694094)(0.31968328,-0.5693171)
\psbezier(4.239683,-0.32931712)(4.259683,-1.2093171)(3.6469872,-0.65047437)(3.3796833,-0.6293171)(3.1123793,-0.6081599)(2.5596833,-1.5093172)(2.3396833,-0.6093171)(2.1196833,0.29068288)(4.219683,0.55068284)(4.239683,-0.32931712)
\psbezier[linecolor=darkblue](4.159683,-0.32931712)(4.219683,-1.0693171)(3.6269872,-0.55047435)(3.3196833,-0.5493171)(3.0123794,-0.5481599)(2.5796833,-1.3493171)(2.4196832,-0.5693171)(2.2596834,0.21068287)(4.0996833,0.41068286)(4.159683,-0.32931712)

\rput(0.7096833,-0.34431714){\color{gdarkgray}$O$}
\psbezier(3.8796833,-0.24931712)(4.3596835,0.05068287)(4.759683,-0.70931715)(5.0596833,-0.12931713)

\rput(5.3796835,0.03568287){\color{darkblue}$\supp u$}
\psbezier(2.5196834,-0.88931715)(1.9414225,-0.94931716)(2.3196833,-1.5893171)(1.8196833,-1.6293172)

\rput(1.1096833,-1.5843171){\color{gdarkgray}$\supp J_\ep u$}
\psline[linecolor=purple]{<->}(4.0596833,-0.6893171)(4.219683,-0.9293171)

\rput(5.1396832,-1.1){\color{purple}$d(\supp u,\partialO)$}
\end{pspicture} 
\caption{Zum Beweis von Satz \ref{prop:7.45}.}
\end{figure}


``\ref{prop:7.45:3}'': $\abs{J_\ep u}(x) = \abs{\int_{\R^n}
j_\ep(x-y)\tilde{u}(y)\dy} \le \norm{u}_\infty\underbrace{\norm{j_\ep}_1}_{=1} = \norm{u}_\infty$.

``\ref{prop:7.45:4}'': Für $x\in K$ gilt
\begin{align*}
\abs{J_\ep u(x)-u(x)} &= 
\abs{\int_{\R^n} j_\ep(x-y)(u(y)-u(x))\dy} \\ &
\le \sup_{\abs{x-y}<\ep}
\abs{u(y)-u(x)}.
\end{align*}
Setzen wir $K_\ep := \overline{\setdef{z\in\R^n}{d(z,K)<\ep}}$, so ist $K_\ep$
kompakt und Teilmenge von $O$ für $\ep<d(K,\partial O)$. Somit
ist $u$ dort gleichmäßig stetig, d.h.
\begin{align*}
\sup_{\atop{\abs{x-y}<\ep}{x,y\in K_\ep}}
\abs{u(y)-u(x)} \to 0,\qquad \ep\downarrow0
\end{align*}
unabhängig vom speziell gewählten $x\in K$.

``\ref{prop:7.45:2}'':
Zu $p\neq 1$ sei $\frac{1}{p}+\frac{1}{q}=1$, so gilt
\begin{align*}
\abs{J_\ep u(x)} &=
\abs{\int_{\R^n}j_\ep(x-y)^{\frac{1}{p}+\frac{1}{q}}\tilde{u}(y)\dy}\\
&\overset{\text{Hölder}}{\le}
\underbrace{\left(\int_{\R^n} j_\ep(x-y)\dy \right)^{1/q}}_{=1}
\left(\int_{\R^n}j_\ep(x-y)\abs{\tilde{u}(y)}^p\dy \right)^{1/p}.
\end{align*}
Für $p=1$ ist diese Ungleichung trivialerweise ebenfalls erfüllt.

Für $1\le p < \infty$ gilt folglich,
\begin{align*}
\int_{O} \abs{J_\ep u(x)}^p \dx
&\le 
\int_{O}\int_{\R^n} j_\ep(x-y)\abs{\tilde{u(y)}}^p\dy \dx\\
&\overset{\text{Fubini}}{=}
\int_{\R^n} \underbrace{\int_{O}j_\ep(x-y) \dx}_{=1} \abs{\tilde{u(y)}}^p
\dy = \norm{u}_p^p.
\end{align*}
Somit wären (a) und (b) bewiesen. Zum Beweis von (c) benötigen wir noch etwas
Vorbereitung.\qedhere
\end{proofenum}
\end{proof}

\begin{bem}
\label{bem:7.46}
Es ist nicht einfach zu zeigen, dass für $O\subset\R^n$ offen und beschränkt
auch
\begin{align*}
\norm{J_\ep \chi_O - \chi_O}_p \to 0,
\end{align*}
denn es gibt solche $O$ mit $\mu(\partial O) > 0$.

Sei z.B. $n=1$ und $(q_j)$ eine Abzählung von $\Q\cap[-2,2]$. Setzen wir
\begin{align*}
O := \bigcup_{j\in\N} \left(q_j-\frac{1}{2^j},q_j+\frac{1}{2^j}\right),
\end{align*}
so ist $O$ offen und
\begin{align*}
\mu(O) \le \sum_{j\in\N} \frac{1}{2^{j-1}} = 2.
\end{align*}
Da $\setd{q_j}\subset O$, folgt $[-2,2]\subset O$, d.h. $\mu(\overline{O}) \ge
4$ also ist
\begin{align*}
\mu(\partial O) = \mu(\overline{O}\setminus O) = \mu(\overline{O})-\mu(O) =
2.\maphere
\end{align*}
\end{bem}

\begin{prop}[Satz von Lusin]
\index{Satz!Lusin}
\label{prop:7.45}
Sei $A\subset\R^n$ messbar, $\mu(A) < \infty$ und $f:\R^n\to\R$ messbar,
$f(x)=0$ für $x\in\R^n\setminus A$. Dann gilt
\begin{align*}
\forall \ep > 0 \exists f_\ep \in C_0(A\to\R) : \norm{f_\ep}_\infty
\le \norm{f}_\infty \land
\mu\setdef{x\in\R^n}{f_\ep(x)\neq f(x)}< \ep.\fishhere
\end{align*}
\end{prop}
\begin{proof}
Siehe \cite{Alt99} Anhang 4.7.\qedhere
\end{proof}

\begin{prop}
\label{prop:7.48}
Sei $1\le p <\infty$.
\begin{propenum}
\item\label{prop:7.48:1} Die Menge $\setdef{s_n : O\to\C}{s_n \text{ einfach und
} \supp s_n \Subset O}$ ist dicht in $L^p(\Omega)$.
\item\label{prop:7.48:2} $C_0(O\to\C)$ ist dicht in $L^p(\Omega)$.\fishhere
\end{propenum}
\end{prop}
\begin{proof}
Sei $f\in L^p(O)$ gegeben. Ohne Einschränkung ist $f$ reellwertig und positiv
ansonsten zerlege $f$ in $(\Re f)_\pm$, $(\Im f)_\pm$.

\begin{proofenum}
\item
Sei $O_k := \setdef{x\in O}{\abs{x} < k \land d(x,\partial O) > \frac{1}{k}}$.
\begin{figure}[!htpb]
\centering
\begin{pspicture}(0,-1.08)(5.3,1.08)
\psbezier[linecolor=darkblue](3.88,0.88)(3.3,0.82)(2.8971407,0.41556767)(2.34,0.32)(1.7828593,0.22443233)(1.32,1.06)(0.66,0.7)(0.0,0.34)(0.48,-0.7)(1.2,-0.82)(1.92,-0.94)(2.76,-0.62)(3.24,-0.72)(3.72,-0.82)(3.78,-1.06)(4.44,-0.68)
\psline[linecolor=darkblue,linestyle=dotted](3.88,0.88)(4.88,0.98)
\psline[linecolor=darkblue,linestyle=dotted](4.44,-0.68)(5.28,-0.22)
\psbezier[linecolor=purple](3.78,0.74)(3.2781081,0.69169813)(2.8569865,0.32068136)(2.36,0.22)(1.8630135,0.11931863)(1.3232433,0.9)(0.72,0.6)(0.11675676,0.3)(0.5902703,-0.62754714)(1.2356757,-0.7316981)(1.8810811,-0.83584905)(2.634054,-0.5181132)(3.08,-0.58)(3.525946,-0.6418868)(3.7083783,-0.96)(4.28,-0.66)
\psline[linecolor=purple](3.76,0.76)(4.28,-0.68)
\rput(4.91,-0.035){\color{darkblue}$O$}
\rput(1.13,-0.275){\color{purple}$O_k$}
\end{pspicture}
\caption{Zur Konstruktion von $O_k$.}
\end{figure}

Setze $f_k = \chi_{O_k}f$, dann ist $f_k$ messbar, $\supp f_k \subset
\overline{O}_k \Subset O$ und auf $O$ gilt
\begin{align*}
0\le f_k(x) \le f(x),\qquad f_k(x)\to f(x).
\end{align*}
Mit dem Satz von der majorisierten Konvergenz erhalten wir somit,
\begin{align*}
\norm{f_k-f}^p_p = \int_\Omega \abs{f_k-f}^p \dmu \to 0.
\end{align*}
Aus der Maßtheorie wissen wir weiterhin dass eine Folge einfacher Funktionen
$(s_n)$ existiert, die auf $O$ monoton geen $f_k$ konvergiert,
\begin{align*}
0\le s_n(x)\le f_k(x),\qquad s_n(x)\to f_k(x).
\end{align*}
Ebenfalls mit majorisierter Konvergenz folgt, dass $\norm{s_n-f_k}_p\to 0$.

Somit folgt \ref{prop:7.48:1}.
\item Sei nun $s=\sum_{j=1}^N a_j \chi_{A_j}$ mit $\norm{f-s}_p < \ep$,
$\supp s\Subset O$ und $A_j\subset \supp s$.
Für jedes der endlich vielen $j\in\setd{1,\ldots,N}$ existiert nach dem Satz von
Lusin ein $g_j\in C_0(A_j\to\C)$ mit $\abs{g_j(x)}\le 1$ auf $A_j$ und
\begin{align*}
\mu\setdef{x\in\R^n}{g_j(x)\neq \chi_{A_j}(x)} < \ep.
\end{align*}
Somit folgt
\begin{align*}
\norm{\chi_{A_j}-g_j}_p^p = \int_{\R^n} \underbrace{\abs{\chi_{A_j}-g_j}}_{\le
2}^p
\dmu
\le 2^p
\mu\setdef{x\in\R^n}{g_j(x)\neq \chi_{A_j}(x)}
< 2^p\ep.
\end{align*}
Da $g_j$ stetig und $\supp g_j \Subset O$ erhalten wir
analog für $s$,
\begin{align*}
\norm{s-\sum_{j=1}^N a_j g_j}_p^p \le N2^p\ep.\qedhere
\end{align*}
\end{proofenum}
\end{proof}

\begin{proof}[Beweis von \ref{prop:7.45} \ref{prop:7.45:2} (c).]
Zu $u\in L^p(\Omega)$ wähle $g\in C_0(\Omega\to\C)$ mit $\norm{u-g}_p < \ep$.
Aus \ref{prop:7.45} \ref{prop:7.45:1} wissen wir
\begin{align*}
J_\ep g\in C_0^\infty(\Omega\to\C),\qquad \text{für } \ep < \ep_0.
\end{align*} 
Für $\ep < \ep_0$ gilt außerdem $J_\ep g \subset \setdef{x\in\Omega}{d(x,\supp
g)<\ep}\Subset \Omega$. Setzen wir daher
\begin{align*}
K:= \setdef{x\in\Omega}{d(x,\supp g)\le \ep_0},
\end{align*}
so folgt mit \ref{prop:7.45} \ref{prop:7.45:4}
\begin{align*}
J_\ep g(x) - \tilde{g}(x) =
\begin{cases}
0, & x\in\R^n\setminus K\\
\to 0,& \text{gleichmäßig auf }K.
\end{cases}
\end{align*}
Nun gilt
\begin{align*}
\norm{J_\ep g - g}_{p}^p = 
\int_K \abs{J_\ep g - g}^p \dmu
\le \sup_{x\in K} \abs{J_\ep g(x)-g(x)}^p \mu(K)
\to 0,\qquad \ep\downarrow 0. 
\end{align*}
Zusammenfassend gilt also
\begin{align*}
\norm{u-J_\ep u}_p \le \norm{u-g}_p + \norm{g-J_\ep g}_p + \norm{J_\ep(g-u)}_p
< \delta,
\end{align*}
für $\ep$ hinreichend klein.\qedhere
\end{proof}

\begin{cor}
\label{cor:7.49}
Sei $1\le p < \infty$. Dann ist $C_0^\infty(\Omega\to\C)$
dicht in $L^p(O)$.\fishhere
\end{cor}
\begin{proof}
Aus dem Beweis von \ref{prop:7.45} verwenden wir
\begin{align*}
\norm{u-J_\ep g}_p \le \norm{u-g}_p + \norm{g-J_\ep g}_p  < 2\delta,
\end{align*}
für $\ep$ hinreichend klein wobei $J_\ep g\in C_0^\infty(O\to\C)$ für $\ep$
hinreichend.\qedhere
\end{proof}

\begin{prop}
\label{prop:7.50}
Sei $1<p<\infty$, $m\in\N$ und $u\in W^{m,p}(\R^n)$. Dann gilt
\begin{align*}
\forall \abs{\alpha} \le m : J_\ep\D^\alpha u = \nabla^\alpha (J_\ep
u).\fishhere
\end{align*}
\end{prop}
\begin{proof}
Sei $\ph\in C_0^\infty(\R^n\to\C)$. Dann gilt
\begin{align*}
\lin{J_\ep \D^\alpha u,\ph}
&= \int_{\R^n}\int_{\R^n} j_\ep(x-y)\D^\alpha u(y) \dy 
\overline{\ph}\dx\\
&\overset{\text{Fubini}}
{=} \int_{\R^n}\D^\alpha u(y)\overline{\int_{\R^n} j_\ep(x-y)\ph(x)\dx} \dy,
\end{align*}
da $j_\ep$ reellwertig. Nun ist $j_\ep(x-\cdot)\ph(\cdot)\in
C_0^\infty(\R^n\to\C)$ wir können somit die Definition der schwachen Ableitung
anwenden und erhalten,
\begin{align*}
\ldots &=
(-1)^{\abs{\alpha}} \int_{\R^n}u(y) \overline{\int_{\R^n}
\nabla^\alpha_y j_\ep(x-y)\ph(x)\dx}
\dy \\ &=
\int_{\R^n}u(y) \overline{\int_{\R^n}
(\nabla^\alpha j_\ep)(x-y)\ph(x)\dx}
\dy\\
&=
\int_{\R^n} \int_{\R^n}
u(y)
(\nabla^\alpha j_\ep)(x-y)\dy\overline{\ph}(x)
\dx\\
&=
\lin{\nabla^\alpha J_\ep u,\ph}
\end{align*}
und da $\ph$ beliebig war, folgt die Behauptung.\qedhere
\end{proof}

\begin{prop}
\label{prop:7.51}
Sei $\Omega=\R^n$, $m\in\N$. Dann gilt
\begin{align*}
W_0^{m,p}(\R^n) = W^{m,p}(\R^n).\fishhere
\end{align*}
\end{prop}
\begin{proof}
``$\subset$'': $W_0^{m,p}(\R^n)\subset W^{m,p}(\R^n)$ gilt nach Definition.\\
``$\supset$'': Sei $u\in W^{m,p}(\R^n)$. Zeige
\begin{align*}
\forall \ep > 0 \exists \ph\in C_0^\infty(\R^n\to\C) : \norm{\ph-u}_{m,p} < \ep.
\end{align*}
\textit{Abschneiden von $u$}. Sei $\psi\in C_0^\infty(\R^n\to\R)$ mit $0\le
\psi\le 1$ und
\begin{align*}
\psi(x) = 
\begin{cases}
1, & \abs{x}< 1,\\
0, & \abs{x} \ge 2.
\end{cases}
\end{align*}
Setzen wir $u_k(x) = \psi(k^{-1}x)u(x)$, dann gilt für $\abs{\alpha}\le m$:
\begin{align*}
\norm{\D^\alpha u-\psi(k^{-1}\cdot)\D^\alpha u}_p^p
&=
\int_{\R^n} \abs{\D^\alpha u(x)}^p\abs{1-\psi(k^{-1}x)}^p\dmu\\
&\to 0
\end{align*}
nach majorisierter Konvergenz. Nun ist
\begin{align*}
\norm{\D^\alpha u - \D^\alpha u_k}_p
\le \norm{\D^\alpha u - \psi(k^{-1}x)\D^\alpha u}_p
+ \sum_{\atop{\beta\le \alpha}{\beta\neq 0}}
\binom{\alpha}{\beta}
\norm{\nabla_x^\beta \psi(k^{-1}x)\D^{\alpha-\beta}u}_p
\end{align*}
und $\abs{\nabla_x^\beta \psi(k^{-1}x)} \le
\frac{1}{k^{\abs{\beta}}}\abs{(\nabla^\beta\psi)(k^{-1}x)}$ also
\begin{align*}
\ldots \le
\norm{\D^\alpha u - \psi(k^{-1}x)\D^\alpha u}_p
+ \frac{1}{k}c \sum_{\beta\le\alpha} \binom{\alpha}{\beta}
\norm{\D^{\alpha-\beta}u}_p
\to \infty.
\end{align*}
Somit existiert ein $k\in\N$, so dass
\begin{align*}
\norm{u-u_k}_{m,p} < \frac{\ep}{2}.
\end{align*}

Sei $\ph=J_\delta u_k$ mit $\delta$ noch zu bestimmen. Nun ist $\supp
u_k\subset K_{2k}(0)$, da $\psi$ außerhalb verschwindet. Somit ist $J_\delta
u_k\in C_0^\infty(\R^n\to\C)$. Für $\abs{\alpha}\le m$ gilt also
\begin{align*}
\norm{\D^\alpha(u_k-J_\delta u_k)}_p \overset{\ref{prop:7.50}}{=}
\norm{\D^\alpha u_k - J_\delta D^\alpha u_k}_p \overset{\ref{prop:7.45}}{\to}0.
\end{align*}
Somit existiert ein $\delta_0> 0$, so dass
\begin{align*}
\norm{u_k-J_{\delta}u_k}_{m,p}<\frac{\ep}{2},\qquad \delta \ge \delta _0.
\end{align*}
Zusammenfassend also
\begin{align*}
\norm{u-J_{\delta_0}u_k}_{m,p} < \ep.\qedhere
\end{align*}
\end{proof}

\begin{cor}
\label{prop:7.52}
Für $m\in\N$ gilt $C_0^m(O\to\C)\subset W_0^{m,p}(O)$.\fishhere
\end{cor}
\begin{proof}
Sei $u\in C_0^m(O\to \C)$. Setze $\ph_\ep:=J_\ep u$, so ist nach
\ref{prop:7.45} $\ph_\ep\in C_0^\infty(O\to\C)$ für $\ep < d(\supp u,\partial
O)$. Wie im Beweis von Satz \ref{prop:7.51} sieht man sofort, dass
\begin{align*}
\norm{\D^\alpha \ph_\ep - \D^\alpha u}_p = \norm{J_\ep \nabla^\alpha u -
\nabla^\alpha u}_0 \to 0.
\end{align*}
Somit $\norm{\ph_\ep - u}_{m,p}\to 0$.\qedhere
\end{proof}

\begin{prop}[Produktregel]
\index{schwache!Produktregel}
Sei $m\in \N$, $u\in W^{m,p}(O)$ und $f\in C^m(O\to\C)$ mit
$\norm{\nabla^\alpha f}_\infty \le c$ für $\abs{\alpha}\le m$. Dann gilt
\begin{align*}
f\cdot u \in W^{m,p}(O)
\end{align*}
und
\begin{align*}
\D^\alpha (f\cdot u) = \sum_{\beta\le \alpha} \nabla^\beta f \cdot
\D^{\alpha-\beta} u.\fishhere
\end{align*}
\end{prop}
\begin{proof}
Wir beweisen lediglich den Fall $m=1$, die übrigen Fälle ergeben sich analog.
Sei $\ph\in C_0^\infty(O\to\C)$, so ist
\begin{align*}
\lin{\D^{e_j}(f\cdot u),\ph} &= - \lin{f\cdot u,\nabla^{e_j}\ph}
= -\lin{u,\overline{f}\nabla^{e_j}\ph}\\ & =
-\lin{u,\nabla^{e_j}(\overline{f}\cdot u)} + \lin{u,(\nabla^{e_j}f)\ph}.
\end{align*}
In Übung 14.1 haben wir bewiesen $-\lin{u,\nabla^{e_j} (\overline{f}\cdot\ph)}
= \lin{(\D^{e_j}u)f,\ph}$ also gilt
\begin{align*}
\lin{\D^{e_j}(f\cdot u),\ph}
&= \lin{f(\D^{e_j}u),\ph} + \lin{(\nabla^{e_j}f)u,\ph}\\
&=
\lin{f(\D^{e_j}u)+(\nabla^{e_j}f)u,\ph}.\qedhere
\end{align*} 
\end{proof}