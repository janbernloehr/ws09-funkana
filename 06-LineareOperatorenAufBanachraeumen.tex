\chapter{Lineare Operatoren auf normierten Räumen}

Sofern nicht anders angegeben seien $E$ und $F$ normierte Räume mit Normen
$\norm{\cdot}_E$, $\norm{\cdot}_F$. Sollte aus dem Zusammenhang hervorgehen,
welcher Raum gemeint ist, wird auch nur $\norm{\cdot}$ verwendet.

\section{Spektraltheorie beschränkter Operatoren}

In diesem Abschnitt studieren wir zunächst die beschränkten Operatoren. Meißt
sind die unbeschränkten Operatoren (wie z.B. Differentialoperatoren)
von größerem Interesse, in einigen Fällen lassen sie sich jedoch zu
einem beschränkten Operator invertieren, auf den wir dann die hier entwickelte
Theorie anwenden können.

\begin{defn}
\label{defn:6.1}
Sei $E$ Banachraum und $A\in\LL(E)$.
\begin{defnenum}
  \item $\lambda\in\K$ heißt \emph{regulär} bezüglich $A$, falls
\begin{align*}
A-\lambda\Id : E\to E
\end{align*}
bijektiv. (Dann ist $(A-\lambda\Id)^{-1}\in\LL(E)$)
\item Die Menge
\begin{align*}
\rho(A) := \setdef{\lambda\in\K}{\lambda\text{ regulär bezüglich $A$}}
\end{align*}
heißt \emph{Resolventenmenge}\index{Resolventen!-Menge} von $A$. Die Abbildung
\begin{align*}
R: \rho(A)\to\LL(E),\quad \lambda\mapsto R(\lambda) = (A-\lambda\Id)^{-1}
\end{align*}
heißt \emph{Resolventenfunktion}\index{Resolventen!-Funktion} von $A$.
\item Die Menge
\begin{align*}
\sigma(A) := \K\setminus\rho(A)
\end{align*}
heißt \emph{Spektrum}\index{Spektrum} von $A$; $\lambda\in\sigma(A)$
heißt \emph{Spektralwert von $A$}. $\lambda\in\K$ heißt \emph{Eigenwert
(EW)}\index{Eigen!-wert} von $A$, falls $\ker (A-\lambda\Id)\neq (0)$. Der Kern
von $A-\lambda\Id$ heißt \emph{Eigenraum}\index{Eigen!-raum} zum EW
$\lambda$.\fishhere
\end{defnenum}
\end{defn}

\begin{bsp}
\label{bsp:6.2}
\begin{bspenum}
  \item Sei $E=\C^n$ und $A\in\LL(E)$, so besitzt $A$ in Koordinaten eine 
  Darstellung als $n\times n$ Matrix. $\lambda$ ist genau dann
  Eigenwert, wenn ein $x\neq 0$ existiert mit $Ax=\lambda x$. Dann ist $A-\lambda \Id$ nicht injektiv also $\lambda\in\sigma(A)$.

Ist $\lambda$ kein Eigenwert, dann ist $\ker(A-\lambda\Id)=(0)$ und damit
$A-\lambda\Id$ injektiv. Die Dimensionsformel besagt nun, dass
$\dim\ker(A-\lambda\Id)+\dim\im (A-\lambda\Id) = n$, also ist $A-\lambda\Id$
auch surjektiv und somit $\lambda\in\rho(A)$.

In diesem Fall ist $\sigma(A) := \setdef{\lambda\in\C}{\lambda \text{ ist
Eigenwert von }A}$ und $\card\sigma(A)\le n$.
  \item Sei $E=l^2$. Betrachte den Shift-Operator 
\begin{align*}
S:(x_1,x_2,\ldots) \mapsto (0,x_1,x_2,\ldots).
\end{align*}
$(1,0,0,\ldots)\notin\im (S)$, also ist $S-0\Id$ nicht surjektiv, d.h.
$0\in\sigma(S)$. Aber $0$ ist kein Eigenwert, denn
\begin{align*}
Sx = 0\Leftrightarrow (0,x_1,x_2,\ldots) = 0\Leftrightarrow x=0,
\end{align*}
Tatsächlich besitzt $S$ keine Eigenwerte aber $\sigma(S) =
\overline{K_1(0)}$.\bsphere 
\end{bspenum}
\end{bsp}

\begin{lem}
\label{prop:6.3}
Seien $E,F$ Banachräume, $T\in\LL(E\to F)$ bijektiv, $S\in\LL(E\to F)$ mit
$\norm{S-T}< \norm{T^{-1}}^{-1}$, dann ist $S$ bijektiv.\fishhere
\end{lem}
\begin{proof}
Der Beweis wird in den Übungen behandelt.\qedhere
\end{proof}

\begin{cor}
\label{prop:6.4}
$\rho(A)$ ist offen.\fishhere
\end{cor}
\begin{proof}
Sei $\lambda\in\rho(A)$, d.h. $T:=A-\lambda\Id$ ist invertierbar. Mit
\ref{prop:6.3} folgt, dass $S=A-z\Id$ invertierbar, also $z\in\rho(A)$, falls
\begin{align*}
\norm{S-T} = \norm{A-z\Id-(A-\lambda\Id)} = \abs{z-\lambda} <
\norm{(A-\lambda\Id)^{-1}}^{-1} =: r > 0.
\end{align*}
Also ist $K_r(\lambda)\subseteq \rho(A)$.\qedhere
\end{proof}

\begin{prop}
\label{prop:6.5}
Sei $E$ Banachraum, $A\in\LL(E)$. Dann ist die Resolventenfunktion $R$
analytisch auf $\rho(A)$, d.h. lokal als Potenzreihe mit Koeffizienten in
$\LL(E)$ entwickelbar:
\begin{align*}
\forall \lambda\in\rho(A) \exists r > 0 \forall z\in K_r(\lambda) :
R(z) = (A-z\Id)^{-1} = \sum\limits_{n=0}^\infty A_n(z-\lambda)^n
\end{align*}
mit $A_n\in\LL(E)$.\fishhere
\end{prop}
\begin{proof}
Der Beweis wird in den Übungen behandelt.\qedhere
\end{proof}

\begin{prop}
\label{prop:6.6}
Sei $E$ ein Banachraum über $\C$ und $A\in\LL(E)$. Dann gilt
\begin{propenum}
  \item\label{prop:6.6:1} $\sigma(A)\neq \varnothing$,
  \item\label{prop:6.6:2} $\sigma(A) \subseteq
  \setdef{z\in\C}{\abs{z}\le\norm{A}}$,
  \item\label{prop:6.6:3} $\sigma(A)$ ist kompakt.\fishhere
\end{propenum}
\end{prop}
\begin{proof}
``\ref{prop:6.6:2}'': Sei $\abs{\lambda}> \norm{A}$. Wir 
zeigen, dass $A-\lambda\Id$ invertierbar ist. Dazu wenden wir Satz
\ref{prop:6.3} auf $T=\Id$ und $S=\Id-\frac{1}{\lambda}A$ an, dann ist $S$
invertierbar, falls
\begin{align*}
\norm{S-T} = \frac{1}{\abs{\lambda}}\norm{A} < \norm{\Id^{-1}}^{-1}
= 1.
\end{align*}
``\ref{prop:6.6:3}'': Aus \ref{prop:6.4} folgt, dass $\sigma(A) = \rho(A)^c$
abgeschlossen und mit \ref{prop:6.6:2} folgt, dass $\sigma(A)$ beschränkt ist.

``\ref{prop:6.6:1}'': Angenommen $\sigma(A)=\varnothing$, d.h. für jedes
$\lambda_0\in\C$ ist $A-\lambda_0\Id$ bijektiv. Wir bezeichnen die
Entwicklung der Resolventenfunktion um $\lambda_0$ als $R_{\lambda_0}$ und
zeigen nun, dass
\begin{align*}
\forall \lambda_0 \in \C \forall f\in E' : f(R_{\lambda_0}(z)) = 0.
\end{align*} 
Mit \ref{bem:4.14} folgt somit $\forall z\in\C :
R_{\lambda_0}(z)=0$, dies ist jedoch ein Widerspruch dazu, dass
$R(\lambda)=(A-z\Id)^{-1}$.

Sei also $f\in E'$ und $\ph(z)=f(R(z))$, so existiert $\ph:\C\to\C$.
\begin{defnenum}
  \item \textit{$\ph$ ist holomorph}. Sei $\lambda_0\in\C$, so existiert ein
  $\ep >0$, so dass nach \ref{prop:6.5}
\begin{align*}
R(\lambda) = \sum\limits_{j=0}^\infty A_n (\lambda-\lambda_0)^j,\qquad
\text{für } \abs{\lambda-\lambda_0}<\ep.
\end{align*}
Da $f$ stetig und linear folgt somit auch
\begin{align*}
\ph(\lambda) = \sum\limits_{j=0}^\infty (\lambda-\lambda_0)^j f(A_n),\qquad
\text{für } \abs{\lambda-\lambda_0}<\ep.
\end{align*}
Somit ist $\ph$ in jedem Punkt in eine Potenzreihe entwickelbar, also holomorph.
\item \textit{$\ph$ ist beschränkt}. Wir zeigen, dass $R(\lambda)$ beschränkt
ist.
\begin{enumerate}
  \item[$\alpha$)]
Sei $M=\setdef{\lambda\in\C}{\abs{\lambda}\le2\norm{A}}$. $R(\lambda)$ ist
stetig, also $\sup\limits_{\lambda\in M} \norm{R(\lambda)}<c$.
  \item[$\beta$)]
Sei nun $\lambda\in\C$ mit $\abs{\lambda}>2\norm{A}$. Betrachte $\Id =
\frac{1}{\lambda}\left(A-(A-\lambda\Id)\right)$, dann ist
\begin{align*}
R(\lambda) = (A-\lambda\Id)^{-1}\Id =
\frac{1}{\lambda}\left(A(A-\lambda\Id)^{-1} - \Id\right)
\end{align*}
und es folgt,
\begin{align*}
&\norm{R(\lambda)} 
\le \frac{1}{\abs{\lambda}}\left(\norm{A\left(A-\lambda\Id\right)^{-1}}+1\right)
< \frac{1}{2}\norm{R(\lambda)} + \frac{1}{\abs{\lambda}}\\
\Rightarrow &\norm{R(\lambda)} < \frac{2}{\abs{\lambda}}\to 0,\qquad
\abs{\lambda}\to
\infty
\end{align*}
\end{enumerate}
$\alpha$) und $\beta$) erlauben es uns, den Satz von Liouville anzuwenden, d.h.
$\ph$ ist konstant. Aus $\beta$) folgt außerdem, $\ph\equiv0$.\qedhere
\end{defnenum} 
\end{proof}

\section{Kompakte Operatoren}

\begin{defn}
\label{defn:6.7}
$T\in\LL(E\to F)$ heißt \emph{kompakt}\index{Operator!kompakt}, falls
$\overline{T(K_1(0))}$ kompakt in $F$.

Wir setzten $\KK(E\to F) := \setdef{T:E\to F}{T\text{ ist kompakt}}$,
$\KK(E):=\KK(E\to E)$.\fishhere
\end{defn}

\begin{prop}
\label{prop:6.8}
\begin{propenum}
  \item Für $T\in \LL(E\to F)$ sind äquivalent:
\begin{equivenum}
  \item\label{prop:6.8:1.1} $T\in\KK(E\to F)$.
  \item\label{prop:6.8:1.2} Für jede beschränkte Folge $(x_n)$ in $E$ besitzt
  $(Tx_n)$ eine in $F$ konvergente Teilfolge.
  \item\label{prop:6.8:1.3} Für jede beschränkte Teilmenge $M\subseteq E$ ist
  $\overline{T(M)}$ kompakt.
\end{equivenum}
\item Seien $T\in\LL(E\to F)$, $S\in\LL(F\to G)$. Ist $T$ oder $S$ kompakt, so
 auch $S\circ T$.
 \item Ist $F$ Banachraum, so ist $\KK(E\to F)$ linearer abgeschlossener
 Teilraum von $\LL(E\to F)$.
 \item Ist $E$ Banachraum, so ist $\KK(E)$ ein abgeschlossenes Ideal von
 $\LL(E)$.\fishhere
\end{propenum}
\end{prop}

\begin{bemn}[Vorbemerkung.]
In einem normierten Raum sind Überdeckungs- und Folgenkompaktheit
äquivalent.\maphere
\end{bemn}

\begin{proof}
\begin{propenum}
  \item ``\ref{prop:6.8:1.1}$\Rightarrow$\ref{prop:6.8:1.2}'': Sei $(x_n)$
  beschränkt in $E$, so ist $(\norm{x_n})$ beschränkt in $\R$ und besitzt daher
  eine konvergente Teilfolge $\norm{x_{n_k}}\to c$.
  Setze $y_k := (1+\norm{x_{n_k}})^{-1}x_{n_k}$, so ist $y_k\in K_1(0)$ und
  daher besitzt $(Ty_k)$ eine konvergente Teilfolge, $y_{k_l}\to y$. Folglich
  ist
\begin{align*}
Tx_{n_{k_l}} = (1+\norm{x_{n_{k_l}}})y_{n_{k_l}}\to cy.
\end{align*}
  
%     Falls $(x_n)$ eine Teilfolge $(x_{n_k})$ besitzt mit
%   $x_{n_k}\to 0$, so konvergiert $Tx_{n_k}\to0$, denn $T$ ist stetig.
%   
% Habe nun
%   $(x_n)$ keine gegen Null konvergente Teilfolge. Setze
%   $y_n:=\frac{x_n}{2\norm{x_n}}$ für $n>N$. $(y_n)$ ist nun Folge in $K_1(0)$
%   und folglich ist $(Ty_n)$ Folge in $\overline{T(K_1(0))}$, besitzt also eine
%   konvergente Teilfolge $(Ty_{n_k})$ mit $Ty_{n_k}\to y$ in $F$, also
% \begin{align*}
% Tx_{n_k} = 2\norm{x_{n_k}}Ty_{n_k}.
% \end{align*}
% $(\norm{x_{n_k}})$ ist beschränkt, besitzt also eine konvergente Teilfolge, also
% \begin{align*}
% Tx_{n_{k_l}} \to 2c y.
% \end{align*} 
``\ref{prop:6.8:1.2}$\Rightarrow$\ref{prop:6.8:1.3}'':
Sei $M\subseteq E$ beschränkt und $(y_n)$ Folge in $\overline{T(M)}$. Zeige es
existiert eine konvergente Teilfolge $(y_{n_k})$. Wähle dazu $\eta_n=Tx_n$ mit
$x_n\in M$ und $\norm{\eta_n-y_n}<\frac{1}{n}$. $(x_n)$ ist Folge in $M$, also
beschränkt und mit \ref{prop:6.8:1.2} folgt $Tx_{n_k}\to y$ in $F$.
Insbesondere $y\in \overline{T(M)}$, also
\begin{align*}
y_{n_k} = \underbrace{y_{n_k} - Tx_{n_k}}_{\to 0} + Tx_{n_k},
\end{align*}
also $y_{n_k}\to y$.

``\ref{prop:6.8:1.3}$\Rightarrow$\ref{prop:6.8:1.1}'': Spezialisierung.
\item Sei $(x_n)$ beschränkt in $E$. Ist $T$ kompakt, dann ist $Tx_{n_k}\to y$
und $S$ ist stetig, also $S\circ Tx_{n_k}\to Sy$. Ist $S$ kompakt, nun ist
$(Tx_{n})$ beschränkt also $S\circ Tx_{n_k}\to z$ in $G$.
\item Seien $S,T\in\KK(E\to F)$, $\alpha\in \K$ und $(x_n)$ beschränkte Folge
in $E$.
\begin{align*}
 &T(x_{n_k}) \to y\Rightarrow \alpha T(x_{n_k}) \to \alpha y\\
 &T(x_{n_k})\to y_1,\quad S(x_{n_k})\to y_2\Rightarrow
 (S+T)(x_{n_k}) \to y_1+y_2.
\end{align*}
Also ist $\KK(E\to F)$ linearer Teilraum von $\LL(E\to F)$.

Sei nun $(T_k)$ Folge in $\KK(E\to F)$, $T\in\LL(E\to F)$ und $\norm{T_k-T}\to
0$. Sei $(x_n)$ beschränkt in $E$, dann ist $T_1$ kompakt, d.h.
\begin{align*}
&T_1(x_n^{(1)})\to y^{(1)}, && (x_n^{(1)}) \text{ Teilfolge von } (x_n).
\end{align*}
Sei nun $T_2$ kompakt, dann
\begin{align*}
&T_2(x_n^{(2)})\to y^{(2)}, && (x_n^{(2)}) \text{ Teilfolge von } (x_n^{(1)}),
\end{align*}
usw. Setze $\xi_n := x_n^{(n)}$ (Diagonalfolge), dann folgt
\begin{align*}
\forall k\in\N : T_k(\xi_n)\to y^{(k)}
\end{align*}
und $(\xi_n)$ ist Teilfolge von $(x_n)$. Zeige nun $(T\xi_n)$ ist Cauchyfolge,
\begin{align*}
\norm{T\xi_n-T\xi_m} &\le \norm{T\xi_n-T_k \xi_n}
+ \norm{T_k\xi_n-T_k\xi_m} + \norm{T_k\xi_m-T\xi_m}\\
&< \ep\norm{\xi_n} +\norm{T_k\xi_n-T_k\xi_m} + \ep \norm{\xi_m},
\end{align*}
für $k$ hinreichend groß. Außerdem ist $(\xi_n)$ konvergent also beschränkt,
d.h. $\norm{\xi_n}\le c$. Für festes $k$ ist $(T_k\xi_n)_n$ Cauchyfolge da
konvergent und daher $\norm{T_k\xi_n-T_k\xi_m}<\ep$ für $n,m>N_\ep$, d.h.
\begin{align*}
\norm{T\xi_n-T\xi_m}  < 2\ep c + \ep = (2c+1)\ep,\qquad n,m> N_\ep.
\end{align*}
\item Folgt direkt aus 2.) und 3.).\qedhere
\end{propenum}
\end{proof}

\begin{lem}[Rieszsches Lemma (fast orthogonales Element)]
\index{Satz!Rieszsches Lemma}
\label{prop:6.9}
Sei $L$ abgeschlossener echter Teilraum von $E$. Dann gilt
\begin{align*}
\forall q \in (0,1) \exists x_q\in E\setminus L : \norm{x_q} = 1 \text{ und }
d(x_q,L)\ge q.\fishhere
\end{align*} 
\end{lem}
\begin{proof}
Sei $y\in E\setminus L$, dann folgt mit \ref{prop:3.4}, dass $d(y,L)>0$. Wähle
nun $x_0\in L$ mit
\begin{align*}
d(y,L) = \inf_{x\in L} \norm{y-x} \le \norm{y-x_0} \le \frac{d(y,L)}{q}.
\end{align*}
Setze nun $x_q = \frac{y-x_0}{\norm{y-x_0}}$, so ist $\norm{x_q}=1$ und für
$x\in L$ gilt,
\begin{align*}
\norm{x_q-x} &= \frac{1}{\norm{y-x_0}}\norm{y-x_0-\norm{y-x_0}x}
\\ &= \frac{1}{\norm{y-x_0}}\norm{y-\underbrace{(x_0+\norm{y-x_0}x)}_{\in L}}
\ge \frac{d(y,L)}{\norm{y-x_0}} \ge q.\qedhere
\end{align*}
\end{proof}

\begin{cor}
\label{prop:6.10}
Sei $E$ normierter Raum, dann ist $\overline{K_1(0)}$ genau dann kompakt, wenn
$\dim E<\infty$.\fishhere
\end{cor}

\begin{bsp}
\label{bsp:6.11}
\begin{bspenum}
  \item $\Id: E\to E$ ist genau dann kompakt, wenn $\dim E<\infty$.
  \item Sei $T\in\LL(E\to F)$ und $m:=\dim \im T < \infty$, so ist $T$ kompakt,
  denn ist $(x_n)$ beschränkt in $E$, so ist $(Tx_n)$ beschränkt in $\im T$.
  $\im T$ ist endlichdimensional und daher isomorph zum $\K^n$, also besitzt
  $(Tx_n)$ eine konvergente Teilfolge.
  \item Sei $E:=(C([a,b]\to\C),\norm{\cdot}_\infty)$ und $K\in
  C([a,b]\times[a,b]\to\C)$, so ist
\begin{align*}
T : E\to E,\quad Tf(x) := \int_a^b K(x,y)f(y)\dy
\end{align*}
kompakt.\bsphere
\end{bspenum}
\end{bsp}

\begin{prop}
\label{prop:6.12}
Sei $T\in\KK(E)$. Dann gelten
\begin{propenum}
  \item $\dim \ker (\Id - T) < \infty$.
  \item $\im (\Id - T)$ ist abgeschlossener linearer Teilraum von $E$.
  \item Ist $\ker (\Id-T) = (0)$, so ist $\im (\Id - T) = E$ und
  $(\Id-T)^{-1}\in\LL(E)$.\fishhere
\end{propenum}
\end{prop}

Für $\Id-T$ mit $T$ kompakt gilt also tatsächlich injektiv $\Rightarrow$
surjektiv.

\begin{proof}
\begin{proofenum}
  \item Sei $M=\ker(\Id-T)$. Wir zeigen, dass $\Id:M\to M$ kompakt
  ist, dann folgt mit \ref{prop:6.10}, dass $M$ endlichdimensional ist.
  
  Sei also $(x_n)$ beschränkt in $M$, dann gilt $Tx_{n_k}\to y$ und damit auch
\begin{align*}
&x_{n_k} = \underbrace{(\Id-T)(x_{n_k})}_{\to 0} + Tx_{n_k} \to y,\\
\Rightarrow\;& (\Id- T)y = \lim\limits_{k\to\infty} (\Id -T)x_{n_k} = 0.
\end{align*}
Also ist $y\in M$.
\item $\im (\Id-T)$ ist stets ein linearer Teilraum von $E$.

Sei nun $y_n$ Folge in $\im(\Id-T)$, d.h. $y_n=(\Id-T)x_n$, mit $y_n\to y\in E$.
\begin{proofenuma}
  \item\textit{$(x_n)$ kann als beschränkte Folge gewählt werden}. Wähle
  $\xi_n \in \ker (\Id-T)$ so, dass
\begin{align*}
&\norm{x_n-\xi_n} \le 2d(x_n,\ker(\Id-T))
\end{align*}
Dann gilt
\begin{align*}
&(\Id-T)(x_n-\xi_n) = y_n,\\
&\norm{x_n-\xi_n} \le 2d(x_n-\xi_n,\ker(\Id-T)) =: d_n.
\end{align*}
Angenommen $d_n\to \infty$. Setze $z_n := \frac{x_n-\xi_n}{d_n}$, so gilt
\begin{align*}
\norm{z_n} \le 1,\qquad d(z_n,\ker(\Id-T)) = \frac{1}{d_n}d(x_n-\xi,\ker(\Id-T))
= \frac{1}{2}.
\end{align*}
Nun ist $z_n$ beschränkt, also $Tz_{n_k}\to z$, wobei
\begin{align*}
z_{n_k} = \underbrace{(\Id-T)(z_{n_k})}_{\to 0} + Tz_{n_k} \to z.
\end{align*}
Somit gilt
\begin{align*}
(\Id-T)z = \lim\limits_{k\to\infty} (\Id-T)z_{n_k} = z-z=0.
\end{align*}
Also $z\in\ker (\Id-T)$, wobei $z_{n_k}\to z$. Dies ist ein Widerspruch zu
$d(z_n,\ker(\Id-T))=\frac{1}{2}$.

Somit ist $\norm{x_n-\xi_n}\le d_n \le c$ mit $(\Id-T)(x_n-\xi_n)=y_n$. Im
Folgenden sei daher $(x_n)$ beschränkte Folge.
\item $(x_n)$ ist beschränkt, also $Tx_{n_k}\to x$, mit
\begin{align*}
&x_{n_k} = (\Id-T)x_{n_k} + Tx_{n_k} = y_{n_k} + Tx_{n_k} \to y + x,\\
\Rightarrow &
(\Id-T)(y+x) = \lim\limits_{k\to\infty} (\Id-T)(x_{n_k}) =
\lim\limits_{k\to\infty} y_{n_k} = y.
\end{align*}
Also ist $y\in \im (\Id-T)$.
\end{proofenuma}
\item Sei $\ker(\Id-T)=(0)$. Angenommen $\Id-T$ ist nicht surjektiv, es gibt
also ein $x\in E\setminus\im (\Id-T)$.
\begin{proofenuma}
\item Wir zeigen nun $\forall n\in \N: (\Id-T)^n(x) \in E\setminus
\im (\Id-T)^{n+1}$ durch Widerspruch. Angenommen
\begin{align*}
\exists n\in\N \exists y\in E : (\Id-T)^n(x) = (\Id-T)^{n+1}y,
\end{align*}
so ist $(\Id-T)^n(x-(\Id-T)y) = 0$, da $\ker (\Id-T) = (0)$ gilt
\begin{align*}
(\Id-T)^{n-1}(x-(\Id-T)y) = 0.
\end{align*}
Wir erhalten induktiv,
\begin{align*}
x-(\Id-T)y = 0 \Rightarrow x\in\im (\Id-T).\dipper
\end{align*}
\item \textit{$\im (\Id-T)^{n+1}$ ist abgeschlossen}.
\begin{align*}
(\Id-T)^{n+1} = \sum\limits_{j=0}^{n+1}\binom{n+1}{j} (-T)^j
= \Id - \underbrace{\left(-\sum\limits_{j=1}^{n+1}\binom{n+1}{j}
(-T)^j\right)}_{\text{kompakt}}.
\end{align*}
Anwendung von 2.) ergibt die Behauptung.
\item Mit a.) und b.) folgt, $d((\Id-T)^nx,\im(\Id-T)^{n+1})>0$. Es existiert
also eine Folge $(a_n)$ in $\im (\Id-T)^{n+1}$ mit
\begin{align*}
0 \le \norm{(\Id-T)^{n}x-a_n} < 2d((\Id-T)^nx, \im (\Id-T)^{n+1}).
\end{align*}
Setze $x_n = \frac{1}{\norm{(\Id-T)^nx-a_n}}((\Id-T)^nx-a_n)$, dann ist
$\norm{x_n}=1$, also ist $(x_n)$ beschränkt, und es gilt $x_n\in \im
(\Id-T)^n$. Jedoch
\begin{align*}
d(x_n,\im(\Id-T)^{n+1}) &= \frac{1}{\norm{(\Id-T)^nx-a_n}}d
\left((\Id-T)^nx -a_n,\im (\Id-T)^{n+1}\right)\\
&=
\frac{1}{\norm{(\Id-T)^nx-a_n}}d
\left((\Id-T)^nx,\im (\Id-T)^{n+1}\right) > \frac{1}{2},
\end{align*}
denn $a_n\in \im (\Id-T)^{n+1}$, also enthält $Tx_n$ auch keine konvergente
Teilfolge, denn für $m>n$ gilt,
\begin{align*}
\norm{Tx_m-Tx_n} &= \norm{(\Id-T)(x_m-x_n)-(x_m-x_n)}
\\ &= \norm{x_n - \left(x_m + (\Id-T)(x_m-x_n) \right)}.
\end{align*}
Nun ist $x_m\in \im (\Id-T)^{n}$ und $(\Id-T)(x_m-x_n)\in \im(\Id-T)^{n+1}$,
also
\begin{align*}
\norm{x_n - \left(x_m + (\Id-T)(x_m-x_n) \right)}
\ge d(x_n, \im (\Id-T)^{n+1}) > \frac{1}{2},
\end{align*}
im Widerspruch zur Kompaktheit von $T$. Also $\im(\Id-T)=E$. 
\item
$(\Id-T)$ ist also bijektiv und daher existiert $(\Id-T)^{-1}$. Wir müssen noch
zeigen, dass $(\Id-T)^{-1}\in\LL(E)$. Angenommen $(\Id-T)^{-1}$ ist
unbeschränkt, d.h.
\begin{align*}
\sup\limits_{y\neq 0} \frac{\norm{(\Id-T)^{-1}}}{\norm{y}} = \infty.
\end{align*}
Wir können also $(y_n)$ in $E$ wählen mit $y_n\to 0$ und
$\norm{(\Id-T)^{-1}y_n}=1$. Setze $x_n=(\Id-T)^{-1}y_n$, so ist $\norm{x_n}=1$
und $y_n=(\Id-T)x_n$. Nach Voraussetzung existiert eine konvergente Teilfolge $(Tx_{n_k})$ mit Grenzwert
$y$. Für diese gilt dann
\begin{align*}
x_{n_k} = y_{n_k}+Tx_{n_k} \to 0 + y = y.
\end{align*}
Somit ist
\begin{align*}
(\Id-T)y &= \lim\limits_{k\to\infty} (\Id-T)x_{n_k}
= \lim\limits_{k\to\infty} x_{n_k}-Tx_{n_k} = 
\lim\limits_{k\to\infty} y_{n_k}+Tx_{n_k}-Tx_{n_k} \\ &= 
\lim\limits_{k\to\infty} y_{n_k} = 0.
\end{align*}
Da $\ker (\Id-T)=(0)$ folgt $y=0$ aber dies steht im Widerspruch zu
$\norm{x_n}=1$ und $x_{n_k}\to y$. Also ist $(\Id-T)^{-1}$ beschränkt.\qedhere
\end{proofenuma}
\end{proofenum}
\end{proof}

\begin{prop}
\label{prop:6.13}
Sei $T\in\KK(E)$. Dann gelten
\begin{propenum}
  \item Sei $\lambda\in\sigma(T)$ und $\lambda\neq 0$, dann ist $\lambda$
  Eigenwert endlicher Vielfachheit.
  \item $\sigma(T)$ hat höchstens $0$ als Häufungspunkt. Insbesondere ist
  $\sigma(T)$ endlich oder abzählbar und jeder Eigenwert $\neq 0$ ist
  isoliert.\fishhere
\end{propenum}
\end{prop}
\begin{proof}
\begin{proofenum}
  \item Sei $\lambda\in\sigma(T)$ und $\lambda\neq 0$, dann ist $T-\lambda \Id$
  nicht bijektiv und somit auch $\frac{1}{\lambda}T-\Id$.
  
  $\lambda$ ist Eigenwert, denn falls $\ker (\frac{1}{\lambda}T-\Id) = (0)$, so
  ist nach \ref{prop:6.12} $\frac{1}{\lambda}T-\Id$ bijektiv. Außerdem ist
  $\dim \ker (\frac{1}{\lambda}T-\Id)< \infty$, also hat $\lambda$ endliche
  Vielfachheit.
  \item Sei $\lambda\neq 0$ Häufungspunkt, so existiert eine Folge
  $(\lambda_n)$ in $\sigma(T)$ mit $\lambda_n\to\lambda$ und $\lambda_n\neq
  \lambda_m$, falls $n\neq m$.
  
  Sei $Tx_n = \lambda_n x_n$ und $\norm{x_n}=1$, so gilt für jedes $N\in\N$,
$\setd{x_1,\ldots,x_N}$
ist linear unabhängig. Sei $V_n=\lin{x_1,\ldots,x_n}$, so ist $\dim V_n = n$
und $V_{n+1}\supsetneq V_n$. Außerdem ist $V_n$ abgeschlossen, wir können also
das Lemma von Riesz anwenden und erhalten so eine Folge $(y_n)$ mit
\begin{align*}
y_n\in V_{n+1}\setminus V_n,\qquad \norm{y_n} = 1,\qquad d(y_n,V_n)=
\frac{1}{2}.
\end{align*}
$(y_n)$ ist beschränkt aber $(Ty_n)$ enthält keine konvergente Teilfolge.
Schreibe dazu $y_n=sx_{n+1} + \eta_n$ mit $s\neq 0$ und $\eta_n\in
V_n$, so gilt für $m>n$
\begin{align*}
&\norm{Ty_m-Ty_n} = \norm{T(sx_{m+1}+\eta_m)-T(sx_{n+1}+\eta_n)}\\
&= \norm{s\lambda_{m+1}x_{m+1}+T\eta_m - s\lambda_{n+1}x_{n+1}-T\eta_n}\\
&= \norm{\lambda_{m+1}y_m - \lambda_{m+1}\eta_m+T\eta_m - \lambda_{n+1}y_n +
\lambda_{n+1}\eta_n -T\eta_n}\\
&= \abs{\lambda_{m+1}}\norm{y_m -
\underbrace{\frac{1}{\lambda_{m+1}}\left(\lambda_{m+1}\eta_m-T\eta_m +
\lambda_{n+1}y_n -
\lambda_{n+1}\eta_n +T\eta_n\right)}_{\in V_m}}\\
&> \abs{\lambda_{m+1}}\frac{1}{2} \ge \frac{1}{2}c > 0.\qedhere
\end{align*}
\end{proofenum}
\end{proof}

\section{Fredholmsche Alternative}

In der linearen Algebra studiert man die Lösbarkeit von linearen
Gleichungssystemen. Sei $A$ eine reelle $n\times n$-Matrix und
\begin{align*}
T: \R^n\to\R^n,\quad x\mapsto Ax,
\end{align*}
der zugehörige Homomorphismus. Gesucht sind nun Lösungen von $Ax=y$, d.h.
$Tx=y$. In endlichdimensionalen Räumen gibt es zwei Fälle zu unterscheiden:
\begin{propenum}
  \item \textit{$T$ ist bijektiv}. So gilt $\forall y\in\R^n \exists ! x\in\R^n
  : Tx=y$. Hier ist außerdem $T$ genau dann bijektiv, wenn $\ker T = (0)$.
  \item \textit{$T$ nicht bijektiv}. Hier ist $T$ also weder surjektiv
  noch injektiv. Lösungen exisieren genau dann, wenn folgende Rangbedingung erfüllt
  ist,
\begin{align*}
\rg(A\mid y) = \rg(A) \Leftrightarrow
\rg\begin{pmatrix}
   A^\top \\ y^\top
   \end{pmatrix}
= \rg(A^\top).
\end{align*}
Dies ist wiederum äquivalent dazu, dass $y^\top$ linear abhängig von den
Zeilenvektoren von $A^\top$ ist, d.h.
\begin{align*}
\forall z\in\R^n : (A^\top z = 0\Rightarrow y^\top z = 0).
\end{align*}
Da $\R^n$ euklidisch, existiert ein Skalarprodukt und wir können dies
schreiben als
\begin{align*}
\forall z\in\R^n : (T^\top z = 0\Rightarrow \lin{y,z} = 0).
\end{align*}
\end{propenum}

Wir wollen diese Fragestellung nun auf unendlichdimensionale normierte
Vektorräume übetragen.
\begin{propenum}
  \item Betrachten wir dazu
\begin{align*}
\Id-T : E\to E,\quad x\mapsto (\Id-T)x,
\end{align*}
so impliziert (nach Satz \ref{prop:6.12}) $\ker (\Id-T)=(0)$, dass $(\Id-T)$
bijektiv und daher $(\Id-T)x=y$ für jedes $y\in E$ eindeutig lösbar ist.
Außerdem hängt die Lösung
\begin{align*}
x = (\Id-T)^{-1}y
\end{align*}
stetig von $y$ ab.
\item Die Bedingung  
\begin{align*}
\forall z\in\R^n : (T^\top z = 0\Rightarrow \lin{y,z} = 0).
\end{align*}
lässt sich jedoch nicht ohne Weiteres auf einen normierten Raum $E$ übertragen,
da weder $T^\top$ noch $\lin{\cdot,\cdot}$ zur Verfügung stehen.

Für Hilbeträume kennen wir bereits die
\begin{prop}[Riesz Abbildung]
\index{Satz!Riesz Abbildung}
\label{prop:6.14}
Sei $H$ ein Hilbertraum.
\begin{align*}
R_H: H\to H',\quad x\mapsto \ph_x,
\end{align*}
mit $\ph_x(y) = \lin{y,x}$ ist eine antilineare, bijektive und isometrische
Abbildung.\fishhere
\end{prop}

Wir wollen nun ein Analogon des $\lin{\cdot,\cdot}$ für normierte Räume durch
Anwendung eines Elementes des Dualraumes finden.
\end{propenum}

\begin{defn}
\label{defn:6.15}
Sei $L\subseteq E$ linearer Teilraum. Dann heißt
\begin{align*}
L^\bot := \setdef{x'\in E'}{x'\big|_L = 0} = \setdef{x'\in E'}{L\subseteq \ker
x'}
\end{align*}
\emph{Annihilator}\index{Annihilator} von $L$.\fishhere
\end{defn}

\begin{prop}
\label{prop:6.16}
\begin{propenum}
  \item $L^\bot$ ist ein abgeschlossener linearer Teilraum von $E'$.
  \item $L^\bot = \overline{L}^\bot$.
  \item Ist $E$ reflexiv, so gilt $(L^\bot)^\bot = J_E(\overline{L}) =
  \overline{J_E(L)}$.\fishhere
\end{propenum}
\end{prop}
\begin{proof}
\begin{proofenum}
  \item Sei $(x_n')$ Folge in $E'$ mit $x_n'(y)=0$ und $x_n'\to x'\in E'$. Sei
  $y\in L$, dann gilt
\begin{align*}
x'(y) = \lim\limits_{n\to\infty} x_n'(y) = 0.
\end{align*}
\item $L\subseteq \overline{L}$, also $(\overline{L})^\bot \subseteq L^\bot$.
Sei weiterhin $x'\in L^\bot$ und $y_n$ Folge in $L$ mit $y_n\to y\in E$, dann
gilt
\begin{align*}
x'(y) = \lim\limits_{n\to\infty} x'(y_n) = 0.
\end{align*}
Somit ist $L^\bot\subseteq \overline{L}^\bot$.
\item Sei $x''\in (L^\bot)^\bot\subseteq E''$, so existiert ein $x\in E$, so
dass $J_E(x) = x''$.
\begin{align*}
x''\in (L^\bot)^\bot 
&\Leftrightarrow \forall x' \in L^\bot : x''(x') = 0\\
&\Leftrightarrow \forall x' \in L^\bot : J_E(x)(x') = x'(x) = 0\\
&\Leftrightarrow \forall x' \in L^\bot : x\in \ker x'.
\end{align*}
Satz \ref{prop:4.17} besagt, dass
\begin{align*}
\overline{L} = \bigcap\setdef{\ker y'}{L\subseteq \ker y'},
\end{align*}
also gilt
\begin{align*}
\forall x' \in L^\bot : x\in \ker x'
\Leftrightarrow x\in \bigcap\setdef{\ker x'}{x'\in L^\bot} = \overline{L}.
\end{align*}
Somit ist $x''=J_E(x) \in (L^\bot)^\bot \Leftrightarrow x\in
\overline{L}\Leftrightarrow x'' = J_E(x) \in J_E(\overline{L})$.\qedhere
\end{proofenum}
\end{proof}

\begin{defn}
\label{defn:6.17}
Sei $T\in\LL(E\to F)$.
\begin{align*}
T' : F'\to E',\quad y'\mapsto y'\circ T,
\end{align*}
heißt \emph{adjungierter Operator}\index{Operator!adjungiert} zu $T$.\fishhere
\end{defn}

\begin{prop}
\label{prop:6.18}
\begin{propenum}
  \item $\norm{T'}_{\LL(F'\to E')}=\norm{T}_{\LL(E\to F)}$.
  \item $(\alpha T_1+\beta T_2)' = \alpha T_1' + \beta T_2'$.
  \item Sei $T'' = (T')' : E''\to F''$. Dann gilt
  \begin{align*}
  T''\big|_{J_E(E)} = J_F\circ T\circ J_E^{-1}
  \end{align*}
bzw.
  \begin{align*}
  T''\circ J_E = J_F\circ T
  \end{align*}
bzw.
  \begin{align*}
  J_F^{-1}\circ T''\circ J_E = T.\fishhere
  \end{align*}
\end{propenum}
\end{prop}
\begin{proof}
Der Beweis findet sich in Übungsaufgabe (6.3).\qedhere
\end{proof}

\begin{bsp}
\label{bsp:6.19}
\begin{bspenum}
  \item $(\Id_{E\to E})' = \Id_{E'\to E'} : E'\to E',\quad y'\mapsto y'\circ
  \Id = y'$.
  \item $T:\C^n\to \C^n,\; x\mapsto Ax$ mit $A\in M^{n\times n}$. Da
  $(\C^n)'=\C^n$ gilt,
\begin{align*}
T': \C^n \to \C^n,\quad x\mapsto A^\top x.\bsphere
\end{align*}
\end{bspenum}
\end{bsp}

\begin{prop}
\label{prop:6.20}
Sei $T\in\LL(E\to F)$. Dann ist $(\im T)^\bot = \ker T'$.\fishhere
\end{prop}
\begin{proof}
Für $y'\in E$ gilt
\begin{align*}
y'\in \ker T' &\Leftrightarrow T'y' = 0 \Leftrightarrow \forall x\in E :
(T'y')(x) = 0 \\ &\Leftrightarrow \forall x\in E : y'\circ Tx  = 0
\Leftrightarrow \forall y\in \im T : y'(y) = 0\\
&\Leftrightarrow \im T \subseteq \ker y' \Leftrightarrow y'\in (\im
T)^\bot.\qedhere
\end{align*}
\end{proof}

\begin{lem}
\label{prop:6.21}
Sei $T\in \KK(E\to F)$. Dann ist $\im T$ separabel.\fishhere
\end{lem}
\begin{proof}
Es gilt $\im T = \bigcup_{n\in\N} T(K_n(0))$. Da $T$ kompakt ist auch
$\overline{T(K_n(0))}$ kompakt, also gilt
\begin{align*}
\forall l\in \N : \overline{T(K_n(0))} \subseteq \bigcup_{j=1}^{J(l)}
K_{1/l}(y_j^{(l)}),\qquad y_j^{(l)}\in \overline{T(K_n(0))},\quad J(l) < \infty.
\end{align*}
Wähle $\eta_j^{l}\in T(K_n(0))$ mit $\norm{\eta_j^{(l)}-y_j^{(l)}} <
\frac{1}{l}$, so gilt
\begin{align*}
\overline{T(K_n(0))} \subseteq \bigcup_{j=1}^{J(l)} K_{2/l}(\eta_j^{(l)}),\qquad
\eta_j^{(l)}\in T(K_n(0)).
\end{align*}
Setze nun $A:= \bigcup_{l=1}^\infty
\setd{\eta_1^{(l)},\ldots,\eta_{J(l)}^{(l)}}$, so ist $A$ abzählbar und dicht
in $\overline{T(K_n(0))}$ und $A\subseteq T(K_n(0))$, also ist $A$ auch dicht
in $T(K_n(0))$.
\begin{align*}
\Rightarrow \bigcup_{n=1}^\infty A_n \text{ ist abzählbar und dicht in
}\bigcup_{n=1}^\infty T(K_n(0)) = \im T.\qedhere
\end{align*}
\end{proof}

\begin{prop}[Satz von Schauder]
\index{Satz!Schauder}
\label{prop:6.22}
Für $T\in \LL(E\to F)$ gilt
\begin{propenum}
\item $T\in \KK(E\to F) \Rightarrow T'\in \KK(F'\to E')$.
\item Falls $F$ Banachraum, dann gilt $T\in \KK(E\to F) \Leftarrow T'\in
\KK(F'\to E')$.\fishhere
\end{propenum}
\end{prop}
\begin{proof}
``$\Rightarrow$'': Sei also $T\in\KK(E\to F)$, $(y_n')$ beschränkte Folge in
$F'$, $\norm{(y_n')} \le C$. Zu zeigen ist nun, dass $(T'y_n')$ eine konvergente
Teilfolge besitzt.
\begin{proofenum}
  \item \textit{Konstruktion eines Grenzelements}. Nach Lemma \ref{prop:6.21}
  ist $\im T$ separierbar, also $\im T = \overline{\setd{y_1,y_2,\ldots}}$.

$(y_n'(y_1))_n$ ist Folge in $\K$ und beschränkt, besitzt also eine konvergente
Teilfolge $(y_n'^{(1)})$.

$(y_n'^{(1)}(y_2))$ ist Folge in $\K$ \ldots, besitzt also eine konvergente
Teilfolge $(y_n'^{(2)})$.

Wähle die Diagonalfolge $y_{n_k}' = y_k'^{(k)}$.

\item \textit{$(y_{n_k}')$ konvergiert auf $\lin{y_1,y_2,\ldots}=:L$}. Sei 
\begin{align*}
y' : L\to \K,\quad y\mapsto \lim\limits_{k\to\infty} y_{n_k}'(y),
\end{align*}
so ist $y'$ linear und beschränkt, denn
\begin{align*}
\norm{y'(y)} \le \limsup\limits_{k\to\infty} \norm{y_{n_k}'}\norm{y} \le
C\norm{y},
\end{align*}
also $y'\in L'$. Mit dem Satz von Hahn-Banach setzen wir $y'$ fort zu $y'\in
F'$.
\item \textit{Es gilt sogar für $y\in \overline{L} = \overline{\im T}$, dass
$y_{n_k}'(y) \to y'(y)$}.

Sei $(y_l)$ Folge in $L$ mit $y_l\to y$, so gilt
\begin{align*}
&\abs{y_{n_k}'(y)-y'(y)}\\ &
 \le \underbrace{\abs{y_{n_k}'(y-y_l)}}_{\text{(1)}} +
\underbrace{\abs{y_{n_k}'(y_l)-y'(y_l)}}_{\text{(2)}} + 
\underbrace{\abs{y'(y_l)-y'(y)}}_{\text{(3)}}.
\end{align*}
(1)$\le \sup\norm{y_{n_k}'}\norm{y-y_l} = c \norm{y-y_l}$,\\
(3)$\le \norm{y'}\norm{y-y_l}$,\\
also (1),(3)$\le \ep$ für $l$ hinreichend groß.\\
(2)$\le \ep$ für $l$ fest und $k$ hinreichend, also
\begin{align*}
\abs{y_{n_k}'(y)-y'(y)} \le 3 \ep,
\end{align*}
für $k$ hinreichend groß.
\item \textit{Für eine Teilfolge $(y_{n_j}')$ von $(y_{n_k}')$ gilt
$T'(y_{n_j}')\to y$}.
\begin{align*}
\norm{T'y_{n_k}' - T'y'} = \sup\limits_{\norm{x}=1}
\abs{(T'y_{n_k}'-T'y')(x)}.
\end{align*}
Wähle $(x_k)$ in $E$ mit $\norm{x_k} = 1$ und
\begin{align*}
\norm{T'y_{n_k}' - T'y'} = \abs{(T'y_{n_k}'-T'y')(x_k)} + \frac{1}{n_k}.
\end{align*}
$T$ ist kompakt also $Tx_{n_j}\to y \in \overline{\im T}$. Somit gilt
\begin{align*}
&\norm{T'y_{n_j}' - T'y'} =
\abs{T'y_{n_j}'(x_{n_j}) - T'y'(x_{n_j})} + \frac{1}{n_j}\\
&= \abs{y_{n_j}'(Tx_{n_j})-y'(Tx_{n_j})} + \frac{1}{n_j}\\
&\le \underbrace{\abs{y_{n_j}'(Tx_{n_j}-y)}}_{\text{(1)}} +
\underbrace{\abs{y_{n_j}'(y)-y'(y)}}_{\text{(2)}} +
\underbrace{\abs{y'(y)-y'(Tx_{n_j})}}_{\text{(3)}} + \frac{1}{n_j}
\end{align*}
(1)$\le \sup\norm{y_{n_j}}\norm{Tx_{n_j}-y} \to 0$,\\
(3)$\le \norm{y'}\norm{Tx_{n_j}-y}\to 0$,\\
(2)$ \to 0$ nach b.), also
\begin{align*}
\norm{T'y_{n_j}' - T'y'} \to 0,\qquad j\to\infty.
\end{align*}
``$\Leftarrow$'': Sei $T'$ kompakt so ist nach ``$\Rightarrow$'' auch $T''\in
\KK(E''\to F'')$. Mit \ref{prop:6.18} folgt
\begin{align*}
T = J_F^{-1}\circ T''\circ J_E.
\end{align*}
Sei $(x_n)$ beschränkt in $E$, so ist $(J_E(x_n))$ beschränkt in $E''$ und
daher
\begin{align*}
T'(J_E(x_{n_k}))\to y''\in F''. 
\end{align*}
Da $F$ vollständig ist $J_F(F)$
abgeschlossen und daher $y''\in J_F(F)$, somit
\begin{align*}
T(x_{n_k}) = J_F^{-1}\circ T''\circ J_E(x_{n_k}) \to J_F^{-1}(y'').\qedhere
\end{align*}
\end{proofenum}
\end{proof}

\begin{prop}[Fredholmsche Alternative]
\index{Fredholmsche!Alternative}
\label{prop:6.23}
Sei $E$ normierter Raum, $T\in\KK(E)$. Dann gilt entweder
\begin{equivenum}
  \item\label{prop:6.23:1} $\ker (\Id-T)=(0)$. Dann besitzt die Gleichung
\begin{align*}
(\Id - T)x = y
\end{align*}
für jedes $y\in E$ eine Lösung. Die Lösung ist eindeutig und hängt stetig von
$y$ ab.

--- oder ---
\item\label{prop:6.23:2} $\ker(\Id-T)\neq 0$. Dann besitzt 
\begin{align*}
(\Id-T)x = y
\end{align*}
genau für die $y\in E$ Lösungen, für die
\begin{align*}
\forall x' \in \ker (\Id-T') : x'(y) = 0.
\end{align*}
Die dadurch gegebene Anzahl von Nebenbedingungen ist endlich. In diesem Fall
ist $x$ nicht eindeutig.\fishhere
\end{equivenum}
\end{prop}
\begin{proof}
``\ref{prop:6.23:1}'': Siehe Satz \ref{prop:6.12}.

``\ref{prop:6.23:2}'': Nach Satz \ref{prop:6.12} ist $\im(\Id-T) =
\overline{\im(\Id-T)}$, also
\begin{align*}
&y\in \im(\Id-T) = \bigcap
\setdef{\ker x'}{x'\in E \land \im (\Id-T) \subseteq \ker (x')}\\
&\Leftrightarrow
\forall z'\in \setdef{x'\in E}{\im
(\Id-T)\subseteq \ker x'} : y\in \ker (z')\tag{*}
\end{align*}
$T$ ist kompakt, also folgt mit \ref{prop:6.22}, dass $T'$ kompakt und daher ist
$\ker(\Id-T')$ endlichdimensional. Sei $\BB=\setd{x_1',\ldots,x_k'}$
Basis von $\ker(\Id-T')$, so ist (*) äquivalent zu
\begin{align*}
\forall j=1,\ldots,k : x_j(y) = 0.\qedhere
\end{align*}
\end{proof}

\begin{defn}
\label{defn:6.24}
Seien $U$, $V$ lineare Teilräume von $L$ mit
\begin{defnenum}
\item $U\cap V = (0)$,
\item $\forall x\in L : \exists u\in U\exists v\in V : x = u+v$.
\end{defnenum}
Dann schreiben wir $L=U\oplus V$ und $U\oplus V$ heißt \emph{direkte
Summe}\index{Vektorraum!Direkte Summe} von $U$ und $V$.\fishhere
\end{defn}

Eine leichte Übung zeigt, dass $u$ und $v$ in obiger Definition eindeutig
bestimmt sind.

Im Gegensatz zur Definition \ref{defn:1.13} muss die Norm auf $U\oplus V$
\textit{nicht} äquivalent zur Norm auf $L$ sein.

\begin{lem}
\label{lem:6.25}
Sei $L=U\oplus V$ und $L=U\oplus \tilde{V}$ mit $\dim V,\dim \tilde{V} <
\infty$, so ist
\begin{align*}
\dim V = \dim \tilde{V}.\fishhere
\end{align*}
\end{lem}
\begin{proof}
Sei $\BB=\setd{x_1,\ldots,x_n}$ Basis von $V$. Dann gilt
\begin{align*}
x_i = u_j + \tilde{v}_j,\qquad u_j\in U,\quad \tilde{v}_j \in \tilde{V}.
\end{align*}
$\setd{\tilde{v}_1,\ldots,\tilde{v}_n}$ ist linear unabhängig, denn sei
\begin{align*}
\sum_{j=1}^n c_j \tilde{v}_j = 0
\Rightarrow
\underbrace{\sum_{j=1}^n c_j x_j}_{\in V} = 
\underbrace{\sum_{j=1}^n c_j u_j}_{\in U} \overset{U\cap V=(0)}{=} 0
\end{align*}
und da $\BB$ Basis, folgt $c_1=\ldots=c_n=0$. Also ist
$\setd{\tilde{v}_1,\ldots,\tilde{v}_n}$ linear unabhängig und daher $\dim
\tilde{V}\ge \dim V$. Die Situation ist jedoch vollkommen symmetrisch, also
gilt auch $\dim V \ge \dim \tilde{V}$ und folglich $\dim V = \dim
\tilde{V}$.\qedhere
\end{proof}

\begin{defn}
\label{defn:6.26}
Sei $L=U\oplus V$ mit $\dim V < \infty$. Dann heißt
\begin{align*}
\codim U := \dim V
\end{align*}
die \emph{Kodimension}\index{Vektorraum!Kodimension} von $U$.\fishhere
\end{defn}

\begin{lem}
\label{prop:6.27}
Sei $E$ normierter Raum, $L\subseteq E$ abgeschlossener linearer Teilraum,
$V\subseteq E$ endlichdimensionaler Teilraum und $L\cap V = (0)$. Dann ist
$L\oplus V$ abgeschlossener linearer Teilraum.\fishhere
\end{lem}
\begin{proof}
Sei $(x_n)$ Folge in $L\oplus V$ mit $x_n\to x$ in $E$. Zeige $x\in L\oplus V$.

Da $x_n\in L\oplus V$ existiert eine eindeutige Zerlegung,
\begin{align*}
x_n = u_n + v_n,\qquad u_n\in L,\quad v_n\in V.
\end{align*}
Wir bilden den Quotientenraum
\begin{align*}
E/L := \setdef{[x]=x+L}{x\in E}.
\end{align*}
Da $L$ abgeschlossen ist
\begin{align*}
\norm{[x]}_0 = \inf\setdef{\norm{x-y}}{y\in L} = d(x,L)
\end{align*}
Norm auf $E/L$. $[x_n]$ ist konvergent bezüglich dieser Norm, denn
\begin{align*}
\norm{[x_n]-[x]}_0 = 
\inf\setdef{\norm{x_n-x-y}}{y\in L} \le \norm{x_n-x}\to 0.
\end{align*}
Außerdem ist $([v_n])$ Cauchyfolge in $[V]:=\setdef{[v]}{v\in V}$, denn
\begin{align*}
\norm{[v_n]-[v_m]}_0 &= \inf\setdef{\norm{v_n-v_m-y}}{y\in L}\\
&\le \norm{v_n+u_n - (v_m+u_m)} = \norm{x_n-x_m} \to 0.
\end{align*}
Da $L\cap V = (0)$, ist $\dim [V] = \dim V < \infty$ und daher ist $[V]$
vollständig und folglich $[v_n]\to [v]$. Insbesondere kann man für den
Grenzwert ein $v\in V$ als Vertreter wählen. Für dieses $v$ gilt
\begin{align*}
\norm{[x]-[v]}_0 = \lim\limits_{n\to\infty} \norm{[x_n]-[v_n]}
= \lim\limits_{n\to\infty} \norm{[u_n]} = 0, 
\end{align*}
denn $u_n\in L$, also ist auch $x-v\in L$. Setze $u:=x-v$, so ist $x=u+v\in
L\oplus V$.\qedhere
\end{proof}

Wir haben nicht bewiesen, dass $u_n\to u$ oder $v_n\to v$ sondern lediglich,
dass $x=u+v$ mit $u\in L$ und $v\in V$. Über die Konvergenz von $(u_n)$ und
$(v_n)$ lassen sich unter diesen allgemeinen Voraussetzungen keine genaueren
Aussagen treffen.

\begin{lem}
\label{prop:6.28}
Sei $E$ normierter Raum, $L\subseteq E$ abgeschlossener linearer Teilraum
endlicher Kodimension. Dann gilt
\begin{align*}
\codim L = \dim L^\bot.
\end{align*}
\end{lem}
\begin{proof}
Sei $E=L\oplus\lin{x_1,\ldots,x_n}$ mit $\setd{x_1,\ldots,x_n}$ linear
unabhängig.
\begin{align*}
L_j := L\oplus \lin{x_1,\ldots,x_{j-1},x_{j+1},\ldots,x_n}
\end{align*}
ist nach Lemma \ref{prop:6.27} abgeschlossen und $x_j\notin L_j$. Nach
\ref{prop:4.16} existieren daher $x_j'\in E'$ mit $L_j\subseteq \ker x_j'$ und
$x_j'(x_j)=1$.

Wir zeigen nun, dass $\setd{x_1',\ldots,x_n'}$ eine Basis von $L^\bot$ bildet.
\begin{proofenum}
\item Die lineare Unabhängigkeit ist klar, denn $x_j'(x_k)=\delta_{jk}$.
\item Sei $x'\in L^\bot$. $x\in E=L\oplus\lin{x_1,\ldots,x_n}$, so besitzt $x$
eine Darstellung,
\begin{align*}
x = u + \sum_{j=1}^n \alpha_j x_j,\qquad u\in L,\; \alpha_j\in\K,
\end{align*}
wobei
\begin{align*}
x'(x) = \underbrace{x'(u)}_{=0} + \sum_{j=1}^n \alpha_j x'(x_j).
\end{align*}
Nun besitzt auch $x_j$ eine Darstellung als
\begin{align*}
x_j = \sum_{l=1}^n x_l x_l'(x_j),
\end{align*}
also
\begin{align*}
x'(x) &= \sum_{j=1}^n \alpha_j x'(x_j)
= \sum_{j=1}^n \alpha_j \sum_{l=1}^n x_l'(x_j) x'(x_l)\\
&= \sum_{l=1}^n x'(x_l) \sum_{j=1}^n \alpha_j  x_l'(x_j)
= \sum_{l=1}^n x'(x_l)  x_l'\left(\underbrace{\sum_{j=1}^n  \alpha_j
x_j}_{x-u}\right)\\
&=
\sum_{l=1}^n x'(x_l)  x_l'(x).\qedhere 
\end{align*}
\end{proofenum}
\end{proof}

\begin{prop}
\label{prop:6.29}
Sei $T\in\KK(E)$. Dann gilt
\begin{align*}
\codim\im(\Id -T) \le \dim\ker (\Id-T).\fishhere
\end{align*}
\end{prop}
\begin{proof}
Sei $n=\dim\ker(\Id-T)$. Angenommen es existiert eine linear unabhängige
Teilmenge $\setd{y_1,\ldots,y_m}$ mit
$\lin{y_1,\ldots,y_m}\cap \im(\Id-T) = (0)$ und
 $m\ge n+1$.
 
 Sei $\setd{x_1,\ldots,x_n}$ Basis von $\ker(\Id-T)$, dann folgt mit
 \ref{prop:4.16}
 \begin{align*}
 \exists x_j'\in E' : x_j'(x_k) = \delta_{jk}.
 \end{align*}
Setze $\tilde{T}x := Tx + \sum_{j=1}^n x_j'(x)y_j$, so liefert scharfes
Hinsehen, dass $\tilde{T}$ kompakt. Betrachte nun
\begin{align*}
(\Id-\tilde{T})x = 0 
&\Leftrightarrow (\Id-T)x - \sum_{j=1}^n x_j'(x)y_j = 0
\Leftrightarrow \underbrace{(\Id-T)x}_{\in \im (\Id-T)} =
\underbrace{\sum_{j=1}^n x_j'(x)y_j}_{\in \lin{y_1,\ldots,y_n}}\\
&\Leftrightarrow (\Id-T)x = 0 \text{ und } \sum_{j=1}^n x_j'(x)y_j = 0
\end{align*}
Für ein solches $x$ folgt, da die $y_j$ linear unabhängig, $x_j'(x)=0$
für $j=1,\ldots,n$. Insbesondere
\begin{align*}
0 = x_j'(x) = \sum_{k=1}^n \alpha_k x_j'(x_k) = \alpha_j
\end{align*}
und daher $x=0$. Somit ist $\ker(\Id-\tilde{T}) = (0)$ und daher nach
\ref{prop:6.12}, $\im(\Id-\tilde{T}) = E$.

Zeige nun $\im(\Id-\tilde{T}) \subseteq \im (\Id-T)\oplus\lin{y_1,\ldots,y_n}$,
denn dann folgt
\begin{align*}
\im(\Id-T)\oplus \lin{y_1,\ldots,y_n} = E,
\end{align*}
im Widerspruch zu $\im(\Id-T)\cap \lin{y_1,\ldots,y_m} = (0)$, da $m\ge n+1$.

Für $x\in E$ gilt jedoch,
\begin{align*}
(\Id-\tilde{T})x = \underbrace{(\Id-T)x}_{\in\im (\Id-T)} - 
\underbrace{\sum_{j=1}^n x_j'(x)y_j}_{\in \lin{y_1,\ldots,y_n}} 
\in  \im(\Id-T)\oplus \lin{y_1,\ldots,y_n}.\qedhere
\end{align*}
\end{proof}

\begin{prop}
\label{prop:6.30}
Sei $T\in\KK(E)$. Dann gilt
\begin{align*}
\dim \ker(\Id-T) = \codim \im(\Id-T) = \dim \ker (\Id-T').\fishhere
\end{align*}
\end{prop}
\begin{proof}
\begin{proofenum}
\item\label{proof:6.30:1} $\dim \ker(\Id-T) \le \dim \ker(\Id-T'')$, denn
$J_E:E\to E''$ ist injektiv und nach \ref{prop:6.18} gilt
\begin{align*}
T''\circ J_E = J_E\circ T.
\end{align*}
Somit ist $(\Id-T'')\circ J_E = J_E\circ (\Id-T)$. (*)

Sei $\setd{x_1,\ldots,x_n}$ Basis von $\ker(\Id-T)$. In Gleichung (*)
verschwindet die rechte Seite für jedes Basiselement,
\begin{align*}
(\Id-T'')J_E(x_j) = 0 \Rightarrow \setd{J_E(x_1),\ldots,J_E(x_n)} \subseteq
\ker (\Id-T'').
\end{align*}
Da $J_E$ injektiv, ist die Menge linear unabhängig, also 
\begin{align*}
\dim \ker(\Id-T)\le\dim \ker(\Id-T'').
\end{align*}
\item\label{proof:6.30:2} Aus \ref{prop:6.29} folgt $\codim\im (\Id-T) \le
\dim\ker(\Id-T)$ also auch
\begin{align*}
\codim\im (\Id-T')\le \dim \ker(\Id-T').
\end{align*}
\item\label{proof:6.30:3} $\dim \ker(\Id-T') \overset{\ref{prop:6.20}}{=} \dim
(\im(\Id-T))^\bot \overset{\ref{prop:6.28}}{=} \codim \im (\Id-T)$. Somit
\begin{align*}
\dim \ker(\Id-T'') = \codim\im (\Id-T').
\end{align*}
\end{proofenum}
Aus \ref{proof:6.30:1}-\ref{proof:6.30:3} folgt,
\begin{align*}
\dim \ker (\Id-T) &\le \dim \ker(\Id-T'') = \codim \im (\Id-T') \le
\dim \ker(\Id-T') \\ &= \codim \im (\Id-T) \le \ker (\Id-T).
\end{align*}
Somit gilt Gleichheit in der gesamten Gleichung.\qedhere
\end{proof}

\begin{prop}[Fredholmsche Integralgleichung]
\index{Fredholmsche!Integralgleichung}
\label{prop:6.31}
Sei $E=C([a,b]\to\C)$. Zu $g\in E$ ist $f\in E$ gesucht mit
\begin{align*}
f(x) - \int_a^b K(x,y)f(y)\dy = g(x),\qquad a\le x\le b,
\end{align*}
wobei $K\in C([a,b]\times[a,b]\to \C)$.\fishhere
\end{prop}

In Übungsaufgabe 4.2 haben wir gezeigt, dass falls
\begin{align*}
\max\limits_{a\le x\le b} \int_a^b K(x,y)\dy < 1,\tag{*}
\end{align*}
$(\Id-T)^{-1}\in\LL(E)$ existiert, d.h.
\begin{align*}
\forall g\in E \exists ! f\in E : (\Id-T)f = g
\end{align*}
und die Lösung $f$ hängt stetig von $g$ ab.

Nun ist $T$ ein kompakter Operator und daher die Fredholm Theorie anwendbar.
Falls $\ker(\Id-T) = (0)$, so gilt dasselbe auch ohne die (*)-Bedingung.

Ist dagegen $\ker(\Id-T)\neq (0)$, so gibt es genau dann Lösungen zu $f$, falls
\begin{align*}
\forall h\in \ker(\Id-T') : h'(g) = 0.
\end{align*}
Dies ist äquivalent zu
\begin{align*}
\forall h\in E : \left(\underbrace{h(x) - \int_a^b \overline{K(y,x)}h(y)\dy =
0}_{(1)} \right) \Rightarrow \int_a^b g(x)\overline{h}(x)\dx = 0 .
\end{align*}
Insbesondere besagt \ref{prop:6.30}, dass (1) genau so viele linear unabhängige
Lösungen besitzt wie
\begin{align*}
f(x)-\int_a^b K(x,y)f(y)\dy = 0.
\end{align*}

\section{Ausblick}

Sei $H$ ein Hilbertraum, $T\in \KK(H)$ und $T$ symmetrisch, d.h.
\begin{align*}
\lin{Tx,y} = \lin{x,Ty}.
\end{align*}

\begin{prop}
$\lambda=\norm{T}$ oder $\lambda=-\norm{T}$ ist Eigenwert von $T$.\fishhere
\end{prop}

Für diesen Satz ist die Symmetrie von $T$ eine notwendige Voraussetzung, denn
betrachte für $H=l^2$ den modifizierten Shift-Operator
\begin{align*}
T(x_1,x_2,\ldots) = (0,\frac{x_1}{2},\frac{x_2}{3}), 
\end{align*}
so ist $T$ kompakt (vgl. Übungsaufgabe 10.2). $T$ besitzt aber keine
Eigenwerte, denn sei $\lambda = 0$ und $Tx = \lambda x$, so ist $x=0$ und sei
$\lambda \neq 0$ und $Tx = \lambda x$, so ist ebenfalls $x=0$ aber
$\norm{T}=\frac{1}{2}$.

\begin{prop}
Sei $T\in \KK(H)$ symmetrisch. Dann existiert ein ONS $(e_j)_{j\in\II}$ mit
$\II$ höchstens abzählbar und
\begin{defnenum}
\item $\forall j\in\II : T e_j = \lambda_j e_j$,\quad $\lambda_j\neq 0$.
\item $(\lambda_j)$ ist monoton fallend.
\item In $(\lambda_j)$ kommt jeder Eigenwert so oft vor, wie es seiner
endlichen Vielfachheit entspricht.
\item $(e_j)_{j\in\II}$ ist vollständig in $\overline{\im T}$, d.h.
\begin{align*}
\forall x\in \overline{\im T} : x = \sum_{j\in\II} \lin{x,e_j}e_j.\fishhere
\end{align*}
\end{defnenum}
\end{prop}