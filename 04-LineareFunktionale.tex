% ==============================================================
% ================== Lineare Funktionale ===========================
% ==============================================================
\chapter{Lineare Funktionale}

Im Folgenden wollen wir zu einem normierten Raum $(E,\norm{\cdot})$ den Dualraum
\begin{align*}
E'=\LL(E\to\K)
\end{align*}
und dessen Elemente $T\in E'$, die
\emph{Funktionale} genauer betrachten.

% ==============================================================
% ================== Verallg. Koords        ===========================
% ==============================================================
\section{Lineare Funktionale als verallgemeinerte Koordinaten}

Ist $v=0$ der Nullvektor, so sind alle Koordinaten von $v$ Null. Sind $v$ und
$w$ identische Vektoren, so sind auch alle Koordinaten identisch.

\begin{bsp}
\label{bsp:4.1}
$E$ sei endlichdimensional mit Basis $\BB=\setd{b_1,\ldots,b_n}$. Somit ist
\begin{align*}
e_k' : E\to \K,\; x =\sum\limits_{j=1}^n x_j b_j \mapsto x_k
\end{align*}
linear und beschränkt, denn
\begin{align*}
\norm{e_k'(x)} = \abs{x_k} \le \sum\limits_{j=1}^m \abs{x_j} = \norm{x}_1
\le c\norm{x}_E.
\end{align*}
$\setd{e_1',\ldots,e_n'}$ ist Basis von $E'$, die \emph{duale Basis} zu
$\setd{b_1,\ldots,b_n}$. (Insbesondere haben $E$ und $E'$ die gleiche
Dimension). Sei $T\in E'$, so gilt
\begin{align*}
&T(x) = T\left(\sum\limits_{j=0}^n x_j b_j\right)
= \sum\limits_{j=1}^n x_j T(b_j) = \sum\limits_{j=1}^n e_j'(x)T(b_j)
= \left(\sum\limits_{j=1}^n T(b_j)e_j' \right)(x)\\
\Rightarrow& T = \sum\limits_{j=1}^n T(b_j)e_j'.
\end{align*}

\textit{Eindeutigkeit}. Sei $T=\sum\limits_{j=1}^n t_j e_j' \Rightarrow T(b_k)
= \sum\limits_{j=1}^n t_j\delta_{jk} = t_k$.

Insbesondere gilt:
\begin{align*}
0=x\in E &\Leftrightarrow \forall j=1,\ldots,n : x_j = 0
\Leftrightarrow \forall j=1,\ldots,n : e_j'(x) = 0 \\ 
& \Leftrightarrow \forall T\in E' : T(x) = 0.
\end{align*}

$T\in E'$ kann also als verallgemeinerte Koordinate betrachtet werden.\bsphere
\end{bsp}

\begin{bsp}
\label{bsp:4.2}
Sei $E=C([0,1]\to\C)$ mit $\norm{\cdot}_E=\norm{\cdot}_\infty$. Zu $x\in[0,1]$
sei 
\begin{align*}
\delta_x : E\to \C : f\mapsto f(x),
\end{align*}
so ist $\delta_x$ linear sowie
\begin{align*}
\abs{\delta_x f} = \abs{f(x)} \le \norm{f}_\infty \Rightarrow \norm{\delta_x}
\le 1.
\end{align*}
Also ist $\delta_x \in E'$.
\begin{align*}
f=0\Leftrightarrow \forall x\in[0,1] : \delta_x(f) = 0
\Leftrightarrow \forall T\in E' : T(f) = 0.
\end{align*}

Es stellt sich nun natürlich die Frage, ob $f=0\Leftrightarrow \forall T\in E'
: T(f) = 0$ in jedem normierten Raum. Zum Beweis benötigen wir jedoch noch
etwas Vorbereitung.\bsphere
\end{bsp}

% ==============================================================
% ================== Hyperebenen  ================================
% ==============================================================
\section{Lineare Funktionale als Hyperebene}

\begin{defn}
\label{defn:4.3}
Sei $L$ ein linearer Raum. Ein linerarer Teilraum $M\subseteq L$ heißt
\emph{Hyperebene}\index{Hyperebene}, falls $\dim L/M = 1$.\fishhere
\end{defn}

$M$ ist also genau dann Hyperebene, wenn eine lineare, bijektive Abbildung $\ph:
L/M\to \K$ existiert.

\begin{prop}
\label{prop:4.4}
\begin{propenum}
  \item Für einen linearen Teilraum $M$ von $L$ sind äquivalent:
  \begin{equivenum}
    \item $M$ ist Hyperebene.
    \item $\exists \ph: L\to \K$ linear, so dass gilt: $M=\ker\ph$ und $\ph\neq
    0$.
  \end{equivenum}
  \item Für $\ph,\psi : L\to\K$ linear sind äquivalent:
  \begin{equivenum}
    \item $\ker\ph = \ker \psi$.
    \item $\exists \lambda\neq 0 : \ph=\lambda\psi$.\fishhere
  \end{equivenum}
  \end{propenum} 
\end{prop}
\begin{proof}
\begin{proofenum}
  \item ``$\Rightarrow$'': Betrachte die Quotientenabbildung $q: L\to L/M :
  x\mapsto [x]$. $q$ ist offensichtlich linear. Weitherhin gilt $\dim L/M = 1$, nach
Definition \ref{defn:4.3} existiert also eine lineare bijektive Abbildung $\psi
: L/M\to \K$.
Setze nun $\ph=\psi\circ q$, dann gilt
\begin{align*}
&\ph(x) = 0 \Leftrightarrow \psi(q(x)) = 0 \Leftrightarrow
q(x) = [0] \Leftrightarrow [x] = [0] \\ & \Leftrightarrow x\in [0]
\Leftrightarrow x\in M.
\end{align*}
Also ist $\ker \ph = M$ und offensichtlich ist $\ph\neq 0$.

``$\Leftarrow$'': Setze $M:=\ker \ph$ und $\psi: L/M\to \K,\; [x]\mapsto
\ph(x)$. $\psi$ ist offensichtlich wohldefiniert und linear. Außerdem ist $\psi$
injektiv, denn falls $\ph(y)=\ph(x)$, so ist $\ph(x-y)=0$, d.h. $x-y\in M
\Leftrightarrow [x]=[y]$. Weiterhin ist $\psi$ surjektiv, da $\ph$ surjektiv.

Also ist $M$ Hyperebene.
\item Für den Spezialfall $\ph=0$ folgt die Behauptung trivialerweise.

 ``$\Rightarrow$'': Sei $x_0\in L$ mit $\ph(x_0) = 1$, dann gilt 
 $\psi(x_0) \neq 0$.
 
 Sei $x\in L\setminus \ker \ph$, dann gilt
 \begin{align*}
 \ph(x-\ph(x)x_0) = \ph(x)-\ph(x)\ph(x_0) = 0.
 \end{align*}
Somit ist $x-\ph(x)x_0\in \ker \ph = \ker \psi$. Insbesondere gilt also,
\begin{align*}
&0 = \psi(x-\ph(x)x_0) = \psi(x) - \ph(x)\psi(x_0)\\
\Rightarrow &\psi(x) = \ph(x)\psi(x_0)\\
\Rightarrow &\ph = \frac{1}{\psi(x_0)}\psi.
\end{align*}

``$\Leftarrow$'': Trivial.\qedhere
\end{proofenum}
\end{proof}

\begin{prop}
\label{prop:4.5}
Für $\ph: E\to \K$ linear sind äquivalent
\begin{equivenum}
  \item\label{prop:4.5:1} $\ph$ ist stetig (also beschränkt),
  \item\label{prop:4.5:2} $\ker \ph$ ist abgeschlossen.
\end{equivenum}
\end{prop}

\begin{proof}
``\ref{prop:4.5:1}$\Rightarrow$\ref{prop:4.5:2}'': $\ker\ph =
\ph^{-1}(\setd{0})$. $\setd{0}$ ist abgeschlossen und daher auch $\ker\ph$.
%  und
% \begin{align*}
% \ph(x) = \lim\limits_{n\to\infty} \ph(x_n) = 0\Rightarrow x\in\ker \ph.
% \end{align*}

``\ref{prop:4.5:2}$\Rightarrow$\ref{prop:4.5:1}'':
  Sei $M=\ker \ph$. Der Fall
$M=L$ ist trivial. Sei also $M\subsetneq L$. Betrachte  erneut die Quotientenabbildung $q: E\to
E/M,\; x\mapsto [x]$. Mit \ref{prop:4.4} folgt, dass eine bijektive Abbildung
$\psi: E/M\to \K$ existiert. Sei
\begin{align*}
\tilde{\ph} := \psi\circ q \overset{\text{Bew. }\ref{prop:4.4}}{\Rightarrow}
\ker \tilde{\ph} = M = \ker \ph \overset{\ref{prop:4.4}}{\Rightarrow} \ph =
\lambda\tilde{\ph}.
\end{align*}
Zeige nun $\tilde{\ph}$ ist stetig. $q$ ist stetig, denn
\begin{align*}
\norm{q(x)}_{E/M} = \inf\limits_{y\in M}\norm{x+y}_E \le \norm{x}_E \Rightarrow
\norm{q} \le 1.
\end{align*}
$\psi$ ist stetig, denn sei $\psi([x_0]) = 1$ und $[x]\in E/M$, so gilt
\begin{align*}
\psi(\psi([x])[x_0]) = \psi([x])\psi([x_0]) \overset{\psi\text{
bij}}{\Rightarrow} \psi([x])[x_0] = [x].
\end{align*}
Insbesondere ist $\setd{[x_0]}$ Basis von $E/M$ und daher $\psi(\alpha[x_0]) =
\alpha\psi([x_0]) =\alpha$, also gilt
\begin{align*}
&\abs{\psi([x])} = \abs{\psi(\alpha[x_0])} = \abs{\alpha} = 
\abs{\alpha}\frac{\norm{[x_0]}_{E/M}}{\norm{[x_0]}_{E/M}}
= \frac{\norm{\alpha[x_0]}_{E/M}}{\norm{[x_0]}_{E/M}}
= \frac{\norm{[x]}_{E/M}}{\norm{[x_0]}_{E/M}}\\
\Rightarrow& \norm{\psi} = \frac{1}{\norm{[x_0]}_{E/M}}.
\end{align*}
Also ist $\psi$ beschränkt. Somit ist $\psi\circ q$ stetig.

\textit{Einfacher}. Betrachte den Spezialfall $T: E\to F$ bijektiv und $E,F$
endlichdimensional. Definiere
\begin{align*}
\norm{x}:=\norm{Tx}_F
\end{align*}
ist Norm auf $E$. Es folgt mit \ref{prop:1.11}, dass
\begin{align*}
&c_1\norm{x}_E \le \norm{x}\le c_2\norm{x}_E\\
\Rightarrow & \norm{Tx}_F = \norm{x} \le c_2 \norm{x}_E.
\end{align*}
Ist also $T:E\to F$ linear und bijektiv und $E,F$ endlichdimensonal, so ist
$T$ Isometrie.\qedhere
\end{proof}

\section{Existenz linearer Funktionale}

\begin{defn}
\label{defn:4.6}
Sei $L$ reeller linearer Raum,
\begin{align*}
p : L\to\R
\end{align*}
heißt \emph{sublinear}\index{Abbildung!sublinear}, falls für $x,y\in L$ und
$\lambda > 0$ gilt,
\begin{defnenum}
  \item $p(x+y)\le p(x)+p(y)$,
  \item $p(\lambda x) = \lambda p(x)$.\fishhere
\end{defnenum}
\end{defn}

Eine sublineare Abbildung erfüllt also die Dreiecksungleichung und ist homogen
für positive Skalare.

\begin{bem}
\label{bem:4.7}
Jede Halbnorm (und damit auch jede Norm) ist sublinear.\\
$p(x) < 0$ ist erlaubt.\maphere
\end{bem}

\begin{prop}[Satz von Banach]
\label{prop:4.8}
Sei $L$ reeller linearer Raum, $p:L\to\R$ sublinear, $M\le L$ linearer
Teilraum und $f: M\to \R$ linear mit
\begin{align*}
\forall x\in M : f(x)\le p(x).
\end{align*}
Dann existiert eine lineare Fortsetzung
\begin{align*}
F: L\to\R,\qquad F|_M = f,
\end{align*}
so dass
\begin{align*}
\forall x\in L : F(x)\le p(x).\fishhere
\end{align*}
\end{prop}

\begin{bem}
\label{bem:4.9}
Im Allgemeinen ist der Nachweis der Existenz einer beliebigen Fortsetzung
einfach, der Nachweis einer $p$-beschränkten Fortsetzung schwierig.\maphere
\end{bem}

\begin{proof}[Beweis von Satz \ref{prop:4.8}.]
\begin{proofenum}
  \item \textit{Fortsetzung auf ``$\dim M+1$''}.\\ Sei $x_0\in L/M$. Angenommen
  es existiert eine Fortsetzung $g$ von $f$ auf $\lin{M,x_0}$, dann hat diese
  die Eigenschaften
  \begin{defnenum}
    \item $g|_M = f$.
    \item $g(y) = g(x+\lambda x_0) = f(x) + \lambda g(x_0)$. Ist die
    Darstellung $y=x+\lambda x_0$ eindeutig, so ist $g$ eindeutig durch $g(x_0)$
    bestimmt.
    
    Sei also $y\in \lin{M,x_0}$ mit $y=x+\lambda x_0$ und
    $y=\tilde{x}+\tilde{\lambda}x_0$, dann ist
\begin{align*}
0 = \underbrace{x-\tilde{x}}_{\in M} + (\lambda-\tilde{\lambda})x_0.
\end{align*}
Da $x_0\notin M$ ist $(\lambda-\tilde{\lambda})x_0$ nur dann Element von $M$,
wenn $\lambda-\tilde{\lambda}=0$, d.h. $\lambda=\tilde{\lambda}$ und
$x=\tilde{x}$.
\end{defnenum}
Sei $g_r : y = x+\lambda x_0 \mapsto f(x) + \lambda r$ für festes $r\in\R$, so
ist $g_r$ linear und Fortsetzung von $f$. Zu zeigen ist nun, dass $r$ so gewählt
werden kann, dass $g_r(y)\le p(y)$, was äquivalent ist zu
\begin{align*}
f(x)+\lambda r \le p(x+\lambda x_0).\tag{*}
\end{align*}
  
\begin{defnenum}
  \item Sei $\lambda = 0$, dann nimmt (*) die Form
\begin{align*}
f(x) \le p(x)
\end{align*}
an und ist nach Voraussetzung erfüllt.
\item Sei $\lambda > 0$, so gilt
\begin{align*}
(*) &\Leftrightarrow f(x)+\lambda r \le p(x+\lambda x_0)\\
&\Leftrightarrow r \le \frac{1}{\lambda}\left(p(x+\lambda x_0) - f(x) \right)\\
&\;\qquad = p\left(\frac{1}{\lambda}p(x)+ x_0\right) -
f\left(\frac{1}{\lambda}x\right)\\
&\Leftrightarrow
r\le \inf\setdef{p(z+x_0)-f(z)}{z\in M}.
\end{align*}
\item Sei $\lambda < 0$, so gilt
\begin{align*}
(*) &\Leftrightarrow f(x)+\lambda r \le p(x+\lambda x_0)\\
&\Leftrightarrow r \ge \frac{1}{\lambda}\left(p(x+\lambda x_0) - f(x) \right)\\
&\;\qquad = -p\left(\frac{1}{-\lambda}p(x)- x_0\right) +
f\left(\frac{1}{-\lambda}x\right)\\
&\Leftrightarrow
r\ge \sup\setdef{-p(z-x_0)+f(z)}{z\in M}.
\end{align*}
\end{defnenum}
Es existiert also ein $p$-beschränktes $g_r$, falls
\begin{align*}
\inf\setdef{p(z+x_0)-f(z)}{z\in M} \ge \sup\setdef{-p(z-x_0)+f(z)}{z\in
M}.\tag{**}
\end{align*}
Seien $x',x''\in M$, dann gilt
\begin{align*}
-p(x'-x_0)+f(x')
&= f(x'+x'')-f(x'') - p(x'-x_0)\\
&\le
p(x'+x'') - f(x'') - p(x'-x_0)\\
&\le
p(x'-x_0)+p(x''+x_0) - f(x'')-p(x'-x_0)\\
&= p(x''+x_0)-f(x'')\Rightarrow (**).
\end{align*}
$g_r$ ist also genau dann $p$ beschränkt, wenn
\begin{align*}
r\in R=\left[\sup_{z\in M}\setd{-p(z-x_0)+f(z)},\inf_{z\in
M}\setd{p(z+x_0)-f(z)}\right].
\end{align*}
Falls $R^\circ\neq \varnothing$, so existieren unendlich viele Fortsetzungen.
\item \textit{Es gibt einen ``größten'' linearen Teilraum $N$ von $L$, auf dem
eine lineare, $p$-beschränkte Fortsetzung existiert}.

Wir zeigen dies durch eine Anwendung des Lemmas von Zorn (engl.
``zornification''). Sei dazu
\begin{align*}
\AA :=
\setdef{(N,g)}{M\leq N \leq L\text{ und $g$ ist $p$-beschr. lin. FS von
$f$ auf $N$}}.
\end{align*}
\begin{defnenum}
  \item $\AA\neq \varnothing$, da $(M,f)\in\AA$.
  \item $\AA$ wird halbgeordnet durch $(N_1,g_1)\le (N_2,g_2)$, falls
\begin{align*}
N_1\le N_2\text{ und } g_2 \text{ FS von } g_1.
\end{align*}
Eine Teilmenge $\BB\subseteq \AA$ heißt \emph{aufsteigende Kette}, wenn $\BB$
durch $\le$ totalgeordnet ist.
\item \textit{Für jede aufsteigende Kette $\BB\subseteq \AA$ existiert eine
obere Schranke in $\AA$}.

Sei $\BB=\setdef{(N_i,g_i)}{i\in\II}$ eine solche Kette. Definiere
\begin{align*}
N:= \bigcup_{i\in\II} N_i,
\end{align*}
so ist $N\leq L$. Setze nun
\begin{align*}
g(x):= g_i(x),\qquad \text{falls } x\in N_i. 
\end{align*}
$g$ ist wohldefiniert, linear und $p$-beschränkt und Forsetzung von $f$, denn
\begin{align*}
\forall i\in\II : g_i\text{ linear und $p$-beschr.},\; g_i|_M = f.
\end{align*}
Also ist $(N,g)\in\AA$. Weiterhin gilt, dass
\begin{align*}
\forall i\in\II : N_i\le N,\quad g|_{N_i} = g_i
\end{align*}
also ist $(N_i,g_i)\le (N,g)$ für jedes $i\in\II$.
\end{defnenum}
(a)-(c) und das Lemma von Zorn ergeben, dass $\AA$ ein maximales Element
enthält, d.h.
\begin{align*}
\exists (N,g)\in \AA : \forall (\tilde{N},\tilde{g})\in \AA :
(\tilde{N},\tilde{g})\le (N,g).
\end{align*}
\item
\textit{Sei $(N,g)$ das maximale Element, dann ist $N=L$}. Angenommen
$N\subsetneq L$, dann $\exists x_0\in L/N$. Mit 1.) folgt, es existiert eine
lineare, $p$-beschränkte Fortsetzung von $g_r$ von $g$ auf $\lin{N,x_0}$. Da
aber $\lin{N,x_0}\le L$ und $g_r$ FS von $g$ ist $(\lin{N,x_0},g_r)\in\AA$ und
$(N,g)\le (\lin{N,x_0},g_r)$.\dipper\qedhere
\end{proofenum}
\end{proof}

\begin{prop}[Fortsetzungssatz von Hahn-Banach]
\index{Satz!Hahn-Banach}
\label{prop:4.10}
Sei $(E,\norm{\cdot})$ normierter Raum $M\le E$ linearer Teilraum, $f\in M'$.
Dann existiert eine Fortsetzung $F\in E'$ von $f$ mit $\norm{F}=\norm{f}$. (Im
Allgemeinen ist diese Fortsetzung nicht eindeutig).\fishhere
\end{prop}

Für den Beweis benötigen wir noch etwas Vorbereitung.

\begin{lem}
\label{lem:4.11}
Sei $L$ linearer Raum über $\C$. Dann gilt:
\begin{propenum}
  \item\label{lem:4.11:1} Äquivalent sind:
  \begin{equivenum}
    \item $g: L\to \C$ ist linear und $f=\Re(g)$.
    \item $f:L\to\R$ reell linear, d.h.
\begin{align*}
&\forall \lambda \in\R : f(\lambda x) = \lambda f(x)\\
&\forall x,y\in L: f(x+y) = f(x)+f(y)
\end{align*}
und $g:L\to \C,\; x\mapsto f(x)-if(x)$.
  \end{equivenum}
  \item\label{lem:4.11:2} Sei $\norm{\cdot}^\sim : L\to\R$ Halbnorm, $g:
  L\to\C$ linear. Dann sind äquivalent
\begin{equivenum}
  \item $\forall x\in L : \abs{g(x)}\le \norm{x}^\sim$.
  \item $\forall x\in L : \abs{\Re g(x)} \le \norm{x}^\sim$.
\end{equivenum}
\item\label{lem:4.11:3} Ist $(E,\norm{\cdot})$ normierter Raum über $\C$, $g\in
E'$ so gilt:
\begin{align*}
\norm{g} = \sup\limits_{x\neq 0} \frac{\abs{\Re g(x)}}{\norm{x}}.\fishhere
\end{align*}
\end{propenum}
\end{lem}

\begin{proof}
\begin{proofenum}
  \item ``$\Rightarrow$'': Sei also $f:=\Re g$, so ist $f$ trivialerweise reell
  linear. Weiterhin gilt $\Im g(x) = -\Re ig(x) = -\Re g(ix)$, also
\begin{align*}
g(x) = \Re g(x) + i\Im g(x) = f(x)-if(ix). 
\end{align*}
``$\Leftarrow$'': Setze $g(x) := f(x) -if(ix)$, so ist $\Re g(x) = f(x)$ und
für $x,y\in L$, $g(x+y) = g(x) + g(y)$. Sei nun $\lambda=\alpha+i\beta \in \C$
mit $\alpha,\beta\in\R$, so ist
\begin{align*}
g(\lambda x) &= f(\lambda x) - if(i\lambda x) \\ &= \alpha f(x) + \beta f(ix) -
i\alpha f(ix) -i\beta f(-x)\\
&= (\alpha+i\beta)(f(x)-if(ix)) = \lambda g(x).
\end{align*}
\item ``$\Rightarrow$'': $\abs{\Re g(x)}\le \abs{g(x)} \le \norm{x}^\sim$.\\
``$\Leftarrow$'': Setze $\ph=\arg g(x)$, so gilt
\begin{align*}
\abs{g(x)} = e^{-i\ph}g(x) = \underbrace{g(e^{i\ph}x)}_{\in\R}
= \Re g(e^{i\ph}x) \le \norm{e^{-i\ph}x}^\sim = \norm{x}^\sim.
\end{align*}
\item Für $g\equiv 0$ folgt die Aussage sofort. Sei $g\neq 0$, so folgt mit
\ref{lem:4.11:1}, $\Re g(x) \neq 0$. Definiere nun
\begin{align*}
&\norm{x}^\sim := \norm{g}\norm{x},\\
&\norm{x}^{\approx} := \norm{\Re g}\norm{x},
\end{align*}
so sind $\norm{\cdot}^\sim$ und $\norm{x}^\approx$ Normen auf $E$.\\
``$\Rightarrow$'': Sei $x\in
E$, so gilt
\begin{align*}
\abs{g(x)} \le \norm{x}^\sim &\overset{\ref{lem:4.11:2}}{\Rightarrow}
\abs{\Re g(x)} \le \norm{x}^\sim  = \norm{g}\norm{x}\\
&\Rightarrow \sup\limits_{x\neq 0} \frac{\abs{\Re g(x)}}{\norm{x}} \le
\norm{g}.
\end{align*}
``$\Leftarrow$'': Analog erhalten wir,
\begin{align*}
\abs{\Re g(x)}\le \norm{x}^\approx 
&\Rightarrow \abs{g(x)} \le \norm{x}^\approx = \norm{\Re g}\norm{x}\\
&\Rightarrow
\sup\limits_{x\neq 0}
\frac{\abs{g(x)}}{\norm{x}} \le \norm{\Re g}.\qedhere
\end{align*}
\end{proofenum}
\end{proof}

\begin{prop}[Satz von Hahn-Banach für Halbnormen]
\index{Satz!Hahn-Banach für Halbnormen}
\label{prop:4.12}
Sei $L$ linearer Raum über $\K$,\\ $\norm{\cdot}^\sim: L\to\R$ Halbnorm, $M\le
L$ und $f:M\to\K$ linear mit
\begin{align*}
\forall x\in M : \abs{f(x)}\le \norm{x}^\sim.
\end{align*}
Dann existiert eine lineare Fortsetzung $F:L\to\K$ mit
\begin{align*}
\forall x\in L : \abs{F(x)} \le \norm{x}^\sim.\fishhere
\end{align*}
\end{prop}
\begin{proof}
\textit{Fall 1}. Sei $\K=\R$ und $p(x) := \norm{x}^\sim$, dann existiert nach
\ref{prop:4.8} eine lineare Fortsetzung $F:L\to\R$, so dass
\begin{align*}
\forall x\in L : F(x)\le \norm{x}^\sim,
\end{align*}
wobei
\begin{align*}
\forall x\in L : -F(x) = F(-x) \le \norm{-x}^\sim = \norm{x}^\sim.
\end{align*}
\textit{Fall 2}. Sei $\K=\C$. Betrachte zunächst $L$ als linearen Raum über
$\R$ und setze
\begin{align*}
g : M\to\R,\quad x\mapsto \Re f(x),
\end{align*}
so ist $g$ linear und $\abs{g(x)}\le \abs{f(x)} \le \norm{x}^\sim$. Wenden wir
Fall 1 an, erhalten wir eine lineare Fortsetzung $G: L\to\R$, so dass
\begin{align*}
\forall x\in L : \abs{G(x)}\le \norm{x}^\sim.
\end{align*}
Betrachte $L$ nun wieder als Raum über $\C$ und
setze $F(x) := G(x) - iG(ix)$, so ist $F$ linear und da
\begin{align*}
\forall x\in L : \abs{\Re F(x)} = \abs{G(x)} \le \norm{x}^\sim,
\end{align*}
gilt ebenfalls $\abs{F(x)} \le \norm{x}^\sim$.

$F$ ist auch Fortsetzung vn $f$, denn für $x\in M$ gilt,
\begin{align*}
F(x) = G(x) - i G(ix) = g(x) - ig(ix) = \Re f(x) - i\Re f(ix) = f(x).\qedhere
\end{align*}
\end{proof}

\begin{proof}[Beweis von Satz \ref{prop:4.10}]
Für $f=0$ folgt die Behauptung trivialerweise. Sei also $f\neq 0$ und
\begin{align*}
\norm{x}^\sim := \norm{f}\norm{x},
\end{align*} 
so ist $\norm{\cdot}^\sim$ Norm auf $E$ und $\abs{f(x)} \le \norm{x}^\sim$ für
$x\in M$. Nach \ref{prop:4.12} existiert somit eine lineare Fortsetzung $F:E\to
\K$, so dass
\begin{align*}
\forall x\in E : \abs{F(x)} \le \norm{x}^\sim = \norm{f}\norm{x}
\Rightarrow \norm{F}\le \norm{f}.
\end{align*}
$F$ ist aber Fortsetzung und daher $\norm{f}\le\norm{F}$, also
$\norm{f}=\norm{F}$.\qedhere
\end{proof}

\begin{cor}
\label{prop:4.13}
\begin{propenum}
  \item $\forall x_0\in E\setminus\setd{0} \exists f \in E' : \norm{f} = 1,
  f(x_0) = \norm{x_0}$.
  \item Seien $\setd{x_1,\ldots,x_n}$ linear unabhängig und seien
  $\alpha_1,\ldots,\alpha_n\in \K$, dann
\begin{align*}
\exists f\in E' \forall j=1,\ldots,n : f(x_j) = \alpha_j.\fishhere
\end{align*}
\end{propenum}
\end{cor}
\begin{proof}
\begin{proofenum}
  \item 
Sei $M:=\lin{x_0}$ und $g: M\to \K,\; \alpha x_0\mapsto \alpha\norm{x_0}$. $g$
ist offensichtlich linear und $\abs{g(\alpha x_0)} = \abs{\alpha}\norm{x_0}$,
also
\begin{align*}
\norm{g} = \sup\limits_{y\neq 0}\frac{\abs{g(y)}}{\norm{y}}
= \sup\limits_{\alpha\neq 0}\frac{\abs{g(\alpha x_0)}}{\norm{\alpha x_0}} = 1.
\end{align*}
Durch Anwendung des Satzes von Hahn-Banach erhalten wir eine lineare
Fortsetzung $G\in E'$ mit $\norm{G}=1$ und $G(x_0) = \norm{x_0}$.
\item $M=\lin{x_1,\ldots,x_n}$ ist endlich dimensional. Aus der linearen
Algebra wissen wir, es existier eine lineare Abbildung
\begin{align*}
g : M \to \K,\qquad g(x_j) = \alpha_j.
\end{align*}
Da $\dim M <\infty$, ist $g$ beschränkt. Wir können also den Fortsetzungssatz
anwenden.\qedhere
\end{proofenum}
\end{proof}

\begin{bem}[Diskussion.]
\label{bem:4.14}
Falls $E\neq 0$, so $\exists x_0\in E\setminus\setd{0}$, so dass
\begin{align*}
\exists f\in E' : f(x) = \norm{x}\neq 0.
\end{align*}
Somit ist der Dualraum $E'=\LL(E\to\K)$ nicht leer. Außerdem gilt
\begin{align*}
x=y \Leftrightarrow x-y=0 \Leftrightarrow \forall f\in E' : f(x-y) =0
\Leftrightarrow \forall f\in E' : f(x) = f(y).
\end{align*}
Man sagt, der Dualraum trennt die Elemente von $E$. Statt $f(x)$ schreibt man
auch $\lin{f,x}$ (Dualitätsprodukt).

\textit{Erklärung}. Betrachte $1< p,q < \infty$ mit
$\frac{1}{p}+\frac{1}{q}=1$.
Sei $E=L^p([a,b])$, so ist $E'=L^q([a,b])$
(später). Dann gilt für $f\in L^p$ und $g\in L^q$,
\begin{align*}
g(f) = \int\limits_a^b f\cdot g \dmu = \lin{f,\overline{g}}.\maphere
\end{align*}
\end{bem}

\begin{cor}
\label{prop:4.15}
Für $x\in E$ gilt,
\begin{align*}
\norm{x} = \sup\setdef{\abs{f(x)}}{f\in E' \land \norm{f}\le 1},
\end{align*}
und das Supremum wird angenommen.\fishhere
\end{cor}
\begin{proof}
$f\in E'$ ist beschränkt, d.h.
$\abs{f(x)}\le \norm{f}\norm{x} \le \norm{x}$. Mit \ref{prop:4.13} folgt
\begin{align*}
\exists f\in E' : \norm{f} = 1 \land f(x) = \norm{x}.\qedhere
\end{align*}
\end{proof}

\begin{cor}
\label{prop:4.16}
Sei $M\le E$ abgeschlossen, $x_0\in E\setminus M$. Dann existiert ein $f\in E'$
mit
\begin{align*}
f\big|_M = 0 \text{ und } f(x_0)\neq 0.\fishhere 
\end{align*}
\end{cor}
\begin{proof}
Betrachte die Quotientenabbildung
\begin{align*}
\pi : E\to E/M,\quad x\mapsto [x].
\end{align*}
Setzen wir $\norm{[x]}_{E/M} := \inf\limits_{y\in M} \norm{x+y}$, so ist $\pi$
stetig (siehe Beweis \ref{prop:4.5}). Weiterhin ist $\pi(x) = [0]$, falls $x\in
M$ und $\pi(x_0)\neq 0$. Mit \ref{prop:4.13} folgt nun
\begin{align*}
\exists \ph\in (E/M)' : \ph([x_0]) = \norm{\pi(x_0)}_{E/M}\neq 0.
\end{align*}
Sei $f=\ph\circ\pi$, so ist $f$ beschränkt und linear, also $f\in E'$, sowie
\begin{align*}
f(x_0) = \ph(\pi(x_0)) \neq 0,\qquad
f(x) = \ph(\pi(x))=0\text{ falls } x\in M.\qedhere
\end{align*}
\end{proof}

\begin{cor}
\label{prop:4.17}
Sei $M\le E$, dann gilt
\begin{propenum}
  \item $\overline{M}=\bigcap \setdef{\ker f}{f\in E'\land M\le \ker f}$.
  \item $M$ liegt genau dann dicht in $E$, wenn
\begin{align*}
\forall f\in E' : \left(f\big|_M=0\Rightarrow f=0\right).\fishhere
\end{align*}
\end{propenum}
\end{cor}
\begin{proof}
\begin{proofenum}
  \item $M\le \ker f$, also ist $\overline{M}\le \ker f$ und da $f$ beliebig
\begin{align*}
\overline{M}\le K:=\bigcap \setdef{\ker f}{f\in E'\land M\le \ker f}.
\end{align*}
$\overline{M}\supseteq K$ ist äquivalent zu $E/\overline{M}\subseteq E/K$. Sei
also $x_0\in E/\overline{M}$, dann folgt mit \ref{prop:4.16}
\begin{align*}
\exists f\in E' : f\big|_{\overline{M}}  = 0\land f(x_0)\neq 0.
\end{align*}
Für dieses $f$ gilt nun $M\le\ker f$ und $x_0\notin \ker f$, also ist
$x_0\notin K$, d.h. $x_0\in E/K$.
\item ``$\Rightarrow$'': Sei $x\in E$, $(x_n)$ in $M$, $x_n\to x$. Für $f\in
E'$ mit $f\big|_M=0$ gilt
\begin{align*}
f(x) = \lim\limits_{n\to\infty} f(x_n) = 0.
\end{align*}
``$\Leftarrow$'': $\overline{M}=\bigcap \setdef{\ker f}{f\in E'\land M\le \ker
f}=E$.\qedhere
\end{proofenum}
\end{proof}

% ==============================================================
% ================== Reflexive Räume ==============================
% ==============================================================
\section{Reflexive Räume}

Auf $\R$ lässt sich der \textit{Satz von Bolzano-Weierstraß} beweisen, der
besagt, dass jede beschränkte Folge eine konvergente Teilfolge besitzt. Dieser
Satz lässt sich dann auf $\C$, $\R^n$ oder allgemeiner jeden
normierten Raum $V$ mit $\dim V < \infty$ ohne zusätzliche Voraussetzungen oder
Abschwächung der Aussage verallgemeinern. Nun stehen in der Funktionalanalysis
die unendlichdimensionalen Räume im Mittelpunkt, jedoch kennen wir hier
bereits Gegenbeispiele, bei denen die Aussage des Satzes falsch wird. Man
betrachte zum Beispiel eine Abzählung der Einheitsvektoren,
\begin{align*}
e_k = (0,\ldots,0,1,0,\ldots),\qquad e_{kn} = \delta_{nk}.
\end{align*}
Diese ist beschränkt, enthält aber sicher keine konvergente Teilfolge.

Ziel dieses Abschnitts ist es nun, den Satz dahingehend auf unendlich
dimensionale Räume zu verallgemeinern, dass jede beschränkte Folge eine
in einem noch zu definierenden Sinne ``schwach konvergente'' Teilfolge enthält.

\begin{defn}
\label{defn:4.18}
$E'' = (E')'$ heißt \emph{Bidualraum}\index{Vektorraum!Bidualraum} von
$E$.\fishhere
\end{defn}

$E''$ ist stets ein Banachraum, da $\K=\R$ oder $\C$ vollständig ist.

\begin{prop}[Einbettung]
\index{Vektorraum!Einbettung}
\label{prop:4.19}
Es existiert eine Isometrie $J_E : E\to E''$.\fishhere
\end{prop}
\begin{proof}
Für $E=(0)$ ist die Aussage trivial, sei also $E\supsetneq (0)$.
Zu $x\in E$ sei
\begin{align*}
T_x : E'\to \K,\; f\mapsto f(x).
\end{align*}
$T_x$ ist offensichtlich
linear und beschränkt, denn
\begin{align*}
\abs{T_x(f)} = \abs{f(x)} \le \norm{x}_E\norm{f}_{E'}.
\end{align*}
Also ist $T_x\in E''$. Da $x\neq 0$,  folgt mit
\ref{prop:4.13} die Existenz einer Abbildung $f_0\in E'$ mit $\norm{f_0}=1$ und
$f_0(x)=\norm{x}$. Für diese gilt,
\begin{align*}
\abs{T_{x}(f_0)} = \abs{f_0(x)} = 1\cdot\norm{x_0}_E,
\end{align*}
also ist $\norm{T_{x}}_{E''}=\norm{x}_E$. Setze nun
\begin{align*}
J_E : E\to E'',\quad x\mapsto T_x,
\end{align*}
so ist $J_E$ Isometrie.\qedhere
\end{proof}

Im Allgemeinen ist jedoch $\im J_E \subsetneq E''$, die Gleichheit gilt nur in
einer speziellen Klasse von Räumen. 

\begin{defn}
\label{defn:4.20}
$E$ heißt \emph{reflexiv}\index{Vektorraum!reflexiv}, falls $J_E$ surjektiv
ist.\fishhere
\end{defn}

\begin{bem}[Bemerkungen.]
\label{bem:4.21}
\begin{bemenum}
\item $E''$ ist vollständig und damit auch $\overline{\im J_E}$. Man erhält
durch diese Einbettung also auch eine Vervollständigung von $E$, denn $\im
J_E$ liegt dicht in $\overline{\im J_E}$ und $\im J_E$ ist isometrisch
isomorph zu $E$ (schreibe $J_E\cong E$).
\item Ist $E$ reflexiv, so ist $E=J_E^{-1}(E'')$ ein Banachraum.
\item Jeder Hilbertraum ist reflexiv.
\item Sei $1<p<\infty$, dann sind $l^p$ und $L^p([a,b])$ reflexiv.\maphere
\end{bemenum}
\end{bem}

\begin{prop}
\label{prop:4.22}
$l^1$ ist \textit{nicht} reflexiv.\fishhere
\end{prop}
\begin{proof}
\begin{proofenum}
  \item 
\textit{$(l^1)'\cong l^\infty$}. Setze dazu
\begin{align*}
\ph: l^\infty\to (l^1)',\quad (a_n)\mapsto \ph(a_n)\text{ mit } \ph(a_n)(x_n) :=
\sum\limits_{n=1}^\infty a_n x_n.
\end{align*}
$\ph(a_n)$ ist offensichtlich linear und beschränkt, denn
\begin{align*}
\abs{\ph(a_n)(x_n)} \le \sum\limits_{n=1}^\infty \abs{a_n}\abs{x_n}
\le \sup_n \abs{a_n} \sum\limits_{n=1}^\infty \abs{x_n}
= \norm{(a_n)}_\infty\norm{(x_n)}_1 < \infty.
\end{align*}
Somit ist $\ph(a_n)\in (l^1)'$ mit $\norm{\ph(a_n)}_{(l^1)'} \le
\norm{(x_n)}_1$, man prüft leicht nach, dass sogar die Gleichheit gilt, also
ist $\ph : l^\infty\to (l^1)'$ Isometrie.

\textit{$\ph$ ist  surjektiv}. Sei $f\in (l^1)'$, $e_k:=(\delta_{nk})_n\in
l^1$. Setze $a_k = f(e_k)$ für $k\in\N$, so gilt
\begin{align*}
\ph(a_n)(e_k) = f(e_k) \Rightarrow
\ph(a_n)(x_l) = f(x_l),
\end{align*}
da $\ph(a_n)$ und $f$ linear und stetig und $l_\text{abb}$ dicht in $l^1$. Also
ist $\ph: l^\infty\to (l^1)'$ bijektive Isometrie.
\item Konstruiere $F\in (l^1)''\setminus J_{l^1}(l^1)$.

Sei $M:=\setdef{(a_n)\in l^\infty}{(a_n)\text{ konvergent}},$ dann ist $M\le
l^\infty$. Setze
\begin{align*}
\lim : M \to \C,\quad (x_n)\mapsto \lim\limits_{n\to\infty} a_n,
\end{align*}
so ist $\lim\in (l^\infty)'$ mit $\norm{\lim}_{M'}\le 1$. Eine Anwendung des
Satzes von Hahn-Banach liefert eine lineare Fortsetzung $\Lim: l^\infty\to \C$
mit $\norm{\Lim}\le 1$.
\begin{figure}[!htbp]
\centering
\begin{pspicture}(-0.7,-0.97)(3.7,1.1)

\rput(0.5,0.615){\color{gdarkgray}$(l^1)'$}
\rput(2.8,0.635){\color{gdarkgray}$l^\infty$}
\rput(1.58,-0.805){\color{gdarkgray}$\C$}

\psline[linecolor=darkblue]{->}(0.9,0.61)(2.5,0.61)
\psline[linecolor=darkblue]{->}(0.5,0.4)(1.46,-0.61)
\psline[linecolor=darkblue]{->}(2.68,0.4)(1.74,-0.61)

\rput(1.7,0.855){\color{gdarkgray}$\ph^{-1}$}
\rput(2.72,-0.125){\color{gdarkgray}$\Lim$}
\rput(0.1,-0.125){\color{gdarkgray}$\Lim\circ\ph^{-1}$}
\end{pspicture} 
\caption{Zur Konstruktion von $\Lim$.}
\end{figure}

Nun ist $\Lim\circ\ph^{-1}\in (l^1)''$. Zeige, dass $\Lim\circ\ph^{-1}\notin
J_{l^\infty}(l^\infty)$.

Angenommen $\Lim\circ\ph^{-1}\in
J_{l^\infty}(l^\infty)$, d.h.
\begin{align*}
&\exists (x_n)\in l^1 : T_{(x_n)} = \Lim\circ\ph^{-1},\\
\Rightarrow &
\forall f\in (l^1)' : f(x_n) = T_{(x_n)}(f) =
\Lim\circ\underbrace{\ph^{-1}(f)}_{:=a_n}.
\end{align*}
Dann ist $f(x_n)= \ph(a_n)(x_n)$
\begin{align*}
\Leftrightarrow
\forall (a_n)\in l^\infty : \ph(a_n)(x_n) = \Lim(a_n)
\end{align*}
Setze also $(a_n):=e_k$, so gilt
\begin{align*}
\ph(e_k)(x_n) = x_k = \Lim((\delta_{nk})_n) = \lim((\delta_{nk})_n) = 0. 
\end{align*}
Somit ist $x_n\equiv 0$ und $\Lim(a_n)\equiv 0$, $\forall (a_n)\in l^\infty$.
Dies ist offensichtlich ein Wiederspruch.\qedhere
\end{proofenum}
\end{proof}

\begin{lem}
\label{prop:4.23}
\begin{propenum}
  \item Sind $E\cong F$ und $E$ reflexiv, so ist $F$ reflexiv.
  \item Ist $E$ reflexiv und $M\le E$ abgeschlossen, so ist auch $M$ reflexiv.
  \item Sei $E$ Banachraum, so ist $E$ genau dann reflexiv, wenn $E'$ reflexiv
  ist. Insbesondere ist $E$ genau dann reflexiv, wenn $E''$ reflexiv ist.
\end{propenum}
\end{lem}

\begin{proof}
\begin{proofenum}
  \item Zu $x''\in F''$ setze
\begin{align*}
p^{**}(x'')(y':E\to\K) := x''(y'\circ\ph),
\end{align*}
so ist $\ph^{**}(x''): E'\to\K$ linear und beschränkt, denn
\begin{align*}
\abs{\ph^{**}(x'')(y')} \le \norm{x''}_{F''}\norm{y'\circ\ph}_{F'}
\le \norm{x''}_{E''}\norm{\ph}_{F\to E}\norm{y'}_{E'}.
\end{align*}
\begin{figure}[!htpb]
\centering
\begin{pspicture}(-0.2,-1.45)(2.9,1.45)

\rput(0.24,1.095){\color{gdarkgray}$F$}
\rput(2.43,1.095){\color{gdarkgray}$E$}

\rput(1.3,1.315){\color{gdarkgray}$\ph$}
\psline[linecolor=darkblue]{->}(0.48,1.07)(2.26,1.07)

%\rput{-270.0}(1.5500001,-3.3700001){\rput(2.46,-0.925){$\supseteq$}}

\psline[linecolor=darkblue]{->}(0.24,0.91)(0.24,-0.35)
\psline[linecolor=darkblue]{->}(2.43,0.93)(2.43,-0.35)


\psline[linecolor=darkblue]{->}(0.7,-0.61)(2,-0.61)

\rput(0.24,-0.625){\color{gdarkgray}$J_F(F)$}
\rput{-270.0}(0.24,-0.96){$\subseteq$}

\rput(2.43,-0.605){\color{gdarkgray}$J_E(E)$}
\rput{-270.0}(2.43,-0.96){$\subseteq$}

\rput(1.3,-1.02){\color{gdarkgray}$\ph^{**}$}
\psline[linecolor=darkblue]{->}(0.48,-1.27)(2.22,-1.27)

\rput(0.24,-1.285){\color{gdarkgray}$F''$}
\rput(2.43,-1.285){\color{gdarkgray}$E''$}



%\rput{-270.0}(-0.66999996,-1.23){\rput(0.28,-0.965){$\supseteq$}}
\end{pspicture} 
\caption{Zur Konstruktion von $\ph^{**}$.}
\end{figure}

Also ist $\ph^{**}(x'')\in E''$. $E$ ist reflexiv, d.h. $\exists y\in
E:\ph^{**}(x'') = J_E(y)$.

Wir zeigen nun, dass $x''=J_F(\ph^{-1}(y))$.

Für $x'\in F'$ gilt,
\begin{align*}
x''(x') &= x''(x\circ\ph^{-1})(\ph)
= \ph^{**}(x'')(x\circ \ph^{-1})\\
&= J_E(y)(x'\circ\ph^{-1})
= x'\circ\ph^{-1}(y)
= J_F(\ph^{-1}(y))(x'),
\end{align*}
d.h. $\forall x''\in F'' \exists x\in F : x'' = J_F(x)$.
\item Sei $x''\in M''$ wir haben zu zeigen, dass ein $x\in M$ existiert so dass
$J_M(x) = x''$, d.h. $T_x(f) = x''(f)$, $\forall f\in M'$.

Zu $y'\in E'$, dann ist $y\big|_M \in M'$, setze $y''(y') := x''(y'\big|_M)$,
so ist $y''$ offensichtlich linear und beschränkt, denn
\begin{align*}
\abs{y''(y')} = \abs{x''(y'\big|_M)} \le \norm{x''}_{M''}\norm{y'\big|_M}_{E'}
\le \norm{x''}_{M''}\norm{y'}_{E'},
\end{align*}
also ist $y''\in E''$.

$E$ ist reflexiv, also existiert ein $y\in E$, sodass
$J_E(y) = T_y = y''$.

Sei nun $y'\in E'$ mit $y\big|_M=0$, dann ist
\begin{align*}
y'(y) = T_y(y') = y''(y') = x''(y'\big|_M) = 0.
\end{align*}
Also ist $y\in \ker y'$. \ref{prop:4.17} besagt
\begin{align*}
\overline{M} := \bigcap\setdef{\ker f}{f\in E'\land M\le \ker f}
\end{align*}
also ist $y\in \overline{M}=M$, da es in jedem dieser Kerne enthalten ist.

Sei nun $x'\in M'$, dann lässt sich $x'$ mit dem Satz von Hahn-Banach
fortsetzen zu $\xi'\in E'$ mit $\norm{\xi'}_{E'}=\norm{x'}_{M'}$. Dann ist
\begin{align*}
&x'(y) = \xi'(y) = T_y(\xi') = y''(\xi) = x''(\xi\big|_M) = x''(x'),\\
\Rightarrow & 
\forall x'\in M' : x''(x')=x'(y) = J_M(y)(x').
\end{align*}
\item
``$\Rightarrow$'': Sei $x'''\in E'''$. Wir haben zu zeigen, dass ein $x'\in E'$
existiert mit $J_{E'}(x') = x'''$. Die Komposition
\begin{align*}
x'''\circ J_{E} : E\to\K 
\end{align*}
ist ein Element in $E'$. Da $E$ reflexiv, ist $J_E$ invertierbar. Für $x''\in
E''$ ist daher,
\begin{align*}
x'''(x'') = x'''(J_E\circ J_E^{-1}(x''))
= \left(x'''\circ J_E\right)( J_E^{-1}(x'')).
\end{align*}
Setze $x:= J_E^{-1}(x'')\in E$ und $f:=x'''\circ J_E\in E'$, dann ist
\begin{align*}
\ldots = f(x)
= T_x(f) = J_{E}(x)(f) = x''(f) = T_{f}(x'') = J_{E'}(f)(x'')
\end{align*}
Also ist $x'''=J_{E'}(f)\in J_{E'}(E')$ und damit ist $J_{E'}$ surjektiv.

``$\Leftarrow$'': $E'$ ist reflexiv, also ist mit dem eben bewiesenen auch
$E''$ reflexiv. $J_E(E)\le E''$ ist als Bild des Banachraumes $E$ unter einer
Isometrie ebenfalls Banachraum und damit abgeschlossen. Also ist $J_E(E)$
reflexiv aber $J_E(E)$ und $E$ sind isometrisch isomorph, also ist auch $E$
reflexiv.\qedhere
\end{proofenum}
\end{proof}

\begin{defn}
\label{defn:4.24}
Eine Folge $(x_n)$ in $E$ heißt \emph{schwach
konvergent}\index{schwache!Konvergenz} gegen $x\in E$, falls
\begin{align*}
\forall f\in E' : f(x_n-x)\to 0.
\end{align*}
Wir schrieben in diesem Fall $x_n\wto x$.\fishhere
\end{defn}

\begin{bsp}
\label{bsp:4.25}
\begin{bspenum}
  \item Sei $E=l^2$ und $e_k = (0,\ldots,0,1,0,\ldots)$.\\ Sei $f\in E'=l^2$,
  d.h. es gibt eine Folge $(a_n)\in l^2$, so dass
\begin{align*}
\forall (x_n)\in l^2 : f((x_n)) := \sum\limits_{n=1}^\infty a_nx_n.
\end{align*}
Somit gilt $f(e_k) = a_k \to 0$, da $\sum_{k\ge 1} \abs{a_k}^2 < \infty$
unabhängig vom gewählten $f$, d.h.
\begin{align*}
\forall f\in E'  : f(e_k) \to 0,
\end{align*}
also $e_k\wto 0$ obwohl $\norm{e_k}_2 = 1$ also $\neg (e_k\to 0)$. Hier gilt
also
\begin{align*}
\norm{0}_2 < \lim\limits_{k\to\infty}\norm{e_k}_2.
\end{align*}
\item
Sei $E=C([0,1]\to\C)$ mit $\norm{\cdot}_\infty$.\\
Man kann zeigen, dass
\begin{align*}
f_n\wto f \Leftrightarrow f_n(x) \to f(x),\qquad \forall x\in[0,1].
\end{align*}
Schwache Konvergenz bezüglich $\norm{\cdot}_\infty$ entspricht also der
punktweisen Konvergenz, während Konvergenz bezüglich $\norm{\cdot}_\infty$ der
gleichmäßigen Konvergenz entspricht. (Siehe \cite{Werner07} S.
107)

\begin{figure}[!htpb]
\centering
\begin{pspicture}(-0.2,-1.5)(3.54,1.28)
\psline{->}(0.32,-1.02)(0.32,1.2)
\psline{->}(0.12,-0.84)(3.52,-0.84)
\psline(1.1,-0.72)(1.1,-0.96)
\psline(1.9,-0.72)(1.9,-0.96)
\psline(3.1,-0.72)(3.1,-0.96)
\psline(0.2,0.98)(0.44,0.98)

\rput(0.01,0.98){\color{gdarkgray}$1$}

\rput(1.06,-1.2){\color{gdarkgray}$\frac{1}{2n}$}

\rput(1.87,-1.2){\color{gdarkgray}$\frac{1}{n}$}

\rput(3.07,-1.2){\color{gdarkgray}$1$}

\psline[linecolor=darkblue](0.34,0.98)(1.1,-0.84)(1.92,0.98)(3.12,0.98)
\end{pspicture}
\caption{Zur Konstruktion von $f_n$.}
\end{figure}

Definiere $f_n$ wie in Skizze, dann $f_n\to 1$ punktweise, d.h. $f_n\wto 1$ mit
\begin{align*}
\norm{f_n}_\infty = 1 = \norm{f}_\infty
\end{align*}
aber $\norm{f-f_n}_\infty = 1$, d.h. $\neg(f_n\to f)$.\bsphere
\end{bspenum}
\end{bsp}

\begin{prop}
\label{prop:4.26}
\begin{propenum}
  \item Der schwache Grenzwert ist eindeutig.
  \item $x_n\to x$ impliziert $x_n\wto x$ und $\norm{x_n}\to\norm{x}$.
  \item Ist $(x_n)$ schwach konvergent, so ist $(x_n)$ beschränkt.
  \item Konvergiert $x_n\wto x$, so gilt
\begin{align*}
\norm{x} \le \liminf\limits_{n\to\infty} \norm{x_n}.\fishhere
\end{align*}
\end{propenum}
\end{prop}
\begin{proof}
\begin{proofenum}
  \item Sei $(x_n)$ Folge in $E$ mit schwachen Grenzwerten $x,y$, d.h.
\begin{align*}
\forall f\in E' : f(x_n-x)\to 0\land f(x_n-y)\to 0,
\end{align*}
d.h. $f(x)-f(y) = \lim\limits_{n\to \infty} f(x-x_n)+f(x_n-y) = 0$, $\forall
f\in E'$. Mit \ref{bem:4.14} folgt somit, dass $x=y$.
\item Übung.
\item Sei $x_n\wto x$, d.h.
\begin{align*}
\forall f\in E' : f(x_n)\to f(x) 
\Leftrightarrow
\forall f\in E' : T_{x_n}(f)\to T_{x}(f). 
\end{align*}
Da $T_{x_n}\to T_x$ punktweise, ist für jedes $f$ die Folge $T_{x_n}(f)$
beschränkt. Der Satz von Banach-Steinhaus sagt somit, dass
$\sup\limits_{k\in\N} \norm{T_{x_n}} =: c < \infty$, d.h. $\forall n\in\N :
\norm{T_{x_n}}=\norm{x_n}<c$.
\item Nach Voraussetzung konvergiert $T_{x_n}(f)\to T_x (f)$, wobei
\begin{align*}
\norm{T_{x_n}(f)} \le \norm{T_{x_n}}\norm{f} = \norm{x_n}\norm{f}.
\end{align*}
Somit gilt für $n\in\N$, $\norm{x} = \norm{T_x} \le \norm{x_n}$ und damit
\begin{align*}
\norm{x} \le \liminf\limits_{n\to\infty}\norm{x_n}.\qedhere
\end{align*}
\end{proofenum}
\end{proof}

\begin{prop}[Beobachtung]
\label{prop:4.27}
Sei $E$ reflexiv und $(x_n)$ Folge in $E$. Dann ist
\begin{align*}
M := \overline{\setd{\lin{x_1,\ldots,x_k}}} = \overline{\setd{\text{ endliche
Linearkombinationen }}} \le E
\end{align*}
abgeschlossen und daher reflexiv. Weiterhin liegt
\begin{align*}
M_\Q := 
\setdef{\sum\limits_{j=1}^n (\alpha_j + i\beta_j) x_j}{n\in\N,\;
\alpha_j,\beta_j\in\Q}
\end{align*}
dicht in $M$ und ist abzählbar.\fishhere
\end{prop}

\begin{defn}
\label{defn:4.28}
$E$ heißt \emph{separabel}\index{Vektorraum!separabel}, wenn $E$ eine abzählbare
dichte Teilmenge besitzt.~\fishhere
\end{defn}

\begin{lem}
\label{prop:4.29}
Ist $E'$ separabel, so ist $E$ separabel.\fishhere
\end{lem}
\begin{proof}
Sei $\setd{x_1',x_2',\ldots}\subseteq E'$ dicht. Zu $n\in\N$ wähle
$x_n \in E$ mit $\norm{x_n}=1$ und $\abs{x_n'(x_n)} \ge
\frac{1}{2}\norm{x_n'}$, dann ist
\begin{align*}
M:=\lin{\setd{x_1,\ldots}}\le E
\end{align*}
Zu zeigen ist nun, dass $\overline{M}=E$.
Sei $x'\in E'$ mit $x'\big|M=0$, dann gilt
\begin{align*}
\norm{x'-x_n'} &\ge \abs{(x'-x_n')(x_n)} = \abs{x_n'(x_n)}
\ge \frac{1}{2}\norm{x_n'} = \frac{1}{2}\norm{x_n'-x'+x'}\\
&\ge \frac{1}{2}\left(\norm{x'} - \norm{x_n'-x'}\right).
\end{align*}
Somit ist
\begin{align*}
&\norm{x'} \le 3\norm{x_n'-x'}, && n\in\N\\
\Rightarrow
&\norm{x'} \le 3\inf_n \norm{x_n'-x'} = 0,
\end{align*}
Somit ist $x'=0$, mit \ref{prop:4.17} folgt nun, dass $\overline{M}=E$. Nun ist
$M_\Q$ abzählbar und dicht in $E$, also ist $E$ separabel.\qedhere
\end{proof} 

\begin{prop}
\label{prop:4.30}
Sei $E$ reflexiv, dann besitzt jede beschränkte Folge in $E$ eine schwach
konvergente Teilfolge.\fishhere
\end{prop}
\begin{proof}
Sei also $(x_n)$ beschränkte Folge in $E$ mit $\sup_n \norm{x_n}_E = c <
\infty$.
\begin{proofenum}
\item Setze $M=\overline{\lin{x_1,x_2,\ldots}}$, dann ist $M\le E$
abgeschlossen, separabel und nach \ref{prop:4.23} außerdem reflexiv, d.h.
$M''=J_M(M)$. Somit ist $M''$ als Bild eines separablen Raumes unter einer
Isometrie ebenfalls separabel und daher ist nach \ref{prop:4.29} auch $M'$
separabel, d.h. $M'=\overline{\setd{y_1',y_2',\ldots}}$.
\item \textit{Konstruktion einer konvergenten Teilfolge von $J_M(x_n)$}. Für
jedes $j\in\N$ gilt
\begin{align*}
\abs{J_M(x_n)(y_j')} \le \norm{J_M(x_n)}_{E''}\norm{y_j'}_{E'} =
\norm{x_n}_{E}\norm{y_j'}_{E'},
\end{align*} 
d.h. die Folge $\left(J_M(x_n)(y_j')\right)$ ist beschränkt in $\K$. Somit
existiert zu $y_1'$ eine konvergente Teilfolge $(J_M(x_n^{(1)})(y_1'))$. Nun ist auch
$(J_M(x_n^{(1)})(y_2'))$ beschränkt und besitzt daher ebenfalls eine Konvergente
Teilfolge $(J_M(x_n^{(2)})(y_2'))$ usw.
Dies können wir nun \textit{ad
infinitum} fortführen.

Setze $\xi_n=x_n^{(n)}$, so ist diese Diagonalfolge
ebenfalls Teilfolge von $(x_n)$ und daher beschränkt und für jedes $j\in\N$ konvergiert
$(J_M(\xi_n)(y_j'))$ für $n\to\infty$.
\item \textit{Definition des Grenzelementes}. Sei
\begin{align*}
x'' : \setd{y_1',y_2',\ldots} \to \K,\quad y_j' \mapsto
\lim\limits_{n\to\infty} J_M(\xi_n)(y_j'),
\end{align*}
so lässt sich $x''$ linear auf die lineare Hülle $\lin{y_1',y_2',\ldots}$ wie
folgt fortsetzen,
\begin{align*}
x''\left(\sum\limits_{k=1}^K \lambda_k y_{j_k}'\right)
:= \sum\limits_{k=1}^K \lambda_k\lim\limits_{n\to\infty} J_M(\xi_n)(y_{j_k}').
\end{align*}
Die Fortsetzung ist ebenfalls beschränkt, denn für $x=\sum_{k=1}^K \lambda_k
y_{j_k}'$ gilt
\begin{align*}
\abs{x''(x)} &= 
\abs{\lim\limits_{n\to\infty} J_M(\xi_n)(x)}
\le \limsup\limits_{n\to\infty} \norm{J_M(\xi_n)}_{E''}\norm{x}_{E'}\\
&= \limsup\limits_{n\to\infty} \norm{\xi_n}_{E''}\norm{x}_{E'}
\le c\norm{x}_{E'}.
\end{align*}
Da $\lin{y_1',y_2',\ldots}$ dicht in $M'$ liegt, können wir den
Fortsetzungssatz \ref{prop:2.5} anwenden und erhalten eine eindeutige
Fortsetzung
\begin{align*}
x'': M'\to \K,\qquad \norm{f}_{M''} = \norm{x''}_{M'} \le c.
\end{align*}
\item Sei $x:=J_M^{-1}(f)$. Wir haben nun zu zeigen, dass $\xi_n\wto x$ in $E$.
Sei also $f\in E'$, dann gilt
\begin{align*}
\abs{f(\xi_n)-f(x)} &= \abs{f\big|_M(\xi_n)-f\big|_M(x)} =
\abs{J_M(\xi_n)(f\big|_M) - J_M(x)(f\big|_M)} \\ &\le
\underbrace{\abs{J_M(\xi_n)(f\big|_M-y_j')}}_{\text{(1)}} +
\underbrace{\abs{J_M(\xi_n)(y_j')-x''(y_j')}}_{\text{(3)}} \\
&\quad+
\underbrace{\abs{x''(y_j'-f\big|_M)}}_{\text{(2)}}.
\end{align*}
Nun gilt
\begin{align*}
&(1) \le \norm{J_M(\xi_n)}_{E''}\norm{f\big|_M-y_j'}_{E'}
\le c\norm{f\big|_M-y_j'}_{E'} < \frac{\ep}{3},\\
&(2) \le \norm{x''}_{M''}\norm{y_j'-f\big|_M}_{E'} \le
c\norm{f\big|_M-y_j'}_{E'} < \frac{\ep}{2}\\
&(3) < \frac{\ep}{2},\qquad n> N_\ep\text{ bei } y_j' \text{ fest}.
\end{align*}
Also ist der gesamte Ausdruck $< \ep$, d.h. $f(\xi_n)\to f(x)$.\qedhere
\end{proofenum}
\end{proof}

\begin{defn}
\label{defn:4.31}
$K\subseteq L$ heißt \emph{konvex}\index{Menge!konvex}, falls für alle $x,y\in
K$ gilt,
\begin{align*}
\forall\lambda\in[0,1] : \lambda x + (1-\lambda)y\in K.\fishhere
\end{align*}
\end{defn}

\begin{bsp}
\label{bsp:4.32}
Sei $E$ ein normierter Raum.
\begin{bspenum}
  \item \textit{Kugeln sind konvex}. Betrachte also die Kugel mit Radius
  $R$ um $x_0$ und $x,y$ seien so gewählt, dass $\norm{x-x_0}_E<R$ und
  $\norm{y-x_0}_E < R$, dann gilt auch
\begin{align*}
\norm{\lambda x + (1-\lambda)y - x_0}_E
&= \norm{\lambda (x-x_0) + (1-\lambda)(y - x_0)}_E\\
&\le \lambda \norm{x-x_0}_E + (1-\lambda)\norm{y-x_0}_E
< R.
\end{align*}
\item \textit{Halbräume sind konvex}. Betrachte
\begin{align*}
K:=\setdef{x\in E}{\Re f(x)\le \alpha},\qquad \alpha\in\R,\quad f\in E'.
\end{align*}
Sind $x,y\in K$, so gilt auch
\begin{align*}
\Re f(\lambda x + (1-\lambda)y)
&= \lambda \Re f(x) + (1-\lambda) \Re f(y) \le \lambda \alpha +
(1-\lambda)\alpha \\ &= \alpha.\bsphere
\end{align*}
\end{bspenum}
\end{bsp}

\begin{prop}[Trennungssatz]
\label{prop:4.33}
Sei $K\subseteq E$ konvex und abgeschlossen, $x_0\in E\setminus K$. Dann gibt
es ein $f\in E'$ und ein $\alpha\in \R$, so dass
\begin{align*}
\Re f(x_0) > \alpha \text{ und } \forall x\in K : \Re f(x) \le \alpha.\fishhere
\end{align*}
\end{prop}

\begin{figure}[!htpb]
\centering
\begin{pspicture}(0,-1.4000001)(4.84,1.4)
\pspolygon[linestyle=none,fillstyle=solid,fillcolor=glightgray]%
	(0.14,1.36)(0.14,-1.06)(4.42,-1.06)(3.02,1.38)
\psline[linecolor=darkblue](4.42,-1.06)(3.02,1.38)
\psdots(4.58,0.28)
\psbezier[fillcolor=white,fillstyle=solid](1.5,0.58)(1.08,0.4)(0.0,1.02)(0.5,-0.18)(1.0,-1.38)(2.5482988,-0.81650954)(2.82,0.18)(3.091701,1.1765095)(1.92,0.76)(1.5,0.58)

\rput(2.38,0.025){\color{gdarkgray}$K$}
\rput(4.62,0.065){\color{gdarkgray}$x_0$}
\rput(3.27,-0.835){$\Re f(x)\le\alpha$}
\end{pspicture} 
\caption{Zum Trennungssatz.}
\end{figure}

\begin{proof}
\begin{proofenum}
\item Sei $\K=\R$.
\begin{defnenum}
  \item Wir können ohne Einschrädnkung annehmen, dass $0\in K^\circ$, denn
  andernfalls verschieben wir $K$ und $x_0$ so, dass $0\in K$ und ersetzen $K$
  durch,
\begin{align*}
\setdef{x\in E}{d(x,K) \ge \frac{1}{2}d(x_0,K)}.
\end{align*}
Man kann in leicht zeigen, dass diese Menge konvex ist.
\item
Sei $p: E\to \R,\; x\mapsto \inf\setdef{r>0}{r^{-1}x\in K}$. $p$ ist
wohldefiniert, da $K_\ep(0)\subseteq K$ und damit
$\left(\frac{2}{\ep}\norm{x}\right)^{-1}x\in K_\ep(0)$, also ist die Menge
nicht leer und das Infimum endlich.

$p$ heißt \emph{Minkowskifunktional} und hat folgende Eigenschaften,
\begin{equivenum}
  \item $0\le p(x) <\infty$.
  \item $\forall x\in K : p(x) \le 1$.
  \item $p(x_0) > 1$.
  \item $p(\lambda x)=\lambda x$, falls $\lambda \ge 0$
  \item $p(x+y) \le p(x) + p(y)$.
\end{equivenum}
(i)-(ii) sind klar.\\
Angenommen $p(x_0)=1$, dann existiert eine Folge in $K$,
die gegen $x_0$ konvergiert, $K$ ist abgeschlossen also $x_0\in K$. \dipper
Ist hingegen $p(x_0)<1$, dann ist $\frac{x_0}{r}\in K$ für ein $r < 1$ und
damit auch $(1-r)\cdot 0  + r \frac{x_0}{r}\in K.\dipper$ Also folgt (iii).\\
Zu (iv): Sei $\lambda > 0$, dann ist $\frac{\lambda x}{\lambda r}\in
K\Leftrightarrow \frac{x}{r}\in K$. $p(0)=0$ ist klar.\\
Zu (v): Seien $r^{-1}x,s^{-1}y\in K$, dann ist
\begin{align*}
\frac{x+y}{r+s} = \frac{r}{r+s}\frac{x}{r} + \frac{s}{r+s}\frac{y}{s}\in K.
\end{align*}
Also ist $p$ sublinear.
\item Setze $T: \lin{x_0}\to \R$, $\lambda x_0\mapsto \lambda p(x_0)$, dann ist
$T$ linear und $p$-beschränkt, denn
\begin{align*}
&\lambda \le 0 : T\lambda x_0 \le 0 \le p(\lambda x_0),\\
&\lambda > 0 : T\lambda x_0 = \lambda p(x_0) = p(\lambda x_0).
\end{align*}
Mit dem Satz von Hahn-Banach folgt nun, dass eine lineare, $p$-beschränkte
Fortsetzung $\tilde{T}: E\to \R$ existiert. Insbesondere gilt
\begin{align*}
&\forall x\in K : \tilde{T}(x) \le p(x) \le 1,\\
&\tilde{T}(x_0) = T(x_0) = p(x_0) > 1.
\end{align*}
\item $\tilde{T}\in E'$, denn $K_\ep(0)\subseteq K$ und daher gilt $\forall
x\in E: \frac{x}{\ep/2 \norm{x}}\in K$,
\begin{align*}
\Rightarrow &\forall x\in E : p(x) \le \frac{\ep}{2}\norm{x},\\
& \tilde{T}(x) \le p(x)  \le \frac{\ep}{2}\norm{x}\\
& -\tilde{T}(x) = \tilde{T}(-x) \le \frac{\ep}{2}\norm{x}.
 \end{align*}
\end{defnenum}
\item Sie $\K=\C$. Fasse $E$ als Raum über $\R$ auf und konstruiere $\tilde{T}$
wie in 1.). Setze anschließend
\begin{align*}
f(x) = \tilde{T}(x)-i\tilde{T}(ix).\qedhere
\end{align*}
\end{proofenum}
\end{proof}

\begin{prop}
\label{prop:4.34}
Eine konvexe abgeschlossene Menge $K\subseteq E$ ist schwach
folgenabgeschlossen, d.h. der schwache Grenzwert $x$ einer Folge $(x_n)$ in $K$,
die schwach in $E$ konvegiert, liegt in $K$.\fishhere
\end{prop}
\begin{proof}
Sei also $(x_n)$ Folge in $K$, die schwach in $E$ konvergiert, d.h.
\begin{align*}
\forall f\in E': f(x_n)\to f(x),\qquad x\in K.
\end{align*}
Angenommen $x\in E\setminus K$. Wir können also den Trennungssatz auf $x$
anwenden und erhalten somit ein $f\in E'$ mit
\begin{align*}
f(x) > \alpha,\qquad \forall y\in K : f(y)\le \alpha.
\end{align*}
Nun gilt aber $x_n\wto x$ und daher auch
\begin{align*}
\forall n\in \N f(x_n)\le \alpha \Rightarrow f(x) \le \alpha.\dipper\qedhere
\end{align*}
\end{proof}

\begin{prop}
\label{prop:4.35}
Sei $E$ reflexiv und $K\subseteq E$ konvex und abgeschlossen, $x_0\in
E\setminus K$. Dann gilt
\begin{align*}
\exists y_0\in K : \norm{x_0-y_0}_E = d(x_0,K).\fishhere
\end{align*}
\end{prop}
\begin{proof}
Sei $(y_n)$ Folge in $K$ mit $\norm{x_0-y_n} \to d(x_0,K)$, dann ist $(y_n)$
beschränkt. Somit existiert eine schwach konvergente Teilfolge $(y_{n_k})$ mit
Grenzwert $y$. Nach \ref{prop:4.34} ist $y\in K$. Außerdem gilt $x_0-y_{n_k}\wto
x_0-y$ also mit \ref{prop:4.26} auch
\begin{align*}
\norm{y-x_0} = \liminf_n \norm{y_{n_k}-x_0} = d(x_0,K).\qedhere
\end{align*}
\end{proof}

\begin{bsp}[Anwendung]
\label{bsp:4.36}
Betrachte eine kreisförmige dünne Membran, die am Rand befestigt ist. Nun wird
ein Hindernis unter die Membran geschoben, so dass diese sich diese nach oben
ausdehnt. Wir interessieren uns nun für die Form, die die Membran annimmt,
abhängig vom Hindernis.

Zur Modellierung der Membran, betrachte eine Funktion $z=u(x,y)$ mit $(x,y)\in
G=K_r(0)$, die die Auslenkung im Punkt $(x,y)$ beschreibt. Die Befestigung am
Rand modellieren wir durch $u\big|_{\partial G} = 0$.

Die Membran beschreibt ein konservatives System. Die Energie ist also erhalten
und aus der Physik wissen wir, dass sie folgende Form hat:
\begin{align*}
&E: C_0^1(G)\to \R,\quad u\mapsto E(u) = \int_G \abs{\partial_x u}^2 +
\abs{\partial_y u}^2\dmu(x,y),\\
&C_0^1(G) := \setdef{f\in C^1(G\to\R)}{f\big|_{\partial G}=0}.
\end{align*}

Gesucht ist nun ein $u\in C_0^1(G)$, so dass $E(u)$ minimal wird. Dazu führen
wir auf $C_0^1(G)$ die Norm
\begin{align*}
\norm{u}_2 = \int_G \abs{u}^2 + \abs{\partial_x u}^2+ \abs{\partial_y
u}\dmu(x,y)
\end{align*}
ein. Auf $C_0^1(G)$ sind $\norm{\cdot}_2$ und $\norm{\cdot}$ äquivalent. Die
Vervollständigungen bezüglich dieser Normen sind also identisch: 
\begin{align*}
W_0^{1,2}(G) = \overline{C_0^1(G)}^{\norm{\cdot}_2} =
\overline{C_0^1(G)}^{\norm{\cdot}}.
\end{align*}
$W_0^{1,2}(G)$ ist ein sogenannter Sobolveraum. Wir werden später zeigen, dass
dieser reflexiv ist.

Modellieren wir das Hinderniss durch die Funktion $z=f(x,y)$, so 
besteht unser Problem darin, $u\in W_0^{1,2}(G)$ zu finden mit
\begin{align*}
u\in K := \overline{\setdef{u\in C_0^1(G)}{u(x,y)\ge f(x,y)}}.
\end{align*}
$K$ ist abgeschlossen und konvex. Ist $\norm{u}$ minimal, dann ist auch
$E(u)=\norm{u}^2$ minimal. Falls $f$ stetig, existiert nun ein $(x,y)\in G$ mit
$f(x,y)> 0$, d.h. $0\notin K$. Mit \ref{prop:4.35} folgt nun
\begin{align*}
&\exists u\in K : \norm{u-0} = d(0,K) = \inf\limits_{v\in K}\norm{v-0},\\
\Rightarrow & \norm{u} = \min\limits_{v\in K}\norm{v}.
\end{align*}
Wir haben somit die Existenz \textit{einer} Lösung nachgewiesen. Ob diese
Eindeutig ist oder wie diese konkret aussieht, ist ein weiteres Problem.\bsphere 
\end{bsp}