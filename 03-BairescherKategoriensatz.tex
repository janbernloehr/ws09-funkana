\chapter{Bairescher Kategoriensatz}

\begin{defn}
\label{defn:3.1}
Sei $M$ Menge. $d: M\times M\to \R$ heißt
\emph{Metrik}\index{Metrik}, falls für $x,y,z\in M$
\begin{defnenum}
  \item $d(x,y) \ge 0$,\qquad Positivität,
  \item $d(x,x) = 0\Leftrightarrow x = 0$,\qquad Definitheit,
  \item $d(x,y)=d(y,x)$,\qquad Symmetrie,
  \item $d(x,z)\le d(x,y)+d(y,z)$,\qquad Dreiecksungleichung.
\end{defnenum}  
$(M,d)$ bezeichnet man als \emph{metrischen
Raum}\index{Metrischer Raum}.\fishhere
\end{defn}

\begin{bem}
\label{bem:3.2}
\begin{bemenum}
  \item\label{bem:3.2:1} $x_n\to x \Leftrightarrow d(x_n,x)\to 0$. $(x_n)$ ist
  genau dann Cauchy, wenn
\begin{align*}
\forall \ep > 0 \exists N\in\N \forall n,m\ge N : d(x_n,x_m) < \ep.
\end{align*}
$x\in M$ ist genau dann innerer Punkt, falls
\begin{align*}
\exists r > 0 : K_r(x):=\setdef{y\in M}{d(x,y)<r}\subseteq M.
\end{align*}
Häufungspunkte, offene und abgeschlossene Mengen sowie Vollständigkeit sind wie
üblich definiert. Folgen- und Überdeckungskompaktheit sind hier äquivalent.
\item Ist $(E,\norm{\cdot})$ normierter Raum, so wird durch
$d(x,y):=\norm{x-y}_E$ eine Metrik induziert. Die Begriffe in
\ref{bem:3.2:1} sind unabhänig davon, ob eine Norm oder die durch sie
induzierte Metrik zugrundegelegt wird.\maphere
\end{bemenum}
\end{bem}

\begin{defn}
\label{defn:3.3}
Sei $N\subseteq M$, $d$ Metrik auf $M$ und $x\in M$.
\begin{align*}
d(x,N) := \inf\setdef{d(x,y)}{y\in N}.\fishhere
\end{align*}
\index{Abstand}
\end{defn}

\begin{prop}
\label{prop:3.4}
Sei $(M,d)$ metrischer Raum, $N\subseteq M$ abgeschlossen und $x\in M\setminus
N$.
\begin{propenum}
  \item $d(x,N)> 0$.
  \item Falls $N$ kompakt, $\exists y\in N : d(x,N)=d(x,y)$.\fishhere
\end{propenum}
\end{prop}

\begin{proof}
\begin{proofenum}
  \item \textit{Kontraposition}. Angenommen $d(x,N)=0$, dann existiert eine
  Folge $(y_n)$ in $N$, so dass $d(x,y_n)\to 0$, d.h. $y_n\to x$. $N$ ist
  abgeschlossen also ist $x\in N$.
  \item Sei $d(x,N)=c$, so existiert eine Folge $(y_n)$ in $N$, so dass
  $d(x,y_n)\to c$. $N$ ist kompakt, d.h. es existiert eine konvergente
  Teilfolge $y_{n_k}\to y\in N$. Dann ist
  $d(x,y)=\lim\limits_{k\to\infty} d(x,y_{n_k})=c$.\qedhere
\end{proofenum}
\end{proof}

\begin{prop}[Bairescher Kategoriensatz]
\label{prop:3.5}
\index{Satz!Bairescher Kategorien-}
Sei $(M,d)$ vollständiger metrischer Raum, $(A_k)$ Folge abgeschlossener Mengen
in $M$ mit
\begin{align*}
M= \bigcup_{n\in\N} A_n.
\end{align*}
Dann existiert ein $k\in\N$, so dass $A_k^\circ \neq \varnothing$.
($A_k^\circ:=\setd{\text{innere Punkte von }A_k}$)\fishhere
\end{prop}
\begin{proof}
Angenommen $\forall k\in \N : A_k^\circ = \varnothing$. Dann ist $M\setminus
A_1$ offen und nichtleer.
\begin{align*}
\Rightarrow \exists \overline{K_{r_1}(x_1)} \subseteq K_{2r_1}(x_1) \subseteq
M\setminus A_1,\qquad && r_1 < 1.
\end{align*}
Dann ist $K_{r_1}(x_1)\setminus A_2$ offen und nichtleer (sonst
$K_{r_1}(x_1)\subseteq A_2^\circ$).
\begin{align*}
&\Rightarrow \exists \overline{K_{r_2}(x_2)}\subseteq K_{2r_2}(x_2)\subseteq
K_{r_1}(x_1)\setminus A_2, && r_2< \frac{1}{2},\\
&\ldots\\
&\Rightarrow \exists \overline{K_{r_n}(x_n)}\subseteq K_{2r_n}(x_n)\subseteq
K_{r_{n-1}}(x_{n-1})\setminus A_{n}, && r_n< \frac{1}{2^{n-1}},\\
\end{align*}
Nun ist $x_n\in K_{r_n}(x_n)\subseteq K_{r_{n-1}}(x_{n-1})\subseteq \ldots
\subseteq K_{r_l}(r_l)$, wobei
\begin{align*}
r_l < \frac{1}{2^l} \Rightarrow d(x_n,x_l) < \frac{1}{2^l}.
\end{align*}
Somit ist $(x_n)$ Cauchy und, also $x_n\to x\in M$. Insbesondere ist
\begin{align*}
\forall k\in \N : x\in \overline{K_{r_k}(x_k)}\subseteq M\setminus A_k
\Rightarrow x\notin A_k\forall k\in\N.
\end{align*}
Aber $x\in\bigcup_{k\in\N}A_k$ \dipper.\qedhere
\end{proof}

\begin{bem}
\label{bem:3.6}
Eine alternative Formulierung des Baireschen Kategoriensatz ist, dass $M$ von
2. Kategorie ist.\maphere
\end{bem}

\begin{prop}[Satz von Banach-Steinhaus (uniform boundedness principle)]
\index{Satz!Banach-Steinhaus}
\label{prop:3.7}
Sei $B$ Banachraum, $E$ normierter Raum und
\begin{align*}
\TT\subseteq \LL(B\to E)
\end{align*}
mit $\forall x\in B: \sup\limits_{T\in\TT}\norm{Tx}_E<\infty$. Dann gilt
\begin{align*}
\sup\limits_{T\in\TT}\norm{T} <\infty.\fishhere
\end{align*}
\end{prop}

Punktweise Beschränktheit einer Familie von Operatoren impliziert also die
gleichmäßige Beschränktheit.

\begin{proof}
Setze
\begin{align*}
A_n := \setdef{x\in B}{\sup\limits_{T\in\TT} \norm{Tx}_E \le n}
= \bigcap_{T\in\TT}\setdef{x\in B}{\norm{Tx}_E\le n}.
\end{align*}
\begin{proofenum}
  \item Zeige $B=\bigcup_{n\in\N} A_n$. Sei dazu $x\in B$, dann folgt
\begin{align*}
\sup\limits_{T\in\TT}\norm{Tx}_E = R<\infty
\Rightarrow x\in A_n \text{ für }n\ge R.
\end{align*}
\item Zeige $A_n$ ist abgeschlossen. Betrachte dazu
\begin{align*}
B\setminus A_n = \bigcup_{T\in\TT} \setdef{x\in B}{\norm{Tx}_E > n}.
\end{align*}
Diese Menge ist als Urbild von $(n,\infty)$ unter der stetigen
Abbildung $\norm{\cdot}_E\circ T$ offen.  Mit dem Satz von Baire folgt nun
\begin{align*}
\exists n_0\in\N \exists K_r(x_0)\subseteq A_{n_0}.
\end{align*}
\item Sei $x\in B$ und $\norm{x}_B=1$, so gilt
\begin{align*}
\norm{Tx}_E &= \frac{2}{r}\norm{T\left(\frac{r}{2}x\right)}_E =
\frac{2}{r}\norm{T\left(\frac{r}{2}x+x_0\right)-Tx_0}_E\\
&\le
\frac{2}{r}\left(\norm{T\left(\frac{r}{2}x+x_0\right)}_E+\norm{Tx_0}_E\right)\\
&\le \frac{2}{r}\left(n_0 + \sup\limits_{T\in\TT} \norm{Tx_0}_E\right)\\
\Rightarrow \norm{T} &\le \frac{2}{r}\left(n_0 + \sup\limits_{T\in\TT}
\norm{Tx_0}_E\right)\\
 \Rightarrow \sup\limits_{T\in\TT}\norm{T} &\le \frac{2}{r}\left(n_0 +
\sup\limits_{T\in\TT} \norm{Tx_0}_E\right).\qedhere
\end{align*}
\end{proofenum}
\end{proof}

\begin{prop}[Satz (open mapping principle)]
\index{Satz!open mapping}
\label{prop:3.8}
Seien $B,\tilde{B}$ Banachräume, $T:B\to\tilde{B}$ linear, beschränkt und
surjektiv. Dann ist $T$ offen, d.h. für $O\subseteq \BB$ offen ist
$T(O)\subseteq \tilde{B}$ offen.\fishhere
\end{prop}
\begin{proof}
\begin{proofenum}
  \item \textit{Anwendung des Satzes von Baire}. Da $T$ surjektiv, gilt
\begin{align*}
B=\bigcup_{n\in\N} K_n(0) \Rightarrow \tilde{B} = \bigcup_{n\in\N}
\overline{T(K_n(0))}^\sim.
\end{align*}
Aufgrund der Vollständigkeit von
 $\tilde{B}$ folgt nun mit dem Satz von Baire
\begin{align*}
\exists n_0\in\N : \exists \tilde{K}_r(y_0)\subseteq
\overline{T(K_{n_0}(0))}^\sim.
\end{align*}
\item
\begin{figure}[!htpb]
\centering
\begin{pspicture}(0,-1.48)(5.3,1.52)
\psellipse(2.42,-0.01)(2.42,1.47)
\pscircle[linecolor=purple](1.63,-0.01){1.29}
\psdots(1.64,0.02)
 
\psline[linecolor=darkblue]{->}(3.32,0.06)(2.64,0.06)
\pscircle[linecolor=darkblue](4.03,0.05){0.63}

\rput(1.79,-0.195){\color{gdarkgray}$0$}
\rput(4.03,0.065){\color{darkblue}$\tilde{K}_r(y_0)$}
\rput(4.51,1.325){\color{gdarkgray}$\overline{T(K_{n_0}(0))}$}

\end{pspicture}
\caption{Zum Verschieben von $\tilde{K}_r(y_0)$}
\end{figure}

\textit{Verschieben von $\tilde{K}_r(y_0)$}. Sei $y\in\tilde{K}_r(0)$ und damit
$\norm{y}_\sim < r$, dann gilt
\begin{align*}
&y+y_0\in \tilde{K}_r(y_0) \subseteq \overline{T(K_{n_0}(0))}^\sim.
\end{align*}
Betrachte nun eine Folge $(x_n)$ in $K_{n_0}(0)$ mit
\begin{align*}
Tx_n\to y+y_0,
\end{align*}
und eine Folge $\xi_n$ in $K_{n_0}(0)$ mit $T\xi_n \to y_0$,
%Sei nun $x_0\in K_{n_0}(0)$ mit $Tx_0 = y_0$, 
so gilt
\begin{align*}
T(x_n-\xi_n) = Tx_n-T\xi_n \to y+y_0-y_0 = y,
\end{align*}
% wobei $\norm{x_n-x_0} \le \norm{x_n}+\norm{x_0} < 2n_0$.
% $y_0$ ist kein Häufungspunkt, denn für eine Folge $(\xi_n)$ in $K_{n_0}(0)$ mit
% $T\xi_n\to y_0$ folgt,
% \begin{align*}
% T(x_n-\xi_n) = Tx_n-T\xi_n \to y + y_0-y_0 = y,
% \end{align*}
wobei $\norm{x_n-\xi_n} \le \norm{x_n}+\norm{\xi_n} < 2n_0$. Also ist $y\in
\overline{T(K_{2n_0}(0)}^\sim$.

Somit ist auch $\tilde{K}_r(0)\subseteq \overline{T(K_{2n_0}(0)}^\sim$.
\item \textit{Skalieren}. Für $a > 0$ ist
\begin{align*}
\tilde{K}_{ar}(0) \subseteq \overline{T(K_{a2n_0}(0))}^\sim,
\end{align*}
denn sei $\norm{y}_\sim < ar$, so ist $\norm{\frac{y}{a}}_\sim < r$ und daher
\begin{align*}
\frac{y}{a}\in \overline{T(K_{2n_0}(0))}^\sim.
\end{align*}
Es existiert also eine Folge $(x_n)$ in $K_{2n_0}(0)$ mit $Tx_n\to
\frac{y}{a}$. Somit  existiert eine Folge $(ax_n)$ in $K_{2an_0}(0)$ mit
$Tax_n\to y$ und daher ist $y\in \overline{T(K_{2an_0}(0)}^\sim$.

Insbesondere $\exists \delta > 0 : \tilde{K}_\delta(0) \subseteq
\overline{T(K_1(0))}^\sim$.
\item \textit{Einbetten} $\overline{T(K_1(0))}^\sim \subseteq T(K_3(0))$. Sei
$y\in \overline{T(K_1(0))}^\sim$.
\begin{align*}
&\Rightarrow \exists x_0\in K_1(0) : \norm{y-Tx_0}_\sim < \frac{\delta}{2}
\end{align*}
also $y-Tx_0\in\tilde{K}_{\delta/2}(0)\subseteq
\overline{T(K_{1/2}(0))}^\sim$.
\begin{align*}
&\Rightarrow \exists x_1\in K_{1/2}(0) : \norm{y-Tx_0-Tx_1}_\sim <
\frac{\delta}{4},
\end{align*}
also $y-Tx_0-Tx_1\in\tilde{K}_{\delta/4}(0)\subseteq \overline{T(K_{1/4}(0))}$,
usw. 
Wir erhalten somit eine Folge $(x_n)$ mit
\begin{align*}
&\norm{y-\sum_{j=0}^n Tx_j} < \frac{\delta}{2^n},\qquad
\norm{x_j} < \frac{1}{2^j}\\
\Rightarrow
&\sum_{j=0}^n Tx_j = T\left(\sum\limits_{j=0}^n x_j\right) \to y
\end{align*}
Weiterhin gilt
\begin{align*}
&\norm{\sum\limits_{j=0}^n x_j} \le
\sum\limits_{j=0}^n \norm{x_j} < \sum\limits_{j=1}^\infty \frac{1}{2^j} =
\frac{1}{1-\frac{1}{2}} = 2.
\end{align*}
Nun ist $B$ Banachraum und daher ist $x=\sum\limits_{j=0}^\infty x_j\in B$ mit
$\norm{x}\le 2$, also $x\in \overline{K_2(0)}\subseteq K_3(0)$. $T$ ist
beschränkt also stetig und daher gilt,
\begin{align*}
&y=\sum\limits_{j=0}^\infty Tx_j = T\left(\sum_{j=0}^\infty x_j\right) = T(x),\\
\Rightarrow &\overline{T(K_1(0))}^\sim \subseteq T(K_3(0)). 
\end{align*}
\item \textit{Skalieren}. Aus $\tilde{K}_\delta(0)\subseteq T(K_3(0))$ folgt:
\begin{align*}
\tilde{K}_{\tilde{\delta}}(0) \subseteq T(K_\ep(0)),\qquad
\text{mit } \tilde{\delta} = \delta\frac{\ep}{3}.
\end{align*}
\item \textit{Abschluss (Offenheit)}. Sei $O\subseteq B$ offen und $y\in
T(O)$ ($y=Tx$, $x\in O$). Zeige nun
\begin{align*}
\exists \tilde{\delta} > 0 : K_{\tilde{\delta}}(y) \subseteq T(O).
\end{align*}
$O$ ist offen, also $\exists \ep > 0 : K_\ep(0)\subseteq O$.
\begin{align*}
\Rightarrow
\tilde{K}_{\tilde{\delta}}(T(x)) &= T(x) + \tilde{K}_{\tilde{\delta}}(0)
\subseteq T(x)+T(K_\ep(0)) = T(x+K_\ep(0)) \\ &= T(K_\ep(x)) \subseteq
T(O).\qedhere
\end{align*} 
\end{proofenum}
\end{proof}

\begin{prop}[Satz (inverse mapping theorem)]
\index{Satz!inverse mapping}
\label{prop:3.9}
Seien $B,\tilde{B}$ Banachräume, $T:B\to\tilde{B}$ linear, beschränkt und
bijektiv. Dann ist $T^{-1}$ beschränkt.\fishhere
\end{prop}

``$T$ stetig $\Rightarrow$ $T^{-1}$ beschränkt''.

\begin{proof}
Zeige $T^{-1}$ ist stetig in $y=0$, d.h.
\begin{align*}
\forall \ep > 0 \exists \delta_\ep > 0 : \norm{y}_{\tilde{B}} < \delta_\ep
\Rightarrow \norm{T^{-1}y}_B < \ep.
\end{align*}
Dies ist äquivalent zu
\begin{align*}
\forall \ep > 0 \exists \delta_\ep > 0 : \tilde{K}_{\delta_\ep}(0)\subseteq
T(K_\ep(0)).
\end{align*}
Dies folgt aber direkt aus der Offenheit von $T$.\qedhere
\end{proof}

\begin{cor}
\label{prop:3.10}
Sei $B$ vollständig bezüglich $\norm{\cdot}$ und $\norm{\cdot}_\sim$ und
$\norm{\cdot}$ feiner als $\norm{\cdot}_\sim$. Dann sind die Normen
äquivalent.\qedhere
\end{cor}
\begin{proof}
Die Abbildung $\Id : (B,\norm{\cdot}) \to (B,\norm{\cdot}_\sim), x\mapsto x$
ist beschränkt (also stetig). Mit Satz \ref{prop:3.9} folgt, dass auch
$\Id^{-1}$ beschränkt (also stetig) ist. Damit ist $\norm{\cdot}_\sim$ feiner als
$\norm{\cdot}$.\qedhere
\end{proof}