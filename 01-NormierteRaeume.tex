\chapter{Normierte Räume}

\section{Grundlagen}
 
 Im Folgenden sollen die grundlegenden Begriffe der normierten Räume wiederholt
 werden.
 
 \begin{defn}
 \label{defn:1.1}
 Sei $\K = \C$ bzw. $\K=\R$. Eine abelsche Gruppe $(L,+)$ heißt 
 \emph{linearer Raum} oder \emph{Vektorraum}\index{Vektorraum}, falls eine
 skalare Multiplikation
 \begin{align*}
 \cdot : \K\times L \to L
 \end{align*}
definiert ist, so dass für $\alpha,\beta\in\K$, $x,y\in L$ gilt
\begin{align*}
&\alpha\cdot (x+y) = \alpha\cdot x + \alpha\cdot y,\\
&(\alpha+\beta)\cdot x = \alpha\cdot x + \beta \cdot x,\\
&\alpha\cdot(\beta\cdot x) = (\alpha\beta)\cdot x.
\end{align*}
Falls $\K=\R$ heißt $L$, \emph{reeller linearer Raum}. Lineare Unabhängigkeit
und Dimension sind wie üblich definiert.\fishhere
\end{defn}
\begin{proof}
Aus $\alpha\cdot x = (\alpha+0)\cdot x = \alpha\cdot x + 0\cdot x$ folgt
unmittelbar, $0\cdot x = 0$.\qedhere
\end{proof}

\begin{bsp}
\label{bps:1.2}
Beispiele für lineare Räume sind,
\begin{bspenum}
  \item $L=\C^n$.
  \item $L=\setdef{(x_n)_{n\in\N}}{(x_n)\text{ komplexe Folge}}$. Die Addition
  ist hier definiert als $(x_n)+(y_n):= (x_n+y_n)$, die skalare Multiplikation als
  $\alpha\cdot (x_n):= (\alpha x_n)$.
  \item $L=C(\R\to\C)= \setdef{f:\R\to\C}{f\text{ ist stetig}}$.\bsphere
\end{bspenum}
\end{bsp}

Nun wollen wir die Begriffe Konvergenz und Abstand erklären.

\begin{defn}
\label{defn:1.3}
Eine Abbildung $\norm{\cdot}: L\to\R$ heißt \emph{Norm (auf $L$)}\index{Norm},
falls sie für $x\in L$ und $\alpha\in\K$ folgende Eigenschaften erfüllt.
\begin{enumerate}[label=(N\arabic{*})]
  \item $\norm{x}\ge 0$,\qquad Positivität,
  \item $\norm{x} = 0\Leftrightarrow x=0$,\qquad Definitheit,
  \item $\norm{\alpha x} = \abs{\alpha}\norm{x}$,\qquad Homogentität,
  \item $\norm{x+y} \le \norm{x}+\norm{y}$,\qquad Dreiecksungleichung.
\end{enumerate}
$(L,\norm{\cdot})$ heißt \emph{normierter Raum}\index{Norm!Normierter Raum},
falls (N1)-(N4) erfüllt sind. Gilt nur (N2) nicht, so heißt $\norm{\cdot}$
\emph{Halbnorm}\index{Norm!Halb-} und $(L,\norm{\cdot})$ \emph{halbnormierter Raum}.\fishhere
\end{defn}

Die geometrische Interpretation von $\norm{x}$ ist die ``Länge'' von $x$, die
von $\norm{x-y}$ ist der ``Abstand'' von $x$ und $y$.

Im Folgenden sei $E$ stets ein normierter Raum.

\begin{defn}
\label{defn:1.4}
Sei $(x_n)$ eine Folge in $E$.
\begin{defnenum}
  \item $(x_n)$ heißt \emph{konvergent}\index{Konvergenz} gegen $x\in E$, falls
\begin{align*}
\forall \ep > 0 \exists N_\ep \in \N \forall n> N_\ep : \norm{x-x_n} < \ep.
\end{align*}
\item $(x_n)$ heißt \emph{Cauchyfolge}\index{Cauchyfolge}, falls
\begin{align*}
\forall \ep > 0 \exists N_\ep \in\N \forall n,m> N_\ep : \norm{x_n-x_m} < \ep.
\end{align*}
\item $E$ heißt \emph{vollständig} oder
\emph{Banachraum}\index{Banach!-raum}, falls gilt,
\begin{align*}
(x_n)\text{ Cauchy} \Rightarrow (x_n) \text{ konvergent}.\fishhere
\end{align*}
\end{defnenum}
\end{defn}

\begin{bemn}[Übung.]
Zeigen Sie, dass jede konvergente Folge auch eine Cauchyfolge ist.\maphere
\end{bemn}

\begin{bsp}
\label{bsp:1.5}
\begin{bspenum}
  \item $E=\C^n$ lässt sich mit der $p$-Norm versehen,
\begin{align*}
\norm{x}_p = \begin{cases}
              \left(\abs{x_1}^p + \ldots +
\abs{x_n}^p\right)^{1/p}, & p\le 1 < \infty,\\
\max\limits_{j=1,\ldots,n} \abs{x_j},& p = \infty.
             \end{cases}
\end{align*}
Wichtige Spezialfälle sind, die euklidische Norm
\begin{align*}
\norm{x}_2 = \left(\abs{x_1}^2 + \ldots + \abs{x_n}^2\right)^{1/2},
\end{align*}
die Summennorm
\begin{align*}
\norm{x}_1 = \abs{x_1} + \ldots + \abs{x_n},
\end{align*}
und die Supremumsnorm
\begin{align*}
\norm{x}_\infty = \max\limits_{j=1,\ldots,n} \abs{x_j}.
\end{align*}
Einen Nachweis der Normeigenschaften findet man in jedem Standard Analysis
Werk. Für $p<1$ lässt sich so keine $p$-Norm definieren, da die
Dreiecksungleichung nicht erfüllt werden kann. Für $1\le p<\infty$ ist
$(\C^n,\norm{\cdot}_p)$ ein Banachraum.
\item Für eine komplexe Folge sei
\begin{align*}
&\norm{(x_n)}_p := \left(\sum\limits_{j=1}^\infty \abs{x_j}^p \right)^{1/p},\\
&\norm{(x_n)}_\infty := \max\limits_{j\in\N}\abs{x_j}.
\end{align*}
Der Raum, auf dem sich diese Norm definieren lässt, heißt
\begin{align*}
l^p := \setdef{(x_n)\text{ komplexe Folge}}{\norm{(x_n)}_p < \infty}.
\end{align*}
Die Normeigenschaften (N1)-(N3) sind klar, (N4) folgt mittels der Minkowski
Ungleichung. Jeder $l^p$ ist ein Banachraum.
\item $E=C([0,1]\to\C)$, der Raum der stetigen Funktionen auf dem Intervall
$[0,1]$, lässt sich ebenfalls mit einer $p$-Norm versehen,
\begin{align*}
\norm{f}_p :=
\begin{cases}
 \left(\int_0^1 \abs{f(x)}^p \dx\right)^{1/p},& 1\le p <\infty,\\
 \max\limits_{0\le x\le 1} \abs{f(x)}, & p = \infty.
\end{cases}
\end{align*}
Da $[0,1]$ kompakt und $f\in E$ stetig ist, ist die $p$-Norm stets endlich.
Die $2$-Norm bildet einen wichtigen Spezialfall in der Theorie der Hilberträume.

Eine leichte Übung zeigt, dass $(E,\norm{\cdot}_\infty)$ Banachraum ist,
während $(E,\norm{\cdot}_p)$ für $1\le p <\infty$ kein Banachraum ist.
\item Sei $k\in\N$ fest, $E:=C^k([0,1]\to\C)$ und
\begin{align*}
\norm{f}_{j,p}^{\sim} := \norm{\frac{\diffd^j f}{\dx^j}}_p,\qquad
j=1,\ldots,k.
\end{align*}
Offensichtlich ist $\norm{f}_{j,p}^{\sim}$ lediglich eine
Halbnorm. Mittels der Konstruktion
\begin{align*}
\norm{f}_{k,p} := \sum\limits_{j=1}^k \norm{f}_{j,p}^{\sim} + \norm{f}_p
\end{align*}
erhalten wir eine Norm auf $E$.\bsphere
\end{bspenum}
\end{bsp}

\begin{lem}
\label{prop:1.6}
In $(E,\norm{\cdot})$ gelten die Aussagen:
\begin{propenum}
  \item Der Grenzwert einer Folge ist eindeutig.
  \item\label{prop:1.6:2} Umgekehrte Dreiecksungleichung,
\begin{align*}
\abs{\norm{x}-\norm{y}}\le\norm{x-y},\qquad x,y\in E.
\end{align*}
\item Seien $(x_n)$, $(y_n)$ Folgen in $E$ und $\alpha\in\K$. Gilt
$\lim\limits_{n\to\infty} x_n = x$, $\lim\limits_{n\to\infty} y_n = y$, so gilt
ebenfalls
\begin{equivenum}
  \item $\lim\limits_{n\to\infty} (x_n+y_n) = x+y$,
  \item $\lim\limits_{n\to\infty} (\alpha\cdot x_n) = \alpha\lim\limits_{n\to\infty}
x_n$.
\end{equivenum} 
D.h. Addition und skalare Multiplikation sind stetig bezüglich der Norm.
\item $\lim\limits_{n\to\infty} x_n = x \Rightarrow \lim\limits_{n\to\infty}
\norm{x_n} = \norm{x}$.
\item Sei $(x_n)$ Cauchyfolge, so ist $(\norm{x_n})$ konvergent.\fishhere
\end{propenum}
\end{lem}
\begin{proof}
\begin{proofenum}
  \item Angenommen $x_n\to x$ und $x_n\to y$, so gilt
\begin{align*}
\norm{x-y} = \norm{x-x_n-(y-x_n)} \le \norm{x-x_n} + \norm{y-x_n} \to 0.
\end{align*}
Da die linke Seite unabhängig von $n$ ist, gilt $\norm{x-y}=0$, d.h. $x=y$.
\item Klar.
\item Betrachte $\norm{(x+y)-(x_n+y_n)} \le \norm{x-x_n}+\norm{y-y_n} \to
0.$
\item $\abs{\norm{x_n}-\norm{x}} \overset{\ref{prop:1.6:2}}{\le} \norm{x_n-x}
\to 0$.
\item $\abs{\norm{x_n}-\norm{x_m}} \le \norm{x_n-x_m} < \ep$ für $n,m$
hinreichend groß. Also ist $(\norm{x_n})$ Cauchy und daher konvergent.\qedhere
\end{proofenum}
\end{proof}

\begin{lem}
\label{lem:1.7}
Für $(E,\norm{\cdot})$ sind äquivalent,
\begin{equivenum}
  \item\label{lem:1.7:1} $E$ ist vollständig,
  \item\label{lem:1.7:2} $(x_n)\in E^\N$ und $\sum\limits_{n=1}^\infty
  \norm{x_n} < \infty$
$\Rightarrow \sum\limits_{j=1}^n x_j$\text{ konvergent}.\fishhere
\end{equivenum}
\end{lem}
\begin{proof}
\ref{lem:1.7:1}$\Rightarrow$\ref{lem:1.7:2} : Sei $y_k :=
\sum\limits_{n=1}^k x_n$ und $k>l$, dann ist
\begin{align*}
\norm{y_k-y_l} = \norm{\sum\limits_{n=l}^k x_n} \le \sum\limits_{n=l}^k
\norm{x_n} < \ep
\end{align*}
für $k,l$ hinreichend groß, da die Reihe $\sum\limits_{n=1}^\infty \norm{x_n}$
konvergiert. Also ist $(y_k)$ Cauchy und aufgrund von \ref{lem:1.7:1}
konvergent.

\ref{lem:1.7:2}$\Rightarrow$\ref{lem:1.7:1}: Sei $(y_k)$ Cauchy. Wähle eine
Teilfolge $(y_{k_l})_{l\in\N}$ mit
\begin{align*}
&\norm{y_{k_1}-y_k} < \frac{1}{2}, && \text{für }k>k_1,\\
&\norm{y_{k_2}-y_k} < \frac{1}{4}, && \text{für }k>k_2, k_2 >k_1,\\
&\quad\vdots\\
&\norm{y_{k_l}-y_{k}} < \frac{1}{2^l} , && \text{für }k>k_l, k_l >k_{l-1}.
\end{align*}
Setze nun $x_l := y_{k_{l+1}} - y_{k_l}$, dann ist
\begin{align*}
\sum\limits_{l=1}^\infty \norm{x_l} \le \sum\limits_{l=1}^\infty \frac{1}{2^l}
< \infty,
\end{align*}
also existiert $\sum\limits_{k=1}^\infty x_k\in E$.
Weiterhin gilt
\begin{align*}
\sum\limits_{l=1}^n x_l = y_{k_{n+1}} - y_{k_1},
\end{align*}
also konvergiert auch $y_{k_{n+1}} = \sum\limits_{l=1}^n x_k + y_{k_1}$ in $E$.
$(y_k)$ ist Cauchy und besitzt eine konvergente Teilfolge, ist also selbst
konvergent.\qedhere
\end{proof}

\section{Vergleich von Normen}

In diesem kurzen Abschnitt soll alles Notwendige erarbeitet werden, um mit
mehreren Normen auf demselben Raum zu arbeiten, sowie Kriterien, um Ergebnisse
bezüglich der einen Norm auf die andere zu übertragen. 

\begin{defn}
\label{defn:1.8}
Seien $\norm{\cdot}$, $\norm{\cdot}^\sim$ zwei Normen  auf $E$.
\begin{defnenum}
  \item $\norm{\cdot}$ heißt \emph{feiner}\index{Norm!Gröber, feiner} als
  $\norm{\cdot}^\sim$, falls
\begin{align*}
\exists c > 0 \forall x\in E : \norm{x}^\sim \le c \norm{x}.
\end{align*}
\item $\norm{\cdot}$ und $\norm{\cdot}^\sim$ heißen
\emph{äquivalent}\emph{Norm!Äquivalenz}, wenn $\norm{\cdot}$ feiner als
$\norm{\cdot}^\sim$ und $\norm{\cdot}^\sim$ feiner als $\norm{\cdot}$, d.h.
\begin{align*}
\exists c_1,c_2 > 0 \forall x\in E : c_1 \norm{x} \le \norm{x}^\sim \le
c_2\norm{x}.\fishhere
\end{align*}
\end{defnenum}
\end{defn}

\begin{bemn}
Die Äquivalenz von Normen bildet eine Äquivalenzrelation.\maphere
\end{bemn}

\begin{prop}
\label{prop:1.9}
\begin{propenum}
  \item Ist $\norm{\cdot}$ feiner als $\norm{\cdot}^\sim$ und $(x_n)$
  konvergent (Cauchy) bezüglich $\norm{\cdot}$, dann auch bezüglich
  $\norm{\cdot}^\sim$.
  \item Sind $\norm{\cdot}$ und $\norm{\cdot}^\sim$ äquivalent, so ist $(x_n)$
  genau dann konvergent (Cauchy) bezüglich $\norm{\cdot}$, falls $(x_n)$
  konvergent (Cauchy) bezüglich $\norm{\cdot}^\sim$.\fishhere
\end{propenum}
\end{prop}
\begin{proof}
Sei $x_n\to x$ bezüglich $\norm{\cdot}$, dann gilt
\begin{align*}
\norm{x_n-x}^\sim \le c \norm{x_n-x} \to 0.\qedhere
\end{align*}
\end{proof}

\begin{bsp}
\label{bsp:1:10}
Sei $E:= C([0,1]\to\C)$, sowie
\begin{align*}
&\norm{f}_1 := \int\limits_0^1 \abs{f(x)}\dx,\\
&\norm{f}_\infty := \sup\limits_{x\in[0,1]} \abs{f(x)},\\
&f_n = x^n.
\end{align*}
\begin{bspenum}
  \item $\norm{\cdot}_\infty$ ist feiner als $\norm{\cdot}_1$, denn
\begin{align*}
\norm{f}_1 = \int\limits_0^1 \abs{f(x)}\dx \le \norm{f}_\infty.
\end{align*}
\item $\norm{\cdot}_1$ ist nicht feiner als $\norm{\cdot}_\infty$. Zeige dazu,
\begin{align*}
\forall c > 0 \exists f\in E : \norm{f}_\infty > c\norm{f}_1. 
\end{align*}
Betrachte dazu die Funktionenfolge $(f_n)$
\begin{align*}
\Rightarrow \norm{f_n}_\infty = 1,\quad \norm{f_n}_1 =
\frac{1}{n+1}.
\end{align*}
\item $(f_n)$ konvergiert bezüglich $\norm{\cdot}_1$ aber nicht bezüglich
$\norm{\cdot}_\infty$.

$f(x) = 0$ ist die Grenzfunktion bezüglich $\norm{\cdot}_1$, denn
\begin{align*}
\norm{f_n-f} = \frac{1}{n+1}\to 0.
\end{align*}
Zeige nun, dass $f_n$ bezüglich $\norm{\cdot}_\infty$ nicht Cauchy ist, d.h.
\begin{align*}
\exists \ep > 0 \forall N\in\N \exists n,m > N : \norm{f_n-f_m}_\infty > \ep.
\end{align*}
Setze $\ep =\frac{1}{4}$. Zu beliebigem $n>N$ wähle $x_0\in[0,1)$ mit $f_n(x_0)
= x_0^n > \frac{1}{2}$. Wähle $m>n$ mit $f_m(x_0) = x_0^m < \frac{1}{4}$, dann
gilt $\norm{f_n-f_m} > \frac{1}{4}$.\bsphere
\end{bspenum}
\end{bsp}

\begin{prop}
\label{prop:1.11}
Ist $L$ endlichdimensional, dann sind alle Normen auf $L$ äquivalent.\fishhere
\end{prop}
\begin{proof}
\begin{proofenum}
  \item Sei $\BB=\setd{b_1,\ldots,b_n}$ Basis von $L$. Setze
\begin{align*}
\norm{x}_1 = \norm{\sum\limits_{j=1}^n x_j b_j}_1 := \sum \abs{x_j}
\end{align*}
als Vergleichsnorm.
\item Sei $\norm{\cdot}$ Norm auf $L$. Dann ist $\norm{\cdot}_1$ feiner
\begin{align*}
\norm{x} = \norm{\sum\limits_{j=1}^n x_j b_j} \le
\sum\limits_{j=1}^n \abs{x_j} \norm{b_j} \le
\underbrace{\max_j \norm{b_j}}_{>0} \cdot \sum\limits_{j=1}^n \abs{x_j}.
\end{align*}
\item Sei $\norm{\cdot}$ Norm auf $L$. Dann ist $\norm{\cdot}$ feiner als
$\norm{\cdot}_1$.
Die Norm $\norm{\cdot}$ ist als Abbildung
\begin{align*}
f : (L,\norm{\cdot}_1) \to \R, \quad x\mapsto \norm{x} =
\norm{\sum\limits_{j=1}^n x_j b_j}
\end{align*}
stetig, denn für $x,y\in L$ gilt,
\begin{align*}
\abs{f(x)-f(y)} &= 
\abs{\norm{\sum\limits_{j=1}^n x_j b_j}-\norm{\sum\limits_{j=1}^n y_j b_j}}
\le \norm{\sum\limits_{j=1}^n (x_j-y_j)b_j}\\ & \le
\sum\limits_{j=1}^n \abs{x_j-y_j}\norm{b_j} \le
\max_j \norm{b_j}\cdot\sum\limits_{j=1}^n \abs{x_j-y_j} \\ &= c \norm{x-y}_1
\end{align*}
Nun ist die Menge $S=\setdef{x\in L}{\norm{x}_1 = 1}$ beschränkt und
abgeschlossen bezüglich der $\norm{\cdot}_1$ Norm, also kompakt. $f$
nimmt daher auf $S$ sein Minimum an, d.h.
\begin{align*}
\exists \xi\in S \forall x\in S : f(x) \ge f(\xi). 
\end{align*}
Da $\xi\in S$ gilt $\xi\neq 0$ und damit folgt für $0\neq x\in L$,
\begin{align*}
\norm{x} = \norm{\sum\limits_{j=1}^n x_j b_j} = \left(\sum\limits_{l=1}^n
\abs{x_l}\right)
\norm{\sum\limits_{j=1}^n \frac{x_j}{\sum\limits_{l=1}^n \abs{x_l}}b_j}
\ge \norm{x}_1 \underbrace{f(\xi)}_{> 0}.
\end{align*}
\item Seien nun $\norm{\cdot}$ und $\norm{\cdot}^\sim$ zwei Normen auf $L$,
dann gilt
\begin{align*}
\exists c_1,c_2,c_3,c_4 > 0 :
\norm{x} \le c_1 \norm{x}_1 \le c_2 \norm{x}^\sim \le c_3 \norm{x}_1 \le c_4
\norm{x}.\qedhere
\end{align*}
\end{proofenum}
\end{proof}

\section{Topologische Grundbegriffe}

Es folgen einige topologische Begriffe für normierte Räume. Nicht alle Aussagen
lassen sich auf einen metrischen bzw. allgemeinen topologischen Raum übertragen.

\begin{defn}
\label{defn:1.12}
Sei $(E,\norm{\cdot})$ ein normierter Raum, $X\subseteq E$.
\begin{defnenum}
  \item $K_r(x) := \setdef{y\in E}{\norm{x-y} < r}$
  \emph{offene Kugel} mit Radius $r (>0)$.
  \item $x\in E$ heißt \emph{innerer Punkt}\index{Innerer Punkt} von $X$,
  falls
\begin{align*}
\exists r > 0 : K_r(x) \subseteq X.
\end{align*}
\item $X$ heißt \emph{offen}\index{Menge!offen}, falls
\begin{align*}
\forall x\in X : x\text{ ist innerer Punkt von $X$}.
\end{align*}
\item $x\in E$ heißt \emph{Häufungspunkt}\index{Häufungspunkt}\item  von $X$, falls
\begin{align*}
\forall r>0 : K_r(x)\cap X\setminus \setd{x} \neq \emptyset
\end{align*}
oder äquivalent: Es existiert eine Folge $(x_n)$ in $X$ mit $x_n\to x$ und
$x_n\neq x$.
\item $X$ heißt \emph{abgeschlossen}\index{Menge!abgeschlossen}, wenn $X$ alle
seine Häufungspunkte enthält.
\item $\overline{X}:=X\cup\setd{\text{Häufungspunkte von }X}$ heißt
\emph{Abschluss} von $X$. \\
$\overline{X}$ ist abgeschlossen und die kleinste abgeschlossene Menge, die $X$
enthält.
\item $Y\subseteq X\subseteq E$ heißt \emph{dicht}\index{Menge!dicht} in $X$,
falls $\overline{Y}\supseteq X$.
\item $X$ heißt \emph{kompakt}\index{Menge!kompakt}, wenn jede offene
Überdeckung $(\OO_\alpha)$, (d.h. ein System offener Mengen $O_\alpha$, $\alpha\in\AA$ Indexmenge, mit
 $X\subseteq \bigcup_\alpha O_\alpha$) eine endliche Teilmenge 
 $\setd{O_{\alpha_1},\ldots,O_{\alpha_n}}$ besitzt, die bereits $X$ überdeckt.\\
 Oder hier äquivalent: Wenn jede Folge $(x_n)$ in $X$ eine in $X$ konvergente
 Teilfolge enthält.
 \item Seien $(E_1,\norm{\cdot}_1)$, $(E_2,\norm{\cdot}_2)$ normierte Räume.
 Eine Abbildung
 \begin{align*}
 f: E_1\to E_2,
 \end{align*}
heißt \emph{stetig}\index{Abbildung!stetig} in $x_0\in E$, falls
\begin{align*}
\forall \ep > 0 \exists \delta > 0 \forall x \in E_1: \norm{x-x_0}_1 < \delta
\Rightarrow \norm{f(x)-f(x_0)}_2 < \ep.
\end{align*}
Oder äquivalent: Wenn für jede Folge $(x_n)$ in $E_1$ gilt
\begin{align*}
x_n \to x_0 \Rightarrow f(x_n)\to f(x_0).
\end{align*}
$f$ heißt \emph{stetig}, wenn $f$ in jedem $x\in E_1$ stetig
ist.\fishhere
\end{defnenum}
\end{defn}

\section{Neue Räume aus alten}

\begin{defn}[Direkte Summe]
\label{defn:1.13}
Seien $(E_1,\norm{\cdot}_1),(E_2,\norm{\cdot}_2)$ normierte Räume, so ist
\begin{align*}
&E_1\oplus E_2 := \setdef{(x,y)}{x\in E_1,y\in E_2},\\
&\norm{(x,y)} := \norm{x}_1 + \norm{x}_2
\end{align*}
ein normierter Raum\index{Vektorraum!Direkte Summe}. Sind $E_1,E_2$ Banachräume,
so ist auch $E_1\oplus E_2$ Banachraum.\fishhere
\end{defn}

\begin{defn}[Quotientenraum]
\label{defn:1.14}
Sei $F$ ein linearer Unterraum von $E$. Setze $x\sim y\Leftrightarrow x-y\in
F$, so ist $\sim$ eine Äquivalenzrelation.
\begin{align*}
E/F  := \setdef{[x]}{x\in E}
\end{align*}
wird zum Quotientenraum\index{Vektorraum!Quotientenraum} mit Addition
$[x]+[y]:=[x+y]$ und skalarer Multiplikation $\alpha[x]=[\alpha x]$.\fishhere
\end{defn}

\begin{prop}
\label{prop:1.15}
Sei $(E,\norm{\cdot})$ ein normierter Raum, $F$ abgeschlossener linearer
Unterraum von $E$ und
\begin{align*}
\norm{[x]}_0 := \inf\setdef{\norm{x+z}}{z\in F} = \inf\setdef{\norm{y}}{y\in
[x]}.
\end{align*}
Dann gelten
\begin{propenum}
  \item $\norm{\cdot}_0$ ist Norm auf $E/F$.
  \item Ist $E$ Banachraum, so ist auch $E/F$ Banachraum.\fishhere
\end{propenum}
\end{prop}
\begin{proof}
\begin{proofenum}
  \item $\norm{[x]}_0$ ist unabhängig vom gewählten Vertreter von $x$,
  denn
\begin{align*}
\norm{[x]}_0 :=  \inf\setdef{\norm{y}}{y\in
[x]]}.
\end{align*}
$\norm{[x]}_0$ ist Norm. (N1) ist offensichtlich erfüllt. (N2): Sei
$\norm{[x]}_0 = 0$, dann existiert eine Folge $(z_n)$ in $F$ mit
$\norm{x+z_n}\to 0$, d.h. $z_n\to-x\in F$ also ist $[x]=[0]$.
(N3): Zunächst ist $\norm{0\cdot[x]}_0 = \norm{[0\cdot x]}_0 = 0$. Für
$\alpha\neq0$ gilt weiterhin,
\begin{align*}
\norm{\alpha[x]}_0 &= \norm{[\alpha x]}_0 = \inf\setdef{\norm{\alpha x +
z}}{z\in F} \\ &= \abs{\alpha}\setdef{\norm{x+\frac{1}{\alpha}z}}{z\in F} =
\abs{\alpha}\norm{[x]}_0.
\end{align*}
(N4): Sei $\ep>0$, dann existieren $z_1,z_2\in F$ mit
\begin{align*}
 \norm{x+z_1} \le \norm{[x]}_0+\ep,\quad
\norm{y+z_2} \le \norm{[y]}_0 + \ep.
\end{align*}
Somit erhalten wir
\begin{align*}
\norm{[x]+[y]}_0 &= \norm{[x+y]}_0
= \inf\setdef{\norm{x+y+z}}{z\in F}\\
&\le \norm{x+z_1+y+z_2}
\le \norm{x+z_1} + \norm{y+z_2}
\\ &\le \norm{[x]}_0 + \norm{[y]}_0 + 2\ep
\end{align*}
Da $\ep$ beliebig war, folgt die Dreiecksungleichung.
\item Sei $[x_n]$ Folge in $E/F$ mit $\sum\limits_{j=1}^\infty \norm{[x_n]}_0
<\infty$. Zu $n\in\N$ wähle jeweils einen Vertreter $x_n\in E$, so dass
\begin{align*}
&\norm{x_n} \le \norm{[x_n]_n}_0 + \frac{1}{2^n},\\
\Rightarrow & \sum\limits_{n=1}^\infty \norm{x_n} < \infty. 
\end{align*} 
$E$ ist Banachraum, daher ist nach Lemma \ref{lem:1.7} $y=\sum\limits_{j=1}^n
x_n$ konvergent. Sei nun $N\in\N$, so gilt
\begin{align*}
\norm{\sum\limits_{j=1}^N [x_n] - [y]}_0 = \norm{\nrm{\sum\limits_{j=1}^N x_n
- y}}_0 \le \norm{\sum\limits_{j=1}^N x_n-y} \to 0,\quad N\to \infty,
\end{align*}
d.h. $\sum\limits_{j=1}^N [x_n] = [y]$ bezüglich $\norm{\cdot}_0$.
Ebenfalls mit Lemma \ref{lem:1.7} folgt, $E/F$ ist vollständig.\qedhere
\end{proofenum}
\end{proof}

\begin{prop}[Vervollständigung]
\index{Vervollständigung}
\label{prop:1.16}
Sei $(E,\norm{\cdot})$ ein normierter Raum. Dann existiert ein Banachraum
$(F,\norm{\cdot}_F)$, so dass $E$ mit einem dichten linearen Unterraum
identifiziert werden kann. D.h. es existiert eine lineare, normerhaltende
Abbildung,
\begin{align*}
j: E\to F,
\end{align*}
mit $j(E)$ ist dicht in $F$. $j$ ist dann insbesondere injektiv.\fishhere
\end{prop}
\begin{proof}
\begin{proofenum}
  \item Betrachte dazu den Raum der Cauchyfolgen auf $E$,
\begin{align*}
\hat{F} := \setdef{(x_n)\in E^\N}{(x_n)\text{ ist Cauchy}}.
\end{align*}
Dieser Raum ist linear und lässt sich mit der Halbnorm
\begin{align*}
\norm{(x_n)}^\land := \lim\limits_{n\to\infty} \norm{x_n}_E
\end{align*}
versehen, denn für jede Cauchyfolge konvergiert die Folge $(\norm{x_n})$.
\item Sei
\begin{align*}
\hat{N} := \setdef{(x_n)\in E^\N}{(x_n)\text{ Nullfolge}}
\end{align*}
der Raum der Nullfolgen. Setze $F=\hat{F}/\hat{N}$ und
\begin{align*}
\norm{[x_n]}_F &:= \inf\setdef{\norm{(x_n)-(z_n)}^\land}{(z_n)\in\hat{N}}\\
&= \inf\setdef{\lim\limits_{n\to\infty}\norm{y_n}_E}{(x_n)-(y_n)\in\hat{N}}.
\end{align*}
Eine äquivalente Formulierung erhalten wir, wenn wir $\hat{x}=(x_n)$ setzen,
\begin{align*}
\norm{[\hat{x}]}_F :=
\inf\setdef{\norm{\hat{y}}^\land}{\hat{x}-\hat{y}\in\hat{N}} =
\norm{\hat{x}}^\land.
\end{align*}
\textit{$\norm{\cdot}_F$ ist Norm}. (N1),(N3),(N4) sind klar, da
$\norm{\cdot}^\land$ Halbnorm. (N2): Des Weiteren gilt $\norm{[\hat{x}]}_F = 0\Leftrightarrow
\norm{\hat{x}}^\land = 0$, d.h. $\hat{x}\in\hat{N}$ und daher $[\hat{x}] = [0]$.
\item
Sei $j: E\to F, x\mapsto (x,x,x,\ldots)$. $j$ ist offensichtlich linear und
\begin{align*}
\norm{j(x)}_F = \lim\limits_{n\to\infty} \norm{x}_E = \norm{x}_E.
\end{align*}
\item \textit{$j(E)$ ist dicht in $E$}. Sei $[\hat{x}]\in F=\hat{F}/\hat{N}$,
dann ist $\hat{x}=(x_n)$ Cauchy in $E$. Nun ist
\begin{align*}
\norm{j(x_m)-[\hat{x}]}_F = \norm{[\hat{y}_m] - [\hat{x}]}_F
= \norm{\nrm{\hat{y}_m-\hat{x}}}_F = \norm{(x_m-x_n)}^{\land}.
\end{align*}
Es gilt
\begin{align*}
\lim\limits_{n\to\infty} \norm{x_m-x_n} < \ep,
\end{align*}
für $m> N_\ep$, also liegt $j(E)$ dicht in $F$.
\item \textit{$F$ ist vollständig}. Sei $([\hat{x}_n])_{n\in\N}$ Cauchyfolge in
$F$, so ist $\hat{x}_n=(y_k^{(n)})_{k\in\N}$ Cauchyfolge in $E$ und da
$\norm{[\hat{x}]}_F = \norm{\hat{x}}^\land$, ist $\hat{x}_n$ Cauchyfolge in
$\hat{F}$. Somit gilt,
\begin{align*}
&\forall n\in\N \forall \ep > 0 \exists K_{n,\ep} \forall k,l \ge K_{n,\ep}
\norm{y_k^{(n)}-y_l^{(n)}}_E < \ep,\tag{*}\\
&\forall \ep > 0 \exists N_\ep \in\N \forall n,m>N_\ep :
\norm{\hat{x}_n-\hat{x}_m}^\land <\ep\tag{**}.
\end{align*}
Setze nun $y_k:=y_{K_{k,1/k}}^{(k)}$, $y:=(y_k)_{k\in\N}$, so ist $(y_k)$
Cauchyfolge in $E$, denn
\begin{align*}
\norm{y_k-y_l}_E &= \norm{y_{K_{k,1/k}}^{(k)}-y_{K_{l,1/l}}^{(l)}}_E\\
&\le \underbrace{\norm{y_{K_{k,1/k}}^{(k)}-y_j^{(k)}}_E}_{(1)} +
\underbrace{\norm{y_j^{(k)}-y_j^{(l)}}_E}_{(2)} +
\underbrace{\norm{y_j^{(l)}-y_{K_{l,1/l}}^{(l)}}_E}_{(3)}.
\end{align*}
Sei $\ep > 0$, wähle $j$ so, dass $j> \max\setd{K_{k,1/k},K_{l,1/l},N_\ep}$, so
sind nach (*) $(1)<\frac{1}{k}$ und $(3)<\frac{1}{l}$. Für eventuell noch
größeres $j$ gilt außerdem
\begin{align*}
(2) = \norm{y_j^{(k)}-y_j^{(l)}}_E \le \lim\limits_{j\to\infty}
\norm{y_j^{(k)}-y_j^{(l)}}_E + \ep = \norm{\hat{x}_k - \hat{x}_l}^\land + \ep.
\end{align*}
Wählen wir nun $k,l>N_\ep$ so erhalten wir nach (**) 
\begin{align*}
\norm{y_k-y_l}_E \le \frac{1}{k} + 2\ep + \frac{1}{l},\qquad \text{für }k,l >
N_\ep.
\end{align*}
also ist $(y_k)$ Cauchy in $E$ und daher $[\hat{y}]\in F$.

Um die Konvergenz zu zeigen, betrachte
\begin{align*}
\norm{\hat{x}_n - \hat{y}}^\sim &= \lim\limits_{k\to\infty}
\norm{y_k^{(n)}-y_{K_{k,1/k}}^{(k)}}_E\\
&\le\lim\limits_{k\to\infty}
\underbrace{\norm{y_k^{(n)}-y_{K_{k,1/k}}^{(n)}}_E}_{(*): < \frac{1}{k}} +
\underbrace{\norm{y_{K_{k,1/k}}^{(n)}- y_{K_{k,1/k}}^{(k)}}_E}_{(**): <
\frac{1}{k}} < \frac{2}{k}
\end{align*}
für $k,l$ hinreichend groß. Also $[\hat{x}_n]\to [\hat{y}]$ in $F$.\qedhere
\end{proofenum}
\end{proof}

\begin{figure}[!htpb]
\centering
\begin{pspicture}(0,-2.64)(6.6,2.68)
\pscircle(0.62,1.48){0.58}
\pscircle(4.26,0.92){1.28}
\pscircle(1.62,-1.36){1.28}
\psbezier[linecolor=darkblue]{->}(1.18,1.94)(1.66,2.4)(2.7,2.4)(3.26,2.0)

\rput(2.21,2.485){\color{gdarkgray}Bilde Cauchyfolgen}
\psbezier[linecolor=darkblue]{->}(0.46,1.14)(0.96,0.54)(0.54,-0.1)(0.96,-0.5)

\rput(0.56,0.505){\color{gdarkgray}$j$}
\psline[linecolor=yellow](3.26,0.14)(5.2,1.76)

\rput(4.63,2.405){\color{gdarkgray}$\hat{F}$}
\rput(4.61,0.885){\color{gdarkgray}$\hat{N}$}
\psbezier[linecolor=darkblue]{->}(4.06,-0.42)(3.8,-1.16)(3.44,-1.3)(2.98,-1.34)

\psline[linecolor=yellow](0.38,-1.14)(1.38,-0.12)
\psline[linecolor=yellow](0.36,-1.52)(1.8,-0.12)
\psline[linecolor=yellow](0.44,-1.82)(2.12,-0.2)
\psline[linecolor=yellow](0.58,-2.1)(2.38,-0.36)
\psline[linecolor=yellow](0.76,-2.3)(2.58,-0.54)
\psline[linecolor=yellow](1.02,-2.48)(2.76,-0.8)
\psline[linecolor=yellow](1.28,-2.58)(2.86,-1.08)
\psline[linecolor=yellow](1.66,-2.62)(2.88,-1.48)

\rput(4.47,-0.575){\color{gdarkgray}bilde}
\rput(5.2,-1.015){\color{gdarkgray}Äquivalenzklassen}
\psdots[dotsize=0.12](0.4,1.2)

\rput(0.11,2.105){\color{gdarkgray}$E$}
\rput(2.64,-2.495){\color{gdarkgray}$F$}
\rput(1.66,0.14){\color{gdarkgray}$\hat{x}+\hat{N}$}
\end{pspicture} 
\caption{Zur Konstruktion des Raumes $F=\hat{F}/\hat{N}$.}
\end{figure}


\begin{bem}
\label{bem:1.17}
Wir werden im nächsten Kapitel sehen, dass diese
Vervollständigung bis auf Isomorphie auch eindeutig ist.\maphere
\end{bem}

\begin{bsp}
\label{bsp:1.18}
\begin{bspenum}
  \item $E=l_\text{abb}:=\setdef{(x_n)\in \C^\N}{\exists N\in\N \forall n> N :
  x_n =0}$. Vervollständigung bezüglich $\norm{\cdot}_p$ ist $l_p$. $(1\le p <
  \infty)$. Für $p=\infty$ ist die Vervollstängigung bezüglich
  $\norm{\cdot}_\infty$ der Raum der Nullfolgen.
  \item $P([a,b]):=\text{Menge der komplexen Polynome als Funktion
  }[a,b]\to\C$. Die Vervollständigung bezüglich $\norm{\cdot}_\infty$ ist
  $C([a,b]\to\C)$ (siehe Weierstraßscher Approximationssatz).
  \item Sei $O\subseteq\R^n$ offen. Für stetiges $f:O\to\C$ setze
\begin{align*}
\norm{f}_p &:= \left(\int\limits_O \abs{f(x)}^p \right)^{1/p},\\
E&:=C_0^\infty(O\to C) \\ &:= \setdef{f\in C^\infty(O\to\C)}{\supp f
\text{ kompakt und }\supp f \subseteq O}.
\end{align*}
Die Vervollständigung ist der $\LL^p(O)$.
\item Sei $O\subseteq \R^n$ offen, $1\le p <\infty$ und $k\in\N$.
\begin{align*}
&E:= \setdef{f\in C^k(\overline{O}\to C)}{\norm{f}_{k,p}< \infty},\\
&\norm{f}_{k,p} := \sum\limits_{\atop{(\alpha_1,\ldots,\alpha_k)\in
\N_0}{\abs{\alpha_1}+\ldots+\abs{\alpha_k}\le k}}
\norm{\frac{\partial^{\alpha_1}\cdots \partial^{\alpha_k}}{\partial
x_1^{\alpha_1}\cdots \partial x_n^{\alpha_n}} f}_p. 
\end{align*}
Die Vervollständigung ist der Sobolevraum $W^{k,p}(O)$. Die Elemente dieses
Raumes sind in einem verallgemeinerten Sinne differenzierbar (schwache
Ableitung, Distributionenableitung, starke Ableitung).\\
Sobolevsche Einbettung:
Für $k>\frac{n}{2}$ gilt $W^{k,2}(O)\subseteq C(\overline{O}\to\C)$, falls
$\partial O$ ``genügend gut''.\bsphere
\end{bspenum}
\end{bsp}

