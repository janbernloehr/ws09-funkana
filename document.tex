% =============================================================================
% Titel:		Funktionalanalysis - Mitschrieb
% Erstellt:	WS 09
% Dozent:	PD. P.H. Lesky
% Autor:	Jan-Cornelius Molnar
% =============================================================================
\documentclass[%
	paper=a5,%
	fleqn,%
	DIV=calc,%
	headings=openleft,
	titlepage=false,%
	twoside=true]%
	{scrbook}

% =============================================================================
% 					Benötigte Pakete
% =============================================================================
\usepackage{janmcommon}
\usepackage{janmscript}
\usepackage{fancyhdr}
\usepackage{float}
\restylefloat{figure}
\usepackage{marginnote}
\usepackage{makeidx}

\makeindex

% =============================================================================
% 					Theorem-Style
% =============================================================================
% Theorem Umgebungen *MIT* Numerierung
\theoremstyle{graymarginwithblueheader}
\theorembodyfont{\itshape}
\theoremseparator{}
\theoremsymbol{}

\newtheorem{prop}{Satz}[chapter]
\newtheorem{lem}[prop]{Lemma}
\newtheorem{defn}[prop]{Definition}
\newtheorem{cor}[prop]{Korollar}

\theoremstyle{graymarginwithyellowheader}
\theorembodyfont{\normalfont}
\theoremseparator{}
\theoremsymbol{}

\newtheorem{bsp}[prop]{Bsp}

\theoremstyle{graymarginwithitblackheader}
\theorembodyfont{\normalfont}
\theoremseparator{}
\theoremsymbol{}

\newtheorem{bem}[prop]{Bemerkung.}

% Theorem Umgebungen *OHNE* Numerierung
\theoremstyle{graymarginwithblueheadern}
\theorembodyfont{\itshape}
\theoremseparator{}
\theoremsymbol{}

\newtheorem{propn}{Satz}
\newtheorem{lemn}{Lemma}
\newtheorem{corn}{Korollar}
\newtheorem{defnn}{Definition}

\theoremstyle{graymarginwithyellowheadern}
\theorembodyfont{\normalfont}
\theoremseparator{}
\theoremsymbol{}

\newtheorem{bspn}{Bsp}

\theoremstyle{graymarginwithitblackheadern}
\theorembodyfont{\normalfont}
\theoremseparator{}
\theoremsymbol{}

\newtheorem{bemn}{Bemerkung.}

% =============================================================================
% 					Überschriften-Style
% =============================================================================
\renewcommand\thesection{\arabic{chapter}-\Alph{section}}
\renewcommand\thesubsection{\small\ensuremath{\blacksquare}\normalsize}
\renewcommand\thebsp{\arabic{bsp}}

\setkomafont{chapter}{\normalfont\bfseries\Huge\color{darkblue}}
\setkomafont{section}{\normalfont\bfseries\Large\color{darkblue}}
\setkomafont{subsection}{\normalfont\bfseries\color{darkblue}}

% =============================================================================
% 					PSTricks-Standards
% =============================================================================
\psset{linecolor=gdarkgray}
\psset{tickcolor=gdarkgray}
\psset{fillcolor=glightgray}

% =============================================================================
% 					Header/Footer-Style
% =============================================================================
% Select page style
\pagestyle{fancyplain}

% Reset header & footer
\fancyhf{}

% Reset chapter & sectionmark
\renewcommand{\chaptermark}[1]{\markboth{\textsc{#1}}{}}
\renewcommand{\sectionmark}[1]{\markright{\textsc{#1}}{}}

% Clear headrule
\renewcommand{\headrule}{}

% =============================================================================
% 					Listen-Style
% =============================================================================
\newenvironment{bemenum}%
	{\begin{enumerate}[label=\textsc{\alph{*}}.,leftmargin=17pt]}{\end{enumerate}}
\newenvironment{defnenum}%
	{\begin{enumerate}[label={\rm(\alph{*})}]}{\end{enumerate}}
\newenvironment{propenum}%
	{\begin{enumerate}[label=\arabic{*})]}{\end{enumerate}}
\newenvironment{equivenum}%
	{\begin{enumerate}[label=(\roman{*})]}{\end{enumerate}}
\newenvironment{bspenum}%
	{\begin{enumerate}[label=\alph{*}.),leftmargin=17pt]}{\end{enumerate}}
\newenvironment{proofenumarabicbr}%
	{\begin{enumerate}[label=(\arabic{*}),leftmargin=17pt]}{\end{enumerate}}
\newenvironment{proofenumroman}%
	{\begin{enumerate}[label=(\roman{*}),leftmargin=17pt]}{\end{enumerate}}
\newenvironment{proofenum}%
	{\begin{enumerate}[label=\arabic{*}),leftmargin=17pt]}{\end{enumerate}}
\newenvironment{proofenuma}%
	{\begin{enumerate}[label=\alph{*}.),leftmargin=2pt]}{\end{enumerate}}

% =============================================================================
% 					Eigene Operatoren
% =============================================================================
\renewcommand{\labelenumi}{{\normalfont(\alph{enumi})}}
\renewcommand{\AA}{\mathcal{A}}
\newcommand{\Lim}{\mathrm{Lim}}

\renewcommand{\Id}{\mathbf{I}}

\renewcommand{\card}{\#}

\newcommand{\codim}{\mathrm{codim}}


% =============================================================================
% 					Document-Body
% =============================================================================
\begin{document}

% Titel
\begin{titlepage}
\begin{center}
{\huge\bf Funktionalanalysis - Mitschrieb}

bei PD. Dr. P. H. Lesky

Jan-Cornelius Molnar, Version: \today\ \thistime
\end{center}
\end{titlepage}

In der Funktionalanalysisvorlesung konzentrieren wir uns auf bekannte
Ergebnisse der Analysis und der linearen Algebra und versuchen diese auf
unendlichdimensionale Räume zu verallgemeinern bzw. zu erweitern. Dazu gehören
z.B. die Eigenwerttheorie, Lösbarkeit von linearen Gleichungssystemen,
Stetigkeitsbegriffe, uvm.
Anwendung findet die Funktionalanalysis beispielsweise bei der Lösung von
partiellen Differentialgleichungen, der Spektraltheorie und in anderen
angewandten Bereichen der Mathematik wie z.B. der Wahrscheinlichkeitstheorie
und stochastischen Analysis, der mathematischen Physik oder der Numerik.

% Inhaltsverzeichnis
\tableofcontents

\fancyhead[RO]{\footnotesize\color{gdarkgray}%
	\marginnote{\Big|\;\textbf{\thesection}}\rightmark}
\fancyhead[LE]{\footnotesize\color{gdarkgray}%
	\marginnote{\;\textbf{\thechapter}\Big|}\leftmark}

% Zweiseitig
\fancyfoot[LE]{\footnotesize\color{gdarkgray}%
 	\thepage}%
 \fancyfoot[RO]{\footnotesize\color{gdarkgray}%
 	\thepage}%
 \fancyfoot[RE,LO]{\tiny\color{gdarkgray}\today\; \thistime}

% Inhalt
\chapter{Normierte Räume}

\section{Grundlagen}
 
 Im Folgenden sollen die grundlegenden Begriffe der normierten Räume wiederholt
 werden.
 
 \begin{defn}
 \label{defn:1.1}
 Sei $\K = \C$ bzw. $\K=\R$. Eine abelsche Gruppe $(L,+)$ heißt 
 \emph{linearer Raum} oder \emph{Vektorraum}\index{Vektorraum}, falls eine
 skalare Multiplikation
 \begin{align*}
 \cdot : \K\times L \to L
 \end{align*}
definiert ist, so dass für $\alpha,\beta\in\K$, $x,y\in L$ gilt
\begin{align*}
&\alpha\cdot (x+y) = \alpha\cdot x + \alpha\cdot y,\\
&(\alpha+\beta)\cdot x = \alpha\cdot x + \beta \cdot x,\\
&\alpha\cdot(\beta\cdot x) = (\alpha\beta)\cdot x.
\end{align*}
Falls $\K=\R$ heißt $L$, \emph{reeller linearer Raum}. Lineare Unabhängigkeit
und Dimension sind wie üblich definiert.\fishhere
\end{defn}
\begin{proof}
Aus $\alpha\cdot x = (\alpha+0)\cdot x = \alpha\cdot x + 0\cdot x$ folgt
unmittelbar, $0\cdot x = 0$.\qedhere
\end{proof}

\begin{bsp}
\label{bps:1.2}
Beispiele für lineare Räume sind,
\begin{bspenum}
  \item $L=\C^n$.
  \item $L=\setdef{(x_n)_{n\in\N}}{(x_n)\text{ komplexe Folge}}$. Die Addition
  ist hier definiert als $(x_n)+(y_n):= (x_n+y_n)$, die skalare Multiplikation als
  $\alpha\cdot (x_n):= (\alpha x_n)$.
  \item $L=C(\R\to\C)= \setdef{f:\R\to\C}{f\text{ ist stetig}}$.\bsphere
\end{bspenum}
\end{bsp}

Nun wollen wir die Begriffe Konvergenz und Abstand erklären.

\begin{defn}
\label{defn:1.3}
Eine Abbildung $\norm{\cdot}: L\to\R$ heißt \emph{Norm (auf $L$)}\index{Norm},
falls sie für $x\in L$ und $\alpha\in\K$ folgende Eigenschaften erfüllt.
\begin{enumerate}[label=(N\arabic{*})]
  \item $\norm{x}\ge 0$,\qquad Positivität,
  \item $\norm{x} = 0\Leftrightarrow x=0$,\qquad Definitheit,
  \item $\norm{\alpha x} = \abs{\alpha}\norm{x}$,\qquad Homogentität,
  \item $\norm{x+y} \le \norm{x}+\norm{y}$,\qquad Dreiecksungleichung.
\end{enumerate}
$(L,\norm{\cdot})$ heißt \emph{normierter Raum}\index{Norm!Normierter Raum},
falls (N1)-(N4) erfüllt sind. Gilt nur (N2) nicht, so heißt $\norm{\cdot}$
\emph{Halbnorm}\index{Norm!Halb-} und $(L,\norm{\cdot})$ \emph{halbnormierter Raum}.\fishhere
\end{defn}

Die geometrische Interpretation von $\norm{x}$ ist die ``Länge'' von $x$, die
von $\norm{x-y}$ ist der ``Abstand'' von $x$ und $y$.

Im Folgenden sei $E$ stets ein normierter Raum.

\begin{defn}
\label{defn:1.4}
Sei $(x_n)$ eine Folge in $E$.
\begin{defnenum}
  \item $(x_n)$ heißt \emph{konvergent}\index{Konvergenz} gegen $x\in E$, falls
\begin{align*}
\forall \ep > 0 \exists N_\ep \in \N \forall n> N_\ep : \norm{x-x_n} < \ep.
\end{align*}
\item $(x_n)$ heißt \emph{Cauchyfolge}\index{Cauchyfolge}, falls
\begin{align*}
\forall \ep > 0 \exists N_\ep \in\N \forall n,m> N_\ep : \norm{x_n-x_m} < \ep.
\end{align*}
\item $E$ heißt \emph{vollständig} oder
\emph{Banachraum}\index{Banach!-raum}, falls gilt,
\begin{align*}
(x_n)\text{ Cauchy} \Rightarrow (x_n) \text{ konvergent}.\fishhere
\end{align*}
\end{defnenum}
\end{defn}

\begin{bemn}[Übung.]
Zeigen Sie, dass jede konvergente Folge auch eine Cauchyfolge ist.\maphere
\end{bemn}

\begin{bsp}
\label{bsp:1.5}
\begin{bspenum}
  \item $E=\C^n$ lässt sich mit der $p$-Norm versehen,
\begin{align*}
\norm{x}_p = \begin{cases}
              \left(\abs{x_1}^p + \ldots +
\abs{x_n}^p\right)^{1/p}, & p\le 1 < \infty,\\
\max\limits_{j=1,\ldots,n} \abs{x_j},& p = \infty.
             \end{cases}
\end{align*}
Wichtige Spezialfälle sind, die euklidische Norm
\begin{align*}
\norm{x}_2 = \left(\abs{x_1}^2 + \ldots + \abs{x_n}^2\right)^{1/2},
\end{align*}
die Summennorm
\begin{align*}
\norm{x}_1 = \abs{x_1} + \ldots + \abs{x_n},
\end{align*}
und die Supremumsnorm
\begin{align*}
\norm{x}_\infty = \max\limits_{j=1,\ldots,n} \abs{x_j}.
\end{align*}
Einen Nachweis der Normeigenschaften findet man in jedem Standard Analysis
Werk. Für $p<1$ lässt sich so keine $p$-Norm definieren, da die
Dreiecksungleichung nicht erfüllt werden kann. Für $1\le p<\infty$ ist
$(\C^n,\norm{\cdot}_p)$ ein Banachraum.
\item Für eine komplexe Folge sei
\begin{align*}
&\norm{(x_n)}_p := \left(\sum\limits_{j=1}^\infty \abs{x_j}^p \right)^{1/p},\\
&\norm{(x_n)}_\infty := \max\limits_{j\in\N}\abs{x_j}.
\end{align*}
Der Raum, auf dem sich diese Norm definieren lässt, heißt
\begin{align*}
l^p := \setdef{(x_n)\text{ komplexe Folge}}{\norm{(x_n)}_p < \infty}.
\end{align*}
Die Normeigenschaften (N1)-(N3) sind klar, (N4) folgt mittels der Minkowski
Ungleichung. Jeder $l^p$ ist ein Banachraum.
\item $E=C([0,1]\to\C)$, der Raum der stetigen Funktionen auf dem Intervall
$[0,1]$, lässt sich ebenfalls mit einer $p$-Norm versehen,
\begin{align*}
\norm{f}_p :=
\begin{cases}
 \left(\int_0^1 \abs{f(x)}^p \dx\right)^{1/p},& 1\le p <\infty,\\
 \max\limits_{0\le x\le 1} \abs{f(x)}, & p = \infty.
\end{cases}
\end{align*}
Da $[0,1]$ kompakt und $f\in E$ stetig ist, ist die $p$-Norm stets endlich.
Die $2$-Norm bildet einen wichtigen Spezialfall in der Theorie der Hilberträume.

Eine leichte Übung zeigt, dass $(E,\norm{\cdot}_\infty)$ Banachraum ist,
während $(E,\norm{\cdot}_p)$ für $1\le p <\infty$ kein Banachraum ist.
\item Sei $k\in\N$ fest, $E:=C^k([0,1]\to\C)$ und
\begin{align*}
\norm{f}_{j,p}^{\sim} := \norm{\frac{\diffd^j f}{\dx^j}}_p,\qquad
j=1,\ldots,k.
\end{align*}
Offensichtlich ist $\norm{f}_{j,p}^{\sim}$ lediglich eine
Halbnorm. Mittels der Konstruktion
\begin{align*}
\norm{f}_{k,p} := \sum\limits_{j=1}^k \norm{f}_{j,p}^{\sim} + \norm{f}_p
\end{align*}
erhalten wir eine Norm auf $E$.\bsphere
\end{bspenum}
\end{bsp}

\begin{lem}
\label{prop:1.6}
In $(E,\norm{\cdot})$ gelten die Aussagen:
\begin{propenum}
  \item Der Grenzwert einer Folge ist eindeutig.
  \item\label{prop:1.6:2} Umgekehrte Dreiecksungleichung,
\begin{align*}
\abs{\norm{x}-\norm{y}}\le\norm{x-y},\qquad x,y\in E.
\end{align*}
\item Seien $(x_n)$, $(y_n)$ Folgen in $E$ und $\alpha\in\K$. Gilt
$\lim\limits_{n\to\infty} x_n = x$, $\lim\limits_{n\to\infty} y_n = y$, so gilt
ebenfalls
\begin{equivenum}
  \item $\lim\limits_{n\to\infty} (x_n+y_n) = x+y$,
  \item $\lim\limits_{n\to\infty} (\alpha\cdot x_n) = \alpha\lim\limits_{n\to\infty}
x_n$.
\end{equivenum} 
D.h. Addition und skalare Multiplikation sind stetig bezüglich der Norm.
\item $\lim\limits_{n\to\infty} x_n = x \Rightarrow \lim\limits_{n\to\infty}
\norm{x_n} = \norm{x}$.
\item Sei $(x_n)$ Cauchyfolge, so ist $(\norm{x_n})$ konvergent.\fishhere
\end{propenum}
\end{lem}
\begin{proof}
\begin{proofenum}
  \item Angenommen $x_n\to x$ und $x_n\to y$, so gilt
\begin{align*}
\norm{x-y} = \norm{x-x_n-(y-x_n)} \le \norm{x-x_n} + \norm{y-x_n} \to 0.
\end{align*}
Da die linke Seite unabhängig von $n$ ist, gilt $\norm{x-y}=0$, d.h. $x=y$.
\item Klar.
\item Betrachte $\norm{(x+y)-(x_n+y_n)} \le \norm{x-x_n}+\norm{y-y_n} \to
0.$
\item $\abs{\norm{x_n}-\norm{x}} \overset{\ref{prop:1.6:2}}{\le} \norm{x_n-x}
\to 0$.
\item $\abs{\norm{x_n}-\norm{x_m}} \le \norm{x_n-x_m} < \ep$ für $n,m$
hinreichend groß. Also ist $(\norm{x_n})$ Cauchy und daher konvergent.\qedhere
\end{proofenum}
\end{proof}

\begin{lem}
\label{lem:1.7}
Für $(E,\norm{\cdot})$ sind äquivalent,
\begin{equivenum}
  \item\label{lem:1.7:1} $E$ ist vollständig,
  \item\label{lem:1.7:2} $(x_n)\in E^\N$ und $\sum\limits_{n=1}^\infty
  \norm{x_n} < \infty$
$\Rightarrow \sum\limits_{j=1}^n x_j$\text{ konvergent}.\fishhere
\end{equivenum}
\end{lem}
\begin{proof}
\ref{lem:1.7:1}$\Rightarrow$\ref{lem:1.7:2} : Sei $y_k :=
\sum\limits_{n=1}^k x_n$ und $k>l$, dann ist
\begin{align*}
\norm{y_k-y_l} = \norm{\sum\limits_{n=l}^k x_n} \le \sum\limits_{n=l}^k
\norm{x_n} < \ep
\end{align*}
für $k,l$ hinreichend groß, da die Reihe $\sum\limits_{n=1}^\infty \norm{x_n}$
konvergiert. Also ist $(y_k)$ Cauchy und aufgrund von \ref{lem:1.7:1}
konvergent.

\ref{lem:1.7:2}$\Rightarrow$\ref{lem:1.7:1}: Sei $(y_k)$ Cauchy. Wähle eine
Teilfolge $(y_{k_l})_{l\in\N}$ mit
\begin{align*}
&\norm{y_{k_1}-y_k} < \frac{1}{2}, && \text{für }k>k_1,\\
&\norm{y_{k_2}-y_k} < \frac{1}{4}, && \text{für }k>k_2, k_2 >k_1,\\
&\quad\vdots\\
&\norm{y_{k_l}-y_{k}} < \frac{1}{2^l} , && \text{für }k>k_l, k_l >k_{l-1}.
\end{align*}
Setze nun $x_l := y_{k_{l+1}} - y_{k_l}$, dann ist
\begin{align*}
\sum\limits_{l=1}^\infty \norm{x_l} \le \sum\limits_{l=1}^\infty \frac{1}{2^l}
< \infty,
\end{align*}
also existiert $\sum\limits_{k=1}^\infty x_k\in E$.
Weiterhin gilt
\begin{align*}
\sum\limits_{l=1}^n x_l = y_{k_{n+1}} - y_{k_1},
\end{align*}
also konvergiert auch $y_{k_{n+1}} = \sum\limits_{l=1}^n x_k + y_{k_1}$ in $E$.
$(y_k)$ ist Cauchy und besitzt eine konvergente Teilfolge, ist also selbst
konvergent.\qedhere
\end{proof}

\section{Vergleich von Normen}

In diesem kurzen Abschnitt soll alles Notwendige erarbeitet werden, um mit
mehreren Normen auf demselben Raum zu arbeiten, sowie Kriterien, um Ergebnisse
bezüglich der einen Norm auf die andere zu übertragen. 

\begin{defn}
\label{defn:1.8}
Seien $\norm{\cdot}$, $\norm{\cdot}^\sim$ zwei Normen  auf $E$.
\begin{defnenum}
  \item $\norm{\cdot}$ heißt \emph{feiner}\index{Norm!Gröber, feiner} als
  $\norm{\cdot}^\sim$, falls
\begin{align*}
\exists c > 0 \forall x\in E : \norm{x}^\sim \le c \norm{x}.
\end{align*}
\item $\norm{\cdot}$ und $\norm{\cdot}^\sim$ heißen
\emph{äquivalent}\emph{Norm!Äquivalenz}, wenn $\norm{\cdot}$ feiner als
$\norm{\cdot}^\sim$ und $\norm{\cdot}^\sim$ feiner als $\norm{\cdot}$, d.h.
\begin{align*}
\exists c_1,c_2 > 0 \forall x\in E : c_1 \norm{x} \le \norm{x}^\sim \le
c_2\norm{x}.\fishhere
\end{align*}
\end{defnenum}
\end{defn}

\begin{bemn}
Die Äquivalenz von Normen bildet eine Äquivalenzrelation.\maphere
\end{bemn}

\begin{prop}
\label{prop:1.9}
\begin{propenum}
  \item Ist $\norm{\cdot}$ feiner als $\norm{\cdot}^\sim$ und $(x_n)$
  konvergent (Cauchy) bezüglich $\norm{\cdot}$, dann auch bezüglich
  $\norm{\cdot}^\sim$.
  \item Sind $\norm{\cdot}$ und $\norm{\cdot}^\sim$ äquivalent, so ist $(x_n)$
  genau dann konvergent (Cauchy) bezüglich $\norm{\cdot}$, falls $(x_n)$
  konvergent (Cauchy) bezüglich $\norm{\cdot}^\sim$.\fishhere
\end{propenum}
\end{prop}
\begin{proof}
Sei $x_n\to x$ bezüglich $\norm{\cdot}$, dann gilt
\begin{align*}
\norm{x_n-x}^\sim \le c \norm{x_n-x} \to 0.\qedhere
\end{align*}
\end{proof}

\begin{bsp}
\label{bsp:1:10}
Sei $E:= C([0,1]\to\C)$, sowie
\begin{align*}
&\norm{f}_1 := \int\limits_0^1 \abs{f(x)}\dx,\\
&\norm{f}_\infty := \sup\limits_{x\in[0,1]} \abs{f(x)},\\
&f_n = x^n.
\end{align*}
\begin{bspenum}
  \item $\norm{\cdot}_\infty$ ist feiner als $\norm{\cdot}_1$, denn
\begin{align*}
\norm{f}_1 = \int\limits_0^1 \abs{f(x)}\dx \le \norm{f}_\infty.
\end{align*}
\item $\norm{\cdot}_1$ ist nicht feiner als $\norm{\cdot}_\infty$. Zeige dazu,
\begin{align*}
\forall c > 0 \exists f\in E : \norm{f}_\infty > c\norm{f}_1. 
\end{align*}
Betrachte dazu die Funktionenfolge $(f_n)$
\begin{align*}
\Rightarrow \norm{f_n}_\infty = 1,\quad \norm{f_n}_1 =
\frac{1}{n+1}.
\end{align*}
\item $(f_n)$ konvergiert bezüglich $\norm{\cdot}_1$ aber nicht bezüglich
$\norm{\cdot}_\infty$.

$f(x) = 0$ ist die Grenzfunktion bezüglich $\norm{\cdot}_1$, denn
\begin{align*}
\norm{f_n-f} = \frac{1}{n+1}\to 0.
\end{align*}
Zeige nun, dass $f_n$ bezüglich $\norm{\cdot}_\infty$ nicht Cauchy ist, d.h.
\begin{align*}
\exists \ep > 0 \forall N\in\N \exists n,m > N : \norm{f_n-f_m}_\infty > \ep.
\end{align*}
Setze $\ep =\frac{1}{4}$. Zu beliebigem $n>N$ wähle $x_0\in[0,1)$ mit $f_n(x_0)
= x_0^n > \frac{1}{2}$. Wähle $m>n$ mit $f_m(x_0) = x_0^m < \frac{1}{4}$, dann
gilt $\norm{f_n-f_m} > \frac{1}{4}$.\bsphere
\end{bspenum}
\end{bsp}

\begin{prop}
\label{prop:1.11}
Ist $L$ endlichdimensional, dann sind alle Normen auf $L$ äquivalent.\fishhere
\end{prop}
\begin{proof}
\begin{proofenum}
  \item Sei $\BB=\setd{b_1,\ldots,b_n}$ Basis von $L$. Setze
\begin{align*}
\norm{x}_1 = \norm{\sum\limits_{j=1}^n x_j b_j}_1 := \sum \abs{x_j}
\end{align*}
als Vergleichsnorm.
\item Sei $\norm{\cdot}$ Norm auf $L$. Dann ist $\norm{\cdot}_1$ feiner
\begin{align*}
\norm{x} = \norm{\sum\limits_{j=1}^n x_j b_j} \le
\sum\limits_{j=1}^n \abs{x_j} \norm{b_j} \le
\underbrace{\max_j \norm{b_j}}_{>0} \cdot \sum\limits_{j=1}^n \abs{x_j}.
\end{align*}
\item Sei $\norm{\cdot}$ Norm auf $L$. Dann ist $\norm{\cdot}$ feiner als
$\norm{\cdot}_1$.
Die Norm $\norm{\cdot}$ ist als Abbildung
\begin{align*}
f : (L,\norm{\cdot}_1) \to \R, \quad x\mapsto \norm{x} =
\norm{\sum\limits_{j=1}^n x_j b_j}
\end{align*}
stetig, denn für $x,y\in L$ gilt,
\begin{align*}
\abs{f(x)-f(y)} &= 
\abs{\norm{\sum\limits_{j=1}^n x_j b_j}-\norm{\sum\limits_{j=1}^n y_j b_j}}
\le \norm{\sum\limits_{j=1}^n (x_j-y_j)b_j}\\ & \le
\sum\limits_{j=1}^n \abs{x_j-y_j}\norm{b_j} \le
\max_j \norm{b_j}\cdot\sum\limits_{j=1}^n \abs{x_j-y_j} \\ &= c \norm{x-y}_1
\end{align*}
Nun ist die Menge $S=\setdef{x\in L}{\norm{x}_1 = 1}$ beschränkt und
abgeschlossen bezüglich der $\norm{\cdot}_1$ Norm, also kompakt. $f$
nimmt daher auf $S$ sein Minimum an, d.h.
\begin{align*}
\exists \xi\in S \forall x\in S : f(x) \ge f(\xi). 
\end{align*}
Da $\xi\in S$ gilt $\xi\neq 0$ und damit folgt für $0\neq x\in L$,
\begin{align*}
\norm{x} = \norm{\sum\limits_{j=1}^n x_j b_j} = \left(\sum\limits_{l=1}^n
\abs{x_l}\right)
\norm{\sum\limits_{j=1}^n \frac{x_j}{\sum\limits_{l=1}^n \abs{x_l}}b_j}
\ge \norm{x}_1 \underbrace{f(\xi)}_{> 0}.
\end{align*}
\item Seien nun $\norm{\cdot}$ und $\norm{\cdot}^\sim$ zwei Normen auf $L$,
dann gilt
\begin{align*}
\exists c_1,c_2,c_3,c_4 > 0 :
\norm{x} \le c_1 \norm{x}_1 \le c_2 \norm{x}^\sim \le c_3 \norm{x}_1 \le c_4
\norm{x}.\qedhere
\end{align*}
\end{proofenum}
\end{proof}

\section{Topologische Grundbegriffe}

Es folgen einige topologische Begriffe für normierte Räume. Nicht alle Aussagen
lassen sich auf einen metrischen bzw. allgemeinen topologischen Raum übertragen.

\begin{defn}
\label{defn:1.12}
Sei $(E,\norm{\cdot})$ ein normierter Raum, $X\subseteq E$.
\begin{defnenum}
  \item $K_r(x) := \setdef{y\in E}{\norm{x-y} < r}$
  \emph{offene Kugel} mit Radius $r (>0)$.
  \item $x\in E$ heißt \emph{innerer Punkt}\index{Innerer Punkt} von $X$,
  falls
\begin{align*}
\exists r > 0 : K_r(x) \subseteq X.
\end{align*}
\item $X$ heißt \emph{offen}\index{Menge!offen}, falls
\begin{align*}
\forall x\in X : x\text{ ist innerer Punkt von $X$}.
\end{align*}
\item $x\in E$ heißt \emph{Häufungspunkt}\index{Häufungspunkt}\item  von $X$, falls
\begin{align*}
\forall r>0 : K_r(x)\cap X\setminus \setd{x} \neq \emptyset
\end{align*}
oder äquivalent: Es existiert eine Folge $(x_n)$ in $X$ mit $x_n\to x$ und
$x_n\neq x$.
\item $X$ heißt \emph{abgeschlossen}\index{Menge!abgeschlossen}, wenn $X$ alle
seine Häufungspunkte enthält.
\item $\overline{X}:=X\cup\setd{\text{Häufungspunkte von }X}$ heißt
\emph{Abschluss} von $X$. \\
$\overline{X}$ ist abgeschlossen und die kleinste abgeschlossene Menge, die $X$
enthält.
\item $Y\subseteq X\subseteq E$ heißt \emph{dicht}\index{Menge!dicht} in $X$,
falls $\overline{Y}\supseteq X$.
\item $X$ heißt \emph{kompakt}\index{Menge!kompakt}, wenn jede offene
Überdeckung $(\OO_\alpha)$, (d.h. ein System offener Mengen $O_\alpha$, $\alpha\in\AA$ Indexmenge, mit
 $X\subseteq \bigcup_\alpha O_\alpha$) eine endliche Teilmenge 
 $\setd{O_{\alpha_1},\ldots,O_{\alpha_n}}$ besitzt, die bereits $X$ überdeckt.\\
 Oder hier äquivalent: Wenn jede Folge $(x_n)$ in $X$ eine in $X$ konvergente
 Teilfolge enthält.
 \item Seien $(E_1,\norm{\cdot}_1)$, $(E_2,\norm{\cdot}_2)$ normierte Räume.
 Eine Abbildung
 \begin{align*}
 f: E_1\to E_2,
 \end{align*}
heißt \emph{stetig}\index{Abbildung!stetig} in $x_0\in E$, falls
\begin{align*}
\forall \ep > 0 \exists \delta > 0 \forall x \in E_1: \norm{x-x_0}_1 < \delta
\Rightarrow \norm{f(x)-f(x_0)}_2 < \ep.
\end{align*}
Oder äquivalent: Wenn für jede Folge $(x_n)$ in $E_1$ gilt
\begin{align*}
x_n \to x_0 \Rightarrow f(x_n)\to f(x_0).
\end{align*}
$f$ heißt \emph{stetig}, wenn $f$ in jedem $x\in E_1$ stetig
ist.\fishhere
\end{defnenum}
\end{defn}

\section{Neue Räume aus alten}

\begin{defn}[Direkte Summe]
\label{defn:1.13}
Seien $(E_1,\norm{\cdot}_1),(E_2,\norm{\cdot}_2)$ normierte Räume, so ist
\begin{align*}
&E_1\oplus E_2 := \setdef{(x,y)}{x\in E_1,y\in E_2},\\
&\norm{(x,y)} := \norm{x}_1 + \norm{x}_2
\end{align*}
ein normierter Raum\index{Vektorraum!Direkte Summe}. Sind $E_1,E_2$ Banachräume,
so ist auch $E_1\oplus E_2$ Banachraum.\fishhere
\end{defn}

\begin{defn}[Quotientenraum]
\label{defn:1.14}
Sei $F$ ein linearer Unterraum von $E$. Setze $x\sim y\Leftrightarrow x-y\in
F$, so ist $\sim$ eine Äquivalenzrelation.
\begin{align*}
E/F  := \setdef{[x]}{x\in E}
\end{align*}
wird zum Quotientenraum\index{Vektorraum!Quotientenraum} mit Addition
$[x]+[y]:=[x+y]$ und skalarer Multiplikation $\alpha[x]=[\alpha x]$.\fishhere
\end{defn}

\begin{prop}
\label{prop:1.15}
Sei $(E,\norm{\cdot})$ ein normierter Raum, $F$ abgeschlossener linearer
Unterraum von $E$ und
\begin{align*}
\norm{[x]}_0 := \inf\setdef{\norm{x+z}}{z\in F} = \inf\setdef{\norm{y}}{y\in
[x]}.
\end{align*}
Dann gelten
\begin{propenum}
  \item $\norm{\cdot}_0$ ist Norm auf $E/F$.
  \item Ist $E$ Banachraum, so ist auch $E/F$ Banachraum.\fishhere
\end{propenum}
\end{prop}
\begin{proof}
\begin{proofenum}
  \item $\norm{[x]}_0$ ist unabhängig vom gewählten Vertreter von $x$,
  denn
\begin{align*}
\norm{[x]}_0 :=  \inf\setdef{\norm{y}}{y\in
[x]]}.
\end{align*}
$\norm{[x]}_0$ ist Norm. (N1) ist offensichtlich erfüllt. (N2): Sei
$\norm{[x]}_0 = 0$, dann existiert eine Folge $(z_n)$ in $F$ mit
$\norm{x+z_n}\to 0$, d.h. $z_n\to-x\in F$ also ist $[x]=[0]$.
(N3): Zunächst ist $\norm{0\cdot[x]}_0 = \norm{[0\cdot x]}_0 = 0$. Für
$\alpha\neq0$ gilt weiterhin,
\begin{align*}
\norm{\alpha[x]}_0 &= \norm{[\alpha x]}_0 = \inf\setdef{\norm{\alpha x +
z}}{z\in F} \\ &= \abs{\alpha}\setdef{\norm{x+\frac{1}{\alpha}z}}{z\in F} =
\abs{\alpha}\norm{[x]}_0.
\end{align*}
(N4): Sei $\ep>0$, dann existieren $z_1,z_2\in F$ mit
\begin{align*}
 \norm{x+z_1} \le \norm{[x]}_0+\ep,\quad
\norm{y+z_2} \le \norm{[y]}_0 + \ep.
\end{align*}
Somit erhalten wir
\begin{align*}
\norm{[x]+[y]}_0 &= \norm{[x+y]}_0
= \inf\setdef{\norm{x+y+z}}{z\in F}\\
&\le \norm{x+z_1+y+z_2}
\le \norm{x+z_1} + \norm{y+z_2}
\\ &\le \norm{[x]}_0 + \norm{[y]}_0 + 2\ep
\end{align*}
Da $\ep$ beliebig war, folgt die Dreiecksungleichung.
\item Sei $[x_n]$ Folge in $E/F$ mit $\sum\limits_{j=1}^\infty \norm{[x_n]}_0
<\infty$. Zu $n\in\N$ wähle jeweils einen Vertreter $x_n\in E$, so dass
\begin{align*}
&\norm{x_n} \le \norm{[x_n]_n}_0 + \frac{1}{2^n},\\
\Rightarrow & \sum\limits_{n=1}^\infty \norm{x_n} < \infty. 
\end{align*} 
$E$ ist Banachraum, daher ist nach Lemma \ref{lem:1.7} $y=\sum\limits_{j=1}^n
x_n$ konvergent. Sei nun $N\in\N$, so gilt
\begin{align*}
\norm{\sum\limits_{j=1}^N [x_n] - [y]}_0 = \norm{\nrm{\sum\limits_{j=1}^N x_n
- y}}_0 \le \norm{\sum\limits_{j=1}^N x_n-y} \to 0,\quad N\to \infty,
\end{align*}
d.h. $\sum\limits_{j=1}^N [x_n] = [y]$ bezüglich $\norm{\cdot}_0$.
Ebenfalls mit Lemma \ref{lem:1.7} folgt, $E/F$ ist vollständig.\qedhere
\end{proofenum}
\end{proof}

\begin{prop}[Vervollständigung]
\index{Vervollständigung}
\label{prop:1.16}
Sei $(E,\norm{\cdot})$ ein normierter Raum. Dann existiert ein Banachraum
$(F,\norm{\cdot}_F)$, so dass $E$ mit einem dichten linearen Unterraum
identifiziert werden kann. D.h. es existiert eine lineare, normerhaltende
Abbildung,
\begin{align*}
j: E\to F,
\end{align*}
mit $j(E)$ ist dicht in $F$. $j$ ist dann insbesondere injektiv.\fishhere
\end{prop}
\begin{proof}
\begin{proofenum}
  \item Betrachte dazu den Raum der Cauchyfolgen auf $E$,
\begin{align*}
\hat{F} := \setdef{(x_n)\in E^\N}{(x_n)\text{ ist Cauchy}}.
\end{align*}
Dieser Raum ist linear und lässt sich mit der Halbnorm
\begin{align*}
\norm{(x_n)}^\land := \lim\limits_{n\to\infty} \norm{x_n}_E
\end{align*}
versehen, denn für jede Cauchyfolge konvergiert die Folge $(\norm{x_n})$.
\item Sei
\begin{align*}
\hat{N} := \setdef{(x_n)\in E^\N}{(x_n)\text{ Nullfolge}}
\end{align*}
der Raum der Nullfolgen. Setze $F=\hat{F}/\hat{N}$ und
\begin{align*}
\norm{[x_n]}_F &:= \inf\setdef{\norm{(x_n)-(z_n)}^\land}{(z_n)\in\hat{N}}\\
&= \inf\setdef{\lim\limits_{n\to\infty}\norm{y_n}_E}{(x_n)-(y_n)\in\hat{N}}.
\end{align*}
Eine äquivalente Formulierung erhalten wir, wenn wir $\hat{x}=(x_n)$ setzen,
\begin{align*}
\norm{[\hat{x}]}_F :=
\inf\setdef{\norm{\hat{y}}^\land}{\hat{x}-\hat{y}\in\hat{N}} =
\norm{\hat{x}}^\land.
\end{align*}
\textit{$\norm{\cdot}_F$ ist Norm}. (N1),(N3),(N4) sind klar, da
$\norm{\cdot}^\land$ Halbnorm. (N2): Des Weiteren gilt $\norm{[\hat{x}]}_F = 0\Leftrightarrow
\norm{\hat{x}}^\land = 0$, d.h. $\hat{x}\in\hat{N}$ und daher $[\hat{x}] = [0]$.
\item
Sei $j: E\to F, x\mapsto (x,x,x,\ldots)$. $j$ ist offensichtlich linear und
\begin{align*}
\norm{j(x)}_F = \lim\limits_{n\to\infty} \norm{x}_E = \norm{x}_E.
\end{align*}
\item \textit{$j(E)$ ist dicht in $E$}. Sei $[\hat{x}]\in F=\hat{F}/\hat{N}$,
dann ist $\hat{x}=(x_n)$ Cauchy in $E$. Nun ist
\begin{align*}
\norm{j(x_m)-[\hat{x}]}_F = \norm{[\hat{y}_m] - [\hat{x}]}_F
= \norm{\nrm{\hat{y}_m-\hat{x}}}_F = \norm{(x_m-x_n)}^{\land}.
\end{align*}
Es gilt
\begin{align*}
\lim\limits_{n\to\infty} \norm{x_m-x_n} < \ep,
\end{align*}
für $m> N_\ep$, also liegt $j(E)$ dicht in $F$.
\item \textit{$F$ ist vollständig}. Sei $([\hat{x}_n])_{n\in\N}$ Cauchyfolge in
$F$, so ist $\hat{x}_n=(y_k^{(n)})_{k\in\N}$ Cauchyfolge in $E$ und da
$\norm{[\hat{x}]}_F = \norm{\hat{x}}^\land$, ist $\hat{x}_n$ Cauchyfolge in
$\hat{F}$. Somit gilt,
\begin{align*}
&\forall n\in\N \forall \ep > 0 \exists K_{n,\ep} \forall k,l \ge K_{n,\ep}
\norm{y_k^{(n)}-y_l^{(n)}}_E < \ep,\tag{*}\\
&\forall \ep > 0 \exists N_\ep \in\N \forall n,m>N_\ep :
\norm{\hat{x}_n-\hat{x}_m}^\land <\ep\tag{**}.
\end{align*}
Setze nun $y_k:=y_{K_{k,1/k}}^{(k)}$, $y:=(y_k)_{k\in\N}$, so ist $(y_k)$
Cauchyfolge in $E$, denn
\begin{align*}
\norm{y_k-y_l}_E &= \norm{y_{K_{k,1/k}}^{(k)}-y_{K_{l,1/l}}^{(l)}}_E\\
&\le \underbrace{\norm{y_{K_{k,1/k}}^{(k)}-y_j^{(k)}}_E}_{(1)} +
\underbrace{\norm{y_j^{(k)}-y_j^{(l)}}_E}_{(2)} +
\underbrace{\norm{y_j^{(l)}-y_{K_{l,1/l}}^{(l)}}_E}_{(3)}.
\end{align*}
Sei $\ep > 0$, wähle $j$ so, dass $j> \max\setd{K_{k,1/k},K_{l,1/l},N_\ep}$, so
sind nach (*) $(1)<\frac{1}{k}$ und $(3)<\frac{1}{l}$. Für eventuell noch
größeres $j$ gilt außerdem
\begin{align*}
(2) = \norm{y_j^{(k)}-y_j^{(l)}}_E \le \lim\limits_{j\to\infty}
\norm{y_j^{(k)}-y_j^{(l)}}_E + \ep = \norm{\hat{x}_k - \hat{x}_l}^\land + \ep.
\end{align*}
Wählen wir nun $k,l>N_\ep$ so erhalten wir nach (**) 
\begin{align*}
\norm{y_k-y_l}_E \le \frac{1}{k} + 2\ep + \frac{1}{l},\qquad \text{für }k,l >
N_\ep.
\end{align*}
also ist $(y_k)$ Cauchy in $E$ und daher $[\hat{y}]\in F$.

Um die Konvergenz zu zeigen, betrachte
\begin{align*}
\norm{\hat{x}_n - \hat{y}}^\sim &= \lim\limits_{k\to\infty}
\norm{y_k^{(n)}-y_{K_{k,1/k}}^{(k)}}_E\\
&\le\lim\limits_{k\to\infty}
\underbrace{\norm{y_k^{(n)}-y_{K_{k,1/k}}^{(n)}}_E}_{(*): < \frac{1}{k}} +
\underbrace{\norm{y_{K_{k,1/k}}^{(n)}- y_{K_{k,1/k}}^{(k)}}_E}_{(**): <
\frac{1}{k}} < \frac{2}{k}
\end{align*}
für $k,l$ hinreichend groß. Also $[\hat{x}_n]\to [\hat{y}]$ in $F$.\qedhere
\end{proofenum}
\end{proof}

\begin{figure}[!htpb]
\centering
\begin{pspicture}(0,-2.64)(6.6,2.68)
\pscircle(0.62,1.48){0.58}
\pscircle(4.26,0.92){1.28}
\pscircle(1.62,-1.36){1.28}
\psbezier[linecolor=darkblue]{->}(1.18,1.94)(1.66,2.4)(2.7,2.4)(3.26,2.0)

\rput(2.21,2.485){\color{gdarkgray}Bilde Cauchyfolgen}
\psbezier[linecolor=darkblue]{->}(0.46,1.14)(0.96,0.54)(0.54,-0.1)(0.96,-0.5)

\rput(0.56,0.505){\color{gdarkgray}$j$}
\psline[linecolor=yellow](3.26,0.14)(5.2,1.76)

\rput(4.63,2.405){\color{gdarkgray}$\hat{F}$}
\rput(4.61,0.885){\color{gdarkgray}$\hat{N}$}
\psbezier[linecolor=darkblue]{->}(4.06,-0.42)(3.8,-1.16)(3.44,-1.3)(2.98,-1.34)

\psline[linecolor=yellow](0.38,-1.14)(1.38,-0.12)
\psline[linecolor=yellow](0.36,-1.52)(1.8,-0.12)
\psline[linecolor=yellow](0.44,-1.82)(2.12,-0.2)
\psline[linecolor=yellow](0.58,-2.1)(2.38,-0.36)
\psline[linecolor=yellow](0.76,-2.3)(2.58,-0.54)
\psline[linecolor=yellow](1.02,-2.48)(2.76,-0.8)
\psline[linecolor=yellow](1.28,-2.58)(2.86,-1.08)
\psline[linecolor=yellow](1.66,-2.62)(2.88,-1.48)

\rput(4.47,-0.575){\color{gdarkgray}bilde}
\rput(5.2,-1.015){\color{gdarkgray}Äquivalenzklassen}
\psdots[dotsize=0.12](0.4,1.2)

\rput(0.11,2.105){\color{gdarkgray}$E$}
\rput(2.64,-2.495){\color{gdarkgray}$F$}
\rput(1.66,0.14){\color{gdarkgray}$\hat{x}+\hat{N}$}
\end{pspicture} 
\caption{Zur Konstruktion des Raumes $F=\hat{F}/\hat{N}$.}
\end{figure}


\begin{bem}
\label{bem:1.17}
Wir werden im nächsten Kapitel sehen, dass diese
Vervollständigung bis auf Isomorphie auch eindeutig ist.\maphere
\end{bem}

\begin{bsp}
\label{bsp:1.18}
\begin{bspenum}
  \item $E=l_\text{abb}:=\setdef{(x_n)\in \C^\N}{\exists N\in\N \forall n> N :
  x_n =0}$. Vervollständigung bezüglich $\norm{\cdot}_p$ ist $l_p$. $(1\le p <
  \infty)$. Für $p=\infty$ ist die Vervollstängigung bezüglich
  $\norm{\cdot}_\infty$ der Raum der Nullfolgen.
  \item $P([a,b]):=\text{Menge der komplexen Polynome als Funktion
  }[a,b]\to\C$. Die Vervollständigung bezüglich $\norm{\cdot}_\infty$ ist
  $C([a,b]\to\C)$ (siehe Weierstraßscher Approximationssatz).
  \item Sei $O\subseteq\R^n$ offen. Für stetiges $f:O\to\C$ setze
\begin{align*}
\norm{f}_p &:= \left(\int\limits_O \abs{f(x)}^p \right)^{1/p},\\
E&:=C_0^\infty(O\to C) \\ &:= \setdef{f\in C^\infty(O\to\C)}{\supp f
\text{ kompakt und }\supp f \subseteq O}.
\end{align*}
Die Vervollständigung ist der $\LL^p(O)$.
\item Sei $O\subseteq \R^n$ offen, $1\le p <\infty$ und $k\in\N$.
\begin{align*}
&E:= \setdef{f\in C^k(\overline{O}\to C)}{\norm{f}_{k,p}< \infty},\\
&\norm{f}_{k,p} := \sum\limits_{\atop{(\alpha_1,\ldots,\alpha_k)\in
\N_0}{\abs{\alpha_1}+\ldots+\abs{\alpha_k}\le k}}
\norm{\frac{\partial^{\alpha_1}\cdots \partial^{\alpha_k}}{\partial
x_1^{\alpha_1}\cdots \partial x_n^{\alpha_n}} f}_p. 
\end{align*}
Die Vervollständigung ist der Sobolevraum $W^{k,p}(O)$. Die Elemente dieses
Raumes sind in einem verallgemeinerten Sinne differenzierbar (schwache
Ableitung, Distributionenableitung, starke Ableitung).\\
Sobolevsche Einbettung:
Für $k>\frac{n}{2}$ gilt $W^{k,2}(O)\subseteq C(\overline{O}\to\C)$, falls
$\partial O$ ``genügend gut''.\bsphere
\end{bspenum}
\end{bsp}



\clearpage
\chapter{Lineare Abbildungen}

\begin{defn}
\label{defn:2.1}
Seien $(E,\norm{\cdot}_E)$, $(F,\norm{\cdot}_F)$ normierte Räume und $T: E\to
F$ linear.
\begin{defnenum}
  \item $T$ heißt auch \emph{linearer
  Operator}\index{Operator}\index{Abbildung!linear}. Falls
  $(F,\norm{\cdot}_F)=(\K,\abs{\cdot}_\K)$, heißt $T$ auch \emph{lineares Funktional}\index{Funktional}.
  \item $T$ heißt \emph{beschränkt}\index{Abbildung!beschränkt}, falls
\begin{align*}
\exists c > 0 \forall x\in E : \norm{x}_E = 1 \Rightarrow \norm{Tx}_F \le c.
\end{align*}
\item Sei $T$ beschränkt, dann heißt
\begin{align*}
\norm{T} &:= \sup\setdef{\norm{Tx}_F}{\norm{x}_E =1}\\
&= \inf\setdef{c > 0}{\forall x\in E : \norm{x}_E =1 \Rightarrow \norm{Tx}_F \le
c}\\
&= \sup\setdef{\norm{Tx}_F}{\norm{x}_E =1}
= \sup\setdef{\frac{\norm{Tx}_F}{\norm{x}_e}}{x\in E\setminus\setd{0}}\\
&= \inf\setdef{c > 0}{\forall x\in E : \norm{Tx}_F \le c\norm{x}_E}
\end{align*}
\emph{Operatornorm} von $T$. Insbesondere gilt
\begin{align*}
\forall x\in E : \norm{Tx}_F \le \norm{T}\norm{x}_E.
\end{align*}
\item Ist $T$ bijektiv und sind sowohl $T$ also auch $T^{-1}$ beschränkt, so
heißt $T$ \emph{Isomorphie}\index{Abbildung!Isomorphie} oder
\emph{Isomorphismus}. $(E,\norm{\cdot}_E)$ und $(F,\norm{\cdot}_F)$ heißen
dann \emph{isomorph}\index{Vektorraum!isomorph}.
\item Falls $\forall x\in E : \norm{Tx}_F =\norm{x}_E$, so heißt $T$
\emph{Isometrie}\index{Abbildung!Isometrie}. Jede surjektive Isometrie ist ein
Isomorphismus.\fishhere
\end{defnenum}
\end{defn}

\begin{prop}
\label{prop:2.2}
Für $T: E\to F$ linear sind äquivalent
\begin{equivenum}
  \item\label{prop:2.2:1} $\exists x_0\in E : T $ ist stetig in $x_0$.
  \item\label{prop:2.2:2} $\forall x_0\in E : T$ ist stetig in $x_0$. Also $T$
  ist stetig.
  \item\label{prop:2.2:3} $T$ ist beschränkt.
  \item\label{prop:2.2:4} $T$ bildet Cauchyfolgen in $E$ auf Cauchyfolgen in
  $F$ ab.\fishhere
\end{equivenum}
\end{prop}
\begin{proof}
\ref{prop:2.2:1}$\Rightarrow$\ref{prop:2.2:2}: Sei $y\in E$ und $(y_n)$ Folge
in $E$ mit $y_n\to y$,
\begin{align*}
\Rightarrow T(y_n) = T(\underbrace{y_n -y+x_0}_{\to x_0})+ T(y-x_0) \to T(x_0)
+ T(y-x_0)  = T(y).
\end{align*}
\ref{prop:2.2:2}$\Rightarrow$\ref{prop:2.2:3}: Kontraposition: Sei
$T$ nicht beschränkt, d.h.
\begin{align*}
\forall c > 0 \exists x\in E : \left( \norm{Tx}_F > c \land \norm{x}_E =1
\right).
\end{align*}
Wähle $(x_n)$ mit $\norm{x_n}_E =1$ und $\norm{Tx_n}_F > n$. Setze $y_n:=
\frac{1}{n}x_n$, dann geht $y_n\to 0$ aber $Ty_n > 1$ also $\neg(Ty_n\to 0)$ und
daher ist $T$ nicht stetig in $x_0=0$.\\
\ref{prop:2.2:3}$\Rightarrow$\ref{prop:2.2:4}:
Sei $(x_n)$ Cauchyfolge in $E$, so gilt
\begin{align*}
\norm{Tx_n - Tx_m}_F \le \norm{T}\norm{x_n-x_m}_E \to 0,\qquad n,m\to \infty.
\end{align*}
Also ist $(Tx_n)$ Cauchyfolge.\\
\ref{prop:2.2:4}$\Rightarrow$\ref{prop:2.2:1}:
Sei $(x_n)$ Folge mit $x_n\to 0$. Setze $y_n := (x_1,0,x_2,0,\ldots)$, dann
gilt $y_n\to 0$ und $(y_n)$ ist Cauchy. Nach \ref{prop:2.2:4} ist $(Ty_n)$
ebenfalls Cauchy und besitzt besitzt eine konvergente Teilfolge $(Ty_{2n})\to
0$. D.h. $(Ty_n)$ ist konvergent gegen $\lim\limits_{n\to\infty} Ty_{2n} = 0$.
Dann ist aber auch $(Tx_n)$ als Teilfolge konvergent mit demselben
Grenzwert.\qedhere
\end{proof}

\begin{prop}
\label{prop:2.3}
\begin{propenum}
  \item Seien $E,F$ isomorph und $E$ Banachraum, dann ist auch $F$ Banachraum.
  \item Falls $\dim E = n < \infty$, so sind $E$ und $\K^n$ isomorph und $E$
  Banachraum.\fishhere
\end{propenum}
\end{prop}
\begin{proof}
\begin{proofenum}
  \item Sei $T: E\to F$ Isomorphie und $(y_n)$ Cauchy in $F$. $T^{-1}$ ist
  beschränkt, also ist $(T^{-1}y_n)$ Cauchy in $E$. Da $E$ vollständig,
  konvergiert $T^{-1}y_n$ gegen $x\in E$. Nun ist $y_n := T(T^{-1}y_n)$ und
  daher konvergent gegen $Tx\in F$.
  \item Sei $\setd{b_1,\ldots,b_n}$ Basis von $E$. Setze
\begin{align*}
T: E\to\K^n,\; x=\sum\limits_{j=1}^n x_j b_j \mapsto
\begin{pmatrix}x_1\\\vdots\\x_n\end{pmatrix}.
\end{align*}
$T$ ist linear und bijektiv, denn
\begin{align*}
T^{-1}: \begin{pmatrix}x_1\\\vdots\\x_n\end{pmatrix} \mapsto
\sum\limits_{j=1}^n x_j b_j.
\end{align*}
Wähle als Norm für $\K^n$:
\begin{align*}
\norm{\begin{pmatrix}x_1\\\vdots\\x_n\end{pmatrix}} := \sum\limits_{j=1}^n
\abs{x_j}.
\end{align*}
Da auf $E$ alle Normen äquivalent sind, gilt
\begin{align*}
\norm{Tx} = \sum\limits_{j=1}^n \abs{x_j} = \norm{x}_1 
\begin{cases}
\le c_1 \norm{x}_E,\\
\ge c_2 \norm{x}_E.
\end{cases}
\end{align*}
Da $\norm{Tx}\le c_1 \norm{x}_E$ ist $T$ beschränkt. Setze $Tx:= y$, so ist $x=
T^{-1}y$ und es gilt
\begin{align*}
\norm{y} \ge c_2 \norm{T^{-1}y}_E \Rightarrow \norm{T^{-1}y}_E \le
\frac{1}{c_2}\norm{y}.\qedhere
\end{align*}
\end{proofenum}
\end{proof}

\begin{bsp}
\label{bsp:2.4}
\begin{bspenum}
  \item $T: (C([0,1]\to\C),\norm{\cdot}) \to \C : f\mapsto f(1)$.
\begin{propenum}
  \item $\norm{\cdot} = \norm{\cdot}_\infty$: $\abs{Tf} = \abs{Tf(1)} \le
  \norm{f}_\infty \Rightarrow \norm{T} \le 1$, also ist $T$ beschränkt. Für
  $f=1$ gilt weiterhin $\norm{T} \ge 1$, d.h. $\norm{T}=1$.
  \item $\norm{\cdot} = \norm{\cdot}_1$: Setze $f_n(x) = x^n$,
\begin{align*}
\Rightarrow \begin{cases} Tf_n = 1,\\ \norm{f_n}_1 = \frac{1}{n+1}\to
0,\end{cases}
\end{align*}
d.h. $T$ ist nicht beschränkt.
\end{propenum}
\item \textit{Multiplikationsoperator}. $E:=(C([0,1]\to\C),\norm{\cdot}_p)$,
$g\in E$,
\begin{align*}
&T: E\to E,\quad f\mapsto g\cdot f,\\
&\norm{Tf}_p^p = \int\limits_0^1 \abs{g\cdot f}^p\dmu \le \norm{g}_\infty^p
\norm{f}_p^p\\
\Rightarrow & \norm{Tf}_p \le \norm{g}_\infty\norm{f}_p.
\end{align*}
D.h. $T$ ist beschränkt und $\norm{T}\le \norm{g}_\infty$.

\item \textit{Differentialoperator}. $T:(C^1([0,1]\to\C),\norm{\cdot}) \to
(C([0,1]\to \C),\norm{\cdot}_\infty)$
\begin{propenum}
  \item $\norm{\cdot}=\norm{\cdot}_\infty$: $T$ ist unstetig. Um dies
  einzusehen, betrachte $f_n(x)=x^n$, so ist $\norm{f}_\infty = 1$ aber
  $\norm{Tf_n}_\infty = \norm{n\cdot x^{n-1}}_\infty = n$.
  \item $\norm{f}:=\norm{f}_\infty+\norm{f'}_\infty$: Hier ist 
 $T$ beschränkt, denn $\norm{Tf}_\infty = \norm{f'}_\infty \le \norm{f}$.
\end{propenum}
\item $K=(K_{ij})_{1\le i,j\le n}$, $T:\C^n\to \C^n,\; x\mapsto Kx$.
\begin{propenum}
  \item $\norm{x}=\norm{x}_2 = \sqrt{\sum\limits_{j=1}^n \abs{x_j}^2}$: Für $K$
  normal ($KK^* = K^*K$) existiert eine ONB aus Eigenvektoren. Die Operatornorm
  heißt hier \emph{Spektralnorm}
\begin{align*}
\norm{K} = \max\setdef{\abs{\lambda}}{\lambda\text{ ist
  Eigenwert von }K}.
\end{align*}
\item $\norm{x}=\norm{x}_\infty = \max_j \abs{x_j}$: Die Operatornorm heißt
hier \emph{Zeilensummennorm},
\begin{align*}
&\abs{(Tx)_i} = \abs{\sum\limits_{j=1}^n K_{ij}x_j} \le \sum\limits_{j=1}^n
\abs{K_{ij}}\abs{x_j} \le \norm{x}_\infty \sum\limits_{j=1}^\infty
\abs{K_{ij}},\\
\Rightarrow & \norm{T} \le \max\limits_{1\le i\le n}
\sum\limits_{j=1}^n\abs{K_{ij}}.
\end{align*}
\end{propenum}
\item $K:\N\times\N \to \C$ mit $\sum\limits_{j=1}^\infty \abs{K(i,j)} \le c$
für alle $i\in\N$. ``Unendliche Matrix''
\begin{align*}
T: l^\infty \to l^\infty,\; (x_n)\mapsto (y_n) = \left(\sum\limits_{j=1}^\infty
K(n,j)x_j \right)_{n\in\N}.
\end{align*}
$y_n$ ist wohldefiniert, denn
\begin{align*}
\abs{\sum\limits_{j=1}^\infty K(n,j)x_j } \le \sup\limits_{j\in\N} \abs{x_j}
\sum\limits_{j=1}^\infty \abs{K(n,j)} \le c \norm{(x_n)}_\infty.
\end{align*}
Weiterhin gilt,
\begin{align*}
\norm{T(x_n)}_\infty = \sup\limits_{n\in\N} \abs{y_n} \le c\norm{(x_n)}_\infty <
\infty \Rightarrow (y_n)\in l^\infty,
\end{align*}
sowie $\norm{T}\le c$.
\item $(E,\norm{\cdot}) = (C([0,1]\to\C),\norm{\cdot}_\infty)$,
\begin{align*}
Tf = \int\limits_0^1 K(\cdot,y)f(y)\dy,
\end{align*}
wobei $K\in C([0,1]\times[0,1]\to\C)$, d.h. $T: E\to E$. Nun ist
\begin{align*}
\norm{Tf(x)} &= \max\limits_{0\le x\le 1} \abs{\int\limits_0^1 K(x,y)f(y)\dy}
\\ 
&\le \max\limits_{0\le x\le 1}\int\limits_0^1  \abs{K(x,y)f(y)} \dy \le
\norm{f}_\infty \max\limits_{0\le x\le 1} \int\limits_{0}^1 \abs{K(x,y)}\dy.
\end{align*}
Somit ist $\norm{T} \le \max\limits_{0\le x\le 1} \int\limits_0^1
\abs{K(x,y)}\dy$.\bsphere
\end{bspenum}
\end{bsp}

\begin{prop}[Fortsetzungssatz]
\label{prop:2.5}
Sei $(E,\norm{\cdot})$ normierter Raum, $\tilde{E}\subseteq E$ dicht,
$(B,\norm{\cdot})$ Banachraum. $\tilde{T}: \tilde{E}\to B$ linear und
beschränkt. Dann existiert eine eindeutige lineare beschränkte Fortsetzung,
\begin{align*}
T: E\to B,\qquad T\big|_{\tilde{E}} = \tilde{T},
\end{align*}
und es gilt $\norm{T} = \norm{\tilde{T}}$.\fishhere
\end{prop}
\begin{proof}
\begin{proofenum}
  \item Zu $x\in E$ sei $(x_n)$ Folge in $\tilde{E}$ mit $x_n\to x$. $(x_n)$
  ist Cauchy und daher ist auch $(Tx_n)$ Cauchyfolge. $B$ ist vollständig, d.h.
  $\tilde{T}x_n \to y\in B$. Setze $Tx:= y = \lim\limits_{n\to\infty}
  \tilde{T}x_n$.
  \item \textit{Wohldefiniertheit}. Falls ebenso $x_n'\to x$, so gilt
  $x_n'-x_n\to 0$ und daher $\tilde{T}(x_n'-x_n) \to 0$, d.h.
\begin{align*}
\lim\limits_{n\to\infty} \tilde{T}x_n' = \lim\limits_{n\to\infty} \tilde{T}x_n.
\end{align*}
\item \textit{Linearität}. Offensichtlich.
\item \textit{Einschränkung}. $Tx=\tilde{T}x$ gilt offensichtlich für jedes
$x\in\tilde{E}$. (Betrachte die konstante Folge $(x_n)=(x,x,x,\ldots)$).
\item \textit{Beschränktheit}.
\begin{align*}
\norm{Tx}_B = \norm{\lim\limits_{n\to\infty} \tilde{T}x_n}_B =
\lim\limits_{n\to\infty} \norm{\tilde{T}x_n}_B \le 
\norm{\tilde{T}}\lim\limits_{n\to\infty} \norm{x_n}_E =
\norm{\tilde{T}}\norm{x}_E.
\end{align*}
D.h. $\norm{T}\le \norm{\tilde{T}}$. Andererseits ist $T$ Fortsetzung und daher
$\norm{T}\ge \norm{\tilde{T}}$. Somit ist $T$ beschränkt und
$\norm{T}=\norm{\tilde{T}}$.
\item \textit{Eindeutigkeit}. Seien $T,S: E\to B$ beschränkt und linear mit
$T\big|_{\tilde{E}} = S\big|_{\tilde{E}} = \tilde{T}$. Sei $x\in E$,
$(x_n)\in\tilde{E}$ mit $x_n\to x$.
\begin{align*}
\Rightarrow Tx = \lim\limits_{n\to\infty} Tx_n = \lim\limits_{n\to\infty}
\tilde{T}x_n  = \lim\limits_{n\to\infty} Sx_n = Sx.\qedhere
\end{align*}
\end{proofenum}
\end{proof}

\begin{bem}
\label{bem:2.6}
Wenn $\tilde{E}$ nicht dicht liegt, kann man $\tilde{T}$ dennoch auf
$\overline{\tilde{E}}$ fortsetzen. Im Allgemeinen lässt sich $\tilde{T}$ jedoch
nicht weiter fortsetzen. Im Spezialfall $B=\K$ lässt sich jedoch der
Fortsetzungssatz von Hahn-Banach anwenden.\maphere
\end{bem}

\begin{cor}
\label{prop:2.7}
Die Vervollständigung eines normierten Raumes ist bis auf Isomorphie
eindeutig.\fishhere
\end{cor}
\begin{proof}
Seien $F,\tilde{F}$ zwei Vervollständigungen. Sei $T: j(E)\to \tilde{j}(E),
x\mapsto \tilde{j}\circ j^{-1}(x)$.
\begin{proofenum}
  \item \textit{$T$ ist Isometrie}. $T$ ist offensichtlich linear und es gilt
\begin{align*}
\norm{Tx}_{\tilde{F}} = \norm{\tilde{j}\circ j^{-1}(x)}_{\tilde{F}} =
  \norm{j^{-1}(x)}_E = \norm{x}_F.
\end{align*}
Sei $\tilde{T}:F\to\tilde{F}$ die beschränkte Fortsetzung von $T$, dann
ist $\tilde{T}$ ebenfalls Isometrie, denn zu $x\in F$ sei $(x_n)$ in $j(E)$ mit
$x_n\to x$, dann gilt $\tilde{T}x = \lim\limits_{n\to\infty} Tx_n$, also
\begin{align*}
\norm{\tilde{T}x}_{\tilde{F}} = \lim\limits_{n\to\infty}\norm{Tx_n}_{\tilde{F}}
= \lim\limits_{n\to\infty} \norm{x_n}_F = \norm{x}_F.
\end{align*}
Insbesondere ist $\tilde{T}$ injektiv.
\item \textit{$\tilde{T}$ ist surjektiv}. $\tilde{T}$ ist beschränkt und $F$
Banachraum, daher ist $\tilde{T}(F)$ ebenfalls Banachraum. $\tilde{j}(E)$ liegt
dicht in $\tilde{F}$, d.h. $\overline{\tilde{T}(F)} = \tilde{F}$. Aber
$\tilde{T}(F)$ ist Banachraum und daher bereits abgeschlossen, also ist
$\tilde{T}(F) = \tilde{F}$.\qedhere
\end{proofenum}
\end{proof}

\begin{defn}
\label{defn:2.8}
Seien $E,F$ normierte Räume über $\K$. Bezeichne
\begin{align*}
\LL(E\to F) := \setdef{T:E\to F}{T\text{ linear und beschränkt}}.
\end{align*}
$\LL(E):=\LL(E\to E)$\index{Abbildung!$\LL(E\to F)$}.\fishhere
\end{defn}

\begin{prop}
\label{prop:2.9}
\begin{propenum}
  \item $\LL(E\to F)$ ist linearer Raum über $\K$.
  \item Die Operatornorm ist eine Norm auf $\LL(E\to F)$.
  \item Ist $F$ Banachraum, so ist $\LL(E\to F)$ Banachraum.\fishhere
\end{propenum}
\end{prop}
\begin{proof}
\begin{proofenum}
  \item Seien $T,S\in\LL(E\to F)$, $\alpha\in \K$.
\begin{align*}
\norm{(T+S)(x)}_F =\norm{Tx+Sx}_F \le \norm{Tx}_F + \norm{Sx}_F \le
\left(\norm{T}+\norm{S}\right)\norm{x}_E,\\
\norm{(\alpha T)(x)}_F = \norm{\alpha Tx}_F = \abs{\alpha}\norm{Tx}_F \le
\abs{\alpha}\norm{T}\norm{x}_E.
\end{align*}
D.h. $T+S$ und $\alpha\cdot T\in \LL(E\to F)$. Weiterhin gilt,
\begin{align*}
\norm{(\alpha T)} = \sup\limits_{\norm{x}=1} \norm{(\alpha T)x}_F = 
\abs{\alpha}\sup\limits_{\norm{x}=1}\norm{Tx}_F =\abs{\alpha}\norm{T}. 
\end{align*} 
  \item (N1) ist klar. (N2): Sei $0=\norm{T} = \sup\limits_{x\neq 0}
  \frac{\norm{Tx}_F}{\norm{x}_E}$, d.h. $\forall x\neq 0: Tx = 0$ also ist
  $T=0$. (N3),(N4): Siehe 1.).
  \item Sei $F$ Banachraum und $(T_n)$ Cauchy in $\LL(E\to F)$. Für $x\in E$
  gilt somit,
\begin{align*}
\norm{T_nx - T_mx}_F = \norm{(T_n-T_m)x}_F \le \norm{T_n-T_m}\norm{x}\to 0.
\end{align*}
D.h. $(T_nx)$ ist Cauchy in $F$. Setze $Tx := \lim\limits_{n\to\infty} T_nx$
punktweise. Offensichtlich ist $T$ linear. Weiterhin gilt,
\begin{align*}
\norm{Tx}_F = \lim\limits_{n\to\infty} \norm{T_nx}_F \le
\left(\lim\limits_{n\to\infty}\norm{T_n}\right)\norm{x}_E.
\end{align*}
Der Grenzwert existiert, da $T_n$ Cauchy. Setze
$\norm{T}=\lim\limits_{n\to\infty}\norm{T_n}$.

Nun ist
\begin{align*}
\norm{(T_n-T)x}_F = \lim\limits_{m\to\infty}\norm{T_nx-T_mx} \le
\lim\limits_{m\to\infty}\norm{T_n-T_m}\norm{x} <\ep\norm{x}_E,
\end{align*}
für $n$ hinreichend. Also ist auch $\norm{T_n-T} <\ep$ für $n> N_\ep$.\qedhere
\end{proofenum}
\end{proof}

\begin{defn}
\label{defn:2.10}
Der Banachraum $\LL(E\to\K)$ heißt \emph{Dualraum}\index{Vektorraum!Dualraum}
von $E$ und wird meißt mit $E^*$ oder $E'$ bezeichnet.\fishhere
\end{defn}

Die Frage ob $E'\neq(0)$ wird der Satz von Hahn-Banach positiv beantworten.
$E'$ ist vom algebraischen Dualraum zu unterscheiden, denn dessen Abbildungen
sind nicht notwendigerweise beschränkt.

\begin{bem}
\label{bem:2.11}
Wenn $T\in \LL(E\to F)$, $S\in\LL(F\to G)$, so ist $T\circ S\in\LL(E\to G)$.
Auf $\LL(E)$ lässt sich daher ein Produkt definieren $T\cdot S := T\circ
S$.\maphere
\end{bem}

\begin{defn}
\label{defn:2.12}
Ein Banachraum $E$ heißt \emph{Banachalgebra}\index{Banach!-algebra},
falls auf $E$ ein Produkt
\begin{align*}
E\times E\to E
\end{align*}
definiert ist, mit folgenden Eigenschaften:
\begin{defnenum}
  \item $x\cdot(y\cdot z) = (x\cdot y)\cdot z$,
  \item $x\cdot(y+z) = x\cdot y + x\cdot z$,
  \item $(x+y)\cdot z = x\cdot z + y\cdot z$,
  \item $\alpha\cdot(x\cdot y) = (\alpha x)\cdot y = x\cdot(\alpha y)$,
  \item $\norm{x\cdot y}\le \norm{x}\norm{y}$.
\end{defnenum}
Insbesondere ist das Produkt stetig, d.h. $x_n\to x$, $y_n\to y$ $\Rightarrow
x_n\cdot y_n \to x\cdot y$.\fishhere
\end{defn}

\clearpage
\chapter{Bairescher Kategoriensatz}

\begin{defn}
\label{defn:3.1}
Sei $M$ Menge. $d: M\times M\to \R$ heißt
\emph{Metrik}\index{Metrik}, falls für $x,y,z\in M$
\begin{defnenum}
  \item $d(x,y) \ge 0$,\qquad Positivität,
  \item $d(x,x) = 0\Leftrightarrow x = 0$,\qquad Definitheit,
  \item $d(x,y)=d(y,x)$,\qquad Symmetrie,
  \item $d(x,z)\le d(x,y)+d(y,z)$,\qquad Dreiecksungleichung.
\end{defnenum}  
$(M,d)$ bezeichnet man als \emph{metrischen
Raum}\index{Metrischer Raum}.\fishhere
\end{defn}

\begin{bem}
\label{bem:3.2}
\begin{bemenum}
  \item\label{bem:3.2:1} $x_n\to x \Leftrightarrow d(x_n,x)\to 0$. $(x_n)$ ist
  genau dann Cauchy, wenn
\begin{align*}
\forall \ep > 0 \exists N\in\N \forall n,m\ge N : d(x_n,x_m) < \ep.
\end{align*}
$x\in M$ ist genau dann innerer Punkt, falls
\begin{align*}
\exists r > 0 : K_r(x):=\setdef{y\in M}{d(x,y)<r}\subseteq M.
\end{align*}
Häufungspunkte, offene und abgeschlossene Mengen sowie Vollständigkeit sind wie
üblich definiert. Folgen- und Überdeckungskompaktheit sind hier äquivalent.
\item Ist $(E,\norm{\cdot})$ normierter Raum, so wird durch
$d(x,y):=\norm{x-y}_E$ eine Metrik induziert. Die Begriffe in
\ref{bem:3.2:1} sind unabhänig davon, ob eine Norm oder die durch sie
induzierte Metrik zugrundegelegt wird.\maphere
\end{bemenum}
\end{bem}

\begin{defn}
\label{defn:3.3}
Sei $N\subseteq M$, $d$ Metrik auf $M$ und $x\in M$.
\begin{align*}
d(x,N) := \inf\setdef{d(x,y)}{y\in N}.\fishhere
\end{align*}
\index{Abstand}
\end{defn}

\begin{prop}
\label{prop:3.4}
Sei $(M,d)$ metrischer Raum, $N\subseteq M$ abgeschlossen und $x\in M\setminus
N$.
\begin{propenum}
  \item $d(x,N)> 0$.
  \item Falls $N$ kompakt, $\exists y\in N : d(x,N)=d(x,y)$.\fishhere
\end{propenum}
\end{prop}

\begin{proof}
\begin{proofenum}
  \item \textit{Kontraposition}. Angenommen $d(x,N)=0$, dann existiert eine
  Folge $(y_n)$ in $N$, so dass $d(x,y_n)\to 0$, d.h. $y_n\to x$. $N$ ist
  abgeschlossen also ist $x\in N$.
  \item Sei $d(x,N)=c$, so existiert eine Folge $(y_n)$ in $N$, so dass
  $d(x,y_n)\to c$. $N$ ist kompakt, d.h. es existiert eine konvergente
  Teilfolge $y_{n_k}\to y\in N$. Dann ist
  $d(x,y)=\lim\limits_{k\to\infty} d(x,y_{n_k})=c$.\qedhere
\end{proofenum}
\end{proof}

\begin{prop}[Bairescher Kategoriensatz]
\label{prop:3.5}
\index{Satz!Bairescher Kategorien-}
Sei $(M,d)$ vollständiger metrischer Raum, $(A_k)$ Folge abgeschlossener Mengen
in $M$ mit
\begin{align*}
M= \bigcup_{n\in\N} A_n.
\end{align*}
Dann existiert ein $k\in\N$, so dass $A_k^\circ \neq \varnothing$.
($A_k^\circ:=\setd{\text{innere Punkte von }A_k}$)\fishhere
\end{prop}
\begin{proof}
Angenommen $\forall k\in \N : A_k^\circ = \varnothing$. Dann ist $M\setminus
A_1$ offen und nichtleer.
\begin{align*}
\Rightarrow \exists \overline{K_{r_1}(x_1)} \subseteq K_{2r_1}(x_1) \subseteq
M\setminus A_1,\qquad && r_1 < 1.
\end{align*}
Dann ist $K_{r_1}(x_1)\setminus A_2$ offen und nichtleer (sonst
$K_{r_1}(x_1)\subseteq A_2^\circ$).
\begin{align*}
&\Rightarrow \exists \overline{K_{r_2}(x_2)}\subseteq K_{2r_2}(x_2)\subseteq
K_{r_1}(x_1)\setminus A_2, && r_2< \frac{1}{2},\\
&\ldots\\
&\Rightarrow \exists \overline{K_{r_n}(x_n)}\subseteq K_{2r_n}(x_n)\subseteq
K_{r_{n-1}}(x_{n-1})\setminus A_{n}, && r_n< \frac{1}{2^{n-1}},\\
\end{align*}
Nun ist $x_n\in K_{r_n}(x_n)\subseteq K_{r_{n-1}}(x_{n-1})\subseteq \ldots
\subseteq K_{r_l}(r_l)$, wobei
\begin{align*}
r_l < \frac{1}{2^l} \Rightarrow d(x_n,x_l) < \frac{1}{2^l}.
\end{align*}
Somit ist $(x_n)$ Cauchy und, also $x_n\to x\in M$. Insbesondere ist
\begin{align*}
\forall k\in \N : x\in \overline{K_{r_k}(x_k)}\subseteq M\setminus A_k
\Rightarrow x\notin A_k\forall k\in\N.
\end{align*}
Aber $x\in\bigcup_{k\in\N}A_k$ \dipper.\qedhere
\end{proof}

\begin{bem}
\label{bem:3.6}
Eine alternative Formulierung des Baireschen Kategoriensatz ist, dass $M$ von
2. Kategorie ist.\maphere
\end{bem}

\begin{prop}[Satz von Banach-Steinhaus (uniform boundedness principle)]
\index{Satz!Banach-Steinhaus}
\label{prop:3.7}
Sei $B$ Banachraum, $E$ normierter Raum und
\begin{align*}
\TT\subseteq \LL(B\to E)
\end{align*}
mit $\forall x\in B: \sup\limits_{T\in\TT}\norm{Tx}_E<\infty$. Dann gilt
\begin{align*}
\sup\limits_{T\in\TT}\norm{T} <\infty.\fishhere
\end{align*}
\end{prop}

Punktweise Beschränktheit einer Familie von Operatoren impliziert also die
gleichmäßige Beschränktheit.

\begin{proof}
Setze
\begin{align*}
A_n := \setdef{x\in B}{\sup\limits_{T\in\TT} \norm{Tx}_E \le n}
= \bigcap_{T\in\TT}\setdef{x\in B}{\norm{Tx}_E\le n}.
\end{align*}
\begin{proofenum}
  \item Zeige $B=\bigcup_{n\in\N} A_n$. Sei dazu $x\in B$, dann folgt
\begin{align*}
\sup\limits_{T\in\TT}\norm{Tx}_E = R<\infty
\Rightarrow x\in A_n \text{ für }n\ge R.
\end{align*}
\item Zeige $A_n$ ist abgeschlossen. Betrachte dazu
\begin{align*}
B\setminus A_n = \bigcup_{T\in\TT} \setdef{x\in B}{\norm{Tx}_E > n}.
\end{align*}
Diese Menge ist als Urbild von $(n,\infty)$ unter der stetigen
Abbildung $\norm{\cdot}_E\circ T$ offen.  Mit dem Satz von Baire folgt nun
\begin{align*}
\exists n_0\in\N \exists K_r(x_0)\subseteq A_{n_0}.
\end{align*}
\item Sei $x\in B$ und $\norm{x}_B=1$, so gilt
\begin{align*}
\norm{Tx}_E &= \frac{2}{r}\norm{T\left(\frac{r}{2}x\right)}_E =
\frac{2}{r}\norm{T\left(\frac{r}{2}x+x_0\right)-Tx_0}_E\\
&\le
\frac{2}{r}\left(\norm{T\left(\frac{r}{2}x+x_0\right)}_E+\norm{Tx_0}_E\right)\\
&\le \frac{2}{r}\left(n_0 + \sup\limits_{T\in\TT} \norm{Tx_0}_E\right)\\
\Rightarrow \norm{T} &\le \frac{2}{r}\left(n_0 + \sup\limits_{T\in\TT}
\norm{Tx_0}_E\right)\\
 \Rightarrow \sup\limits_{T\in\TT}\norm{T} &\le \frac{2}{r}\left(n_0 +
\sup\limits_{T\in\TT} \norm{Tx_0}_E\right).\qedhere
\end{align*}
\end{proofenum}
\end{proof}

\begin{prop}[Satz (open mapping principle)]
\index{Satz!open mapping}
\label{prop:3.8}
Seien $B,\tilde{B}$ Banachräume, $T:B\to\tilde{B}$ linear, beschränkt und
surjektiv. Dann ist $T$ offen, d.h. für $O\subseteq \BB$ offen ist
$T(O)\subseteq \tilde{B}$ offen.\fishhere
\end{prop}
\begin{proof}
\begin{proofenum}
  \item \textit{Anwendung des Satzes von Baire}. Da $T$ surjektiv, gilt
\begin{align*}
B=\bigcup_{n\in\N} K_n(0) \Rightarrow \tilde{B} = \bigcup_{n\in\N}
\overline{T(K_n(0))}^\sim.
\end{align*}
Aufgrund der Vollständigkeit von
 $\tilde{B}$ folgt nun mit dem Satz von Baire
\begin{align*}
\exists n_0\in\N : \exists \tilde{K}_r(y_0)\subseteq
\overline{T(K_{n_0}(0))}^\sim.
\end{align*}
\item
\begin{figure}[!htpb]
\centering
\begin{pspicture}(0,-1.48)(5.3,1.52)
\psellipse(2.42,-0.01)(2.42,1.47)
\pscircle[linecolor=purple](1.63,-0.01){1.29}
\psdots(1.64,0.02)
 
\psline[linecolor=darkblue]{->}(3.32,0.06)(2.64,0.06)
\pscircle[linecolor=darkblue](4.03,0.05){0.63}

\rput(1.79,-0.195){\color{gdarkgray}$0$}
\rput(4.03,0.065){\color{darkblue}$\tilde{K}_r(y_0)$}
\rput(4.51,1.325){\color{gdarkgray}$\overline{T(K_{n_0}(0))}$}

\end{pspicture}
\caption{Zum Verschieben von $\tilde{K}_r(y_0)$}
\end{figure}

\textit{Verschieben von $\tilde{K}_r(y_0)$}. Sei $y\in\tilde{K}_r(0)$ und damit
$\norm{y}_\sim < r$, dann gilt
\begin{align*}
&y+y_0\in \tilde{K}_r(y_0) \subseteq \overline{T(K_{n_0}(0))}^\sim.
\end{align*}
Betrachte nun eine Folge $(x_n)$ in $K_{n_0}(0)$ mit
\begin{align*}
Tx_n\to y+y_0,
\end{align*}
und eine Folge $\xi_n$ in $K_{n_0}(0)$ mit $T\xi_n \to y_0$,
%Sei nun $x_0\in K_{n_0}(0)$ mit $Tx_0 = y_0$, 
so gilt
\begin{align*}
T(x_n-\xi_n) = Tx_n-T\xi_n \to y+y_0-y_0 = y,
\end{align*}
% wobei $\norm{x_n-x_0} \le \norm{x_n}+\norm{x_0} < 2n_0$.
% $y_0$ ist kein Häufungspunkt, denn für eine Folge $(\xi_n)$ in $K_{n_0}(0)$ mit
% $T\xi_n\to y_0$ folgt,
% \begin{align*}
% T(x_n-\xi_n) = Tx_n-T\xi_n \to y + y_0-y_0 = y,
% \end{align*}
wobei $\norm{x_n-\xi_n} \le \norm{x_n}+\norm{\xi_n} < 2n_0$. Also ist $y\in
\overline{T(K_{2n_0}(0)}^\sim$.

Somit ist auch $\tilde{K}_r(0)\subseteq \overline{T(K_{2n_0}(0)}^\sim$.
\item \textit{Skalieren}. Für $a > 0$ ist
\begin{align*}
\tilde{K}_{ar}(0) \subseteq \overline{T(K_{a2n_0}(0))}^\sim,
\end{align*}
denn sei $\norm{y}_\sim < ar$, so ist $\norm{\frac{y}{a}}_\sim < r$ und daher
\begin{align*}
\frac{y}{a}\in \overline{T(K_{2n_0}(0))}^\sim.
\end{align*}
Es existiert also eine Folge $(x_n)$ in $K_{2n_0}(0)$ mit $Tx_n\to
\frac{y}{a}$. Somit  existiert eine Folge $(ax_n)$ in $K_{2an_0}(0)$ mit
$Tax_n\to y$ und daher ist $y\in \overline{T(K_{2an_0}(0)}^\sim$.

Insbesondere $\exists \delta > 0 : \tilde{K}_\delta(0) \subseteq
\overline{T(K_1(0))}^\sim$.
\item \textit{Einbetten} $\overline{T(K_1(0))}^\sim \subseteq T(K_3(0))$. Sei
$y\in \overline{T(K_1(0))}^\sim$.
\begin{align*}
&\Rightarrow \exists x_0\in K_1(0) : \norm{y-Tx_0}_\sim < \frac{\delta}{2}
\end{align*}
also $y-Tx_0\in\tilde{K}_{\delta/2}(0)\subseteq
\overline{T(K_{1/2}(0))}^\sim$.
\begin{align*}
&\Rightarrow \exists x_1\in K_{1/2}(0) : \norm{y-Tx_0-Tx_1}_\sim <
\frac{\delta}{4},
\end{align*}
also $y-Tx_0-Tx_1\in\tilde{K}_{\delta/4}(0)\subseteq \overline{T(K_{1/4}(0))}$,
usw. 
Wir erhalten somit eine Folge $(x_n)$ mit
\begin{align*}
&\norm{y-\sum_{j=0}^n Tx_j} < \frac{\delta}{2^n},\qquad
\norm{x_j} < \frac{1}{2^j}\\
\Rightarrow
&\sum_{j=0}^n Tx_j = T\left(\sum\limits_{j=0}^n x_j\right) \to y
\end{align*}
Weiterhin gilt
\begin{align*}
&\norm{\sum\limits_{j=0}^n x_j} \le
\sum\limits_{j=0}^n \norm{x_j} < \sum\limits_{j=1}^\infty \frac{1}{2^j} =
\frac{1}{1-\frac{1}{2}} = 2.
\end{align*}
Nun ist $B$ Banachraum und daher ist $x=\sum\limits_{j=0}^\infty x_j\in B$ mit
$\norm{x}\le 2$, also $x\in \overline{K_2(0)}\subseteq K_3(0)$. $T$ ist
beschränkt also stetig und daher gilt,
\begin{align*}
&y=\sum\limits_{j=0}^\infty Tx_j = T\left(\sum_{j=0}^\infty x_j\right) = T(x),\\
\Rightarrow &\overline{T(K_1(0))}^\sim \subseteq T(K_3(0)). 
\end{align*}
\item \textit{Skalieren}. Aus $\tilde{K}_\delta(0)\subseteq T(K_3(0))$ folgt:
\begin{align*}
\tilde{K}_{\tilde{\delta}}(0) \subseteq T(K_\ep(0)),\qquad
\text{mit } \tilde{\delta} = \delta\frac{\ep}{3}.
\end{align*}
\item \textit{Abschluss (Offenheit)}. Sei $O\subseteq B$ offen und $y\in
T(O)$ ($y=Tx$, $x\in O$). Zeige nun
\begin{align*}
\exists \tilde{\delta} > 0 : K_{\tilde{\delta}}(y) \subseteq T(O).
\end{align*}
$O$ ist offen, also $\exists \ep > 0 : K_\ep(0)\subseteq O$.
\begin{align*}
\Rightarrow
\tilde{K}_{\tilde{\delta}}(T(x)) &= T(x) + \tilde{K}_{\tilde{\delta}}(0)
\subseteq T(x)+T(K_\ep(0)) = T(x+K_\ep(0)) \\ &= T(K_\ep(x)) \subseteq
T(O).\qedhere
\end{align*} 
\end{proofenum}
\end{proof}

\begin{prop}[Satz (inverse mapping theorem)]
\index{Satz!inverse mapping}
\label{prop:3.9}
Seien $B,\tilde{B}$ Banachräume, $T:B\to\tilde{B}$ linear, beschränkt und
bijektiv. Dann ist $T^{-1}$ beschränkt.\fishhere
\end{prop}

``$T$ stetig $\Rightarrow$ $T^{-1}$ beschränkt''.

\begin{proof}
Zeige $T^{-1}$ ist stetig in $y=0$, d.h.
\begin{align*}
\forall \ep > 0 \exists \delta_\ep > 0 : \norm{y}_{\tilde{B}} < \delta_\ep
\Rightarrow \norm{T^{-1}y}_B < \ep.
\end{align*}
Dies ist äquivalent zu
\begin{align*}
\forall \ep > 0 \exists \delta_\ep > 0 : \tilde{K}_{\delta_\ep}(0)\subseteq
T(K_\ep(0)).
\end{align*}
Dies folgt aber direkt aus der Offenheit von $T$.\qedhere
\end{proof}

\begin{cor}
\label{prop:3.10}
Sei $B$ vollständig bezüglich $\norm{\cdot}$ und $\norm{\cdot}_\sim$ und
$\norm{\cdot}$ feiner als $\norm{\cdot}_\sim$. Dann sind die Normen
äquivalent.\qedhere
\end{cor}
\begin{proof}
Die Abbildung $\Id : (B,\norm{\cdot}) \to (B,\norm{\cdot}_\sim), x\mapsto x$
ist beschränkt (also stetig). Mit Satz \ref{prop:3.9} folgt, dass auch
$\Id^{-1}$ beschränkt (also stetig) ist. Damit ist $\norm{\cdot}_\sim$ feiner als
$\norm{\cdot}$.\qedhere
\end{proof}

\clearpage
% ==============================================================
% ================== Lineare Funktionale ===========================
% ==============================================================
\chapter{Lineare Funktionale}

Im Folgenden wollen wir zu einem normierten Raum $(E,\norm{\cdot})$ den Dualraum
\begin{align*}
E'=\LL(E\to\K)
\end{align*}
und dessen Elemente $T\in E'$, die
\emph{Funktionale} genauer betrachten.

% ==============================================================
% ================== Verallg. Koords        ===========================
% ==============================================================
\section{Lineare Funktionale als verallgemeinerte Koordinaten}

Ist $v=0$ der Nullvektor, so sind alle Koordinaten von $v$ Null. Sind $v$ und
$w$ identische Vektoren, so sind auch alle Koordinaten identisch.

\begin{bsp}
\label{bsp:4.1}
$E$ sei endlichdimensional mit Basis $\BB=\setd{b_1,\ldots,b_n}$. Somit ist
\begin{align*}
e_k' : E\to \K,\; x =\sum\limits_{j=1}^n x_j b_j \mapsto x_k
\end{align*}
linear und beschränkt, denn
\begin{align*}
\norm{e_k'(x)} = \abs{x_k} \le \sum\limits_{j=1}^m \abs{x_j} = \norm{x}_1
\le c\norm{x}_E.
\end{align*}
$\setd{e_1',\ldots,e_n'}$ ist Basis von $E'$, die \emph{duale Basis} zu
$\setd{b_1,\ldots,b_n}$. (Insbesondere haben $E$ und $E'$ die gleiche
Dimension). Sei $T\in E'$, so gilt
\begin{align*}
&T(x) = T\left(\sum\limits_{j=0}^n x_j b_j\right)
= \sum\limits_{j=1}^n x_j T(b_j) = \sum\limits_{j=1}^n e_j'(x)T(b_j)
= \left(\sum\limits_{j=1}^n T(b_j)e_j' \right)(x)\\
\Rightarrow& T = \sum\limits_{j=1}^n T(b_j)e_j'.
\end{align*}

\textit{Eindeutigkeit}. Sei $T=\sum\limits_{j=1}^n t_j e_j' \Rightarrow T(b_k)
= \sum\limits_{j=1}^n t_j\delta_{jk} = t_k$.

Insbesondere gilt:
\begin{align*}
0=x\in E &\Leftrightarrow \forall j=1,\ldots,n : x_j = 0
\Leftrightarrow \forall j=1,\ldots,n : e_j'(x) = 0 \\ 
& \Leftrightarrow \forall T\in E' : T(x) = 0.
\end{align*}

$T\in E'$ kann also als verallgemeinerte Koordinate betrachtet werden.\bsphere
\end{bsp}

\begin{bsp}
\label{bsp:4.2}
Sei $E=C([0,1]\to\C)$ mit $\norm{\cdot}_E=\norm{\cdot}_\infty$. Zu $x\in[0,1]$
sei 
\begin{align*}
\delta_x : E\to \C : f\mapsto f(x),
\end{align*}
so ist $\delta_x$ linear sowie
\begin{align*}
\abs{\delta_x f} = \abs{f(x)} \le \norm{f}_\infty \Rightarrow \norm{\delta_x}
\le 1.
\end{align*}
Also ist $\delta_x \in E'$.
\begin{align*}
f=0\Leftrightarrow \forall x\in[0,1] : \delta_x(f) = 0
\Leftrightarrow \forall T\in E' : T(f) = 0.
\end{align*}

Es stellt sich nun natürlich die Frage, ob $f=0\Leftrightarrow \forall T\in E'
: T(f) = 0$ in jedem normierten Raum. Zum Beweis benötigen wir jedoch noch
etwas Vorbereitung.\bsphere
\end{bsp}

% ==============================================================
% ================== Hyperebenen  ================================
% ==============================================================
\section{Lineare Funktionale als Hyperebene}

\begin{defn}
\label{defn:4.3}
Sei $L$ ein linearer Raum. Ein linerarer Teilraum $M\subseteq L$ heißt
\emph{Hyperebene}\index{Hyperebene}, falls $\dim L/M = 1$.\fishhere
\end{defn}

$M$ ist also genau dann Hyperebene, wenn eine lineare, bijektive Abbildung $\ph:
L/M\to \K$ existiert.

\begin{prop}
\label{prop:4.4}
\begin{propenum}
  \item Für einen linearen Teilraum $M$ von $L$ sind äquivalent:
  \begin{equivenum}
    \item $M$ ist Hyperebene.
    \item $\exists \ph: L\to \K$ linear, so dass gilt: $M=\ker\ph$ und $\ph\neq
    0$.
  \end{equivenum}
  \item Für $\ph,\psi : L\to\K$ linear sind äquivalent:
  \begin{equivenum}
    \item $\ker\ph = \ker \psi$.
    \item $\exists \lambda\neq 0 : \ph=\lambda\psi$.\fishhere
  \end{equivenum}
  \end{propenum} 
\end{prop}
\begin{proof}
\begin{proofenum}
  \item ``$\Rightarrow$'': Betrachte die Quotientenabbildung $q: L\to L/M :
  x\mapsto [x]$. $q$ ist offensichtlich linear. Weitherhin gilt $\dim L/M = 1$, nach
Definition \ref{defn:4.3} existiert also eine lineare bijektive Abbildung $\psi
: L/M\to \K$.
Setze nun $\ph=\psi\circ q$, dann gilt
\begin{align*}
&\ph(x) = 0 \Leftrightarrow \psi(q(x)) = 0 \Leftrightarrow
q(x) = [0] \Leftrightarrow [x] = [0] \\ & \Leftrightarrow x\in [0]
\Leftrightarrow x\in M.
\end{align*}
Also ist $\ker \ph = M$ und offensichtlich ist $\ph\neq 0$.

``$\Leftarrow$'': Setze $M:=\ker \ph$ und $\psi: L/M\to \K,\; [x]\mapsto
\ph(x)$. $\psi$ ist offensichtlich wohldefiniert und linear. Außerdem ist $\psi$
injektiv, denn falls $\ph(y)=\ph(x)$, so ist $\ph(x-y)=0$, d.h. $x-y\in M
\Leftrightarrow [x]=[y]$. Weiterhin ist $\psi$ surjektiv, da $\ph$ surjektiv.

Also ist $M$ Hyperebene.
\item Für den Spezialfall $\ph=0$ folgt die Behauptung trivialerweise.

 ``$\Rightarrow$'': Sei $x_0\in L$ mit $\ph(x_0) = 1$, dann gilt 
 $\psi(x_0) \neq 0$.
 
 Sei $x\in L\setminus \ker \ph$, dann gilt
 \begin{align*}
 \ph(x-\ph(x)x_0) = \ph(x)-\ph(x)\ph(x_0) = 0.
 \end{align*}
Somit ist $x-\ph(x)x_0\in \ker \ph = \ker \psi$. Insbesondere gilt also,
\begin{align*}
&0 = \psi(x-\ph(x)x_0) = \psi(x) - \ph(x)\psi(x_0)\\
\Rightarrow &\psi(x) = \ph(x)\psi(x_0)\\
\Rightarrow &\ph = \frac{1}{\psi(x_0)}\psi.
\end{align*}

``$\Leftarrow$'': Trivial.\qedhere
\end{proofenum}
\end{proof}

\begin{prop}
\label{prop:4.5}
Für $\ph: E\to \K$ linear sind äquivalent
\begin{equivenum}
  \item\label{prop:4.5:1} $\ph$ ist stetig (also beschränkt),
  \item\label{prop:4.5:2} $\ker \ph$ ist abgeschlossen.
\end{equivenum}
\end{prop}

\begin{proof}
``\ref{prop:4.5:1}$\Rightarrow$\ref{prop:4.5:2}'': $\ker\ph =
\ph^{-1}(\setd{0})$. $\setd{0}$ ist abgeschlossen und daher auch $\ker\ph$.
%  und
% \begin{align*}
% \ph(x) = \lim\limits_{n\to\infty} \ph(x_n) = 0\Rightarrow x\in\ker \ph.
% \end{align*}

``\ref{prop:4.5:2}$\Rightarrow$\ref{prop:4.5:1}'':
  Sei $M=\ker \ph$. Der Fall
$M=L$ ist trivial. Sei also $M\subsetneq L$. Betrachte  erneut die Quotientenabbildung $q: E\to
E/M,\; x\mapsto [x]$. Mit \ref{prop:4.4} folgt, dass eine bijektive Abbildung
$\psi: E/M\to \K$ existiert. Sei
\begin{align*}
\tilde{\ph} := \psi\circ q \overset{\text{Bew. }\ref{prop:4.4}}{\Rightarrow}
\ker \tilde{\ph} = M = \ker \ph \overset{\ref{prop:4.4}}{\Rightarrow} \ph =
\lambda\tilde{\ph}.
\end{align*}
Zeige nun $\tilde{\ph}$ ist stetig. $q$ ist stetig, denn
\begin{align*}
\norm{q(x)}_{E/M} = \inf\limits_{y\in M}\norm{x+y}_E \le \norm{x}_E \Rightarrow
\norm{q} \le 1.
\end{align*}
$\psi$ ist stetig, denn sei $\psi([x_0]) = 1$ und $[x]\in E/M$, so gilt
\begin{align*}
\psi(\psi([x])[x_0]) = \psi([x])\psi([x_0]) \overset{\psi\text{
bij}}{\Rightarrow} \psi([x])[x_0] = [x].
\end{align*}
Insbesondere ist $\setd{[x_0]}$ Basis von $E/M$ und daher $\psi(\alpha[x_0]) =
\alpha\psi([x_0]) =\alpha$, also gilt
\begin{align*}
&\abs{\psi([x])} = \abs{\psi(\alpha[x_0])} = \abs{\alpha} = 
\abs{\alpha}\frac{\norm{[x_0]}_{E/M}}{\norm{[x_0]}_{E/M}}
= \frac{\norm{\alpha[x_0]}_{E/M}}{\norm{[x_0]}_{E/M}}
= \frac{\norm{[x]}_{E/M}}{\norm{[x_0]}_{E/M}}\\
\Rightarrow& \norm{\psi} = \frac{1}{\norm{[x_0]}_{E/M}}.
\end{align*}
Also ist $\psi$ beschränkt. Somit ist $\psi\circ q$ stetig.

\textit{Einfacher}. Betrachte den Spezialfall $T: E\to F$ bijektiv und $E,F$
endlichdimensional. Definiere
\begin{align*}
\norm{x}:=\norm{Tx}_F
\end{align*}
ist Norm auf $E$. Es folgt mit \ref{prop:1.11}, dass
\begin{align*}
&c_1\norm{x}_E \le \norm{x}\le c_2\norm{x}_E\\
\Rightarrow & \norm{Tx}_F = \norm{x} \le c_2 \norm{x}_E.
\end{align*}
Ist also $T:E\to F$ linear und bijektiv und $E,F$ endlichdimensonal, so ist
$T$ Isometrie.\qedhere
\end{proof}

\section{Existenz linearer Funktionale}

\begin{defn}
\label{defn:4.6}
Sei $L$ reeller linearer Raum,
\begin{align*}
p : L\to\R
\end{align*}
heißt \emph{sublinear}\index{Abbildung!sublinear}, falls für $x,y\in L$ und
$\lambda > 0$ gilt,
\begin{defnenum}
  \item $p(x+y)\le p(x)+p(y)$,
  \item $p(\lambda x) = \lambda p(x)$.\fishhere
\end{defnenum}
\end{defn}

Eine sublineare Abbildung erfüllt also die Dreiecksungleichung und ist homogen
für positive Skalare.

\begin{bem}
\label{bem:4.7}
Jede Halbnorm (und damit auch jede Norm) ist sublinear.\\
$p(x) < 0$ ist erlaubt.\maphere
\end{bem}

\begin{prop}[Satz von Banach]
\label{prop:4.8}
Sei $L$ reeller linearer Raum, $p:L\to\R$ sublinear, $M\le L$ linearer
Teilraum und $f: M\to \R$ linear mit
\begin{align*}
\forall x\in M : f(x)\le p(x).
\end{align*}
Dann existiert eine lineare Fortsetzung
\begin{align*}
F: L\to\R,\qquad F|_M = f,
\end{align*}
so dass
\begin{align*}
\forall x\in L : F(x)\le p(x).\fishhere
\end{align*}
\end{prop}

\begin{bem}
\label{bem:4.9}
Im Allgemeinen ist der Nachweis der Existenz einer beliebigen Fortsetzung
einfach, der Nachweis einer $p$-beschränkten Fortsetzung schwierig.\maphere
\end{bem}

\begin{proof}[Beweis von Satz \ref{prop:4.8}.]
\begin{proofenum}
  \item \textit{Fortsetzung auf ``$\dim M+1$''}.\\ Sei $x_0\in L/M$. Angenommen
  es existiert eine Fortsetzung $g$ von $f$ auf $\lin{M,x_0}$, dann hat diese
  die Eigenschaften
  \begin{defnenum}
    \item $g|_M = f$.
    \item $g(y) = g(x+\lambda x_0) = f(x) + \lambda g(x_0)$. Ist die
    Darstellung $y=x+\lambda x_0$ eindeutig, so ist $g$ eindeutig durch $g(x_0)$
    bestimmt.
    
    Sei also $y\in \lin{M,x_0}$ mit $y=x+\lambda x_0$ und
    $y=\tilde{x}+\tilde{\lambda}x_0$, dann ist
\begin{align*}
0 = \underbrace{x-\tilde{x}}_{\in M} + (\lambda-\tilde{\lambda})x_0.
\end{align*}
Da $x_0\notin M$ ist $(\lambda-\tilde{\lambda})x_0$ nur dann Element von $M$,
wenn $\lambda-\tilde{\lambda}=0$, d.h. $\lambda=\tilde{\lambda}$ und
$x=\tilde{x}$.
\end{defnenum}
Sei $g_r : y = x+\lambda x_0 \mapsto f(x) + \lambda r$ für festes $r\in\R$, so
ist $g_r$ linear und Fortsetzung von $f$. Zu zeigen ist nun, dass $r$ so gewählt
werden kann, dass $g_r(y)\le p(y)$, was äquivalent ist zu
\begin{align*}
f(x)+\lambda r \le p(x+\lambda x_0).\tag{*}
\end{align*}
  
\begin{defnenum}
  \item Sei $\lambda = 0$, dann nimmt (*) die Form
\begin{align*}
f(x) \le p(x)
\end{align*}
an und ist nach Voraussetzung erfüllt.
\item Sei $\lambda > 0$, so gilt
\begin{align*}
(*) &\Leftrightarrow f(x)+\lambda r \le p(x+\lambda x_0)\\
&\Leftrightarrow r \le \frac{1}{\lambda}\left(p(x+\lambda x_0) - f(x) \right)\\
&\;\qquad = p\left(\frac{1}{\lambda}p(x)+ x_0\right) -
f\left(\frac{1}{\lambda}x\right)\\
&\Leftrightarrow
r\le \inf\setdef{p(z+x_0)-f(z)}{z\in M}.
\end{align*}
\item Sei $\lambda < 0$, so gilt
\begin{align*}
(*) &\Leftrightarrow f(x)+\lambda r \le p(x+\lambda x_0)\\
&\Leftrightarrow r \ge \frac{1}{\lambda}\left(p(x+\lambda x_0) - f(x) \right)\\
&\;\qquad = -p\left(\frac{1}{-\lambda}p(x)- x_0\right) +
f\left(\frac{1}{-\lambda}x\right)\\
&\Leftrightarrow
r\ge \sup\setdef{-p(z-x_0)+f(z)}{z\in M}.
\end{align*}
\end{defnenum}
Es existiert also ein $p$-beschränktes $g_r$, falls
\begin{align*}
\inf\setdef{p(z+x_0)-f(z)}{z\in M} \ge \sup\setdef{-p(z-x_0)+f(z)}{z\in
M}.\tag{**}
\end{align*}
Seien $x',x''\in M$, dann gilt
\begin{align*}
-p(x'-x_0)+f(x')
&= f(x'+x'')-f(x'') - p(x'-x_0)\\
&\le
p(x'+x'') - f(x'') - p(x'-x_0)\\
&\le
p(x'-x_0)+p(x''+x_0) - f(x'')-p(x'-x_0)\\
&= p(x''+x_0)-f(x'')\Rightarrow (**).
\end{align*}
$g_r$ ist also genau dann $p$ beschränkt, wenn
\begin{align*}
r\in R=\left[\sup_{z\in M}\setd{-p(z-x_0)+f(z)},\inf_{z\in
M}\setd{p(z+x_0)-f(z)}\right].
\end{align*}
Falls $R^\circ\neq \varnothing$, so existieren unendlich viele Fortsetzungen.
\item \textit{Es gibt einen ``größten'' linearen Teilraum $N$ von $L$, auf dem
eine lineare, $p$-beschränkte Fortsetzung existiert}.

Wir zeigen dies durch eine Anwendung des Lemmas von Zorn (engl.
``zornification''). Sei dazu
\begin{align*}
\AA :=
\setdef{(N,g)}{M\leq N \leq L\text{ und $g$ ist $p$-beschr. lin. FS von
$f$ auf $N$}}.
\end{align*}
\begin{defnenum}
  \item $\AA\neq \varnothing$, da $(M,f)\in\AA$.
  \item $\AA$ wird halbgeordnet durch $(N_1,g_1)\le (N_2,g_2)$, falls
\begin{align*}
N_1\le N_2\text{ und } g_2 \text{ FS von } g_1.
\end{align*}
Eine Teilmenge $\BB\subseteq \AA$ heißt \emph{aufsteigende Kette}, wenn $\BB$
durch $\le$ totalgeordnet ist.
\item \textit{Für jede aufsteigende Kette $\BB\subseteq \AA$ existiert eine
obere Schranke in $\AA$}.

Sei $\BB=\setdef{(N_i,g_i)}{i\in\II}$ eine solche Kette. Definiere
\begin{align*}
N:= \bigcup_{i\in\II} N_i,
\end{align*}
so ist $N\leq L$. Setze nun
\begin{align*}
g(x):= g_i(x),\qquad \text{falls } x\in N_i. 
\end{align*}
$g$ ist wohldefiniert, linear und $p$-beschränkt und Forsetzung von $f$, denn
\begin{align*}
\forall i\in\II : g_i\text{ linear und $p$-beschr.},\; g_i|_M = f.
\end{align*}
Also ist $(N,g)\in\AA$. Weiterhin gilt, dass
\begin{align*}
\forall i\in\II : N_i\le N,\quad g|_{N_i} = g_i
\end{align*}
also ist $(N_i,g_i)\le (N,g)$ für jedes $i\in\II$.
\end{defnenum}
(a)-(c) und das Lemma von Zorn ergeben, dass $\AA$ ein maximales Element
enthält, d.h.
\begin{align*}
\exists (N,g)\in \AA : \forall (\tilde{N},\tilde{g})\in \AA :
(\tilde{N},\tilde{g})\le (N,g).
\end{align*}
\item
\textit{Sei $(N,g)$ das maximale Element, dann ist $N=L$}. Angenommen
$N\subsetneq L$, dann $\exists x_0\in L/N$. Mit 1.) folgt, es existiert eine
lineare, $p$-beschränkte Fortsetzung von $g_r$ von $g$ auf $\lin{N,x_0}$. Da
aber $\lin{N,x_0}\le L$ und $g_r$ FS von $g$ ist $(\lin{N,x_0},g_r)\in\AA$ und
$(N,g)\le (\lin{N,x_0},g_r)$.\dipper\qedhere
\end{proofenum}
\end{proof}

\begin{prop}[Fortsetzungssatz von Hahn-Banach]
\index{Satz!Hahn-Banach}
\label{prop:4.10}
Sei $(E,\norm{\cdot})$ normierter Raum $M\le E$ linearer Teilraum, $f\in M'$.
Dann existiert eine Fortsetzung $F\in E'$ von $f$ mit $\norm{F}=\norm{f}$. (Im
Allgemeinen ist diese Fortsetzung nicht eindeutig).\fishhere
\end{prop}

Für den Beweis benötigen wir noch etwas Vorbereitung.

\begin{lem}
\label{lem:4.11}
Sei $L$ linearer Raum über $\C$. Dann gilt:
\begin{propenum}
  \item\label{lem:4.11:1} Äquivalent sind:
  \begin{equivenum}
    \item $g: L\to \C$ ist linear und $f=\Re(g)$.
    \item $f:L\to\R$ reell linear, d.h.
\begin{align*}
&\forall \lambda \in\R : f(\lambda x) = \lambda f(x)\\
&\forall x,y\in L: f(x+y) = f(x)+f(y)
\end{align*}
und $g:L\to \C,\; x\mapsto f(x)-if(x)$.
  \end{equivenum}
  \item\label{lem:4.11:2} Sei $\norm{\cdot}^\sim : L\to\R$ Halbnorm, $g:
  L\to\C$ linear. Dann sind äquivalent
\begin{equivenum}
  \item $\forall x\in L : \abs{g(x)}\le \norm{x}^\sim$.
  \item $\forall x\in L : \abs{\Re g(x)} \le \norm{x}^\sim$.
\end{equivenum}
\item\label{lem:4.11:3} Ist $(E,\norm{\cdot})$ normierter Raum über $\C$, $g\in
E'$ so gilt:
\begin{align*}
\norm{g} = \sup\limits_{x\neq 0} \frac{\abs{\Re g(x)}}{\norm{x}}.\fishhere
\end{align*}
\end{propenum}
\end{lem}

\begin{proof}
\begin{proofenum}
  \item ``$\Rightarrow$'': Sei also $f:=\Re g$, so ist $f$ trivialerweise reell
  linear. Weiterhin gilt $\Im g(x) = -\Re ig(x) = -\Re g(ix)$, also
\begin{align*}
g(x) = \Re g(x) + i\Im g(x) = f(x)-if(ix). 
\end{align*}
``$\Leftarrow$'': Setze $g(x) := f(x) -if(ix)$, so ist $\Re g(x) = f(x)$ und
für $x,y\in L$, $g(x+y) = g(x) + g(y)$. Sei nun $\lambda=\alpha+i\beta \in \C$
mit $\alpha,\beta\in\R$, so ist
\begin{align*}
g(\lambda x) &= f(\lambda x) - if(i\lambda x) \\ &= \alpha f(x) + \beta f(ix) -
i\alpha f(ix) -i\beta f(-x)\\
&= (\alpha+i\beta)(f(x)-if(ix)) = \lambda g(x).
\end{align*}
\item ``$\Rightarrow$'': $\abs{\Re g(x)}\le \abs{g(x)} \le \norm{x}^\sim$.\\
``$\Leftarrow$'': Setze $\ph=\arg g(x)$, so gilt
\begin{align*}
\abs{g(x)} = e^{-i\ph}g(x) = \underbrace{g(e^{i\ph}x)}_{\in\R}
= \Re g(e^{i\ph}x) \le \norm{e^{-i\ph}x}^\sim = \norm{x}^\sim.
\end{align*}
\item Für $g\equiv 0$ folgt die Aussage sofort. Sei $g\neq 0$, so folgt mit
\ref{lem:4.11:1}, $\Re g(x) \neq 0$. Definiere nun
\begin{align*}
&\norm{x}^\sim := \norm{g}\norm{x},\\
&\norm{x}^{\approx} := \norm{\Re g}\norm{x},
\end{align*}
so sind $\norm{\cdot}^\sim$ und $\norm{x}^\approx$ Normen auf $E$.\\
``$\Rightarrow$'': Sei $x\in
E$, so gilt
\begin{align*}
\abs{g(x)} \le \norm{x}^\sim &\overset{\ref{lem:4.11:2}}{\Rightarrow}
\abs{\Re g(x)} \le \norm{x}^\sim  = \norm{g}\norm{x}\\
&\Rightarrow \sup\limits_{x\neq 0} \frac{\abs{\Re g(x)}}{\norm{x}} \le
\norm{g}.
\end{align*}
``$\Leftarrow$'': Analog erhalten wir,
\begin{align*}
\abs{\Re g(x)}\le \norm{x}^\approx 
&\Rightarrow \abs{g(x)} \le \norm{x}^\approx = \norm{\Re g}\norm{x}\\
&\Rightarrow
\sup\limits_{x\neq 0}
\frac{\abs{g(x)}}{\norm{x}} \le \norm{\Re g}.\qedhere
\end{align*}
\end{proofenum}
\end{proof}

\begin{prop}[Satz von Hahn-Banach für Halbnormen]
\index{Satz!Hahn-Banach für Halbnormen}
\label{prop:4.12}
Sei $L$ linearer Raum über $\K$,\\ $\norm{\cdot}^\sim: L\to\R$ Halbnorm, $M\le
L$ und $f:M\to\K$ linear mit
\begin{align*}
\forall x\in M : \abs{f(x)}\le \norm{x}^\sim.
\end{align*}
Dann existiert eine lineare Fortsetzung $F:L\to\K$ mit
\begin{align*}
\forall x\in L : \abs{F(x)} \le \norm{x}^\sim.\fishhere
\end{align*}
\end{prop}
\begin{proof}
\textit{Fall 1}. Sei $\K=\R$ und $p(x) := \norm{x}^\sim$, dann existiert nach
\ref{prop:4.8} eine lineare Fortsetzung $F:L\to\R$, so dass
\begin{align*}
\forall x\in L : F(x)\le \norm{x}^\sim,
\end{align*}
wobei
\begin{align*}
\forall x\in L : -F(x) = F(-x) \le \norm{-x}^\sim = \norm{x}^\sim.
\end{align*}
\textit{Fall 2}. Sei $\K=\C$. Betrachte zunächst $L$ als linearen Raum über
$\R$ und setze
\begin{align*}
g : M\to\R,\quad x\mapsto \Re f(x),
\end{align*}
so ist $g$ linear und $\abs{g(x)}\le \abs{f(x)} \le \norm{x}^\sim$. Wenden wir
Fall 1 an, erhalten wir eine lineare Fortsetzung $G: L\to\R$, so dass
\begin{align*}
\forall x\in L : \abs{G(x)}\le \norm{x}^\sim.
\end{align*}
Betrachte $L$ nun wieder als Raum über $\C$ und
setze $F(x) := G(x) - iG(ix)$, so ist $F$ linear und da
\begin{align*}
\forall x\in L : \abs{\Re F(x)} = \abs{G(x)} \le \norm{x}^\sim,
\end{align*}
gilt ebenfalls $\abs{F(x)} \le \norm{x}^\sim$.

$F$ ist auch Fortsetzung vn $f$, denn für $x\in M$ gilt,
\begin{align*}
F(x) = G(x) - i G(ix) = g(x) - ig(ix) = \Re f(x) - i\Re f(ix) = f(x).\qedhere
\end{align*}
\end{proof}

\begin{proof}[Beweis von Satz \ref{prop:4.10}]
Für $f=0$ folgt die Behauptung trivialerweise. Sei also $f\neq 0$ und
\begin{align*}
\norm{x}^\sim := \norm{f}\norm{x},
\end{align*} 
so ist $\norm{\cdot}^\sim$ Norm auf $E$ und $\abs{f(x)} \le \norm{x}^\sim$ für
$x\in M$. Nach \ref{prop:4.12} existiert somit eine lineare Fortsetzung $F:E\to
\K$, so dass
\begin{align*}
\forall x\in E : \abs{F(x)} \le \norm{x}^\sim = \norm{f}\norm{x}
\Rightarrow \norm{F}\le \norm{f}.
\end{align*}
$F$ ist aber Fortsetzung und daher $\norm{f}\le\norm{F}$, also
$\norm{f}=\norm{F}$.\qedhere
\end{proof}

\begin{cor}
\label{prop:4.13}
\begin{propenum}
  \item $\forall x_0\in E\setminus\setd{0} \exists f \in E' : \norm{f} = 1,
  f(x_0) = \norm{x_0}$.
  \item Seien $\setd{x_1,\ldots,x_n}$ linear unabhängig und seien
  $\alpha_1,\ldots,\alpha_n\in \K$, dann
\begin{align*}
\exists f\in E' \forall j=1,\ldots,n : f(x_j) = \alpha_j.\fishhere
\end{align*}
\end{propenum}
\end{cor}
\begin{proof}
\begin{proofenum}
  \item 
Sei $M:=\lin{x_0}$ und $g: M\to \K,\; \alpha x_0\mapsto \alpha\norm{x_0}$. $g$
ist offensichtlich linear und $\abs{g(\alpha x_0)} = \abs{\alpha}\norm{x_0}$,
also
\begin{align*}
\norm{g} = \sup\limits_{y\neq 0}\frac{\abs{g(y)}}{\norm{y}}
= \sup\limits_{\alpha\neq 0}\frac{\abs{g(\alpha x_0)}}{\norm{\alpha x_0}} = 1.
\end{align*}
Durch Anwendung des Satzes von Hahn-Banach erhalten wir eine lineare
Fortsetzung $G\in E'$ mit $\norm{G}=1$ und $G(x_0) = \norm{x_0}$.
\item $M=\lin{x_1,\ldots,x_n}$ ist endlich dimensional. Aus der linearen
Algebra wissen wir, es existier eine lineare Abbildung
\begin{align*}
g : M \to \K,\qquad g(x_j) = \alpha_j.
\end{align*}
Da $\dim M <\infty$, ist $g$ beschränkt. Wir können also den Fortsetzungssatz
anwenden.\qedhere
\end{proofenum}
\end{proof}

\begin{bem}[Diskussion.]
\label{bem:4.14}
Falls $E\neq 0$, so $\exists x_0\in E\setminus\setd{0}$, so dass
\begin{align*}
\exists f\in E' : f(x) = \norm{x}\neq 0.
\end{align*}
Somit ist der Dualraum $E'=\LL(E\to\K)$ nicht leer. Außerdem gilt
\begin{align*}
x=y \Leftrightarrow x-y=0 \Leftrightarrow \forall f\in E' : f(x-y) =0
\Leftrightarrow \forall f\in E' : f(x) = f(y).
\end{align*}
Man sagt, der Dualraum trennt die Elemente von $E$. Statt $f(x)$ schreibt man
auch $\lin{f,x}$ (Dualitätsprodukt).

\textit{Erklärung}. Betrachte $1< p,q < \infty$ mit
$\frac{1}{p}+\frac{1}{q}=1$.
Sei $E=L^p([a,b])$, so ist $E'=L^q([a,b])$
(später). Dann gilt für $f\in L^p$ und $g\in L^q$,
\begin{align*}
g(f) = \int\limits_a^b f\cdot g \dmu = \lin{f,\overline{g}}.\maphere
\end{align*}
\end{bem}

\begin{cor}
\label{prop:4.15}
Für $x\in E$ gilt,
\begin{align*}
\norm{x} = \sup\setdef{\abs{f(x)}}{f\in E' \land \norm{f}\le 1},
\end{align*}
und das Supremum wird angenommen.\fishhere
\end{cor}
\begin{proof}
$f\in E'$ ist beschränkt, d.h.
$\abs{f(x)}\le \norm{f}\norm{x} \le \norm{x}$. Mit \ref{prop:4.13} folgt
\begin{align*}
\exists f\in E' : \norm{f} = 1 \land f(x) = \norm{x}.\qedhere
\end{align*}
\end{proof}

\begin{cor}
\label{prop:4.16}
Sei $M\le E$ abgeschlossen, $x_0\in E\setminus M$. Dann existiert ein $f\in E'$
mit
\begin{align*}
f\big|_M = 0 \text{ und } f(x_0)\neq 0.\fishhere 
\end{align*}
\end{cor}
\begin{proof}
Betrachte die Quotientenabbildung
\begin{align*}
\pi : E\to E/M,\quad x\mapsto [x].
\end{align*}
Setzen wir $\norm{[x]}_{E/M} := \inf\limits_{y\in M} \norm{x+y}$, so ist $\pi$
stetig (siehe Beweis \ref{prop:4.5}). Weiterhin ist $\pi(x) = [0]$, falls $x\in
M$ und $\pi(x_0)\neq 0$. Mit \ref{prop:4.13} folgt nun
\begin{align*}
\exists \ph\in (E/M)' : \ph([x_0]) = \norm{\pi(x_0)}_{E/M}\neq 0.
\end{align*}
Sei $f=\ph\circ\pi$, so ist $f$ beschränkt und linear, also $f\in E'$, sowie
\begin{align*}
f(x_0) = \ph(\pi(x_0)) \neq 0,\qquad
f(x) = \ph(\pi(x))=0\text{ falls } x\in M.\qedhere
\end{align*}
\end{proof}

\begin{cor}
\label{prop:4.17}
Sei $M\le E$, dann gilt
\begin{propenum}
  \item $\overline{M}=\bigcap \setdef{\ker f}{f\in E'\land M\le \ker f}$.
  \item $M$ liegt genau dann dicht in $E$, wenn
\begin{align*}
\forall f\in E' : \left(f\big|_M=0\Rightarrow f=0\right).\fishhere
\end{align*}
\end{propenum}
\end{cor}
\begin{proof}
\begin{proofenum}
  \item $M\le \ker f$, also ist $\overline{M}\le \ker f$ und da $f$ beliebig
\begin{align*}
\overline{M}\le K:=\bigcap \setdef{\ker f}{f\in E'\land M\le \ker f}.
\end{align*}
$\overline{M}\supseteq K$ ist äquivalent zu $E/\overline{M}\subseteq E/K$. Sei
also $x_0\in E/\overline{M}$, dann folgt mit \ref{prop:4.16}
\begin{align*}
\exists f\in E' : f\big|_{\overline{M}}  = 0\land f(x_0)\neq 0.
\end{align*}
Für dieses $f$ gilt nun $M\le\ker f$ und $x_0\notin \ker f$, also ist
$x_0\notin K$, d.h. $x_0\in E/K$.
\item ``$\Rightarrow$'': Sei $x\in E$, $(x_n)$ in $M$, $x_n\to x$. Für $f\in
E'$ mit $f\big|_M=0$ gilt
\begin{align*}
f(x) = \lim\limits_{n\to\infty} f(x_n) = 0.
\end{align*}
``$\Leftarrow$'': $\overline{M}=\bigcap \setdef{\ker f}{f\in E'\land M\le \ker
f}=E$.\qedhere
\end{proofenum}
\end{proof}

% ==============================================================
% ================== Reflexive Räume ==============================
% ==============================================================
\section{Reflexive Räume}

Auf $\R$ lässt sich der \textit{Satz von Bolzano-Weierstraß} beweisen, der
besagt, dass jede beschränkte Folge eine konvergente Teilfolge besitzt. Dieser
Satz lässt sich dann auf $\C$, $\R^n$ oder allgemeiner jeden
normierten Raum $V$ mit $\dim V < \infty$ ohne zusätzliche Voraussetzungen oder
Abschwächung der Aussage verallgemeinern. Nun stehen in der Funktionalanalysis
die unendlichdimensionalen Räume im Mittelpunkt, jedoch kennen wir hier
bereits Gegenbeispiele, bei denen die Aussage des Satzes falsch wird. Man
betrachte zum Beispiel eine Abzählung der Einheitsvektoren,
\begin{align*}
e_k = (0,\ldots,0,1,0,\ldots),\qquad e_{kn} = \delta_{nk}.
\end{align*}
Diese ist beschränkt, enthält aber sicher keine konvergente Teilfolge.

Ziel dieses Abschnitts ist es nun, den Satz dahingehend auf unendlich
dimensionale Räume zu verallgemeinern, dass jede beschränkte Folge eine
in einem noch zu definierenden Sinne ``schwach konvergente'' Teilfolge enthält.

\begin{defn}
\label{defn:4.18}
$E'' = (E')'$ heißt \emph{Bidualraum}\index{Vektorraum!Bidualraum} von
$E$.\fishhere
\end{defn}

$E''$ ist stets ein Banachraum, da $\K=\R$ oder $\C$ vollständig ist.

\begin{prop}[Einbettung]
\index{Vektorraum!Einbettung}
\label{prop:4.19}
Es existiert eine Isometrie $J_E : E\to E''$.\fishhere
\end{prop}
\begin{proof}
Für $E=(0)$ ist die Aussage trivial, sei also $E\supsetneq (0)$.
Zu $x\in E$ sei
\begin{align*}
T_x : E'\to \K,\; f\mapsto f(x).
\end{align*}
$T_x$ ist offensichtlich
linear und beschränkt, denn
\begin{align*}
\abs{T_x(f)} = \abs{f(x)} \le \norm{x}_E\norm{f}_{E'}.
\end{align*}
Also ist $T_x\in E''$. Da $x\neq 0$,  folgt mit
\ref{prop:4.13} die Existenz einer Abbildung $f_0\in E'$ mit $\norm{f_0}=1$ und
$f_0(x)=\norm{x}$. Für diese gilt,
\begin{align*}
\abs{T_{x}(f_0)} = \abs{f_0(x)} = 1\cdot\norm{x_0}_E,
\end{align*}
also ist $\norm{T_{x}}_{E''}=\norm{x}_E$. Setze nun
\begin{align*}
J_E : E\to E'',\quad x\mapsto T_x,
\end{align*}
so ist $J_E$ Isometrie.\qedhere
\end{proof}

Im Allgemeinen ist jedoch $\im J_E \subsetneq E''$, die Gleichheit gilt nur in
einer speziellen Klasse von Räumen. 

\begin{defn}
\label{defn:4.20}
$E$ heißt \emph{reflexiv}\index{Vektorraum!reflexiv}, falls $J_E$ surjektiv
ist.\fishhere
\end{defn}

\begin{bem}[Bemerkungen.]
\label{bem:4.21}
\begin{bemenum}
\item $E''$ ist vollständig und damit auch $\overline{\im J_E}$. Man erhält
durch diese Einbettung also auch eine Vervollständigung von $E$, denn $\im
J_E$ liegt dicht in $\overline{\im J_E}$ und $\im J_E$ ist isometrisch
isomorph zu $E$ (schreibe $J_E\cong E$).
\item Ist $E$ reflexiv, so ist $E=J_E^{-1}(E'')$ ein Banachraum.
\item Jeder Hilbertraum ist reflexiv.
\item Sei $1<p<\infty$, dann sind $l^p$ und $L^p([a,b])$ reflexiv.\maphere
\end{bemenum}
\end{bem}

\begin{prop}
\label{prop:4.22}
$l^1$ ist \textit{nicht} reflexiv.\fishhere
\end{prop}
\begin{proof}
\begin{proofenum}
  \item 
\textit{$(l^1)'\cong l^\infty$}. Setze dazu
\begin{align*}
\ph: l^\infty\to (l^1)',\quad (a_n)\mapsto \ph(a_n)\text{ mit } \ph(a_n)(x_n) :=
\sum\limits_{n=1}^\infty a_n x_n.
\end{align*}
$\ph(a_n)$ ist offensichtlich linear und beschränkt, denn
\begin{align*}
\abs{\ph(a_n)(x_n)} \le \sum\limits_{n=1}^\infty \abs{a_n}\abs{x_n}
\le \sup_n \abs{a_n} \sum\limits_{n=1}^\infty \abs{x_n}
= \norm{(a_n)}_\infty\norm{(x_n)}_1 < \infty.
\end{align*}
Somit ist $\ph(a_n)\in (l^1)'$ mit $\norm{\ph(a_n)}_{(l^1)'} \le
\norm{(x_n)}_1$, man prüft leicht nach, dass sogar die Gleichheit gilt, also
ist $\ph : l^\infty\to (l^1)'$ Isometrie.

\textit{$\ph$ ist  surjektiv}. Sei $f\in (l^1)'$, $e_k:=(\delta_{nk})_n\in
l^1$. Setze $a_k = f(e_k)$ für $k\in\N$, so gilt
\begin{align*}
\ph(a_n)(e_k) = f(e_k) \Rightarrow
\ph(a_n)(x_l) = f(x_l),
\end{align*}
da $\ph(a_n)$ und $f$ linear und stetig und $l_\text{abb}$ dicht in $l^1$. Also
ist $\ph: l^\infty\to (l^1)'$ bijektive Isometrie.
\item Konstruiere $F\in (l^1)''\setminus J_{l^1}(l^1)$.

Sei $M:=\setdef{(a_n)\in l^\infty}{(a_n)\text{ konvergent}},$ dann ist $M\le
l^\infty$. Setze
\begin{align*}
\lim : M \to \C,\quad (x_n)\mapsto \lim\limits_{n\to\infty} a_n,
\end{align*}
so ist $\lim\in (l^\infty)'$ mit $\norm{\lim}_{M'}\le 1$. Eine Anwendung des
Satzes von Hahn-Banach liefert eine lineare Fortsetzung $\Lim: l^\infty\to \C$
mit $\norm{\Lim}\le 1$.
\begin{figure}[!htbp]
\centering
\begin{pspicture}(-0.7,-0.97)(3.7,1.1)

\rput(0.5,0.615){\color{gdarkgray}$(l^1)'$}
\rput(2.8,0.635){\color{gdarkgray}$l^\infty$}
\rput(1.58,-0.805){\color{gdarkgray}$\C$}

\psline[linecolor=darkblue]{->}(0.9,0.61)(2.5,0.61)
\psline[linecolor=darkblue]{->}(0.5,0.4)(1.46,-0.61)
\psline[linecolor=darkblue]{->}(2.68,0.4)(1.74,-0.61)

\rput(1.7,0.855){\color{gdarkgray}$\ph^{-1}$}
\rput(2.72,-0.125){\color{gdarkgray}$\Lim$}
\rput(0.1,-0.125){\color{gdarkgray}$\Lim\circ\ph^{-1}$}
\end{pspicture} 
\caption{Zur Konstruktion von $\Lim$.}
\end{figure}

Nun ist $\Lim\circ\ph^{-1}\in (l^1)''$. Zeige, dass $\Lim\circ\ph^{-1}\notin
J_{l^\infty}(l^\infty)$.

Angenommen $\Lim\circ\ph^{-1}\in
J_{l^\infty}(l^\infty)$, d.h.
\begin{align*}
&\exists (x_n)\in l^1 : T_{(x_n)} = \Lim\circ\ph^{-1},\\
\Rightarrow &
\forall f\in (l^1)' : f(x_n) = T_{(x_n)}(f) =
\Lim\circ\underbrace{\ph^{-1}(f)}_{:=a_n}.
\end{align*}
Dann ist $f(x_n)= \ph(a_n)(x_n)$
\begin{align*}
\Leftrightarrow
\forall (a_n)\in l^\infty : \ph(a_n)(x_n) = \Lim(a_n)
\end{align*}
Setze also $(a_n):=e_k$, so gilt
\begin{align*}
\ph(e_k)(x_n) = x_k = \Lim((\delta_{nk})_n) = \lim((\delta_{nk})_n) = 0. 
\end{align*}
Somit ist $x_n\equiv 0$ und $\Lim(a_n)\equiv 0$, $\forall (a_n)\in l^\infty$.
Dies ist offensichtlich ein Wiederspruch.\qedhere
\end{proofenum}
\end{proof}

\begin{lem}
\label{prop:4.23}
\begin{propenum}
  \item Sind $E\cong F$ und $E$ reflexiv, so ist $F$ reflexiv.
  \item Ist $E$ reflexiv und $M\le E$ abgeschlossen, so ist auch $M$ reflexiv.
  \item Sei $E$ Banachraum, so ist $E$ genau dann reflexiv, wenn $E'$ reflexiv
  ist. Insbesondere ist $E$ genau dann reflexiv, wenn $E''$ reflexiv ist.
\end{propenum}
\end{lem}

\begin{proof}
\begin{proofenum}
  \item Zu $x''\in F''$ setze
\begin{align*}
p^{**}(x'')(y':E\to\K) := x''(y'\circ\ph),
\end{align*}
so ist $\ph^{**}(x''): E'\to\K$ linear und beschränkt, denn
\begin{align*}
\abs{\ph^{**}(x'')(y')} \le \norm{x''}_{F''}\norm{y'\circ\ph}_{F'}
\le \norm{x''}_{E''}\norm{\ph}_{F\to E}\norm{y'}_{E'}.
\end{align*}
\begin{figure}[!htpb]
\centering
\begin{pspicture}(-0.2,-1.45)(2.9,1.45)

\rput(0.24,1.095){\color{gdarkgray}$F$}
\rput(2.43,1.095){\color{gdarkgray}$E$}

\rput(1.3,1.315){\color{gdarkgray}$\ph$}
\psline[linecolor=darkblue]{->}(0.48,1.07)(2.26,1.07)

%\rput{-270.0}(1.5500001,-3.3700001){\rput(2.46,-0.925){$\supseteq$}}

\psline[linecolor=darkblue]{->}(0.24,0.91)(0.24,-0.35)
\psline[linecolor=darkblue]{->}(2.43,0.93)(2.43,-0.35)


\psline[linecolor=darkblue]{->}(0.7,-0.61)(2,-0.61)

\rput(0.24,-0.625){\color{gdarkgray}$J_F(F)$}
\rput{-270.0}(0.24,-0.96){$\subseteq$}

\rput(2.43,-0.605){\color{gdarkgray}$J_E(E)$}
\rput{-270.0}(2.43,-0.96){$\subseteq$}

\rput(1.3,-1.02){\color{gdarkgray}$\ph^{**}$}
\psline[linecolor=darkblue]{->}(0.48,-1.27)(2.22,-1.27)

\rput(0.24,-1.285){\color{gdarkgray}$F''$}
\rput(2.43,-1.285){\color{gdarkgray}$E''$}



%\rput{-270.0}(-0.66999996,-1.23){\rput(0.28,-0.965){$\supseteq$}}
\end{pspicture} 
\caption{Zur Konstruktion von $\ph^{**}$.}
\end{figure}

Also ist $\ph^{**}(x'')\in E''$. $E$ ist reflexiv, d.h. $\exists y\in
E:\ph^{**}(x'') = J_E(y)$.

Wir zeigen nun, dass $x''=J_F(\ph^{-1}(y))$.

Für $x'\in F'$ gilt,
\begin{align*}
x''(x') &= x''(x\circ\ph^{-1})(\ph)
= \ph^{**}(x'')(x\circ \ph^{-1})\\
&= J_E(y)(x'\circ\ph^{-1})
= x'\circ\ph^{-1}(y)
= J_F(\ph^{-1}(y))(x'),
\end{align*}
d.h. $\forall x''\in F'' \exists x\in F : x'' = J_F(x)$.
\item Sei $x''\in M''$ wir haben zu zeigen, dass ein $x\in M$ existiert so dass
$J_M(x) = x''$, d.h. $T_x(f) = x''(f)$, $\forall f\in M'$.

Zu $y'\in E'$, dann ist $y\big|_M \in M'$, setze $y''(y') := x''(y'\big|_M)$,
so ist $y''$ offensichtlich linear und beschränkt, denn
\begin{align*}
\abs{y''(y')} = \abs{x''(y'\big|_M)} \le \norm{x''}_{M''}\norm{y'\big|_M}_{E'}
\le \norm{x''}_{M''}\norm{y'}_{E'},
\end{align*}
also ist $y''\in E''$.

$E$ ist reflexiv, also existiert ein $y\in E$, sodass
$J_E(y) = T_y = y''$.

Sei nun $y'\in E'$ mit $y\big|_M=0$, dann ist
\begin{align*}
y'(y) = T_y(y') = y''(y') = x''(y'\big|_M) = 0.
\end{align*}
Also ist $y\in \ker y'$. \ref{prop:4.17} besagt
\begin{align*}
\overline{M} := \bigcap\setdef{\ker f}{f\in E'\land M\le \ker f}
\end{align*}
also ist $y\in \overline{M}=M$, da es in jedem dieser Kerne enthalten ist.

Sei nun $x'\in M'$, dann lässt sich $x'$ mit dem Satz von Hahn-Banach
fortsetzen zu $\xi'\in E'$ mit $\norm{\xi'}_{E'}=\norm{x'}_{M'}$. Dann ist
\begin{align*}
&x'(y) = \xi'(y) = T_y(\xi') = y''(\xi) = x''(\xi\big|_M) = x''(x'),\\
\Rightarrow & 
\forall x'\in M' : x''(x')=x'(y) = J_M(y)(x').
\end{align*}
\item
``$\Rightarrow$'': Sei $x'''\in E'''$. Wir haben zu zeigen, dass ein $x'\in E'$
existiert mit $J_{E'}(x') = x'''$. Die Komposition
\begin{align*}
x'''\circ J_{E} : E\to\K 
\end{align*}
ist ein Element in $E'$. Da $E$ reflexiv, ist $J_E$ invertierbar. Für $x''\in
E''$ ist daher,
\begin{align*}
x'''(x'') = x'''(J_E\circ J_E^{-1}(x''))
= \left(x'''\circ J_E\right)( J_E^{-1}(x'')).
\end{align*}
Setze $x:= J_E^{-1}(x'')\in E$ und $f:=x'''\circ J_E\in E'$, dann ist
\begin{align*}
\ldots = f(x)
= T_x(f) = J_{E}(x)(f) = x''(f) = T_{f}(x'') = J_{E'}(f)(x'')
\end{align*}
Also ist $x'''=J_{E'}(f)\in J_{E'}(E')$ und damit ist $J_{E'}$ surjektiv.

``$\Leftarrow$'': $E'$ ist reflexiv, also ist mit dem eben bewiesenen auch
$E''$ reflexiv. $J_E(E)\le E''$ ist als Bild des Banachraumes $E$ unter einer
Isometrie ebenfalls Banachraum und damit abgeschlossen. Also ist $J_E(E)$
reflexiv aber $J_E(E)$ und $E$ sind isometrisch isomorph, also ist auch $E$
reflexiv.\qedhere
\end{proofenum}
\end{proof}

\begin{defn}
\label{defn:4.24}
Eine Folge $(x_n)$ in $E$ heißt \emph{schwach
konvergent}\index{schwache!Konvergenz} gegen $x\in E$, falls
\begin{align*}
\forall f\in E' : f(x_n-x)\to 0.
\end{align*}
Wir schrieben in diesem Fall $x_n\wto x$.\fishhere
\end{defn}

\begin{bsp}
\label{bsp:4.25}
\begin{bspenum}
  \item Sei $E=l^2$ und $e_k = (0,\ldots,0,1,0,\ldots)$.\\ Sei $f\in E'=l^2$,
  d.h. es gibt eine Folge $(a_n)\in l^2$, so dass
\begin{align*}
\forall (x_n)\in l^2 : f((x_n)) := \sum\limits_{n=1}^\infty a_nx_n.
\end{align*}
Somit gilt $f(e_k) = a_k \to 0$, da $\sum_{k\ge 1} \abs{a_k}^2 < \infty$
unabhängig vom gewählten $f$, d.h.
\begin{align*}
\forall f\in E'  : f(e_k) \to 0,
\end{align*}
also $e_k\wto 0$ obwohl $\norm{e_k}_2 = 1$ also $\neg (e_k\to 0)$. Hier gilt
also
\begin{align*}
\norm{0}_2 < \lim\limits_{k\to\infty}\norm{e_k}_2.
\end{align*}
\item
Sei $E=C([0,1]\to\C)$ mit $\norm{\cdot}_\infty$.\\
Man kann zeigen, dass
\begin{align*}
f_n\wto f \Leftrightarrow f_n(x) \to f(x),\qquad \forall x\in[0,1].
\end{align*}
Schwache Konvergenz bezüglich $\norm{\cdot}_\infty$ entspricht also der
punktweisen Konvergenz, während Konvergenz bezüglich $\norm{\cdot}_\infty$ der
gleichmäßigen Konvergenz entspricht. (Siehe \cite{Werner07} S.
107)

\begin{figure}[!htpb]
\centering
\begin{pspicture}(-0.2,-1.5)(3.54,1.28)
\psline{->}(0.32,-1.02)(0.32,1.2)
\psline{->}(0.12,-0.84)(3.52,-0.84)
\psline(1.1,-0.72)(1.1,-0.96)
\psline(1.9,-0.72)(1.9,-0.96)
\psline(3.1,-0.72)(3.1,-0.96)
\psline(0.2,0.98)(0.44,0.98)

\rput(0.01,0.98){\color{gdarkgray}$1$}

\rput(1.06,-1.2){\color{gdarkgray}$\frac{1}{2n}$}

\rput(1.87,-1.2){\color{gdarkgray}$\frac{1}{n}$}

\rput(3.07,-1.2){\color{gdarkgray}$1$}

\psline[linecolor=darkblue](0.34,0.98)(1.1,-0.84)(1.92,0.98)(3.12,0.98)
\end{pspicture}
\caption{Zur Konstruktion von $f_n$.}
\end{figure}

Definiere $f_n$ wie in Skizze, dann $f_n\to 1$ punktweise, d.h. $f_n\wto 1$ mit
\begin{align*}
\norm{f_n}_\infty = 1 = \norm{f}_\infty
\end{align*}
aber $\norm{f-f_n}_\infty = 1$, d.h. $\neg(f_n\to f)$.\bsphere
\end{bspenum}
\end{bsp}

\begin{prop}
\label{prop:4.26}
\begin{propenum}
  \item Der schwache Grenzwert ist eindeutig.
  \item $x_n\to x$ impliziert $x_n\wto x$ und $\norm{x_n}\to\norm{x}$.
  \item Ist $(x_n)$ schwach konvergent, so ist $(x_n)$ beschränkt.
  \item Konvergiert $x_n\wto x$, so gilt
\begin{align*}
\norm{x} \le \liminf\limits_{n\to\infty} \norm{x_n}.\fishhere
\end{align*}
\end{propenum}
\end{prop}
\begin{proof}
\begin{proofenum}
  \item Sei $(x_n)$ Folge in $E$ mit schwachen Grenzwerten $x,y$, d.h.
\begin{align*}
\forall f\in E' : f(x_n-x)\to 0\land f(x_n-y)\to 0,
\end{align*}
d.h. $f(x)-f(y) = \lim\limits_{n\to \infty} f(x-x_n)+f(x_n-y) = 0$, $\forall
f\in E'$. Mit \ref{bem:4.14} folgt somit, dass $x=y$.
\item Übung.
\item Sei $x_n\wto x$, d.h.
\begin{align*}
\forall f\in E' : f(x_n)\to f(x) 
\Leftrightarrow
\forall f\in E' : T_{x_n}(f)\to T_{x}(f). 
\end{align*}
Da $T_{x_n}\to T_x$ punktweise, ist für jedes $f$ die Folge $T_{x_n}(f)$
beschränkt. Der Satz von Banach-Steinhaus sagt somit, dass
$\sup\limits_{k\in\N} \norm{T_{x_n}} =: c < \infty$, d.h. $\forall n\in\N :
\norm{T_{x_n}}=\norm{x_n}<c$.
\item Nach Voraussetzung konvergiert $T_{x_n}(f)\to T_x (f)$, wobei
\begin{align*}
\norm{T_{x_n}(f)} \le \norm{T_{x_n}}\norm{f} = \norm{x_n}\norm{f}.
\end{align*}
Somit gilt für $n\in\N$, $\norm{x} = \norm{T_x} \le \norm{x_n}$ und damit
\begin{align*}
\norm{x} \le \liminf\limits_{n\to\infty}\norm{x_n}.\qedhere
\end{align*}
\end{proofenum}
\end{proof}

\begin{prop}[Beobachtung]
\label{prop:4.27}
Sei $E$ reflexiv und $(x_n)$ Folge in $E$. Dann ist
\begin{align*}
M := \overline{\setd{\lin{x_1,\ldots,x_k}}} = \overline{\setd{\text{ endliche
Linearkombinationen }}} \le E
\end{align*}
abgeschlossen und daher reflexiv. Weiterhin liegt
\begin{align*}
M_\Q := 
\setdef{\sum\limits_{j=1}^n (\alpha_j + i\beta_j) x_j}{n\in\N,\;
\alpha_j,\beta_j\in\Q}
\end{align*}
dicht in $M$ und ist abzählbar.\fishhere
\end{prop}

\begin{defn}
\label{defn:4.28}
$E$ heißt \emph{separabel}\index{Vektorraum!separabel}, wenn $E$ eine abzählbare
dichte Teilmenge besitzt.~\fishhere
\end{defn}

\begin{lem}
\label{prop:4.29}
Ist $E'$ separabel, so ist $E$ separabel.\fishhere
\end{lem}
\begin{proof}
Sei $\setd{x_1',x_2',\ldots}\subseteq E'$ dicht. Zu $n\in\N$ wähle
$x_n \in E$ mit $\norm{x_n}=1$ und $\abs{x_n'(x_n)} \ge
\frac{1}{2}\norm{x_n'}$, dann ist
\begin{align*}
M:=\lin{\setd{x_1,\ldots}}\le E
\end{align*}
Zu zeigen ist nun, dass $\overline{M}=E$.
Sei $x'\in E'$ mit $x'\big|M=0$, dann gilt
\begin{align*}
\norm{x'-x_n'} &\ge \abs{(x'-x_n')(x_n)} = \abs{x_n'(x_n)}
\ge \frac{1}{2}\norm{x_n'} = \frac{1}{2}\norm{x_n'-x'+x'}\\
&\ge \frac{1}{2}\left(\norm{x'} - \norm{x_n'-x'}\right).
\end{align*}
Somit ist
\begin{align*}
&\norm{x'} \le 3\norm{x_n'-x'}, && n\in\N\\
\Rightarrow
&\norm{x'} \le 3\inf_n \norm{x_n'-x'} = 0,
\end{align*}
Somit ist $x'=0$, mit \ref{prop:4.17} folgt nun, dass $\overline{M}=E$. Nun ist
$M_\Q$ abzählbar und dicht in $E$, also ist $E$ separabel.\qedhere
\end{proof} 

\begin{prop}
\label{prop:4.30}
Sei $E$ reflexiv, dann besitzt jede beschränkte Folge in $E$ eine schwach
konvergente Teilfolge.\fishhere
\end{prop}
\begin{proof}
Sei also $(x_n)$ beschränkte Folge in $E$ mit $\sup_n \norm{x_n}_E = c <
\infty$.
\begin{proofenum}
\item Setze $M=\overline{\lin{x_1,x_2,\ldots}}$, dann ist $M\le E$
abgeschlossen, separabel und nach \ref{prop:4.23} außerdem reflexiv, d.h.
$M''=J_M(M)$. Somit ist $M''$ als Bild eines separablen Raumes unter einer
Isometrie ebenfalls separabel und daher ist nach \ref{prop:4.29} auch $M'$
separabel, d.h. $M'=\overline{\setd{y_1',y_2',\ldots}}$.
\item \textit{Konstruktion einer konvergenten Teilfolge von $J_M(x_n)$}. Für
jedes $j\in\N$ gilt
\begin{align*}
\abs{J_M(x_n)(y_j')} \le \norm{J_M(x_n)}_{E''}\norm{y_j'}_{E'} =
\norm{x_n}_{E}\norm{y_j'}_{E'},
\end{align*} 
d.h. die Folge $\left(J_M(x_n)(y_j')\right)$ ist beschränkt in $\K$. Somit
existiert zu $y_1'$ eine konvergente Teilfolge $(J_M(x_n^{(1)})(y_1'))$. Nun ist auch
$(J_M(x_n^{(1)})(y_2'))$ beschränkt und besitzt daher ebenfalls eine Konvergente
Teilfolge $(J_M(x_n^{(2)})(y_2'))$ usw.
Dies können wir nun \textit{ad
infinitum} fortführen.

Setze $\xi_n=x_n^{(n)}$, so ist diese Diagonalfolge
ebenfalls Teilfolge von $(x_n)$ und daher beschränkt und für jedes $j\in\N$ konvergiert
$(J_M(\xi_n)(y_j'))$ für $n\to\infty$.
\item \textit{Definition des Grenzelementes}. Sei
\begin{align*}
x'' : \setd{y_1',y_2',\ldots} \to \K,\quad y_j' \mapsto
\lim\limits_{n\to\infty} J_M(\xi_n)(y_j'),
\end{align*}
so lässt sich $x''$ linear auf die lineare Hülle $\lin{y_1',y_2',\ldots}$ wie
folgt fortsetzen,
\begin{align*}
x''\left(\sum\limits_{k=1}^K \lambda_k y_{j_k}'\right)
:= \sum\limits_{k=1}^K \lambda_k\lim\limits_{n\to\infty} J_M(\xi_n)(y_{j_k}').
\end{align*}
Die Fortsetzung ist ebenfalls beschränkt, denn für $x=\sum_{k=1}^K \lambda_k
y_{j_k}'$ gilt
\begin{align*}
\abs{x''(x)} &= 
\abs{\lim\limits_{n\to\infty} J_M(\xi_n)(x)}
\le \limsup\limits_{n\to\infty} \norm{J_M(\xi_n)}_{E''}\norm{x}_{E'}\\
&= \limsup\limits_{n\to\infty} \norm{\xi_n}_{E''}\norm{x}_{E'}
\le c\norm{x}_{E'}.
\end{align*}
Da $\lin{y_1',y_2',\ldots}$ dicht in $M'$ liegt, können wir den
Fortsetzungssatz \ref{prop:2.5} anwenden und erhalten eine eindeutige
Fortsetzung
\begin{align*}
x'': M'\to \K,\qquad \norm{f}_{M''} = \norm{x''}_{M'} \le c.
\end{align*}
\item Sei $x:=J_M^{-1}(f)$. Wir haben nun zu zeigen, dass $\xi_n\wto x$ in $E$.
Sei also $f\in E'$, dann gilt
\begin{align*}
\abs{f(\xi_n)-f(x)} &= \abs{f\big|_M(\xi_n)-f\big|_M(x)} =
\abs{J_M(\xi_n)(f\big|_M) - J_M(x)(f\big|_M)} \\ &\le
\underbrace{\abs{J_M(\xi_n)(f\big|_M-y_j')}}_{\text{(1)}} +
\underbrace{\abs{J_M(\xi_n)(y_j')-x''(y_j')}}_{\text{(3)}} \\
&\quad+
\underbrace{\abs{x''(y_j'-f\big|_M)}}_{\text{(2)}}.
\end{align*}
Nun gilt
\begin{align*}
&(1) \le \norm{J_M(\xi_n)}_{E''}\norm{f\big|_M-y_j'}_{E'}
\le c\norm{f\big|_M-y_j'}_{E'} < \frac{\ep}{3},\\
&(2) \le \norm{x''}_{M''}\norm{y_j'-f\big|_M}_{E'} \le
c\norm{f\big|_M-y_j'}_{E'} < \frac{\ep}{2}\\
&(3) < \frac{\ep}{2},\qquad n> N_\ep\text{ bei } y_j' \text{ fest}.
\end{align*}
Also ist der gesamte Ausdruck $< \ep$, d.h. $f(\xi_n)\to f(x)$.\qedhere
\end{proofenum}
\end{proof}

\begin{defn}
\label{defn:4.31}
$K\subseteq L$ heißt \emph{konvex}\index{Menge!konvex}, falls für alle $x,y\in
K$ gilt,
\begin{align*}
\forall\lambda\in[0,1] : \lambda x + (1-\lambda)y\in K.\fishhere
\end{align*}
\end{defn}

\begin{bsp}
\label{bsp:4.32}
Sei $E$ ein normierter Raum.
\begin{bspenum}
  \item \textit{Kugeln sind konvex}. Betrachte also die Kugel mit Radius
  $R$ um $x_0$ und $x,y$ seien so gewählt, dass $\norm{x-x_0}_E<R$ und
  $\norm{y-x_0}_E < R$, dann gilt auch
\begin{align*}
\norm{\lambda x + (1-\lambda)y - x_0}_E
&= \norm{\lambda (x-x_0) + (1-\lambda)(y - x_0)}_E\\
&\le \lambda \norm{x-x_0}_E + (1-\lambda)\norm{y-x_0}_E
< R.
\end{align*}
\item \textit{Halbräume sind konvex}. Betrachte
\begin{align*}
K:=\setdef{x\in E}{\Re f(x)\le \alpha},\qquad \alpha\in\R,\quad f\in E'.
\end{align*}
Sind $x,y\in K$, so gilt auch
\begin{align*}
\Re f(\lambda x + (1-\lambda)y)
&= \lambda \Re f(x) + (1-\lambda) \Re f(y) \le \lambda \alpha +
(1-\lambda)\alpha \\ &= \alpha.\bsphere
\end{align*}
\end{bspenum}
\end{bsp}

\begin{prop}[Trennungssatz]
\label{prop:4.33}
Sei $K\subseteq E$ konvex und abgeschlossen, $x_0\in E\setminus K$. Dann gibt
es ein $f\in E'$ und ein $\alpha\in \R$, so dass
\begin{align*}
\Re f(x_0) > \alpha \text{ und } \forall x\in K : \Re f(x) \le \alpha.\fishhere
\end{align*}
\end{prop}

\begin{figure}[!htpb]
\centering
\begin{pspicture}(0,-1.4000001)(4.84,1.4)
\pspolygon[linestyle=none,fillstyle=solid,fillcolor=glightgray]%
	(0.14,1.36)(0.14,-1.06)(4.42,-1.06)(3.02,1.38)
\psline[linecolor=darkblue](4.42,-1.06)(3.02,1.38)
\psdots(4.58,0.28)
\psbezier[fillcolor=white,fillstyle=solid](1.5,0.58)(1.08,0.4)(0.0,1.02)(0.5,-0.18)(1.0,-1.38)(2.5482988,-0.81650954)(2.82,0.18)(3.091701,1.1765095)(1.92,0.76)(1.5,0.58)

\rput(2.38,0.025){\color{gdarkgray}$K$}
\rput(4.62,0.065){\color{gdarkgray}$x_0$}
\rput(3.27,-0.835){$\Re f(x)\le\alpha$}
\end{pspicture} 
\caption{Zum Trennungssatz.}
\end{figure}

\begin{proof}
\begin{proofenum}
\item Sei $\K=\R$.
\begin{defnenum}
  \item Wir können ohne Einschrädnkung annehmen, dass $0\in K^\circ$, denn
  andernfalls verschieben wir $K$ und $x_0$ so, dass $0\in K$ und ersetzen $K$
  durch,
\begin{align*}
\setdef{x\in E}{d(x,K) \ge \frac{1}{2}d(x_0,K)}.
\end{align*}
Man kann in leicht zeigen, dass diese Menge konvex ist.
\item
Sei $p: E\to \R,\; x\mapsto \inf\setdef{r>0}{r^{-1}x\in K}$. $p$ ist
wohldefiniert, da $K_\ep(0)\subseteq K$ und damit
$\left(\frac{2}{\ep}\norm{x}\right)^{-1}x\in K_\ep(0)$, also ist die Menge
nicht leer und das Infimum endlich.

$p$ heißt \emph{Minkowskifunktional} und hat folgende Eigenschaften,
\begin{equivenum}
  \item $0\le p(x) <\infty$.
  \item $\forall x\in K : p(x) \le 1$.
  \item $p(x_0) > 1$.
  \item $p(\lambda x)=\lambda x$, falls $\lambda \ge 0$
  \item $p(x+y) \le p(x) + p(y)$.
\end{equivenum}
(i)-(ii) sind klar.\\
Angenommen $p(x_0)=1$, dann existiert eine Folge in $K$,
die gegen $x_0$ konvergiert, $K$ ist abgeschlossen also $x_0\in K$. \dipper
Ist hingegen $p(x_0)<1$, dann ist $\frac{x_0}{r}\in K$ für ein $r < 1$ und
damit auch $(1-r)\cdot 0  + r \frac{x_0}{r}\in K.\dipper$ Also folgt (iii).\\
Zu (iv): Sei $\lambda > 0$, dann ist $\frac{\lambda x}{\lambda r}\in
K\Leftrightarrow \frac{x}{r}\in K$. $p(0)=0$ ist klar.\\
Zu (v): Seien $r^{-1}x,s^{-1}y\in K$, dann ist
\begin{align*}
\frac{x+y}{r+s} = \frac{r}{r+s}\frac{x}{r} + \frac{s}{r+s}\frac{y}{s}\in K.
\end{align*}
Also ist $p$ sublinear.
\item Setze $T: \lin{x_0}\to \R$, $\lambda x_0\mapsto \lambda p(x_0)$, dann ist
$T$ linear und $p$-beschränkt, denn
\begin{align*}
&\lambda \le 0 : T\lambda x_0 \le 0 \le p(\lambda x_0),\\
&\lambda > 0 : T\lambda x_0 = \lambda p(x_0) = p(\lambda x_0).
\end{align*}
Mit dem Satz von Hahn-Banach folgt nun, dass eine lineare, $p$-beschränkte
Fortsetzung $\tilde{T}: E\to \R$ existiert. Insbesondere gilt
\begin{align*}
&\forall x\in K : \tilde{T}(x) \le p(x) \le 1,\\
&\tilde{T}(x_0) = T(x_0) = p(x_0) > 1.
\end{align*}
\item $\tilde{T}\in E'$, denn $K_\ep(0)\subseteq K$ und daher gilt $\forall
x\in E: \frac{x}{\ep/2 \norm{x}}\in K$,
\begin{align*}
\Rightarrow &\forall x\in E : p(x) \le \frac{\ep}{2}\norm{x},\\
& \tilde{T}(x) \le p(x)  \le \frac{\ep}{2}\norm{x}\\
& -\tilde{T}(x) = \tilde{T}(-x) \le \frac{\ep}{2}\norm{x}.
 \end{align*}
\end{defnenum}
\item Sie $\K=\C$. Fasse $E$ als Raum über $\R$ auf und konstruiere $\tilde{T}$
wie in 1.). Setze anschließend
\begin{align*}
f(x) = \tilde{T}(x)-i\tilde{T}(ix).\qedhere
\end{align*}
\end{proofenum}
\end{proof}

\begin{prop}
\label{prop:4.34}
Eine konvexe abgeschlossene Menge $K\subseteq E$ ist schwach
folgenabgeschlossen, d.h. der schwache Grenzwert $x$ einer Folge $(x_n)$ in $K$,
die schwach in $E$ konvegiert, liegt in $K$.\fishhere
\end{prop}
\begin{proof}
Sei also $(x_n)$ Folge in $K$, die schwach in $E$ konvergiert, d.h.
\begin{align*}
\forall f\in E': f(x_n)\to f(x),\qquad x\in K.
\end{align*}
Angenommen $x\in E\setminus K$. Wir können also den Trennungssatz auf $x$
anwenden und erhalten somit ein $f\in E'$ mit
\begin{align*}
f(x) > \alpha,\qquad \forall y\in K : f(y)\le \alpha.
\end{align*}
Nun gilt aber $x_n\wto x$ und daher auch
\begin{align*}
\forall n\in \N f(x_n)\le \alpha \Rightarrow f(x) \le \alpha.\dipper\qedhere
\end{align*}
\end{proof}

\begin{prop}
\label{prop:4.35}
Sei $E$ reflexiv und $K\subseteq E$ konvex und abgeschlossen, $x_0\in
E\setminus K$. Dann gilt
\begin{align*}
\exists y_0\in K : \norm{x_0-y_0}_E = d(x_0,K).\fishhere
\end{align*}
\end{prop}
\begin{proof}
Sei $(y_n)$ Folge in $K$ mit $\norm{x_0-y_n} \to d(x_0,K)$, dann ist $(y_n)$
beschränkt. Somit existiert eine schwach konvergente Teilfolge $(y_{n_k})$ mit
Grenzwert $y$. Nach \ref{prop:4.34} ist $y\in K$. Außerdem gilt $x_0-y_{n_k}\wto
x_0-y$ also mit \ref{prop:4.26} auch
\begin{align*}
\norm{y-x_0} = \liminf_n \norm{y_{n_k}-x_0} = d(x_0,K).\qedhere
\end{align*}
\end{proof}

\begin{bsp}[Anwendung]
\label{bsp:4.36}
Betrachte eine kreisförmige dünne Membran, die am Rand befestigt ist. Nun wird
ein Hindernis unter die Membran geschoben, so dass diese sich diese nach oben
ausdehnt. Wir interessieren uns nun für die Form, die die Membran annimmt,
abhängig vom Hindernis.

Zur Modellierung der Membran, betrachte eine Funktion $z=u(x,y)$ mit $(x,y)\in
G=K_r(0)$, die die Auslenkung im Punkt $(x,y)$ beschreibt. Die Befestigung am
Rand modellieren wir durch $u\big|_{\partial G} = 0$.

Die Membran beschreibt ein konservatives System. Die Energie ist also erhalten
und aus der Physik wissen wir, dass sie folgende Form hat:
\begin{align*}
&E: C_0^1(G)\to \R,\quad u\mapsto E(u) = \int_G \abs{\partial_x u}^2 +
\abs{\partial_y u}^2\dmu(x,y),\\
&C_0^1(G) := \setdef{f\in C^1(G\to\R)}{f\big|_{\partial G}=0}.
\end{align*}

Gesucht ist nun ein $u\in C_0^1(G)$, so dass $E(u)$ minimal wird. Dazu führen
wir auf $C_0^1(G)$ die Norm
\begin{align*}
\norm{u}_2 = \int_G \abs{u}^2 + \abs{\partial_x u}^2+ \abs{\partial_y
u}\dmu(x,y)
\end{align*}
ein. Auf $C_0^1(G)$ sind $\norm{\cdot}_2$ und $\norm{\cdot}$ äquivalent. Die
Vervollständigungen bezüglich dieser Normen sind also identisch: 
\begin{align*}
W_0^{1,2}(G) = \overline{C_0^1(G)}^{\norm{\cdot}_2} =
\overline{C_0^1(G)}^{\norm{\cdot}}.
\end{align*}
$W_0^{1,2}(G)$ ist ein sogenannter Sobolveraum. Wir werden später zeigen, dass
dieser reflexiv ist.

Modellieren wir das Hinderniss durch die Funktion $z=f(x,y)$, so 
besteht unser Problem darin, $u\in W_0^{1,2}(G)$ zu finden mit
\begin{align*}
u\in K := \overline{\setdef{u\in C_0^1(G)}{u(x,y)\ge f(x,y)}}.
\end{align*}
$K$ ist abgeschlossen und konvex. Ist $\norm{u}$ minimal, dann ist auch
$E(u)=\norm{u}^2$ minimal. Falls $f$ stetig, existiert nun ein $(x,y)\in G$ mit
$f(x,y)> 0$, d.h. $0\notin K$. Mit \ref{prop:4.35} folgt nun
\begin{align*}
&\exists u\in K : \norm{u-0} = d(0,K) = \inf\limits_{v\in K}\norm{v-0},\\
\Rightarrow & \norm{u} = \min\limits_{v\in K}\norm{v}.
\end{align*}
Wir haben somit die Existenz \textit{einer} Lösung nachgewiesen. Ob diese
Eindeutig ist oder wie diese konkret aussieht, ist ein weiteres Problem.\bsphere 
\end{bsp}

\clearpage
\chapter{Hilberträume}

``\ldots sind die schönsten Banach-Räume'' --- B. Kümmerer

\begin{defn}
\label{defn:5.1}
Sei $L$ ein linearer Raum über $\K$. Eine Abbildung
\begin{align*}
\lin{\cdot,\cdot} : L\times L \to \K
\end{align*}
heißt \emph{Skalarprodukt}\index{Skalarprodukt}, falls sie die folgenden
Eigenschaften erfüllt.
\begin{defnenum}
  \item\label{defn:5.1:1} $\lin{\alpha x_1+\beta x_2,y} = \alpha\lin{x_1,y}+
  \beta\lin{x_2,y}$.
  \item\label{defn:5.1:2} $\lin{x,y}=\overline{\lin{y,x}}$.
  \item\label{defn:5.1:3} $\lin{x,x} \ge 0$
  \item\label{defn:5.1:4} $\lin{x,x}=0\Rightarrow x=0$.
\end{defnenum}
Sind nur \ref{defn:5.1:1}-\ref{defn:5.1:3} erfüllt, so heißt
$\lin{\cdot,\cdot}$ \emph{Semiskalarprodukt}\index{Skalarprodukt!Semi-}.\fishhere
\end{defn}

Aus \ref{defn:5.1:1} folgt insbesondere, dass $\lin{0,y}=0$.\\
\ref{defn:5.1:1} und \ref{defn:5.1:2} besagen weiterhin, dass $\lin{x,\alpha
y_1+\beta y_2}=\overline{\alpha}\lin{x,y_1}+\overline{\beta}\lin{x,y_2}$.\\
\ref{defn:5.1:3} ist sinnvoll, da nach \ref{defn:5.1:2} $\lin{x,x}\in\R$.

\begin{prop}[Cauchy-Schwarz-Bunjakowski-Ungleichung]
\index{Skalarprodukt!Cauchy-Schwartz}
\label{prop:5.2}
Sei $\lin{\cdot,\cdot}$ ein Semi-Skalarprodukt. Dann gilt
\begin{align*}
\abs{\lin{x,y}}^2 \le \lin{x,x}\lin{y,y}.
\end{align*}
Falls $\lin{\cdot,\cdot}$ ein Skalarprodukt, dann gilt die Gleichheit genau
dann, wenn $\setd{x,y}$ linear abhängig.\fishhere
\end{prop}
\begin{proof}
\begin{proofenum}
\item Der Fall $\lin{x,y}=0$ ist klar, sei also $\lin{x,y}\neq 0$ und
$\lambda\in\R$, so gilt
\begin{align*}
0 &\le \lin{x+\lambda \lin{x,y}y,x+\lambda\lin{x,y}y}\\
&= \lin{x,x} + \lambda\overline{\lin{x,y}}\lin{x,y} + \lambda\lin{x,y}\lin{y,x}
+ \lambda^2\abs{\lin{x,y}}^2\lin{y,y}\\
&= \lin{x,x} + 2\lambda\abs{\lin{x,y}}^2 + \lambda^2\abs{\lin{x,y}}^2\lin{y,y}
=: p(\lambda).
\end{align*}
Da $p$ Polynom 2. Grades und $\ge 0$, hat $p$ höchstens eine Nullstelle und
daher ist die Diskriminante der Mitternachtsformel $\le 0$, d.h.
\begin{align*}
&0 \ge 4\abs{\lin{x,y}}^4 - 4 \abs{\lin{x,y}}^2\lin{y,y}\lin{x,x}\\
\Leftrightarrow\;
&0 \ge \abs{\lin{x,y}}^2 - \lin{y,y}\lin{x,x}.
\end{align*}
\item ``$\Leftarrow$'': Sei $\setd{x,y}$ linear abhängig, d.h. $x = \alpha y$.
Dann gilt
\begin{align*}
&\abs{\lin{x,y}}^2 = \abs{\lin{x,\alpha y}}^2 = \abs{\alpha}^2
\abs{\lin{y,y}}^2,\\
&\lin{x,x}\lin{y,y} = \alpha\overline{\alpha}\lin{y,y}^2
\abs{\alpha}^2\abs{\lin{y,y}}^2.
\end{align*}
``$\Rightarrow$'': Seien $\lin{x,y}$ linear unabhängig, dann ist insbesondere
$x\neq 0$. Nun existiert ein $\alpha\in\K$, so dass
\begin{align*}
\lin{y,x}-\alpha\lin{x,x} = 0,
\end{align*}
wobei $y-\alpha x \neq 0$ aber
$\lin{y-\alpha x,x}=0$. Somit ist
\begin{align*}
\lin{y,y} &= \lin{y-\alpha x+\alpha x,y-\alpha x + \alpha x}\\
&= \underbrace{\lin{y-\alpha x,y-\alpha x}}_{>0} + \abs{\alpha}^2 \lin{x,x}.
\end{align*}
Es gilt also $\lin{y,y} > \abs{\alpha}^2\lin{x,x}$ und wir erhalten
\begin{align*}
&\abs{\lin{x,y}}^2 = \abs{\lin{x,y-\alpha x + \alpha x}}^2 =
\abs{\alpha}^2\abs{\lin{x,x}}^2 < \lin{x,x}\lin{y,y}.\qedhere
\end{align*}
\end{proofenum}
\end{proof}

\begin{prop}
\label{prop:5.3}
Sei $\lin{\cdot,\cdot}$ ein (Semi)-Skalarprodukt auf $L$. Dann definiert
\begin{align*}
\norm{\cdot}: L\to \R,\quad x\mapsto \sqrt{\lin{x,x}}
\end{align*}
eine (Halb)-Norm.\fishhere
\end{prop}
\begin{proof}
Die Abbildung ist wohldefiniert und (N1), (N2) und (N3) sind klar. Zu (N4)
betrachte,
\begin{align*}
\norm{x+y}^2  &=\lin{x+y,x+y} = \lin{x,x} + 2\Re \lin{x,y} + \lin{y,y}\\
&\le \norm{x}^2 + \norm{y}^2 + 2\abs{\lin{x,y}}
=  \norm{x}^2 + \norm{y}^2 + 2\sqrt{\lin{x,x}\lin{y,y}}\\
&\le \norm{x}^2 + \norm{y}^2 + 2\norm{x}\norm{y}
= \left(\norm{x}+\norm{y}\right)^2.\qedhere
\end{align*}
\end{proof}

\begin{bem}[Nota Bene.]
\label{bem:5.4}
$\abs{\lin{x,y}}\le \norm{x}\norm{y}$.\maphere
\end{bem}

\begin{defn}
\label{defn:5.5}
Sei $H$ ein linearer Raum, auf dem ein Skalarprodukt $\lin{x,x}$
existiert, versehen mit der durch das Skalarprodukt induzierten Norm
\begin{align*}
\norm{x}_H := \sqrt{\lin{x,x}}. 
\end{align*}
Dann heißt $H$ \emph{Prähilbertraum}\index{Hilbertraum!Prä-}. Ist $H$ außerdem
Banachraum, so heißt $H$ \emph{Hilbertraum}\index{Hilbertraum}. Schreibe auch
$(H,\lin{\cdot,\cdot})$.\fishhere
\end{defn}

\begin{bsp}
\label{bsp:5.6}
\begin{bspenum}
  \item $H=\C^n$ mit Skalarprodukt $\lin{x,y}=\sum_{j=1}^n
  x_j\overline{y_j}$ und der dadurch induzierten Norm $\norm{x}=
  \left(\sum_{j=1}^n \abs{x_j}^2\right)^{1/2}$ ist ein Hilbertraum.
  \item $H = l^2$ mit dem Skalarprodukt $\lin{(x_n),(y_n)} =
  \sum_{j=1}^\infty x_j \overline{y_j}$ und der dadurch induzierten
  2-Norm $\norm{(x_n)}_2 = \left(\sum_{j=1}^\infty
  \abs{x_j}^2\right)^{1/2}$ ist ebenfalls ein Hilbertraum.
  \item $H=C([0,1]\to\C)$ mit dem Skalarprodukt $\lin{f,g} = \int_{[0,1]}
  f\overline{g}\dmu$ und der dadurch induzierten 2-Norm
  $\norm{f} = \left(\int_{[0,1]} \abs{f}^2 \dmu\right)^{1/2}$ ist lediglich
  ein Prähilbertraum. Seine Vervollständigung ist der $L^2([0,1]\to\C)$.\bsphere
\end{bspenum}
\end{bsp}

\begin{lem}
\label{lem:5.7}
Das Skalarprodukt ist in beiden Argumenten stetig bezüglich der induzierten
Norm.\fishhere
\end{lem}
\begin{proof}
Seien also $(x_n)$, $(y_n)$ Folgen in $H$, $x,y\in H$ sowie
$\norm{x_n-x},\norm{y_n-y}\to 0$. Dann gilt
\begin{align*}
\abs{\lin{x_n,y_n}-\lin{x,y}} &=\abs{\lin{x_n-x,y_n}- \lin{x,y_n-y}}
\\ &\le \underbrace{\norm{x_n-x}}_{\to0}\underbrace{\norm{y_n}}_{\le C} +
\norm{x}\underbrace{\norm{y_n-y}}_{\to0} \to 0.\qedhere
\end{align*}
\end{proof}

\begin{prop}
\label{prop:5.8}
Ist $H$ ein Prähilbertraum, so ist die Vervollständigung $\overline{H}$ (siehe
\ref{prop:1.16}) ein Hilbertraum.\fishhere
\end{prop}
\begin{proof}
$H$ wird mit der induzierten Norm zum normierten Raum. Mit Satz \ref{prop:1.16}
erhalten wir somit die eindeutige Vervollständigung $\overline{H}$. Definiere
nun für $x,y\in\overline{H}$
\begin{align*}
\lin{x,y}_{\overline{H}} := \lim\limits_{n\to\infty}\lin{x_n,y_n}_H, &&
x_n\to x,\quad y_n\to y.
\end{align*}
Der Limes ist unabhängig von den gewählten Folgen $(x_n)$ und $(y_n)$, also ist
$\lin{\cdot,\cdot}_{\overline{H}}$ ein Skalarprodukt und die Norm auf
$\overline{H}$ induziert durch ebendieses.\qedhere
\end{proof}

Der folgende Satz macht eine Aussage darüber, wie man --- falls ein
Skalarprodukt auf $H$ existiert --- dieses aus der induzierten Norm
zurückerhält.
\begin{prop}
\label{prop:5.9}
\begin{propenum}
  \item 
Sei $H$ ein Prähilbertraum. Dann gilt die \emph{Polarisationsformel}
\begin{align*}
\lin{x,y} :=
%\begin{cases}
%\frac{1}{4}\left(\norm{x+y}^2 - \norm{x-y}^2\right), & \K=\R,\\
\frac{1}{4}\left(\norm{x+y}^2 - \norm{x-y}^2 + i\left(\norm{x+iy}^2 -
\norm{x-iy}^2\right)\right),% & \K=\C,
%\end{cases}
\end{align*}
wobei im Fall eines reellen Hilbertraums der Imaginärteil ignoriert wird.

Außerdem gilt die \emph{Parallelogrammgleichung},
\begin{align*}
\norm{x+y}^2 + \norm{x-y}^2 = 2\norm{x}^2 + 2\norm{y}^2.
\end{align*}
\item Ist $(E,\norm{\cdot})$ normierter Raum, dann ist $E$ genau dann ein
Prähilbertraum, wenn $\norm{\cdot}$ die Parallelogrammungleichung
erfüllt.\fishhere
\end{propenum}
\end{prop}

\begin{defn}
\label{defn:5.10}
\begin{defnenum}
  \item $x,y\in H$ heißen \emph{orthogonal}\index{orthogonal}, falls
  $\lin{x,y}=0$.
  \item $A,B\subseteq H$ heißen \emph{orthogonal}\index{Menge!orthogonal}, falls
  \begin{align*}
  \forall x\in A \forall y\in B : \lin{x,y}  = 0.
  \end{align*}
\item Sei $A\subseteq H$, dann heißt $A^\bot := \setdef{y\in H}{\forall x\in A
: \lin{x,y}=0}$ \emph{orthogonales
Komplement}\index{Orthogonales Komplement}.\fishhere
\end{defnenum} 
\end{defn}

\begin{bem}[Bemerkungen.]
\label{bem:5.11}
\begin{bemenum}
  \item Für $x,y\in H$ mit $x\bot y$ gilt der \emph{Satz des
  Pythagoras}\index{Satz!Pythagoras},
\begin{align*}
\norm{x+y}^2 = \lin{x+y,x+y} = \lin{x,x}+\lin{y,y} = \norm{x}^2 + \norm{y}^2.
\end{align*}
\item $A^\bot$ ist abgeschlossener linearer Teilraum von $H$, denn für $y,z\in
A^\top$, $x\in A$ und $\alpha\in \K$  gilt
\begin{align*}
\lin{y+z,x} =\lin{y,x}+\lin{z,x} = 0,\qquad
\lin{\alpha y,x} = \alpha\lin{y,x} =0.
\end{align*}
Sei nun $(y_n)$ Folge in $A^\bot$ mit Grenzwert $y\in H$, dann gilt für $x\in A$
\begin{align*}
\lin{y,x} = \lim\limits_{n\to\infty}\lin{y_n,x} = 0.
\end{align*}
Also ist $y\in A^\bot$, d.h. $A^\bot$ ist abgeschlossen.\maphere
\end{bemenum}
\end{bem}

\begin{prop}
\label{prop:5.12}
Sei $H$ Hilbertraum, $K\subseteq H$ abgeschlossen und konvex, $x_0\in H$ dann
gibt es genau ein $x\in K$, so dass
\begin{align*}
\norm{x-x_0} = d(x_0,K).\fishhere
\end{align*}
\end{prop}
\begin{proof}
\textit{Existenz}. Sei $(x_n)$ Folge in $K$ mit $\norm{x_n-x_0}\to d(x_0,K)$.
Dann ist $(x_n)$ Cauchyfolge, denn
\begin{align*}
\norm{x_n-x_m}^2 &= \norm{x_n-x_0-(x_m-x_0)}^2 \\ &\overset{\text{PG}}{=}
2\norm{x_n-x_0}^2 + 2\norm{x_m-x_0}^2 - 4\norm{x_0 - \frac{x_n+x_m}{2}}^2\\
&\le 2(d(x_0,K)^2 + \ep) + 2(d(x_0,K)^2 + \ep) - 4 d(x_0,K)^2
= 4\ep.  
\end{align*}
Also $x_n\to x$ in $H$, da $K$ abgeschlossen ist $x\in K$ und es gilt
\begin{align*}
\norm{x-x_0} = \lim\limits_{n\to\infty} \norm{x_n-x_0} = d(x_0,K).
\end{align*}
\textit{Eindeutigkeit}. Seien $x, \tilde{x} \in K$ und
$\norm{x-x_0}=\norm{\tilde{x}-x_0}=d(x_0,K)$. Betrachte
$x_n=(x,\tilde{x},x,\tilde{x},\ldots)$, dann ist nach obiger Rechnung $x_n$
Cauchy und hat daher einen eindeutigen Grenzwert, also $x=\tilde{x}$.\qedhere
\end{proof}

\begin{bem}[Vereinbarung.]
\label{bem:5.13}
Im Folgenden sei $H$ immer ein Hilbertraum.\maphere
\end{bem}

\begin{prop}[Projektionssatz]
\label{bem:5.14}
Sei $M$ abgeschlossener linearer Teilraum von $H$ und $x_0\in H$. Dann gilt
\begin{propenum}
  \item Für $y\in M : \norm{y-x_0}=d(x_0,M)$ genau dann, wenn $y-x_0\ \bot\ M$.
  \item Es gibt genau ein $u\in M$ und genau ein $v\in M^\bot$, so dass $x_0 =
  u+v$.
  \item Die Abbildung $P_M: H\to M,\; x_0\mapsto u$ ist linear und stetig.
  Falls $M\neq(0)$, so ist $\norm{P_M}=1$ und $P_M^2=P_M$. $P_M$ heißt
  \emph{orthogonale Projektion} auf $M$.\fishhere
\end{propenum}
\end{prop}
\begin{proof}
\begin{proofenum}
  \item Betrachte für $z\in M\setminus\setd{0}$, $y+\alpha\frac{z}{\norm{z}}$
  mit $\alpha:= \lin{x_0-y,\frac{z}{\norm{z}}}$.
\begin{align*}
d(x_0,M)^2 &\le \norm{x_0-\left(y+\alpha\frac{z}{\norm{z}}\right)}^2
= \norm{(x_0-y)-\alpha\frac{z}{\norm{z}}}^2\\
&= \norm{x_0-y}+\abs{\alpha}^2-2\Re\lin{x_0-y,\alpha\frac{z}{\norm{z}}} \\
&= \norm{x_0-y}^2 - \abs{\alpha}^2.
\end{align*}
Also ist $\norm{x_0-y}^2 = d(x_0,M)$ genau dann, wenn
\begin{align*}
&\forall z\in M\setminus\setd{0} : \lin{x_0-y,z} = 0\\
\Leftrightarrow &
\forall z\in M : \lin{x_0-y,z} = 0\\
\Leftrightarrow &
x_0-y\bot M.
\end{align*}
\item $M$ ist linearer Teilraum also insbesondere konvex. Mit \ref{prop:5.12}
folgt somit, dass genau ein $u\in M$ existiert, so dass
\begin{align*}
d(x_0,M) = \norm{u-x_0}.
\end{align*}
Mit a.) folgt dann auch  $v=u-x_0\in M^\bot$ und $v$ ist eindeutig, da $u$
eindeutig.
\item \textit{Linearität}. Seien $x_1=u_1+v_1$, $x_2=u_2+v_2$ mit der eben
bewiesenen Zerlegung. Dann ist
\begin{align*}
P_M(x_1+x_2) = P_M(u_1+u_2+v_1+v_2),
\end{align*}
wobei $u_1+u_2\in M$ und $v_1+v_2\in M^\bot$, also
\begin{align*}
\ldots = u_1+u_2 = P_M(x_1)+P(x_2).
\end{align*}
\textit{Stetigkeit}. $\norm{P_M(x)}^2 = \norm{u}^2\le \norm{u}^2 + \norm{v}^2
\overset{\text{Pyth.}}{=} \norm{u+v}^2 = \norm{x}^2$, also $\norm{P_M}\le 1$.
Sei weiterhin $x\in M\setminus\setd{0}$, dann ist $\norm{P_M(x)} = \norm{x}$,
also ist $\norm{P_M}=1$.

Sei $x=u+v$, dann ist $P_Mx = u$ und $P_M u = u$, also
$P_M^2=P_M$.\qedhere
\end{proofenum}
\end{proof}

\begin{bem}
\label{bem:5.15}
Zu $x\in H$ sei
\begin{align*}
\ph_x : H\to \K,\quad y\mapsto \lin{y,x},
\end{align*}
dann ist $\ph_x$ offensichtlich linear. $\ph_x$ ist außerdem stetig, denn
\begin{align*}
\abs{\ph_x(y)} = \abs{\lin{y,x}}\le \norm{x}\norm{y}.
\end{align*}
Da $\lin{x,x}=\norm{x}^2$ gilt sogar $\norm{\ph_x}=\norm{x}$ also ist 
$\ph_x$ Element von $H'$.\maphere
\end{bem}

\begin{lem}[Lemma von Riesz]
\label{prop:5.16}
Zu jedem $f\in H'$ existiert genau ein $x\in H$, so dass
\begin{align*}
f = \ph_x =  \lin{\cdot,x}.\fishhere
\end{align*} 
\end{lem}
\begin{proof}
\begin{proofenum}
  \item \textit{Konstruktion von $x$}. Der Fall $f=0$ ist trivial. Sei also
  $f\neq 0$, dann gibt es ein $\tilde{x}\in H$, so dass $f(\tilde{x})\neq 0$.
  Wenden wir nun den Projektionssatz an, erhalten wir $\tilde{x}=u+v$ mit
  $u\in\ker f$ und $v\in \ker f^\bot$.
  
  Zu $\lambda > 0$ soll sein
  \begin{align*}
  \lambda f(v) = f(\lambda v) \overset{!}{=}\ph_{\alpha v}(\lambda v) =
  \lin{\lambda v,\alpha v} = \lambda\overline{\alpha}\norm{v}^2.
  \end{align*}
Setze also $\alpha = \dfrac{\overline{f(v)}}{\norm{v}^2}$ und $x=\alpha v$ dann
ist $f(x)\neq 0$.

\item\textit{Zeige $f=\ph_x$}. Sei $y\in H$, so schreibe $y=y-\beta x+ \beta x$,
wobei $\beta = \frac{f(y)}{f(x)}$, dann ist
\begin{align*}
f(y) = f(y-\beta x) + \beta f(x) = f(y-\beta x) + f(y) \Rightarrow f(y-\beta x)
= 0,
\end{align*}
also ist $y-\beta x\in \ker f$ und somit $0=\lin{y-\beta x,x}$.
\begin{align*}
f(y) &= \beta f(x) = \beta\lin{x,x} = \lin{y,x}.
\end{align*}
\item\textit{Eindeutigkeit}. Seien $x,\tilde{x}\in H$ so , dass
\begin{align*}
\forall y\in H : \lin{y,x} = f(y) = \lin{y,\tilde{x}}.
\end{align*}
Dann gilt auch
\begin{align*}
\forall y\in H : \lin{x-\tilde{x},y} = 0
\end{align*}
also $x-\tilde{x}=0$.\qedhere
\end{proofenum}
\end{proof}

\begin{bem}[Diskussion.]
\label{bem:5.17}
\begin{bemenum}
  \item Ist $f\in H'$, dann existiert ein $x\in H$, so dass $\ker f =
  \setd{x}^\bot$.
  \item Es gilt die \emph{Dimensionsformel},
\begin{align*}
\dim \ker f^\bot =  \dim\K = 1.\maphere 
\end{align*}
\end{bemenum}
\end{bem}

\begin{cor}
\label{prop:5.18}
\begin{propenum}
  \item Die Abbildung $H\to H',\; x\mapsto \ph_x$ ist eine bijektive
  antilineare Isometrie und $H'$ ist ein Hilbertraum mit
\begin{align*}
\lin{\ph_x,\ph_y}_{H'} = \lin{y,x}.
\end{align*}
Man sagt auch $H$ ist selbstdual.
\item Jeder Hilbertraum ist reflexiv.\fishhere
\end{propenum}
\end{cor}
\begin{proof}
\begin{proofenum}
  \item Die Bijektivität folgt aus \ref{bem:5.15}, \ref{prop:5.16}. Zur
  Antilinearität betrachte,
\begin{align*}
\ph_{\alpha x_1+\beta x_2}(y) &= \lin{y,\alpha x_1+\beta x_2}
= \overline{\alpha}\lin{y,x_1}  + \overline{\beta}\lin{y,x_2}\\
&= \overline{\alpha}\ph_{x_1}(y) + \overline{\beta}\ph_{x_2}(y).
\end{align*}
Die Isometrieeigenschaft haben wir in \ref{bem:5.17} nachgewiesen.

$H'$ ist stets ein Banachraum, da $\K$ vollständig. $\lin{\cdot,\cdot}_{H'}$
ist Skalarprodukt, denn die positive Definitheit folgt direkt aus
$\lin{\cdot,\cdot}$. Zur Linearität betrachte beispielsweise,
\begin{align*}
\lin{\alpha \ph_x,\ph_y}_{H'} = \lin{\ph_{\overline{\alpha}x},\ph_y} =
\lin{y,\overline{\alpha}x} = \alpha\lin{y,x} = \alpha\lin{\ph_x,\ph_y}_{H'}.
\end{align*} 
Die Norm auf $H'$ wird durch $\lin{\cdot,\cdot}_{H'}$ induziert, denn
\begin{align*}
\lin{\ph_x,\ph_x}_{H'} = \lin{x,x} = \norm{x}^2 = \norm{\ph_x}^2.
\end{align*}
\item Zeige $J_H: H\to H''$ ist surjektiv. Da $H$ Hilbertraum, gilt
\begin{align*}
J_H(x)(\ph_y) = \ph_y(x) = \lin{x,y}.
\end{align*}
$H'$ ist ebenfalls ein Hilbertraum, also folgt für $\phi\in H''$, dass genau ein
$\ph_x\in H'$ existiert, so dass gilt
\begin{align*}
\forall \ph_y\in H' : \phi(\ph_y) = \lin{\ph_y,\ph_x}_{H'}.
\end{align*}
$x$ ist dadurch eindeutig bestimmt und für $\ph_y\in H'$ folgt,
\begin{align*}
J_H(x)(\ph_y) = \ph_y(x) = \lin{x,y} = \lin{\ph_y,\ph_x}_{H'} =
\phi(x,y).\qedhere
\end{align*}
\end{proofenum}
\end{proof}

\begin{figure}[!htpb]
\centering
\begin{pspicture}(0,-1.69)(5.98,1.71)
\pscircle(0.76,0.27){0.76}
\pscircle(3.02,0.27){0.76}
\pscircle(5.22,0.27){0.76}
\psbezier[linecolor=darkblue]{->}(0.9,0.81)(1.24,1.35)(2.02,1.49)(2.6,0.95)
\psbezier[linecolor=darkblue]{->}(3.14,0.81)(3.48,1.35)(4.26,1.49)(4.84,0.95)

\rput(0.76,0.255){\color{gdarkgray}$H$}
\rput(3.02,0.255){\color{gdarkgray}$H'$}
\rput(5.22,0.255){\color{gdarkgray}$H''$}
\psbezier[linecolor=purple]{->}(0.82,-0.25)(1.76,-1.67)(3.9,-1.43)(4.68,-0.33)
\rput(1.74,1.515){\color{darkblue}$F:x\mapsto \ph_x$}
\rput(3.99,1.515){\color{darkblue}$G:f\mapsto\ph_f$}
\rput(2.8,-0.945){\color{purple}$J_H=G\circ F$}
\end{pspicture} 
\caption{Zum Reflexivitätsbeweis.}
\end{figure}

\begin{corn}[Schwache Konvergenz]
In einem Hilbertraum gilt
\begin{align*}
x_n\wto x \Leftrightarrow \forall y\in H: \lin{x_n,y}\to \lin{x,y}.\fishhere
\end{align*}
\end{corn}

\begin{defn}
\label{defn:5.19}
Sei $H$ ein Prä-Hilbertraum. Eine Menge $\setd{e_i}_{i\in\II}\subseteq H$ mit 
Indexmenge $\II$, heißt \emph{Orthonormalsystem (ONS)}\index{orthogonal!ONS},
falls
\begin{align*}
\forall i,j\in\II : \lin{e_i,e_j} = \delta_{ij}.\fishhere
\end{align*}
\end{defn}

\begin{bsp}
\label{bsp:5.20}
\begin{bspenum}
  \item $H=l^2$ besitzt das ONS $\setd{e_1,e_2,\ldots}$ mit
  $e_k=(0,\ldots,0,1,0,\ldots)$.
  \item Betrachte $H=L^2([-1,1])$ mit dem Skalarprodukt $\lin{f,g}=\int_{[-1,1]}
  f\overline{g}\dmu$.
  
  Orthonormalisiere die Folge $x^n$ mit Hilfe des Gram-Schmidtschen
  Orthonormalisierungsverfahren. Hierbei ergeben sich die
  Legendre-Polynome.\bsphere
\end{bspenum}
\end{bsp}

\begin{prop}
\label{prop:5.21}
Sei $H$ ein Hilbertraum, $\setd{e_i}_{i\in\II}$ ein ONS,
$M:=\overline{\lin{e_i\ :\ i\in \II}}$. Dann gelten
\begin{propenum}
  \item Für $x\in H$ ist
\begin{align*}
I_x := \setdef{i\in I}{\lin{x,e_i}\neq0} 
\end{align*}
höchstens abzählbar und es gilt,
\begin{align*}
\sum\limits_{i\in\II} \abs{\lin{x,e_i}}^2 := \sum\limits_{i\in I_x}
\abs{\lin{x,e_i}}^2 \le \norm{x}^2,\qquad\text{\emph{Besselsche Ungleichung}}.
\end{align*}
\item Die orthogonale Projektion\index{orthogonal!Projektion} $P_M$ auf $M$ ist
gegeben durch,
\begin{align*}
P_M x := \sum\limits_{i\in\II_x} \lin{x,e_i}e_i
\end{align*}
und unabhängig von der Reihenfolge in der Reihe. Die $\lin{x,e_i}$ heißen
\emph{Fourierkoeffizienten}\index{Fourierkoeffizienten} von $x$.\fishhere
\end{propenum}
\end{prop}
\begin{proof}
\begin{proofenum}
  \item \text{Sei $\II=\setd{1,\ldots,n}$ endlich}. Dann hat $M$ die Form,
\begin{align*}
M := \setdef{\sum\limits_{j=1}^n \alpha_j e_j}{\alpha_j\in\K}. 
\end{align*} 
Für $x\in M$ und $j\in\II$ gilt dann
\begin{align*}
\lin{x-\sum\limits_{i=1}^n \lin{x,e_i}e_i,e_j} = \lin{x,e_j}- \lin{x,e_j} = 0.
\end{align*}
Also ist $x-\sum_{i=1}^n \lin{x,e_i}e_i\in M^\bot$, da aber auch
\begin{align*}
x = \underbrace{x-\sum\limits_{i=1}^n \lin{x,e_i}e_i}_{\in M^\bot} +
\underbrace{\sum\limits_{i=1}^n \lin{x,e_i}e_i}_{\in M}
\end{align*}
folgt nach dem Projektionssatz, $P_Mx=\sum\limits_{i=1}^n
\lin{x,e_i}e_i$. Somit ist
\begin{align*}
\norm{x}^2 &\ge \norm{P_M x}^2 = \lin{\sum\limits_{i=1}^n
\lin{x,e_i}e_i,\sum\limits_{j=1}^n \lin{x,e_j}e_j}\\
&= \sum\limits_{i,j=1}^n \lin{x,e_i}\overline{\lin{x,e_j}}\lin{e_i,e_j}
= \sum\limits_{i=1}^n \abs{\lin{x,e_i}}^2.
\end{align*}
und es folgen 1.) und 2.).
\item \textit{Sei $\II$ unendlich}. Setze
\begin{align*}
I_n(x) := \setdef{i\in \II}{\abs{\lin{x,e_i}}^2\ge\frac{1}{n}},
\end{align*}
dann ist $\card I_n(x) \le n \norm{x}^2$, denn falls $\card I_n(x)\ge m >
n\norm{x}^2$, so existieren $i_1,\ldots,i_m\in\II$ mit $\abs{\lin{x,e_{i_k}}}^2
\ge \frac{1}{n}$ und dann gilt
\begin{align*}
\sum\limits_{k=1}^m \abs{\lin{x,e_{i_k}}}^2 \ge \frac{m}{n} > \norm{x}^2
\end{align*}
im Widerspruch zur Besselschen Ungleichung angewandt auf das endliche ONS
$(e_{i_1},\ldots,e_{i_m})$.

Somit ist $I(x) = \bigcup_{n\in\N} I_n(x)$ höchstens abzählbar. Sei nun
$I(x)=\setd{e_{i_1},e_{i_2},\ldots}$, dann folgt aus 1.) für jedes
$n\in\N$,
\begin{align*}
\sum\limits_{j=1}^n \abs{\lin{e_{i_j},x}}^2 \le \norm{x}^2.
\end{align*}
Die rechte Seite ist nun unabhängig von $n$, der Übergang zu $n\to\infty$
liefert die Besselsche Ungleichung.

Mit der Besselschen Ungleichung und dem Satz von Pythagoras, folgt nun
\begin{align*}
\norm{\sum\limits_{j=n}^m \lin{x,e_{i_j}}e_{i_j}}^2 = 
\sum\limits_{j=n}^m \abs{\lin{x,e_{i_j}}}^2 < \ep,
\end{align*}
für $n,m>N_\ep$, somit ist die Reihe Cauchy, also konvergent.

$H$ ist Hilbertraum, also ist $y=\sum\limits_{j=1}^\infty
\lin{x,e_{i_j}}e_{i_j}\in H$ und es gilt,
\begin{align*}
\lin{x-y,e_k}
=
\begin{cases}
0, & \text{falls } \forall j\in I(x) : k\neq j,\\
\lin{x,e_k}-\lin{y,e_k}, & \text{sonst},
\end{cases}
\end{align*}
wobei auch
\begin{align*}
\lin{x,e_k}-\lin{y,e_k}&=
\lin{x,e_k}-\lin{\sum\limits_{j=1}^\infty\lin{x,e_{i_j}}e_{i_j},e_k}\\
&= \lin{x,e_k}-\lin{x,e_k} = 0.
\end{align*}
Somit ist $x=(x-y)+y$, mit $y\in M$ und $x-y\in M^\bot$.
Aufgrund des Projektionssatzes ist $P_M(x) = y$ und die Projektion ist
unabhängig von der Abzählung der $e_{i_j}$, also auch $y$ und man darf in der
Reihe umsortieren.\qedhere
\end{proofenum}
\end{proof}

\begin{prop}
\label{prop:5.22}
Sei $(e_i)_{i\in\II}$ ein ONS im Hilbertraum $H$, dann sind äquivalent
\begin{propenum}
  \item\label{prop:5.22:1} $\lin{e_i : i\in \II}$ liegt dicht in $H$,
  \item\label{prop:5.22:2} $\forall x\in H : x=\sum_{i\in\II}
  \lin{x_i,e_i}e_i$,
  \item\label{prop:5.22:3} $\forall x\in H : \norm{x}^2 = \sum_{i\in\II}
  \abs{\lin{x,e_i}}^2$,\qquad \emph{Parsevallsche Gleichung},
  \item\label{prop:5.22:4} $\forall x,y\in H : \lin{x,y} = \sum_{i\in\II}
  \lin{x,e_i}\lin{e_i,y}$.
\end{propenum}
Ist eine der obigen Aussagen erfüllt, heißt das ONS $(e_i)$
\emph{vollständig (VONS)}\index{orthogonal!VONS} oder
\emph{Orthonormalbasis}\index{orthogonal!Basis}.\fishhere
\end{prop}

Orthonormalbasen in unendlichdimensionalen Räumen unterscheiden sich von denen
in endlichdimensionalen Räumen insbesondere dadurch, dass \textit{nicht} jedes
Element als \textit{endliche} Linearkombination von Basisvektoren dargestellt werden
kann, sondern durchaus unendliche Reihen zugelassen sind.

\begin{proof}
``\ref{prop:5.22:1}$\Leftrightarrow$\ref{prop:5.22:2}'': Sei $M:=
\overline{\lin{e_i:i\in\II}}$, dann ist
\begin{align*}
\text{\ref{prop:5.22:1}} \Leftrightarrow M = H \Leftrightarrow
\forall x\in H : P_Mx = x
\overset{\ref{prop:5.21}}{\Leftrightarrow}
\forall x\in H : \sum\limits_{i\in\II} \lin{x,e_i}e_i = x \Leftrightarrow
\text{\ref{prop:5.22:2}}.
\end{align*}
``\ref{prop:5.22:2}$\Rightarrow$\ref{prop:5.22:3}'': Sei also $x\in H$, so
gilt
\begin{align*}
&x = \sum\limits_{k=1}^\infty \lin{x,e_{i_k}}e_{i_k}
=
\lim\limits_{K\to\infty} \sum\limits_{k=1}^K \lin{x,e_{i_k}}e_{i_k},\\
\Rightarrow & 
\norm{x}^2 =  \lim\limits_{K\to\infty} \norm{\sum\limits_{k=1}^K
\lin{x,e_{i_k}}e_{i_k}}^2 \overset{\text{Pyth.}}{=}
\lim\limits_{K\to\infty} 
\sum\limits_{k=1}^K \abs{\lin{x,e_{i_k}}}^2.
\end{align*}
``\ref{prop:5.22:3}$\Rightarrow$\ref{prop:5.22:2}'':
Analog zur obigen Rechnung folgt,
\begin{align*}
\norm{P_Mx}^2 = \sum\limits_{k=1}^\infty \abs{\lin{x,e_{i_k}}}^2
\end{align*}
und somit insbesondere $\norm{x}^2 =
\norm{P_Mx}^2$. Der Projektionssatz liefert eine Zerlegung $x=u+v$, wobei
$u=P_Mx\in M$, $v\in M^\bot$. Nun ist 
\begin{align*}
\norm{v}^2 =
\norm{x}^2-\norm{u}^2=0\Rightarrow x = u = \sum\limits_{k=1}^\infty
\lin{x,e_{i_k}}e_{i_k}.
\end{align*}

``\ref{prop:5.22:3}$\Rightarrow$\ref{prop:5.22:4}'': Aus \ref{prop:5.22:3}
folgt für $x,y\in H$ die Darstellung,
\begin{align*}
x = \sum\limits_{k=1}^\infty \lin{x,e_{i_k}}e_{i_k},\qquad
y=\sum\limits_{k=1}^\infty \lin{y,e_{i_k}}e_{i_k}.
\end{align*}
Mit der Stetigkeit des Skalarproduktes folgt schließlich
\begin{align*}
\lin{x,y} = \sum\limits_{k=1}^\infty\sum\limits_{l=1}^\infty
\lin{x,e_{i_k}}\overline{\lin{y,e_{i_l}}}\underbrace{\lin{e_{i_k},e_{i_l}}}_{\delta_{il}}
= \sum\limits_{k=1}^\infty \lin{x,e_{i_k}}\lin{e_{i_k},y}.
\end{align*}

``\ref{prop:5.22:4}$\Rightarrow$\ref{prop:5.22:3}'': Klar.\qedhere
\end{proof}

\begin{prop}
\label{prop:5.23}
Ist $H\neq(0)$ ein Hilbertraum, so besitzt $H$ ein VONS.\fishhere
\end{prop}
\begin{proof}
Sei $\AA:= \setdef{E\subseteq H}{E\text{ ist ONS}}$. Dann gilt
\begin{proofenum}
  \item $\AA\neq \varnothing$, denn $\exists x\in H\setminus\setd{0}$ also
  $\setd{\frac{x}{\norm{x}}}\in\AA$.
  \item $\AA$ ist durch $\subseteq$ halbgeordnet.
  \item Sei $\KK\subseteq\AA$ eine Kette (d.h. durch $\subseteq$ total
  geordnet). Setze $E_0 := \bigcup_{E\in\KK} E$, dann ist $E_0$ trivialerweise ein ONS und
  offensichtlich obere Schranke.
\end{proofenum}
Anwendung des Lemmas von Zorn ergibt, dass $\AA$ mindestens ein maximales
Element $E_1$ besitzt, d.h.
\begin{align*}
\forall E\in \AA : E_1\subseteq E \Rightarrow E=E_1.
\end{align*}
Wir haben nun zu zeigen, dass das ONS $E_1$ auch vollständig ist. Angenommen es
gibt ein $x\in H\setminus\overline{\lin{E_1}}$, dann ist $x\neq 0$. Mit dem
Projektionssatz erhalten wir die Zerlegung,
\begin{align*}
x = u+v,\qquad u\in \overline{\lin{E_1}},\quad v\in \overline{\lin{E_1}}^\bot.
\end{align*}
Dann ist $v\neq 0$ und $\tilde{E}=E_1\cup\setd{\frac{v}{\norm{v}}}$ ein ONS,
d.h. es gibt ein $\tilde{E}\in \AA$ mit $E_1\subseteq \tilde{E}$ und
$\tilde{E}\neq E_1$. Dies ist eine Widerspruch zur Maximalität von $E_1$.

Somit ist $E_1$ vollständig. Wir haben tatsächlich mehr gezeigt, 
denn jedes maximale Element von $\AA$ ist liefert ein VONS und umgekehrt ist
jedes VONS ein maximales Element von $\AA$.\qedhere
\end{proof}

\begin{bsp}
\label{bsp:5.24}
\begin{bspenum}
  \item Sei $H=l^2$ und $e_k = (0,\ldots,0,1,0,\ldots)$, dann ist
  $(e_k)_{k\in\N}$ ein VONS.
  \item Sei $H=L^2([-1,1])$ und $e_k : x\mapsto \frac{1}{2\pi}e^{ikx\pi}$, dann
  ist $(e_k)_{k\in\N}$ ein VONS.\bsphere
\end{bspenum}
\end{bsp}

\begin{bem}[Bemerkungen.]
\label{bem:5.25}
Sei $H$ ein Hilbertraum.
\begin{bemenum}
  \item $H$ ist genau dann separierbar, wenn $H$ ein abzählbares VONS besitzt.
\begin{proof}
``$\Rightarrow$'': Sei $H=\overline{\lin{x_1,x_2,\ldots}}$. Wende das
Gram-Schmidtsche Orthonormalisierungsverfahren auf $\setd{x_1,x_2,\ldots}$ an.

``$\Leftarrow$'': Sei $\setd{e_1,e_2,\ldots}$ abzählbares VONS, dann ist
\begin{align*}
M := \bigcup_{n\in\N} \setdef{\sum\limits_{j=1}^n (\alpha_j + i
\beta_j)e_j}{\alpha_j,\beta_j\in\Q}
\end{align*}
abzählbar und dicht.\qedhere 
\end{proof}
  \item Falls $H$ separierbar, existiert ein Hilbertraumisomorphismus
\begin{align*}
\ph: H\to l^2.
\end{align*}
\begin{proof}
Wir erhalten wie in 1.) ein VONS $\setd{e_1,e_2,\ldots}$. Setze dann
\begin{align*}
\phi: x=\sum\limits_{j=1}^\infty \lin{x,e_j}e_j \mapsto
\left(\lin{x,e_j}\right)_{j\in\N} \in l^2.
\end{align*}
Aus \ref{prop:5.22} folgt, $\lin{x,y}_H =
\lin{\phi(x),\phi(y)}_{l^2}$.\qedhere\maphere
\end{proof}
\end{bemenum}
\end{bem}

\clearpage
\chapter{Lineare Operatoren auf normierten Räumen}

Sofern nicht anders angegeben seien $E$ und $F$ normierte Räume mit Normen
$\norm{\cdot}_E$, $\norm{\cdot}_F$. Sollte aus dem Zusammenhang hervorgehen,
welcher Raum gemeint ist, wird auch nur $\norm{\cdot}$ verwendet.

\section{Spektraltheorie beschränkter Operatoren}

In diesem Abschnitt studieren wir zunächst die beschränkten Operatoren. Meißt
sind die unbeschränkten Operatoren (wie z.B. Differentialoperatoren)
von größerem Interesse, in einigen Fällen lassen sie sich jedoch zu
einem beschränkten Operator invertieren, auf den wir dann die hier entwickelte
Theorie anwenden können.

\begin{defn}
\label{defn:6.1}
Sei $E$ Banachraum und $A\in\LL(E)$.
\begin{defnenum}
  \item $\lambda\in\K$ heißt \emph{regulär} bezüglich $A$, falls
\begin{align*}
A-\lambda\Id : E\to E
\end{align*}
bijektiv. (Dann ist $(A-\lambda\Id)^{-1}\in\LL(E)$)
\item Die Menge
\begin{align*}
\rho(A) := \setdef{\lambda\in\K}{\lambda\text{ regulär bezüglich $A$}}
\end{align*}
heißt \emph{Resolventenmenge}\index{Resolventen!-Menge} von $A$. Die Abbildung
\begin{align*}
R: \rho(A)\to\LL(E),\quad \lambda\mapsto R(\lambda) = (A-\lambda\Id)^{-1}
\end{align*}
heißt \emph{Resolventenfunktion}\index{Resolventen!-Funktion} von $A$.
\item Die Menge
\begin{align*}
\sigma(A) := \K\setminus\rho(A)
\end{align*}
heißt \emph{Spektrum}\index{Spektrum} von $A$; $\lambda\in\sigma(A)$
heißt \emph{Spektralwert von $A$}. $\lambda\in\K$ heißt \emph{Eigenwert
(EW)}\index{Eigen!-wert} von $A$, falls $\ker (A-\lambda\Id)\neq (0)$. Der Kern
von $A-\lambda\Id$ heißt \emph{Eigenraum}\index{Eigen!-raum} zum EW
$\lambda$.\fishhere
\end{defnenum}
\end{defn}

\begin{bsp}
\label{bsp:6.2}
\begin{bspenum}
  \item Sei $E=\C^n$ und $A\in\LL(E)$, so besitzt $A$ in Koordinaten eine 
  Darstellung als $n\times n$ Matrix. $\lambda$ ist genau dann
  Eigenwert, wenn ein $x\neq 0$ existiert mit $Ax=\lambda x$. Dann ist $A-\lambda \Id$ nicht injektiv also $\lambda\in\sigma(A)$.

Ist $\lambda$ kein Eigenwert, dann ist $\ker(A-\lambda\Id)=(0)$ und damit
$A-\lambda\Id$ injektiv. Die Dimensionsformel besagt nun, dass
$\dim\ker(A-\lambda\Id)+\dim\im (A-\lambda\Id) = n$, also ist $A-\lambda\Id$
auch surjektiv und somit $\lambda\in\rho(A)$.

In diesem Fall ist $\sigma(A) := \setdef{\lambda\in\C}{\lambda \text{ ist
Eigenwert von }A}$ und $\card\sigma(A)\le n$.
  \item Sei $E=l^2$. Betrachte den Shift-Operator 
\begin{align*}
S:(x_1,x_2,\ldots) \mapsto (0,x_1,x_2,\ldots).
\end{align*}
$(1,0,0,\ldots)\notin\im (S)$, also ist $S-0\Id$ nicht surjektiv, d.h.
$0\in\sigma(S)$. Aber $0$ ist kein Eigenwert, denn
\begin{align*}
Sx = 0\Leftrightarrow (0,x_1,x_2,\ldots) = 0\Leftrightarrow x=0,
\end{align*}
Tatsächlich besitzt $S$ keine Eigenwerte aber $\sigma(S) =
\overline{K_1(0)}$.\bsphere 
\end{bspenum}
\end{bsp}

\begin{lem}
\label{prop:6.3}
Seien $E,F$ Banachräume, $T\in\LL(E\to F)$ bijektiv, $S\in\LL(E\to F)$ mit
$\norm{S-T}< \norm{T^{-1}}^{-1}$, dann ist $S$ bijektiv.\fishhere
\end{lem}
\begin{proof}
Der Beweis wird in den Übungen behandelt.\qedhere
\end{proof}

\begin{cor}
\label{prop:6.4}
$\rho(A)$ ist offen.\fishhere
\end{cor}
\begin{proof}
Sei $\lambda\in\rho(A)$, d.h. $T:=A-\lambda\Id$ ist invertierbar. Mit
\ref{prop:6.3} folgt, dass $S=A-z\Id$ invertierbar, also $z\in\rho(A)$, falls
\begin{align*}
\norm{S-T} = \norm{A-z\Id-(A-\lambda\Id)} = \abs{z-\lambda} <
\norm{(A-\lambda\Id)^{-1}}^{-1} =: r > 0.
\end{align*}
Also ist $K_r(\lambda)\subseteq \rho(A)$.\qedhere
\end{proof}

\begin{prop}
\label{prop:6.5}
Sei $E$ Banachraum, $A\in\LL(E)$. Dann ist die Resolventenfunktion $R$
analytisch auf $\rho(A)$, d.h. lokal als Potenzreihe mit Koeffizienten in
$\LL(E)$ entwickelbar:
\begin{align*}
\forall \lambda\in\rho(A) \exists r > 0 \forall z\in K_r(\lambda) :
R(z) = (A-z\Id)^{-1} = \sum\limits_{n=0}^\infty A_n(z-\lambda)^n
\end{align*}
mit $A_n\in\LL(E)$.\fishhere
\end{prop}
\begin{proof}
Der Beweis wird in den Übungen behandelt.\qedhere
\end{proof}

\begin{prop}
\label{prop:6.6}
Sei $E$ ein Banachraum über $\C$ und $A\in\LL(E)$. Dann gilt
\begin{propenum}
  \item\label{prop:6.6:1} $\sigma(A)\neq \varnothing$,
  \item\label{prop:6.6:2} $\sigma(A) \subseteq
  \setdef{z\in\C}{\abs{z}\le\norm{A}}$,
  \item\label{prop:6.6:3} $\sigma(A)$ ist kompakt.\fishhere
\end{propenum}
\end{prop}
\begin{proof}
``\ref{prop:6.6:2}'': Sei $\abs{\lambda}> \norm{A}$. Wir 
zeigen, dass $A-\lambda\Id$ invertierbar ist. Dazu wenden wir Satz
\ref{prop:6.3} auf $T=\Id$ und $S=\Id-\frac{1}{\lambda}A$ an, dann ist $S$
invertierbar, falls
\begin{align*}
\norm{S-T} = \frac{1}{\abs{\lambda}}\norm{A} < \norm{\Id^{-1}}^{-1}
= 1.
\end{align*}
``\ref{prop:6.6:3}'': Aus \ref{prop:6.4} folgt, dass $\sigma(A) = \rho(A)^c$
abgeschlossen und mit \ref{prop:6.6:2} folgt, dass $\sigma(A)$ beschränkt ist.

``\ref{prop:6.6:1}'': Angenommen $\sigma(A)=\varnothing$, d.h. für jedes
$\lambda_0\in\C$ ist $A-\lambda_0\Id$ bijektiv. Wir bezeichnen die
Entwicklung der Resolventenfunktion um $\lambda_0$ als $R_{\lambda_0}$ und
zeigen nun, dass
\begin{align*}
\forall \lambda_0 \in \C \forall f\in E' : f(R_{\lambda_0}(z)) = 0.
\end{align*} 
Mit \ref{bem:4.14} folgt somit $\forall z\in\C :
R_{\lambda_0}(z)=0$, dies ist jedoch ein Widerspruch dazu, dass
$R(\lambda)=(A-z\Id)^{-1}$.

Sei also $f\in E'$ und $\ph(z)=f(R(z))$, so existiert $\ph:\C\to\C$.
\begin{defnenum}
  \item \textit{$\ph$ ist holomorph}. Sei $\lambda_0\in\C$, so existiert ein
  $\ep >0$, so dass nach \ref{prop:6.5}
\begin{align*}
R(\lambda) = \sum\limits_{j=0}^\infty A_n (\lambda-\lambda_0)^j,\qquad
\text{für } \abs{\lambda-\lambda_0}<\ep.
\end{align*}
Da $f$ stetig und linear folgt somit auch
\begin{align*}
\ph(\lambda) = \sum\limits_{j=0}^\infty (\lambda-\lambda_0)^j f(A_n),\qquad
\text{für } \abs{\lambda-\lambda_0}<\ep.
\end{align*}
Somit ist $\ph$ in jedem Punkt in eine Potenzreihe entwickelbar, also holomorph.
\item \textit{$\ph$ ist beschränkt}. Wir zeigen, dass $R(\lambda)$ beschränkt
ist.
\begin{enumerate}
  \item[$\alpha$)]
Sei $M=\setdef{\lambda\in\C}{\abs{\lambda}\le2\norm{A}}$. $R(\lambda)$ ist
stetig, also $\sup\limits_{\lambda\in M} \norm{R(\lambda)}<c$.
  \item[$\beta$)]
Sei nun $\lambda\in\C$ mit $\abs{\lambda}>2\norm{A}$. Betrachte $\Id =
\frac{1}{\lambda}\left(A-(A-\lambda\Id)\right)$, dann ist
\begin{align*}
R(\lambda) = (A-\lambda\Id)^{-1}\Id =
\frac{1}{\lambda}\left(A(A-\lambda\Id)^{-1} - \Id\right)
\end{align*}
und es folgt,
\begin{align*}
&\norm{R(\lambda)} 
\le \frac{1}{\abs{\lambda}}\left(\norm{A\left(A-\lambda\Id\right)^{-1}}+1\right)
< \frac{1}{2}\norm{R(\lambda)} + \frac{1}{\abs{\lambda}}\\
\Rightarrow &\norm{R(\lambda)} < \frac{2}{\abs{\lambda}}\to 0,\qquad
\abs{\lambda}\to
\infty
\end{align*}
\end{enumerate}
$\alpha$) und $\beta$) erlauben es uns, den Satz von Liouville anzuwenden, d.h.
$\ph$ ist konstant. Aus $\beta$) folgt außerdem, $\ph\equiv0$.\qedhere
\end{defnenum} 
\end{proof}

\section{Kompakte Operatoren}

\begin{defn}
\label{defn:6.7}
$T\in\LL(E\to F)$ heißt \emph{kompakt}\index{Operator!kompakt}, falls
$\overline{T(K_1(0))}$ kompakt in $F$.

Wir setzten $\KK(E\to F) := \setdef{T:E\to F}{T\text{ ist kompakt}}$,
$\KK(E):=\KK(E\to E)$.\fishhere
\end{defn}

\begin{prop}
\label{prop:6.8}
\begin{propenum}
  \item Für $T\in \LL(E\to F)$ sind äquivalent:
\begin{equivenum}
  \item\label{prop:6.8:1.1} $T\in\KK(E\to F)$.
  \item\label{prop:6.8:1.2} Für jede beschränkte Folge $(x_n)$ in $E$ besitzt
  $(Tx_n)$ eine in $F$ konvergente Teilfolge.
  \item\label{prop:6.8:1.3} Für jede beschränkte Teilmenge $M\subseteq E$ ist
  $\overline{T(M)}$ kompakt.
\end{equivenum}
\item Seien $T\in\LL(E\to F)$, $S\in\LL(F\to G)$. Ist $T$ oder $S$ kompakt, so
 auch $S\circ T$.
 \item Ist $F$ Banachraum, so ist $\KK(E\to F)$ linearer abgeschlossener
 Teilraum von $\LL(E\to F)$.
 \item Ist $E$ Banachraum, so ist $\KK(E)$ ein abgeschlossenes Ideal von
 $\LL(E)$.\fishhere
\end{propenum}
\end{prop}

\begin{bemn}[Vorbemerkung.]
In einem normierten Raum sind Überdeckungs- und Folgenkompaktheit
äquivalent.\maphere
\end{bemn}

\begin{proof}
\begin{propenum}
  \item ``\ref{prop:6.8:1.1}$\Rightarrow$\ref{prop:6.8:1.2}'': Sei $(x_n)$
  beschränkt in $E$, so ist $(\norm{x_n})$ beschränkt in $\R$ und besitzt daher
  eine konvergente Teilfolge $\norm{x_{n_k}}\to c$.
  Setze $y_k := (1+\norm{x_{n_k}})^{-1}x_{n_k}$, so ist $y_k\in K_1(0)$ und
  daher besitzt $(Ty_k)$ eine konvergente Teilfolge, $y_{k_l}\to y$. Folglich
  ist
\begin{align*}
Tx_{n_{k_l}} = (1+\norm{x_{n_{k_l}}})y_{n_{k_l}}\to cy.
\end{align*}
  
%     Falls $(x_n)$ eine Teilfolge $(x_{n_k})$ besitzt mit
%   $x_{n_k}\to 0$, so konvergiert $Tx_{n_k}\to0$, denn $T$ ist stetig.
%   
% Habe nun
%   $(x_n)$ keine gegen Null konvergente Teilfolge. Setze
%   $y_n:=\frac{x_n}{2\norm{x_n}}$ für $n>N$. $(y_n)$ ist nun Folge in $K_1(0)$
%   und folglich ist $(Ty_n)$ Folge in $\overline{T(K_1(0))}$, besitzt also eine
%   konvergente Teilfolge $(Ty_{n_k})$ mit $Ty_{n_k}\to y$ in $F$, also
% \begin{align*}
% Tx_{n_k} = 2\norm{x_{n_k}}Ty_{n_k}.
% \end{align*}
% $(\norm{x_{n_k}})$ ist beschränkt, besitzt also eine konvergente Teilfolge, also
% \begin{align*}
% Tx_{n_{k_l}} \to 2c y.
% \end{align*} 
``\ref{prop:6.8:1.2}$\Rightarrow$\ref{prop:6.8:1.3}'':
Sei $M\subseteq E$ beschränkt und $(y_n)$ Folge in $\overline{T(M)}$. Zeige es
existiert eine konvergente Teilfolge $(y_{n_k})$. Wähle dazu $\eta_n=Tx_n$ mit
$x_n\in M$ und $\norm{\eta_n-y_n}<\frac{1}{n}$. $(x_n)$ ist Folge in $M$, also
beschränkt und mit \ref{prop:6.8:1.2} folgt $Tx_{n_k}\to y$ in $F$.
Insbesondere $y\in \overline{T(M)}$, also
\begin{align*}
y_{n_k} = \underbrace{y_{n_k} - Tx_{n_k}}_{\to 0} + Tx_{n_k},
\end{align*}
also $y_{n_k}\to y$.

``\ref{prop:6.8:1.3}$\Rightarrow$\ref{prop:6.8:1.1}'': Spezialisierung.
\item Sei $(x_n)$ beschränkt in $E$. Ist $T$ kompakt, dann ist $Tx_{n_k}\to y$
und $S$ ist stetig, also $S\circ Tx_{n_k}\to Sy$. Ist $S$ kompakt, nun ist
$(Tx_{n})$ beschränkt also $S\circ Tx_{n_k}\to z$ in $G$.
\item Seien $S,T\in\KK(E\to F)$, $\alpha\in \K$ und $(x_n)$ beschränkte Folge
in $E$.
\begin{align*}
 &T(x_{n_k}) \to y\Rightarrow \alpha T(x_{n_k}) \to \alpha y\\
 &T(x_{n_k})\to y_1,\quad S(x_{n_k})\to y_2\Rightarrow
 (S+T)(x_{n_k}) \to y_1+y_2.
\end{align*}
Also ist $\KK(E\to F)$ linearer Teilraum von $\LL(E\to F)$.

Sei nun $(T_k)$ Folge in $\KK(E\to F)$, $T\in\LL(E\to F)$ und $\norm{T_k-T}\to
0$. Sei $(x_n)$ beschränkt in $E$, dann ist $T_1$ kompakt, d.h.
\begin{align*}
&T_1(x_n^{(1)})\to y^{(1)}, && (x_n^{(1)}) \text{ Teilfolge von } (x_n).
\end{align*}
Sei nun $T_2$ kompakt, dann
\begin{align*}
&T_2(x_n^{(2)})\to y^{(2)}, && (x_n^{(2)}) \text{ Teilfolge von } (x_n^{(1)}),
\end{align*}
usw. Setze $\xi_n := x_n^{(n)}$ (Diagonalfolge), dann folgt
\begin{align*}
\forall k\in\N : T_k(\xi_n)\to y^{(k)}
\end{align*}
und $(\xi_n)$ ist Teilfolge von $(x_n)$. Zeige nun $(T\xi_n)$ ist Cauchyfolge,
\begin{align*}
\norm{T\xi_n-T\xi_m} &\le \norm{T\xi_n-T_k \xi_n}
+ \norm{T_k\xi_n-T_k\xi_m} + \norm{T_k\xi_m-T\xi_m}\\
&< \ep\norm{\xi_n} +\norm{T_k\xi_n-T_k\xi_m} + \ep \norm{\xi_m},
\end{align*}
für $k$ hinreichend groß. Außerdem ist $(\xi_n)$ konvergent also beschränkt,
d.h. $\norm{\xi_n}\le c$. Für festes $k$ ist $(T_k\xi_n)_n$ Cauchyfolge da
konvergent und daher $\norm{T_k\xi_n-T_k\xi_m}<\ep$ für $n,m>N_\ep$, d.h.
\begin{align*}
\norm{T\xi_n-T\xi_m}  < 2\ep c + \ep = (2c+1)\ep,\qquad n,m> N_\ep.
\end{align*}
\item Folgt direkt aus 2.) und 3.).\qedhere
\end{propenum}
\end{proof}

\begin{lem}[Rieszsches Lemma (fast orthogonales Element)]
\index{Satz!Rieszsches Lemma}
\label{prop:6.9}
Sei $L$ abgeschlossener echter Teilraum von $E$. Dann gilt
\begin{align*}
\forall q \in (0,1) \exists x_q\in E\setminus L : \norm{x_q} = 1 \text{ und }
d(x_q,L)\ge q.\fishhere
\end{align*} 
\end{lem}
\begin{proof}
Sei $y\in E\setminus L$, dann folgt mit \ref{prop:3.4}, dass $d(y,L)>0$. Wähle
nun $x_0\in L$ mit
\begin{align*}
d(y,L) = \inf_{x\in L} \norm{y-x} \le \norm{y-x_0} \le \frac{d(y,L)}{q}.
\end{align*}
Setze nun $x_q = \frac{y-x_0}{\norm{y-x_0}}$, so ist $\norm{x_q}=1$ und für
$x\in L$ gilt,
\begin{align*}
\norm{x_q-x} &= \frac{1}{\norm{y-x_0}}\norm{y-x_0-\norm{y-x_0}x}
\\ &= \frac{1}{\norm{y-x_0}}\norm{y-\underbrace{(x_0+\norm{y-x_0}x)}_{\in L}}
\ge \frac{d(y,L)}{\norm{y-x_0}} \ge q.\qedhere
\end{align*}
\end{proof}

\begin{cor}
\label{prop:6.10}
Sei $E$ normierter Raum, dann ist $\overline{K_1(0)}$ genau dann kompakt, wenn
$\dim E<\infty$.\fishhere
\end{cor}

\begin{bsp}
\label{bsp:6.11}
\begin{bspenum}
  \item $\Id: E\to E$ ist genau dann kompakt, wenn $\dim E<\infty$.
  \item Sei $T\in\LL(E\to F)$ und $m:=\dim \im T < \infty$, so ist $T$ kompakt,
  denn ist $(x_n)$ beschränkt in $E$, so ist $(Tx_n)$ beschränkt in $\im T$.
  $\im T$ ist endlichdimensional und daher isomorph zum $\K^n$, also besitzt
  $(Tx_n)$ eine konvergente Teilfolge.
  \item Sei $E:=(C([a,b]\to\C),\norm{\cdot}_\infty)$ und $K\in
  C([a,b]\times[a,b]\to\C)$, so ist
\begin{align*}
T : E\to E,\quad Tf(x) := \int_a^b K(x,y)f(y)\dy
\end{align*}
kompakt.\bsphere
\end{bspenum}
\end{bsp}

\begin{prop}
\label{prop:6.12}
Sei $T\in\KK(E)$. Dann gelten
\begin{propenum}
  \item $\dim \ker (\Id - T) < \infty$.
  \item $\im (\Id - T)$ ist abgeschlossener linearer Teilraum von $E$.
  \item Ist $\ker (\Id-T) = (0)$, so ist $\im (\Id - T) = E$ und
  $(\Id-T)^{-1}\in\LL(E)$.\fishhere
\end{propenum}
\end{prop}

Für $\Id-T$ mit $T$ kompakt gilt also tatsächlich injektiv $\Rightarrow$
surjektiv.

\begin{proof}
\begin{proofenum}
  \item Sei $M=\ker(\Id-T)$. Wir zeigen, dass $\Id:M\to M$ kompakt
  ist, dann folgt mit \ref{prop:6.10}, dass $M$ endlichdimensional ist.
  
  Sei also $(x_n)$ beschränkt in $M$, dann gilt $Tx_{n_k}\to y$ und damit auch
\begin{align*}
&x_{n_k} = \underbrace{(\Id-T)(x_{n_k})}_{\to 0} + Tx_{n_k} \to y,\\
\Rightarrow\;& (\Id- T)y = \lim\limits_{k\to\infty} (\Id -T)x_{n_k} = 0.
\end{align*}
Also ist $y\in M$.
\item $\im (\Id-T)$ ist stets ein linearer Teilraum von $E$.

Sei nun $y_n$ Folge in $\im(\Id-T)$, d.h. $y_n=(\Id-T)x_n$, mit $y_n\to y\in E$.
\begin{proofenuma}
  \item\textit{$(x_n)$ kann als beschränkte Folge gewählt werden}. Wähle
  $\xi_n \in \ker (\Id-T)$ so, dass
\begin{align*}
&\norm{x_n-\xi_n} \le 2d(x_n,\ker(\Id-T))
\end{align*}
Dann gilt
\begin{align*}
&(\Id-T)(x_n-\xi_n) = y_n,\\
&\norm{x_n-\xi_n} \le 2d(x_n-\xi_n,\ker(\Id-T)) =: d_n.
\end{align*}
Angenommen $d_n\to \infty$. Setze $z_n := \frac{x_n-\xi_n}{d_n}$, so gilt
\begin{align*}
\norm{z_n} \le 1,\qquad d(z_n,\ker(\Id-T)) = \frac{1}{d_n}d(x_n-\xi,\ker(\Id-T))
= \frac{1}{2}.
\end{align*}
Nun ist $z_n$ beschränkt, also $Tz_{n_k}\to z$, wobei
\begin{align*}
z_{n_k} = \underbrace{(\Id-T)(z_{n_k})}_{\to 0} + Tz_{n_k} \to z.
\end{align*}
Somit gilt
\begin{align*}
(\Id-T)z = \lim\limits_{k\to\infty} (\Id-T)z_{n_k} = z-z=0.
\end{align*}
Also $z\in\ker (\Id-T)$, wobei $z_{n_k}\to z$. Dies ist ein Widerspruch zu
$d(z_n,\ker(\Id-T))=\frac{1}{2}$.

Somit ist $\norm{x_n-\xi_n}\le d_n \le c$ mit $(\Id-T)(x_n-\xi_n)=y_n$. Im
Folgenden sei daher $(x_n)$ beschränkte Folge.
\item $(x_n)$ ist beschränkt, also $Tx_{n_k}\to x$, mit
\begin{align*}
&x_{n_k} = (\Id-T)x_{n_k} + Tx_{n_k} = y_{n_k} + Tx_{n_k} \to y + x,\\
\Rightarrow &
(\Id-T)(y+x) = \lim\limits_{k\to\infty} (\Id-T)(x_{n_k}) =
\lim\limits_{k\to\infty} y_{n_k} = y.
\end{align*}
Also ist $y\in \im (\Id-T)$.
\end{proofenuma}
\item Sei $\ker(\Id-T)=(0)$. Angenommen $\Id-T$ ist nicht surjektiv, es gibt
also ein $x\in E\setminus\im (\Id-T)$.
\begin{proofenuma}
\item Wir zeigen nun $\forall n\in \N: (\Id-T)^n(x) \in E\setminus
\im (\Id-T)^{n+1}$ durch Widerspruch. Angenommen
\begin{align*}
\exists n\in\N \exists y\in E : (\Id-T)^n(x) = (\Id-T)^{n+1}y,
\end{align*}
so ist $(\Id-T)^n(x-(\Id-T)y) = 0$, da $\ker (\Id-T) = (0)$ gilt
\begin{align*}
(\Id-T)^{n-1}(x-(\Id-T)y) = 0.
\end{align*}
Wir erhalten induktiv,
\begin{align*}
x-(\Id-T)y = 0 \Rightarrow x\in\im (\Id-T).\dipper
\end{align*}
\item \textit{$\im (\Id-T)^{n+1}$ ist abgeschlossen}.
\begin{align*}
(\Id-T)^{n+1} = \sum\limits_{j=0}^{n+1}\binom{n+1}{j} (-T)^j
= \Id - \underbrace{\left(-\sum\limits_{j=1}^{n+1}\binom{n+1}{j}
(-T)^j\right)}_{\text{kompakt}}.
\end{align*}
Anwendung von 2.) ergibt die Behauptung.
\item Mit a.) und b.) folgt, $d((\Id-T)^nx,\im(\Id-T)^{n+1})>0$. Es existiert
also eine Folge $(a_n)$ in $\im (\Id-T)^{n+1}$ mit
\begin{align*}
0 \le \norm{(\Id-T)^{n}x-a_n} < 2d((\Id-T)^nx, \im (\Id-T)^{n+1}).
\end{align*}
Setze $x_n = \frac{1}{\norm{(\Id-T)^nx-a_n}}((\Id-T)^nx-a_n)$, dann ist
$\norm{x_n}=1$, also ist $(x_n)$ beschränkt, und es gilt $x_n\in \im
(\Id-T)^n$. Jedoch
\begin{align*}
d(x_n,\im(\Id-T)^{n+1}) &= \frac{1}{\norm{(\Id-T)^nx-a_n}}d
\left((\Id-T)^nx -a_n,\im (\Id-T)^{n+1}\right)\\
&=
\frac{1}{\norm{(\Id-T)^nx-a_n}}d
\left((\Id-T)^nx,\im (\Id-T)^{n+1}\right) > \frac{1}{2},
\end{align*}
denn $a_n\in \im (\Id-T)^{n+1}$, also enthält $Tx_n$ auch keine konvergente
Teilfolge, denn für $m>n$ gilt,
\begin{align*}
\norm{Tx_m-Tx_n} &= \norm{(\Id-T)(x_m-x_n)-(x_m-x_n)}
\\ &= \norm{x_n - \left(x_m + (\Id-T)(x_m-x_n) \right)}.
\end{align*}
Nun ist $x_m\in \im (\Id-T)^{n}$ und $(\Id-T)(x_m-x_n)\in \im(\Id-T)^{n+1}$,
also
\begin{align*}
\norm{x_n - \left(x_m + (\Id-T)(x_m-x_n) \right)}
\ge d(x_n, \im (\Id-T)^{n+1}) > \frac{1}{2},
\end{align*}
im Widerspruch zur Kompaktheit von $T$. Also $\im(\Id-T)=E$. 
\item
$(\Id-T)$ ist also bijektiv und daher existiert $(\Id-T)^{-1}$. Wir müssen noch
zeigen, dass $(\Id-T)^{-1}\in\LL(E)$. Angenommen $(\Id-T)^{-1}$ ist
unbeschränkt, d.h.
\begin{align*}
\sup\limits_{y\neq 0} \frac{\norm{(\Id-T)^{-1}}}{\norm{y}} = \infty.
\end{align*}
Wir können also $(y_n)$ in $E$ wählen mit $y_n\to 0$ und
$\norm{(\Id-T)^{-1}y_n}=1$. Setze $x_n=(\Id-T)^{-1}y_n$, so ist $\norm{x_n}=1$
und $y_n=(\Id-T)x_n$. Nach Voraussetzung existiert eine konvergente Teilfolge $(Tx_{n_k})$ mit Grenzwert
$y$. Für diese gilt dann
\begin{align*}
x_{n_k} = y_{n_k}+Tx_{n_k} \to 0 + y = y.
\end{align*}
Somit ist
\begin{align*}
(\Id-T)y &= \lim\limits_{k\to\infty} (\Id-T)x_{n_k}
= \lim\limits_{k\to\infty} x_{n_k}-Tx_{n_k} = 
\lim\limits_{k\to\infty} y_{n_k}+Tx_{n_k}-Tx_{n_k} \\ &= 
\lim\limits_{k\to\infty} y_{n_k} = 0.
\end{align*}
Da $\ker (\Id-T)=(0)$ folgt $y=0$ aber dies steht im Widerspruch zu
$\norm{x_n}=1$ und $x_{n_k}\to y$. Also ist $(\Id-T)^{-1}$ beschränkt.\qedhere
\end{proofenuma}
\end{proofenum}
\end{proof}

\begin{prop}
\label{prop:6.13}
Sei $T\in\KK(E)$. Dann gelten
\begin{propenum}
  \item Sei $\lambda\in\sigma(T)$ und $\lambda\neq 0$, dann ist $\lambda$
  Eigenwert endlicher Vielfachheit.
  \item $\sigma(T)$ hat höchstens $0$ als Häufungspunkt. Insbesondere ist
  $\sigma(T)$ endlich oder abzählbar und jeder Eigenwert $\neq 0$ ist
  isoliert.\fishhere
\end{propenum}
\end{prop}
\begin{proof}
\begin{proofenum}
  \item Sei $\lambda\in\sigma(T)$ und $\lambda\neq 0$, dann ist $T-\lambda \Id$
  nicht bijektiv und somit auch $\frac{1}{\lambda}T-\Id$.
  
  $\lambda$ ist Eigenwert, denn falls $\ker (\frac{1}{\lambda}T-\Id) = (0)$, so
  ist nach \ref{prop:6.12} $\frac{1}{\lambda}T-\Id$ bijektiv. Außerdem ist
  $\dim \ker (\frac{1}{\lambda}T-\Id)< \infty$, also hat $\lambda$ endliche
  Vielfachheit.
  \item Sei $\lambda\neq 0$ Häufungspunkt, so existiert eine Folge
  $(\lambda_n)$ in $\sigma(T)$ mit $\lambda_n\to\lambda$ und $\lambda_n\neq
  \lambda_m$, falls $n\neq m$.
  
  Sei $Tx_n = \lambda_n x_n$ und $\norm{x_n}=1$, so gilt für jedes $N\in\N$,
$\setd{x_1,\ldots,x_N}$
ist linear unabhängig. Sei $V_n=\lin{x_1,\ldots,x_n}$, so ist $\dim V_n = n$
und $V_{n+1}\supsetneq V_n$. Außerdem ist $V_n$ abgeschlossen, wir können also
das Lemma von Riesz anwenden und erhalten so eine Folge $(y_n)$ mit
\begin{align*}
y_n\in V_{n+1}\setminus V_n,\qquad \norm{y_n} = 1,\qquad d(y_n,V_n)=
\frac{1}{2}.
\end{align*}
$(y_n)$ ist beschränkt aber $(Ty_n)$ enthält keine konvergente Teilfolge.
Schreibe dazu $y_n=sx_{n+1} + \eta_n$ mit $s\neq 0$ und $\eta_n\in
V_n$, so gilt für $m>n$
\begin{align*}
&\norm{Ty_m-Ty_n} = \norm{T(sx_{m+1}+\eta_m)-T(sx_{n+1}+\eta_n)}\\
&= \norm{s\lambda_{m+1}x_{m+1}+T\eta_m - s\lambda_{n+1}x_{n+1}-T\eta_n}\\
&= \norm{\lambda_{m+1}y_m - \lambda_{m+1}\eta_m+T\eta_m - \lambda_{n+1}y_n +
\lambda_{n+1}\eta_n -T\eta_n}\\
&= \abs{\lambda_{m+1}}\norm{y_m -
\underbrace{\frac{1}{\lambda_{m+1}}\left(\lambda_{m+1}\eta_m-T\eta_m +
\lambda_{n+1}y_n -
\lambda_{n+1}\eta_n +T\eta_n\right)}_{\in V_m}}\\
&> \abs{\lambda_{m+1}}\frac{1}{2} \ge \frac{1}{2}c > 0.\qedhere
\end{align*}
\end{proofenum}
\end{proof}

\section{Fredholmsche Alternative}

In der linearen Algebra studiert man die Lösbarkeit von linearen
Gleichungssystemen. Sei $A$ eine reelle $n\times n$-Matrix und
\begin{align*}
T: \R^n\to\R^n,\quad x\mapsto Ax,
\end{align*}
der zugehörige Homomorphismus. Gesucht sind nun Lösungen von $Ax=y$, d.h.
$Tx=y$. In endlichdimensionalen Räumen gibt es zwei Fälle zu unterscheiden:
\begin{propenum}
  \item \textit{$T$ ist bijektiv}. So gilt $\forall y\in\R^n \exists ! x\in\R^n
  : Tx=y$. Hier ist außerdem $T$ genau dann bijektiv, wenn $\ker T = (0)$.
  \item \textit{$T$ nicht bijektiv}. Hier ist $T$ also weder surjektiv
  noch injektiv. Lösungen exisieren genau dann, wenn folgende Rangbedingung erfüllt
  ist,
\begin{align*}
\rg(A\mid y) = \rg(A) \Leftrightarrow
\rg\begin{pmatrix}
   A^\top \\ y^\top
   \end{pmatrix}
= \rg(A^\top).
\end{align*}
Dies ist wiederum äquivalent dazu, dass $y^\top$ linear abhängig von den
Zeilenvektoren von $A^\top$ ist, d.h.
\begin{align*}
\forall z\in\R^n : (A^\top z = 0\Rightarrow y^\top z = 0).
\end{align*}
Da $\R^n$ euklidisch, existiert ein Skalarprodukt und wir können dies
schreiben als
\begin{align*}
\forall z\in\R^n : (T^\top z = 0\Rightarrow \lin{y,z} = 0).
\end{align*}
\end{propenum}

Wir wollen diese Fragestellung nun auf unendlichdimensionale normierte
Vektorräume übetragen.
\begin{propenum}
  \item Betrachten wir dazu
\begin{align*}
\Id-T : E\to E,\quad x\mapsto (\Id-T)x,
\end{align*}
so impliziert (nach Satz \ref{prop:6.12}) $\ker (\Id-T)=(0)$, dass $(\Id-T)$
bijektiv und daher $(\Id-T)x=y$ für jedes $y\in E$ eindeutig lösbar ist.
Außerdem hängt die Lösung
\begin{align*}
x = (\Id-T)^{-1}y
\end{align*}
stetig von $y$ ab.
\item Die Bedingung  
\begin{align*}
\forall z\in\R^n : (T^\top z = 0\Rightarrow \lin{y,z} = 0).
\end{align*}
lässt sich jedoch nicht ohne Weiteres auf einen normierten Raum $E$ übertragen,
da weder $T^\top$ noch $\lin{\cdot,\cdot}$ zur Verfügung stehen.

Für Hilbeträume kennen wir bereits die
\begin{prop}[Riesz Abbildung]
\index{Satz!Riesz Abbildung}
\label{prop:6.14}
Sei $H$ ein Hilbertraum.
\begin{align*}
R_H: H\to H',\quad x\mapsto \ph_x,
\end{align*}
mit $\ph_x(y) = \lin{y,x}$ ist eine antilineare, bijektive und isometrische
Abbildung.\fishhere
\end{prop}

Wir wollen nun ein Analogon des $\lin{\cdot,\cdot}$ für normierte Räume durch
Anwendung eines Elementes des Dualraumes finden.
\end{propenum}

\begin{defn}
\label{defn:6.15}
Sei $L\subseteq E$ linearer Teilraum. Dann heißt
\begin{align*}
L^\bot := \setdef{x'\in E'}{x'\big|_L = 0} = \setdef{x'\in E'}{L\subseteq \ker
x'}
\end{align*}
\emph{Annihilator}\index{Annihilator} von $L$.\fishhere
\end{defn}

\begin{prop}
\label{prop:6.16}
\begin{propenum}
  \item $L^\bot$ ist ein abgeschlossener linearer Teilraum von $E'$.
  \item $L^\bot = \overline{L}^\bot$.
  \item Ist $E$ reflexiv, so gilt $(L^\bot)^\bot = J_E(\overline{L}) =
  \overline{J_E(L)}$.\fishhere
\end{propenum}
\end{prop}
\begin{proof}
\begin{proofenum}
  \item Sei $(x_n')$ Folge in $E'$ mit $x_n'(y)=0$ und $x_n'\to x'\in E'$. Sei
  $y\in L$, dann gilt
\begin{align*}
x'(y) = \lim\limits_{n\to\infty} x_n'(y) = 0.
\end{align*}
\item $L\subseteq \overline{L}$, also $(\overline{L})^\bot \subseteq L^\bot$.
Sei weiterhin $x'\in L^\bot$ und $y_n$ Folge in $L$ mit $y_n\to y\in E$, dann
gilt
\begin{align*}
x'(y) = \lim\limits_{n\to\infty} x'(y_n) = 0.
\end{align*}
Somit ist $L^\bot\subseteq \overline{L}^\bot$.
\item Sei $x''\in (L^\bot)^\bot\subseteq E''$, so existiert ein $x\in E$, so
dass $J_E(x) = x''$.
\begin{align*}
x''\in (L^\bot)^\bot 
&\Leftrightarrow \forall x' \in L^\bot : x''(x') = 0\\
&\Leftrightarrow \forall x' \in L^\bot : J_E(x)(x') = x'(x) = 0\\
&\Leftrightarrow \forall x' \in L^\bot : x\in \ker x'.
\end{align*}
Satz \ref{prop:4.17} besagt, dass
\begin{align*}
\overline{L} = \bigcap\setdef{\ker y'}{L\subseteq \ker y'},
\end{align*}
also gilt
\begin{align*}
\forall x' \in L^\bot : x\in \ker x'
\Leftrightarrow x\in \bigcap\setdef{\ker x'}{x'\in L^\bot} = \overline{L}.
\end{align*}
Somit ist $x''=J_E(x) \in (L^\bot)^\bot \Leftrightarrow x\in
\overline{L}\Leftrightarrow x'' = J_E(x) \in J_E(\overline{L})$.\qedhere
\end{proofenum}
\end{proof}

\begin{defn}
\label{defn:6.17}
Sei $T\in\LL(E\to F)$.
\begin{align*}
T' : F'\to E',\quad y'\mapsto y'\circ T,
\end{align*}
heißt \emph{adjungierter Operator}\index{Operator!adjungiert} zu $T$.\fishhere
\end{defn}

\begin{prop}
\label{prop:6.18}
\begin{propenum}
  \item $\norm{T'}_{\LL(F'\to E')}=\norm{T}_{\LL(E\to F)}$.
  \item $(\alpha T_1+\beta T_2)' = \alpha T_1' + \beta T_2'$.
  \item Sei $T'' = (T')' : E''\to F''$. Dann gilt
  \begin{align*}
  T''\big|_{J_E(E)} = J_F\circ T\circ J_E^{-1}
  \end{align*}
bzw.
  \begin{align*}
  T''\circ J_E = J_F\circ T
  \end{align*}
bzw.
  \begin{align*}
  J_F^{-1}\circ T''\circ J_E = T.\fishhere
  \end{align*}
\end{propenum}
\end{prop}
\begin{proof}
Der Beweis findet sich in Übungsaufgabe (6.3).\qedhere
\end{proof}

\begin{bsp}
\label{bsp:6.19}
\begin{bspenum}
  \item $(\Id_{E\to E})' = \Id_{E'\to E'} : E'\to E',\quad y'\mapsto y'\circ
  \Id = y'$.
  \item $T:\C^n\to \C^n,\; x\mapsto Ax$ mit $A\in M^{n\times n}$. Da
  $(\C^n)'=\C^n$ gilt,
\begin{align*}
T': \C^n \to \C^n,\quad x\mapsto A^\top x.\bsphere
\end{align*}
\end{bspenum}
\end{bsp}

\begin{prop}
\label{prop:6.20}
Sei $T\in\LL(E\to F)$. Dann ist $(\im T)^\bot = \ker T'$.\fishhere
\end{prop}
\begin{proof}
Für $y'\in E$ gilt
\begin{align*}
y'\in \ker T' &\Leftrightarrow T'y' = 0 \Leftrightarrow \forall x\in E :
(T'y')(x) = 0 \\ &\Leftrightarrow \forall x\in E : y'\circ Tx  = 0
\Leftrightarrow \forall y\in \im T : y'(y) = 0\\
&\Leftrightarrow \im T \subseteq \ker y' \Leftrightarrow y'\in (\im
T)^\bot.\qedhere
\end{align*}
\end{proof}

\begin{lem}
\label{prop:6.21}
Sei $T\in \KK(E\to F)$. Dann ist $\im T$ separabel.\fishhere
\end{lem}
\begin{proof}
Es gilt $\im T = \bigcup_{n\in\N} T(K_n(0))$. Da $T$ kompakt ist auch
$\overline{T(K_n(0))}$ kompakt, also gilt
\begin{align*}
\forall l\in \N : \overline{T(K_n(0))} \subseteq \bigcup_{j=1}^{J(l)}
K_{1/l}(y_j^{(l)}),\qquad y_j^{(l)}\in \overline{T(K_n(0))},\quad J(l) < \infty.
\end{align*}
Wähle $\eta_j^{l}\in T(K_n(0))$ mit $\norm{\eta_j^{(l)}-y_j^{(l)}} <
\frac{1}{l}$, so gilt
\begin{align*}
\overline{T(K_n(0))} \subseteq \bigcup_{j=1}^{J(l)} K_{2/l}(\eta_j^{(l)}),\qquad
\eta_j^{(l)}\in T(K_n(0)).
\end{align*}
Setze nun $A:= \bigcup_{l=1}^\infty
\setd{\eta_1^{(l)},\ldots,\eta_{J(l)}^{(l)}}$, so ist $A$ abzählbar und dicht
in $\overline{T(K_n(0))}$ und $A\subseteq T(K_n(0))$, also ist $A$ auch dicht
in $T(K_n(0))$.
\begin{align*}
\Rightarrow \bigcup_{n=1}^\infty A_n \text{ ist abzählbar und dicht in
}\bigcup_{n=1}^\infty T(K_n(0)) = \im T.\qedhere
\end{align*}
\end{proof}

\begin{prop}[Satz von Schauder]
\index{Satz!Schauder}
\label{prop:6.22}
Für $T\in \LL(E\to F)$ gilt
\begin{propenum}
\item $T\in \KK(E\to F) \Rightarrow T'\in \KK(F'\to E')$.
\item Falls $F$ Banachraum, dann gilt $T\in \KK(E\to F) \Leftarrow T'\in
\KK(F'\to E')$.\fishhere
\end{propenum}
\end{prop}
\begin{proof}
``$\Rightarrow$'': Sei also $T\in\KK(E\to F)$, $(y_n')$ beschränkte Folge in
$F'$, $\norm{(y_n')} \le C$. Zu zeigen ist nun, dass $(T'y_n')$ eine konvergente
Teilfolge besitzt.
\begin{proofenum}
  \item \textit{Konstruktion eines Grenzelements}. Nach Lemma \ref{prop:6.21}
  ist $\im T$ separierbar, also $\im T = \overline{\setd{y_1,y_2,\ldots}}$.

$(y_n'(y_1))_n$ ist Folge in $\K$ und beschränkt, besitzt also eine konvergente
Teilfolge $(y_n'^{(1)})$.

$(y_n'^{(1)}(y_2))$ ist Folge in $\K$ \ldots, besitzt also eine konvergente
Teilfolge $(y_n'^{(2)})$.

Wähle die Diagonalfolge $y_{n_k}' = y_k'^{(k)}$.

\item \textit{$(y_{n_k}')$ konvergiert auf $\lin{y_1,y_2,\ldots}=:L$}. Sei 
\begin{align*}
y' : L\to \K,\quad y\mapsto \lim\limits_{k\to\infty} y_{n_k}'(y),
\end{align*}
so ist $y'$ linear und beschränkt, denn
\begin{align*}
\norm{y'(y)} \le \limsup\limits_{k\to\infty} \norm{y_{n_k}'}\norm{y} \le
C\norm{y},
\end{align*}
also $y'\in L'$. Mit dem Satz von Hahn-Banach setzen wir $y'$ fort zu $y'\in
F'$.
\item \textit{Es gilt sogar für $y\in \overline{L} = \overline{\im T}$, dass
$y_{n_k}'(y) \to y'(y)$}.

Sei $(y_l)$ Folge in $L$ mit $y_l\to y$, so gilt
\begin{align*}
&\abs{y_{n_k}'(y)-y'(y)}\\ &
 \le \underbrace{\abs{y_{n_k}'(y-y_l)}}_{\text{(1)}} +
\underbrace{\abs{y_{n_k}'(y_l)-y'(y_l)}}_{\text{(2)}} + 
\underbrace{\abs{y'(y_l)-y'(y)}}_{\text{(3)}}.
\end{align*}
(1)$\le \sup\norm{y_{n_k}'}\norm{y-y_l} = c \norm{y-y_l}$,\\
(3)$\le \norm{y'}\norm{y-y_l}$,\\
also (1),(3)$\le \ep$ für $l$ hinreichend groß.\\
(2)$\le \ep$ für $l$ fest und $k$ hinreichend, also
\begin{align*}
\abs{y_{n_k}'(y)-y'(y)} \le 3 \ep,
\end{align*}
für $k$ hinreichend groß.
\item \textit{Für eine Teilfolge $(y_{n_j}')$ von $(y_{n_k}')$ gilt
$T'(y_{n_j}')\to y$}.
\begin{align*}
\norm{T'y_{n_k}' - T'y'} = \sup\limits_{\norm{x}=1}
\abs{(T'y_{n_k}'-T'y')(x)}.
\end{align*}
Wähle $(x_k)$ in $E$ mit $\norm{x_k} = 1$ und
\begin{align*}
\norm{T'y_{n_k}' - T'y'} = \abs{(T'y_{n_k}'-T'y')(x_k)} + \frac{1}{n_k}.
\end{align*}
$T$ ist kompakt also $Tx_{n_j}\to y \in \overline{\im T}$. Somit gilt
\begin{align*}
&\norm{T'y_{n_j}' - T'y'} =
\abs{T'y_{n_j}'(x_{n_j}) - T'y'(x_{n_j})} + \frac{1}{n_j}\\
&= \abs{y_{n_j}'(Tx_{n_j})-y'(Tx_{n_j})} + \frac{1}{n_j}\\
&\le \underbrace{\abs{y_{n_j}'(Tx_{n_j}-y)}}_{\text{(1)}} +
\underbrace{\abs{y_{n_j}'(y)-y'(y)}}_{\text{(2)}} +
\underbrace{\abs{y'(y)-y'(Tx_{n_j})}}_{\text{(3)}} + \frac{1}{n_j}
\end{align*}
(1)$\le \sup\norm{y_{n_j}}\norm{Tx_{n_j}-y} \to 0$,\\
(3)$\le \norm{y'}\norm{Tx_{n_j}-y}\to 0$,\\
(2)$ \to 0$ nach b.), also
\begin{align*}
\norm{T'y_{n_j}' - T'y'} \to 0,\qquad j\to\infty.
\end{align*}
``$\Leftarrow$'': Sei $T'$ kompakt so ist nach ``$\Rightarrow$'' auch $T''\in
\KK(E''\to F'')$. Mit \ref{prop:6.18} folgt
\begin{align*}
T = J_F^{-1}\circ T''\circ J_E.
\end{align*}
Sei $(x_n)$ beschränkt in $E$, so ist $(J_E(x_n))$ beschränkt in $E''$ und
daher
\begin{align*}
T'(J_E(x_{n_k}))\to y''\in F''. 
\end{align*}
Da $F$ vollständig ist $J_F(F)$
abgeschlossen und daher $y''\in J_F(F)$, somit
\begin{align*}
T(x_{n_k}) = J_F^{-1}\circ T''\circ J_E(x_{n_k}) \to J_F^{-1}(y'').\qedhere
\end{align*}
\end{proofenum}
\end{proof}

\begin{prop}[Fredholmsche Alternative]
\index{Fredholmsche!Alternative}
\label{prop:6.23}
Sei $E$ normierter Raum, $T\in\KK(E)$. Dann gilt entweder
\begin{equivenum}
  \item\label{prop:6.23:1} $\ker (\Id-T)=(0)$. Dann besitzt die Gleichung
\begin{align*}
(\Id - T)x = y
\end{align*}
für jedes $y\in E$ eine Lösung. Die Lösung ist eindeutig und hängt stetig von
$y$ ab.

--- oder ---
\item\label{prop:6.23:2} $\ker(\Id-T)\neq 0$. Dann besitzt 
\begin{align*}
(\Id-T)x = y
\end{align*}
genau für die $y\in E$ Lösungen, für die
\begin{align*}
\forall x' \in \ker (\Id-T') : x'(y) = 0.
\end{align*}
Die dadurch gegebene Anzahl von Nebenbedingungen ist endlich. In diesem Fall
ist $x$ nicht eindeutig.\fishhere
\end{equivenum}
\end{prop}
\begin{proof}
``\ref{prop:6.23:1}'': Siehe Satz \ref{prop:6.12}.

``\ref{prop:6.23:2}'': Nach Satz \ref{prop:6.12} ist $\im(\Id-T) =
\overline{\im(\Id-T)}$, also
\begin{align*}
&y\in \im(\Id-T) = \bigcap
\setdef{\ker x'}{x'\in E \land \im (\Id-T) \subseteq \ker (x')}\\
&\Leftrightarrow
\forall z'\in \setdef{x'\in E}{\im
(\Id-T)\subseteq \ker x'} : y\in \ker (z')\tag{*}
\end{align*}
$T$ ist kompakt, also folgt mit \ref{prop:6.22}, dass $T'$ kompakt und daher ist
$\ker(\Id-T')$ endlichdimensional. Sei $\BB=\setd{x_1',\ldots,x_k'}$
Basis von $\ker(\Id-T')$, so ist (*) äquivalent zu
\begin{align*}
\forall j=1,\ldots,k : x_j(y) = 0.\qedhere
\end{align*}
\end{proof}

\begin{defn}
\label{defn:6.24}
Seien $U$, $V$ lineare Teilräume von $L$ mit
\begin{defnenum}
\item $U\cap V = (0)$,
\item $\forall x\in L : \exists u\in U\exists v\in V : x = u+v$.
\end{defnenum}
Dann schreiben wir $L=U\oplus V$ und $U\oplus V$ heißt \emph{direkte
Summe}\index{Vektorraum!Direkte Summe} von $U$ und $V$.\fishhere
\end{defn}

Eine leichte Übung zeigt, dass $u$ und $v$ in obiger Definition eindeutig
bestimmt sind.

Im Gegensatz zur Definition \ref{defn:1.13} muss die Norm auf $U\oplus V$
\textit{nicht} äquivalent zur Norm auf $L$ sein.

\begin{lem}
\label{lem:6.25}
Sei $L=U\oplus V$ und $L=U\oplus \tilde{V}$ mit $\dim V,\dim \tilde{V} <
\infty$, so ist
\begin{align*}
\dim V = \dim \tilde{V}.\fishhere
\end{align*}
\end{lem}
\begin{proof}
Sei $\BB=\setd{x_1,\ldots,x_n}$ Basis von $V$. Dann gilt
\begin{align*}
x_i = u_j + \tilde{v}_j,\qquad u_j\in U,\quad \tilde{v}_j \in \tilde{V}.
\end{align*}
$\setd{\tilde{v}_1,\ldots,\tilde{v}_n}$ ist linear unabhängig, denn sei
\begin{align*}
\sum_{j=1}^n c_j \tilde{v}_j = 0
\Rightarrow
\underbrace{\sum_{j=1}^n c_j x_j}_{\in V} = 
\underbrace{\sum_{j=1}^n c_j u_j}_{\in U} \overset{U\cap V=(0)}{=} 0
\end{align*}
und da $\BB$ Basis, folgt $c_1=\ldots=c_n=0$. Also ist
$\setd{\tilde{v}_1,\ldots,\tilde{v}_n}$ linear unabhängig und daher $\dim
\tilde{V}\ge \dim V$. Die Situation ist jedoch vollkommen symmetrisch, also
gilt auch $\dim V \ge \dim \tilde{V}$ und folglich $\dim V = \dim
\tilde{V}$.\qedhere
\end{proof}

\begin{defn}
\label{defn:6.26}
Sei $L=U\oplus V$ mit $\dim V < \infty$. Dann heißt
\begin{align*}
\codim U := \dim V
\end{align*}
die \emph{Kodimension}\index{Vektorraum!Kodimension} von $U$.\fishhere
\end{defn}

\begin{lem}
\label{prop:6.27}
Sei $E$ normierter Raum, $L\subseteq E$ abgeschlossener linearer Teilraum,
$V\subseteq E$ endlichdimensionaler Teilraum und $L\cap V = (0)$. Dann ist
$L\oplus V$ abgeschlossener linearer Teilraum.\fishhere
\end{lem}
\begin{proof}
Sei $(x_n)$ Folge in $L\oplus V$ mit $x_n\to x$ in $E$. Zeige $x\in L\oplus V$.

Da $x_n\in L\oplus V$ existiert eine eindeutige Zerlegung,
\begin{align*}
x_n = u_n + v_n,\qquad u_n\in L,\quad v_n\in V.
\end{align*}
Wir bilden den Quotientenraum
\begin{align*}
E/L := \setdef{[x]=x+L}{x\in E}.
\end{align*}
Da $L$ abgeschlossen ist
\begin{align*}
\norm{[x]}_0 = \inf\setdef{\norm{x-y}}{y\in L} = d(x,L)
\end{align*}
Norm auf $E/L$. $[x_n]$ ist konvergent bezüglich dieser Norm, denn
\begin{align*}
\norm{[x_n]-[x]}_0 = 
\inf\setdef{\norm{x_n-x-y}}{y\in L} \le \norm{x_n-x}\to 0.
\end{align*}
Außerdem ist $([v_n])$ Cauchyfolge in $[V]:=\setdef{[v]}{v\in V}$, denn
\begin{align*}
\norm{[v_n]-[v_m]}_0 &= \inf\setdef{\norm{v_n-v_m-y}}{y\in L}\\
&\le \norm{v_n+u_n - (v_m+u_m)} = \norm{x_n-x_m} \to 0.
\end{align*}
Da $L\cap V = (0)$, ist $\dim [V] = \dim V < \infty$ und daher ist $[V]$
vollständig und folglich $[v_n]\to [v]$. Insbesondere kann man für den
Grenzwert ein $v\in V$ als Vertreter wählen. Für dieses $v$ gilt
\begin{align*}
\norm{[x]-[v]}_0 = \lim\limits_{n\to\infty} \norm{[x_n]-[v_n]}
= \lim\limits_{n\to\infty} \norm{[u_n]} = 0, 
\end{align*}
denn $u_n\in L$, also ist auch $x-v\in L$. Setze $u:=x-v$, so ist $x=u+v\in
L\oplus V$.\qedhere
\end{proof}

Wir haben nicht bewiesen, dass $u_n\to u$ oder $v_n\to v$ sondern lediglich,
dass $x=u+v$ mit $u\in L$ und $v\in V$. Über die Konvergenz von $(u_n)$ und
$(v_n)$ lassen sich unter diesen allgemeinen Voraussetzungen keine genaueren
Aussagen treffen.

\begin{lem}
\label{prop:6.28}
Sei $E$ normierter Raum, $L\subseteq E$ abgeschlossener linearer Teilraum
endlicher Kodimension. Dann gilt
\begin{align*}
\codim L = \dim L^\bot.
\end{align*}
\end{lem}
\begin{proof}
Sei $E=L\oplus\lin{x_1,\ldots,x_n}$ mit $\setd{x_1,\ldots,x_n}$ linear
unabhängig.
\begin{align*}
L_j := L\oplus \lin{x_1,\ldots,x_{j-1},x_{j+1},\ldots,x_n}
\end{align*}
ist nach Lemma \ref{prop:6.27} abgeschlossen und $x_j\notin L_j$. Nach
\ref{prop:4.16} existieren daher $x_j'\in E'$ mit $L_j\subseteq \ker x_j'$ und
$x_j'(x_j)=1$.

Wir zeigen nun, dass $\setd{x_1',\ldots,x_n'}$ eine Basis von $L^\bot$ bildet.
\begin{proofenum}
\item Die lineare Unabhängigkeit ist klar, denn $x_j'(x_k)=\delta_{jk}$.
\item Sei $x'\in L^\bot$. $x\in E=L\oplus\lin{x_1,\ldots,x_n}$, so besitzt $x$
eine Darstellung,
\begin{align*}
x = u + \sum_{j=1}^n \alpha_j x_j,\qquad u\in L,\; \alpha_j\in\K,
\end{align*}
wobei
\begin{align*}
x'(x) = \underbrace{x'(u)}_{=0} + \sum_{j=1}^n \alpha_j x'(x_j).
\end{align*}
Nun besitzt auch $x_j$ eine Darstellung als
\begin{align*}
x_j = \sum_{l=1}^n x_l x_l'(x_j),
\end{align*}
also
\begin{align*}
x'(x) &= \sum_{j=1}^n \alpha_j x'(x_j)
= \sum_{j=1}^n \alpha_j \sum_{l=1}^n x_l'(x_j) x'(x_l)\\
&= \sum_{l=1}^n x'(x_l) \sum_{j=1}^n \alpha_j  x_l'(x_j)
= \sum_{l=1}^n x'(x_l)  x_l'\left(\underbrace{\sum_{j=1}^n  \alpha_j
x_j}_{x-u}\right)\\
&=
\sum_{l=1}^n x'(x_l)  x_l'(x).\qedhere 
\end{align*}
\end{proofenum}
\end{proof}

\begin{prop}
\label{prop:6.29}
Sei $T\in\KK(E)$. Dann gilt
\begin{align*}
\codim\im(\Id -T) \le \dim\ker (\Id-T).\fishhere
\end{align*}
\end{prop}
\begin{proof}
Sei $n=\dim\ker(\Id-T)$. Angenommen es existiert eine linear unabhängige
Teilmenge $\setd{y_1,\ldots,y_m}$ mit
$\lin{y_1,\ldots,y_m}\cap \im(\Id-T) = (0)$ und
 $m\ge n+1$.
 
 Sei $\setd{x_1,\ldots,x_n}$ Basis von $\ker(\Id-T)$, dann folgt mit
 \ref{prop:4.16}
 \begin{align*}
 \exists x_j'\in E' : x_j'(x_k) = \delta_{jk}.
 \end{align*}
Setze $\tilde{T}x := Tx + \sum_{j=1}^n x_j'(x)y_j$, so liefert scharfes
Hinsehen, dass $\tilde{T}$ kompakt. Betrachte nun
\begin{align*}
(\Id-\tilde{T})x = 0 
&\Leftrightarrow (\Id-T)x - \sum_{j=1}^n x_j'(x)y_j = 0
\Leftrightarrow \underbrace{(\Id-T)x}_{\in \im (\Id-T)} =
\underbrace{\sum_{j=1}^n x_j'(x)y_j}_{\in \lin{y_1,\ldots,y_n}}\\
&\Leftrightarrow (\Id-T)x = 0 \text{ und } \sum_{j=1}^n x_j'(x)y_j = 0
\end{align*}
Für ein solches $x$ folgt, da die $y_j$ linear unabhängig, $x_j'(x)=0$
für $j=1,\ldots,n$. Insbesondere
\begin{align*}
0 = x_j'(x) = \sum_{k=1}^n \alpha_k x_j'(x_k) = \alpha_j
\end{align*}
und daher $x=0$. Somit ist $\ker(\Id-\tilde{T}) = (0)$ und daher nach
\ref{prop:6.12}, $\im(\Id-\tilde{T}) = E$.

Zeige nun $\im(\Id-\tilde{T}) \subseteq \im (\Id-T)\oplus\lin{y_1,\ldots,y_n}$,
denn dann folgt
\begin{align*}
\im(\Id-T)\oplus \lin{y_1,\ldots,y_n} = E,
\end{align*}
im Widerspruch zu $\im(\Id-T)\cap \lin{y_1,\ldots,y_m} = (0)$, da $m\ge n+1$.

Für $x\in E$ gilt jedoch,
\begin{align*}
(\Id-\tilde{T})x = \underbrace{(\Id-T)x}_{\in\im (\Id-T)} - 
\underbrace{\sum_{j=1}^n x_j'(x)y_j}_{\in \lin{y_1,\ldots,y_n}} 
\in  \im(\Id-T)\oplus \lin{y_1,\ldots,y_n}.\qedhere
\end{align*}
\end{proof}

\begin{prop}
\label{prop:6.30}
Sei $T\in\KK(E)$. Dann gilt
\begin{align*}
\dim \ker(\Id-T) = \codim \im(\Id-T) = \dim \ker (\Id-T').\fishhere
\end{align*}
\end{prop}
\begin{proof}
\begin{proofenum}
\item\label{proof:6.30:1} $\dim \ker(\Id-T) \le \dim \ker(\Id-T'')$, denn
$J_E:E\to E''$ ist injektiv und nach \ref{prop:6.18} gilt
\begin{align*}
T''\circ J_E = J_E\circ T.
\end{align*}
Somit ist $(\Id-T'')\circ J_E = J_E\circ (\Id-T)$. (*)

Sei $\setd{x_1,\ldots,x_n}$ Basis von $\ker(\Id-T)$. In Gleichung (*)
verschwindet die rechte Seite für jedes Basiselement,
\begin{align*}
(\Id-T'')J_E(x_j) = 0 \Rightarrow \setd{J_E(x_1),\ldots,J_E(x_n)} \subseteq
\ker (\Id-T'').
\end{align*}
Da $J_E$ injektiv, ist die Menge linear unabhängig, also 
\begin{align*}
\dim \ker(\Id-T)\le\dim \ker(\Id-T'').
\end{align*}
\item\label{proof:6.30:2} Aus \ref{prop:6.29} folgt $\codim\im (\Id-T) \le
\dim\ker(\Id-T)$ also auch
\begin{align*}
\codim\im (\Id-T')\le \dim \ker(\Id-T').
\end{align*}
\item\label{proof:6.30:3} $\dim \ker(\Id-T') \overset{\ref{prop:6.20}}{=} \dim
(\im(\Id-T))^\bot \overset{\ref{prop:6.28}}{=} \codim \im (\Id-T)$. Somit
\begin{align*}
\dim \ker(\Id-T'') = \codim\im (\Id-T').
\end{align*}
\end{proofenum}
Aus \ref{proof:6.30:1}-\ref{proof:6.30:3} folgt,
\begin{align*}
\dim \ker (\Id-T) &\le \dim \ker(\Id-T'') = \codim \im (\Id-T') \le
\dim \ker(\Id-T') \\ &= \codim \im (\Id-T) \le \ker (\Id-T).
\end{align*}
Somit gilt Gleichheit in der gesamten Gleichung.\qedhere
\end{proof}

\begin{prop}[Fredholmsche Integralgleichung]
\index{Fredholmsche!Integralgleichung}
\label{prop:6.31}
Sei $E=C([a,b]\to\C)$. Zu $g\in E$ ist $f\in E$ gesucht mit
\begin{align*}
f(x) - \int_a^b K(x,y)f(y)\dy = g(x),\qquad a\le x\le b,
\end{align*}
wobei $K\in C([a,b]\times[a,b]\to \C)$.\fishhere
\end{prop}

In Übungsaufgabe 4.2 haben wir gezeigt, dass falls
\begin{align*}
\max\limits_{a\le x\le b} \int_a^b K(x,y)\dy < 1,\tag{*}
\end{align*}
$(\Id-T)^{-1}\in\LL(E)$ existiert, d.h.
\begin{align*}
\forall g\in E \exists ! f\in E : (\Id-T)f = g
\end{align*}
und die Lösung $f$ hängt stetig von $g$ ab.

Nun ist $T$ ein kompakter Operator und daher die Fredholm Theorie anwendbar.
Falls $\ker(\Id-T) = (0)$, so gilt dasselbe auch ohne die (*)-Bedingung.

Ist dagegen $\ker(\Id-T)\neq (0)$, so gibt es genau dann Lösungen zu $f$, falls
\begin{align*}
\forall h\in \ker(\Id-T') : h'(g) = 0.
\end{align*}
Dies ist äquivalent zu
\begin{align*}
\forall h\in E : \left(\underbrace{h(x) - \int_a^b \overline{K(y,x)}h(y)\dy =
0}_{(1)} \right) \Rightarrow \int_a^b g(x)\overline{h}(x)\dx = 0 .
\end{align*}
Insbesondere besagt \ref{prop:6.30}, dass (1) genau so viele linear unabhängige
Lösungen besitzt wie
\begin{align*}
f(x)-\int_a^b K(x,y)f(y)\dy = 0.
\end{align*}

\section{Ausblick}

Sei $H$ ein Hilbertraum, $T\in \KK(H)$ und $T$ symmetrisch, d.h.
\begin{align*}
\lin{Tx,y} = \lin{x,Ty}.
\end{align*}

\begin{prop}
$\lambda=\norm{T}$ oder $\lambda=-\norm{T}$ ist Eigenwert von $T$.\fishhere
\end{prop}

Für diesen Satz ist die Symmetrie von $T$ eine notwendige Voraussetzung, denn
betrachte für $H=l^2$ den modifizierten Shift-Operator
\begin{align*}
T(x_1,x_2,\ldots) = (0,\frac{x_1}{2},\frac{x_2}{3}), 
\end{align*}
so ist $T$ kompakt (vgl. Übungsaufgabe 10.2). $T$ besitzt aber keine
Eigenwerte, denn sei $\lambda = 0$ und $Tx = \lambda x$, so ist $x=0$ und sei
$\lambda \neq 0$ und $Tx = \lambda x$, so ist ebenfalls $x=0$ aber
$\norm{T}=\frac{1}{2}$.

\begin{prop}
Sei $T\in \KK(H)$ symmetrisch. Dann existiert ein ONS $(e_j)_{j\in\II}$ mit
$\II$ höchstens abzählbar und
\begin{defnenum}
\item $\forall j\in\II : T e_j = \lambda_j e_j$,\quad $\lambda_j\neq 0$.
\item $(\lambda_j)$ ist monoton fallend.
\item In $(\lambda_j)$ kommt jeder Eigenwert so oft vor, wie es seiner
endlichen Vielfachheit entspricht.
\item $(e_j)_{j\in\II}$ ist vollständig in $\overline{\im T}$, d.h.
\begin{align*}
\forall x\in \overline{\im T} : x = \sum_{j\in\II} \lin{x,e_j}e_j.\fishhere
\end{align*}
\end{defnenum}
\end{prop}

\clearpage
\chapter{Sobolevräume}

\section{Das Lebesgue-Integral}

\begin{defn}
\label{defn:7.1}
$\Sigma\in\PP(\R^n)$ heißt $\sigma$-Algebra, falls
\begin{defnenum}
\item $\varnothing\in\Sigma$,
\item $A\in\Sigma\Rightarrow A^c\in\Sigma$,
\item $A_j\in \Sigma \Rightarrow \bigcup_{j\in\N} A_j \in \Sigma$.\fishhere
\end{defnenum}
\end{defn}

Es gibt zahllose $\sigma$-Algebren auf dem $\R^n$. Die $\sigma$-Algebra, die
die offenen Mengen enthält, heißt \emph{Borel-$\sigma$-Algebra}.

\begin{cor}
\label{prop:7.2}
\begin{propenum}
\item $\R^n\in\Sigma$.
\item $A_j\in\Sigma \Rightarrow \bigcap_{j\in\N} A_j \in \Sigma$.
\item $A,B\in\Sigma \Rightarrow A\setminus B\in\Sigma$.\fishhere
\end{propenum}
\end{cor}

\begin{defn}
\label{defn:7.3}
Eine Abbildung $\mu: \Sigma \to [0,\infty]$ heißt \emph{Maß}, falls
\begin{defnenum}
\item $\mu(\varnothing) = 0$.
\item $\mu$ \emph{$\sigma$-additiv}, d.h. für $A_j\in\Sigma$ mit $A_j\cap A_k =
\varnothing$ falls $j\neq k$, gilt
\begin{align*}
\mu\left(\dot{\bigcup}_{j\in\N} A_j\right) = \sum_{j=1}^\infty
\mu(A_j).\fishhere
\end{align*}
\end{defnenum}
\end{defn}

Mit $A \dcup B$ bzw. $\dot{\bigcup}_j A_j$ fordern wir implizit, dass $A$ und
$B$ bzw. die $A_j$ disjunkt sind.

\begin{cor}
\label{prop:7.4}
\begin{propenum}
\item Wenn $A,B\in\Sigma$ und $A\subseteq B$, dann folgt $\mu(A)\le\mu(B)$.
\item $A_j\in\Sigma$, $A_1\subseteq A_2\subseteq A_3 \subseteq \ldots$,
\begin{align*}
\Rightarrow \mu\left(\bigcup_{j\in\N} A_j\right) = \lim\limits_{j\to\infty}
\mu(A_j).\fishhere
\end{align*}
\end{propenum}
\end{cor}

\begin{prop}
\label{prop:7.5}
Es existiert eine $\sigma$-Algebra $\Sigma\subseteq \PP(\R^n)$ und ein Maß
$\mu$ auf $\Sigma$ mit den Eigenschaften:
\begin{propenum}
\item\label{prop:7.5:1} $O\subseteq\R^n$ offen $\Rightarrow O\in\Sigma$.
\item\label{prop:7.5:2} $\mu\left([a_1,b_1]\times \ldots \times [a_n,b_n]\right)
=\prod_{j=1}^n (b_j-a_j)$.
\item\label{prop:7.5:3} $N\in\Sigma$ mit $\mu(N) = 0$ und $M\subseteq N$, so ist
$M\in\Sigma$ und $\mu(M) = 0$.
\item\label{prop:7.5:4} $\mu$ ist translationsinvariant, d.h.
\begin{align*}
A\in\Sigma,\; x\in\R^n \Rightarrow
\begin{cases}
x+A\in\Sigma,\\
\mu(x+A) =\mu(A).\fishhere
\end{cases}
\end{align*}
\end{propenum}
\end{prop}

\begin{defn}
\label{defn:7.6}
Für das Paar $(\Sigma,\mu)$, das von allen $(\Sigma,\mu)$, die
\ref{prop:7.5:1}-\ref{prop:7.5:4} aus \ref{prop:7.5} erfüllen, die kleinste
$\sigma$-Algebra besitzt, heißt $\mu$ \emph{Lebesgue-Maß} und $\Sigma$ die
$\sigma$-Algebra der \emph{Lebesgue-messbaren} Mengen.\fishhere
\end{defn}

\begin{bem}
\label{bem:7.7}
$A\subseteq\R^n$ abzählbar $\Rightarrow$ $A\in\Sigma$ und $\mu(A) = 0$.\maphere
\end{bem}

\begin{defn}
\label{defn:7.8}
Sei $B\subseteq A$ mit $\mu(A\setminus B) = 0$. Gilt eine Bedingung auf $B$, so
sagt man die Bedingung gilt auf $A$ \emph{fast überall}.\fishhere
\end{defn}

\begin{defn}
\label{defn:7.9}
\begin{defnenum}
\item $f: D\to \RA$, $D\subseteq \R^n$, $\RA:=\R\cup\setd{-\infty,\infty}$
heißt \emph{messbar}, falls
\begin{align*}
\forall a\in\RA : f^{-1}([a,\infty]) \in \Sigma.
\end{align*}
\item $f: D\to \C$ heißt \emph{messbar}, falls $\Re f$ messbar und $\Im f$
messbar.\fishhere
\end{defnenum}
\end{defn}

\begin{prop}
\label{prop:7.10}
\begin{propenum}
\item Sei $D\in\Sigma$ und $f$ stetig, so ist $f$ messbar.
\item Falls $f,g$ messbar, dann auch
\begin{defnenum}
\item $\min\setd{f,g}$ und $\max\setd{f,g}$.
\item $f\cdot g$ mit der Konvention $0\cdot \infty = 0$.
\item $f+g$, falls $f(x) = \pm\infty \Rightarrow g(x)\neq \mp \infty \mufu$.\\
Insbesondere sind daher $c\cdot f$, $f_+:=\max\setd{f,0}$, $f_-:=
-\min\setd{f,0}$ und $\abs{f}:= f_++f_-$.
\end{defnenum}
\item Sei $f_n$ messbar, dann sind es auch
\begin{defnenum}
\item $\sup_n f_n$, $\inf_n f_n$,
\item $\limsup_n f_n$, $\liminf_n f_n$.
\end{defnenum}
\item Sei $f:\R\to\R$ stetig und $g: D\to\RA$ messbar. Setze
\begin{align*}
f\circ g(x) :=
\begin{cases}
\pm \infty, & g(x) = \pm\infty,\\
f\circ g(x),& \text{sonst}.
\end{cases}
\end{align*}
So ist auch $f\circ g$ messbar.\fishhere
\end{propenum}
\end{prop}

\begin{defn}
\label{defn:7.11}
\begin{defnenum}
\item Sei $A\subseteq \R^n$. Dann heißt
\begin{align*}
\chi_A : x\mapsto
\begin{cases}
1, & x\in A,\\
0, & \text{sonst},
\end{cases}
\end{align*}
\emph{charakteristische Funktion} von $A$.
\item $A_j\in\Sigma$, $\alpha_j\in\R $, $N\in\N$ dann heißt,
\begin{align*}
f = \sum_{j=1}^N \alpha_j \chi_{A_j}
\end{align*} 
\emph{einfache Funktion}.\fishhere
\end{defnenum}
\end{defn}

\begin{prop}
\label{prop:7.12}
Sei $A\in\Sigma$, $f: A\to[0,\infty]$ messbar. Dann existiert eine Folge
einfacher Funktionen $(s_n)$ mit
\begin{align*}
\forall x\in A : s_1(x)\le s_2(x)\le \ldots
\end{align*}
und $f(x) = \lim\limits_{n\to\infty} s_n(x)$. Falls $f$ beschränkt kann $(s_n)$
so gewählt werden, dass $s_n\unito g$ auf $A$.\fishhere
\end{prop}

\begin{prop}
\label{prop:7.13}
Sei $A\in\Sigma$.
\begin{propenum}
\item Für $A_j\in\Sigma$, $s=\sum_{j=1}^N \alpha_j \chi_{A_j}$ mit $\alpha_j\ge
0$ setze
\begin{align*}
\int_A s \dmu := \sum_{j=1}^N \alpha_j \mu(A_j).
\end{align*}
\item Sei $f: A\to [0,\infty]$ messbar. Dann
\begin{align*}
\int_A f \dmu := \sup
\setdef{\int_A s \dmu}{s\text{ einfach, positiv und }s\le f}.
\end{align*}
\item Sei $f: A\to\RA$ messbar.
\begin{align*}
\int_A f\dmu := \int_A f_+ \dmu - \int_A f_-\dmu,
\end{align*}
falls nicht beide Integrale $\infty$. Andernfalls ist das Integral nicht
definiert.
\item $f: A\to\RA$ heißt \emph{Lebesgue-integrierbar}, falls
\begin{align*}
\int_A f_+\dmu,\quad \int_A f_-\dmu < \infty.
\end{align*}
\item $f: A\to \C$ heißt \emph{Lebesgue-integrierbar}, falls $\Re f$ und $\Im
f$ Lebesgue-integrierbar. Schreibe dann $f\in L^1(A)$ und setze
\begin{align*}
\int_A f\dmu = \int_A \Re f \dmu + i\int_A \Im f\dmu.\fishhere
\end{align*}
\end{propenum}
\end{prop}

\begin{prop}[Eigenschaften]
\label{prop:7.14}
Seien $A,B\in\Sigma$, $f,g:A\to\RA$ oder $\C$ messbar.
\begin{propenum}
\item $f\in L^1(A)$ genau dann, wenn $\abs{f}\in L^1(A)$. Dann gilt
\begin{align*}
\abs{\int_A f\dmu} \le \int_A \abs{f}\dmu.
\end{align*}
\item Sind $f\in L^1(A)$ und $\abs{g}\le f$, so ist $g\in L^1(A)$ und
\begin{align*}
\int_A \abs{g}\dmu \le \int_A f\dmu.
\end{align*}
\item Die Abbildung
\begin{align*}
L^1(A) \to \R,\quad f\mapsto \int_A f\dmu 
\end{align*}
ist linear.
\item Falls $\mu(A)<\infty$ und $f$ $\mufu$ beschränkt, so ist $f\in L^1(A)$ und
\begin{align*}
\int_A \abs{f}\dmu \le \norm{f}_\infty \mu(A).
\end{align*}
\item Seien $a,b\in\R$ mit $a\le f(x)\le b\mufu$ auf $A$ und $\mu(A)<\infty$.
Dann ist $f\in L^1(A)$ und
\begin{align*}
a\mu(A) \le \int_A f\dmu \le b\mu(A).
\end{align*}
\item Sei $f\le g\mufu$ auf $A$, so ist
\begin{align*}
\int_A f\dmu \le \int_A g\dmu,
\end{align*}
falls beide Integrale existieren.
\item Sei $f\in L^1(A)$ und $B\subseteq A$, so ist $f\in L^1(B)$ und
\begin{align*}
\int_B \abs{f}\dmu \le \int_A \abs{f}\dmu.
\end{align*}
\item Sei $\mu(A) = 0$, so ist $\int_A f\dmu = 0$.
\item Sei $f\in L^1(A)$ und für jedes $B\in \Sigma$ gelte
\begin{align*}
B\subseteq A \Rightarrow \int_B f\dmu = 0,
\end{align*}
so ist $f=0\mufu$ auf $A$.
\item Sei $A\cap B = \varnothing$, $f\in L^1(A)$ und $f\in L^1(B)$, so ist
$f\in L^1(A\cap B)$ und
\begin{align*}
\int_{A\cup B} f \dmu = \int_A f\dmu + \int_B f\dmu.\fishhere
\end{align*} 
\end{propenum}
\end{prop}

\subsection{Konvergenzsätze}

\begin{prop}[Satz von der monotonen Konvergenz]
\label{prop:7.15}
Seien $A\in\Sigma$, $f_n : A\to [0,\infty]$ messbar und $0\le f_1(x)\le f_2(x)
\le \ldots$ auf $A$. Dann gilt
\begin{align*}
\lim\limits_{n\to\infty}\int_A f_n\dmu = 
\int_A \lim\limits_{n\to\infty }f_n \dmu.\fishhere
\end{align*}
\end{prop}

\begin{prop}[Lemma von Fatou]
\label{prop:7.16}
Seien $A\in\Sigma$, $f_n: A\to [0,\infty]$ messbar. Dann gilt
\begin{align*}
\int_A \liminf_n f_n \dmu \le \liminf_n \int_A f_n\dmu.\fishhere
\end{align*}
\end{prop}

\begin{prop}[Satz von der majorisierten Konvergenz]
\label{prop:7.17}
Seien $A\in\Sigma$, $f_n,f: A\to \RA$ messbar, $f_n\to f$. Existiert ein $g:
A\to \RA$ messbar mit $\int_A g\dmu < \infty$ und $\abs{f_n}\le g$, so gilt
\begin{align*}
\lim\limits_{n\to\infty} \int_A \abs{f_n-f}\dmu = 0.\fishhere 
\end{align*}
\end{prop}

\section{$L^p$-Räume}

Für alles Weitere sei $\Omega\subseteq \R^n$ ein Gebiet.

\begin{defn}
\label{defn:7.18}
\begin{defnenum}
\item Für $1\le p < \infty$ sei
\begin{align*}
L^p(\Omega) := \setdef{u:\Omega\to\C \text{ messbar}}{\norm{u}_p := \left(
\int_\Omega \abs{u}^p \dmu
\right)^{1/p} < \infty}.
\end{align*}
Identifizieren wir Funktionen, die sich nur auf Nullmengen unterscheiden, wird
$\norm{\cdot}_p$ zur Norm (Minkowski Ungleichung).
\item $u: \Omega\to \C$ heißt \emph{wesentlich beschränkt}, falls
\begin{align*}
\exists K > 0 : \abs{u}\le K\mufu
\end{align*} 
In diesem Fall heißt
\begin{align*}
\esssup(\abs{u}) := \inf\setdef{K>0}{\abs{u}\le K \mufu}
\end{align*}
\emph{wesentliches Supremum} von $u$.
\begin{align*}
L^\infty(\Omega) := \setdef{u:\Omega \to \C\text{ messbar}}{\norm{u}_\infty :=
\esssup\abs{u} < \infty}.
\end{align*}
Durch die obige Identifikation wird $\norm{\cdot}_\infty$ zur Norm.\fishhere
\end{defnenum}
\end{defn}

\begin{prop}
\label{prop:7.19}
\begin{propenum}
\item Für $1\le p\le \infty$ ist $\norm{\cdot}_p$ eine Norm.
\item Für $1\le p \le \infty$ ist $L^p(\Omega)$ ein Banachraum. $L^2(\Omega)$
ist mit
\begin{align*}
\lin{u,v} = \int_\Omega u\overline{v}\dmu
\end{align*}
ein Hilbertraum.\fishhere
\end{propenum}
\end{prop}

\begin{prop}
\label{prop:7.20}
\begin{propenum}
\item Die Menge
\begin{align*}
C_0^\infty (\Omega \to \C) :=
\setdef{\ph\in C^\infty(\Omega\to \C)}{\supp \ph := \overline{\setdef{x\in
\Omega}{\ph(x)\neq 0}}\text{ ist kompakt}}
\end{align*}
ist dicht in $L^p(\Omega)$ im Fall $1\le p < \infty$.
\item $L^p(\Omega)$ ist separabel.\fishhere
\end{propenum}
\end{prop}

Damit erhalten wir auch einen alternativen Zugang zum $L^p(\Omega)$ als
Abschluss von $C_0^\infty(\Omega\to\C)$ bezüglich der $\norm{\cdot}_p$-Norm.

\begin{proof}
Den Beweis verschieben wir auf später.\qedhere
\end{proof}

\begin{prop}
Sei $1\le p < \infty$. Die Abbildung
\begin{align*}
\Phi : L^q(\Omega)\to L^p(\Omega)'
\end{align*}
mit
\begin{align*}
\Phi(u)(f) = F_u(f) = \int_\Omega f\overline{u}\dmu
\end{align*}
ist eine isometrische antilineare Bijektion.\fishhere
\end{prop}

\begin{proof}
\begin{proofenum}
\item Seien $f\in L^p(\Omega)$, $u\in L^q(\Omega)$, so folgt mit der
Hölder-Ungleichung, dass $f\cdot \overline{u}\in L^1(\Omega)$, sowie
\begin{align*}
\int_\Omega \abs{f\overline{u}}\dmu \le \norm{f}_p \norm{u}_q.
\end{align*}
Somit ist $\Phi(u)$ wohldefiniert und linear, da das Integral linear ist. Es ist
sofort ersichtlich, dass
\begin{align*}
\norm{\Phi(u)} \le \norm{u}_q,
\end{align*}
insbesondere ist $\Phi(u)\in L^q(\Omega)$.
\item Setzen wir
\begin{align*}
f(x) = 
\begin{cases}
\abs{u(x)}^{\frac{q}{p}-1}u(x), & u(x)\neq 0,\\
0, & \text{sonst},
\end{cases}
\end{align*}
so ist 
\begin{align*}
\norm{f}_p^p = \int_\Omega \abs{f}^p \dmu = \int_\Omega \abs{u}^q \dmu =
\norm{u}_q^p.
\end{align*}
Folglich ist auch
\begin{align*}
\abs{\Phi(u)(f)} &= \abs{\int_\Omega f\overline{u}\dmu}
=
\abs{\int_\Omega \abs{u}^{\frac{q}{p}-1} u\overline{u}\dmu}
= 
\abs{\int_\Omega \abs{u}^{\frac{q}{p}+1}\dmu}\\
&=
\int_\Omega \abs{u}^{q}\dmu
= \norm{u}_q^q \overset{!}{=}\norm{f}_p\norm{u}_q,
\end{align*}
denn $\norm{f}_p\norm{u}_q = \norm{u}_q^{\frac{q}{p}+1} = \norm{u}_q^q$, also
existiert ein $f\in L^p(\Omega)$ mit
\begin{align*}
\abs{\Phi(u)(f)} = \norm{f}_p=\norm{u}_q
\end{align*}
Somit ist $\Phi$ normerhaltend.
\item $\Phi(\alpha u + \beta v) = \overline{\alpha}\Phi(u) +
\overline{\beta}\Phi(v)$ ist klar.
\item $\Phi$ ist injektiv, da isometrisch. Zur Surjektivität siehe \cite{Alt99}
Kapitel 4.3. Der Beweis basiert auf dem Satz von Radon-Nikodym.\qedhere
\end{proofenum}
\end{proof}

\begin{cor}
\label{prop:7.22}
Für $1<p<\infty$ ist $L^p(\Omega)$ reflexiv.\fishhere
\end{cor}
\begin{proof}
Analoger Beweis zu \ref{prop:5.18}.\qedhere
\end{proof}

\begin{prop}
\label{prop:7.23}
$L^p(\Omega)' = (C_0^\infty(\Omega\to\C),\norm{\cdot}_p)'$.\fishhere
\end{prop}

Insbesondere ist $F_u$ (mit $u\in L^q(\Omega)$) und damit auch $u\in
L^q(\Omega)$ eindeutig bestimmt durch die Werte $F_u(\ph)$ für $\ph\in
C_0^\infty(\Omega\to\C)$.

\begin{proof}
Zu $F\in L^p(\Omega)'$ ist
\begin{align*}
F\Big|_{C_0^\infty(\Omega\to\C)} \in 
(C_0^\infty(\Omega\to\C),\norm{\cdot}_p)'.
\end{align*}
Umkehert existiert zu $f\in (C_0^\infty(\Omega\to\C),\norm{\cdot}_p)'$ nach dem
Fortsetzungssatz \ref{prop:2.5} eine eindeutige Fortsetzung $F\in
L^p(\Omega)'$, da $C_0^\infty(\Omega\to\C)$ dicht in $L^p(\Omega)$.\qedhere 
\end{proof}

\section{Verallgemeinerung der Ableitung I}

Sei wieder $\Omega\subset\R^n$ ein Gebiet.

\begin{bsp}
\label{bsp:7.24}
Seien $\Omega=\R^n$, $f\in C^1(\R^n\to \C)$ und $f,\partial_{x_i} f \in
L^p(\Omega)$. Identifiziere $L^p(\Omega)$ mit $L^q(\Omega)'$, also $F_f$ mit
$f$. Dann sind $\partial_{x_i} f$ bzw $F_{\partial_{x_i} f}$ eindeutig bestimmt
durch die Werte auf $C_0^\infty(\Omega\to\C)$.
\begin{align*}
F_{\partial_{x_i}f}(\ph) &=
\int_{\R^n} \ph\overline{\partial_{x_i}f}\dmu = 
\int_{\supp \ph} \ph\overline{\partial_{x_i}f}\dmu
\overset{\text{part.int.}}{=}
-\int_{\supp \ph} \partial_{x_i}\ph\overline{f}\dmu\\
&= -F_{f}(\partial_{x_i}\ph).\bsphere
\end{align*}
\end{bsp}

\begin{defn}
\label{defn:7.25}
Seien $1\le p < \infty$, $f\in L^p(\Omega)$, $\alpha\in \N_0^n$ und $u\in
L^p(\Omega)$ mit
\begin{align*}
\forall \ph \in C_0^\infty(\Omega\to \C) : F_u(\ph) =
(-1)^{\abs{\alpha}}F_f(\nabla^\alpha \ph) 
\end{align*}
bzw.
\begin{align*}
\int_\Omega \ph\overline{u}\dmu =
(-1)^{\abs{\alpha}}\int_\Omega \nabla^\alpha \ph \overline{f}\dmu.
\end{align*}
Dann heißt $u$ \emph{schwache Ableitung}\index{schwache!Ableitung} von $f$ der
Ordnung $\alpha$. Schreibweise
\begin{align*}
u:= \D^\alpha f.\fishhere
\end{align*}
\end{defn}
Für $\ph\in C_0^\infty(\Omega\to\C)$ gilt also
\begin{align*}
\lin{\D^\alpha f,\ph} = 
\int_\Omega (\D^\alpha f)\overline{\ph} \dmu = 
(-1)^{\abs{\alpha}}\int_\Omega f \nabla^\alpha \overline{\ph}\dmu = 
(-1)^{\abs{\alpha}}\lin{f, \nabla^\alpha\ph}.
\end{align*}

Für alles Folgende seien - sofern nicht anders angegeben -
$\alpha,\beta\in\N_0^n$.

\begin{cor}
\label{prop:7.26}
Sei $f\in C^k(\Omega\to\C)$ mit $\nabla^\alpha f \in L^p(\Omega)$ für
$\abs{\alpha}\le k$. Dann gilt
\begin{align*}
\nabla^\alpha f = \D^\alpha f.\fishhere
\end{align*}
\end{cor}
\begin{proof}
Sei $\ph\in C_0^\infty(\Omega\to\C)$,
\begin{align*}
\int_\Omega \ph \overline{\nabla^\alpha f}\dmu
\overset{\text{part.int.}}{=}
 (-1)^{\abs{\alpha}}\int_\Omega \nabla^\alpha \ph \overline{f}\dmu.\qedhere
\end{align*}
\end{proof}

\begin{bsp}
\label{bsp:7.27}
Sei $\Omega=(a,b)$ mit $0\in (a,b)$ und $f(x) = \abs{x}$. Bestimme $\D^1 f$.
\begin{align*}
\int_a^b \ph(x) \overline{D^1 f(x)}\dx
&= - \int_a^b \ph'(x)\overline{f(x)}\dx \\ &= 
\int_a^0 x\ph'(x)\dx - \int_0^b x\ph'(x)\dx\\
&\overset{\text{part.int.}}{=}
-\int_a^0 \ph(x)\dx + \int_0^b \ph(x)\dx\\
&= \int_a^b \sign(x)\ph(x)\dx,
\end{align*}
wobei
\begin{align*}
\sign(x) = 
\begin{cases}
-1, & x < 0,\\
1, &  x > 0.
\end{cases}
\end{align*}
Die Definition von $\sign$ im Nullpunkt ist irrelevant.

Somit ist für jedes $\ph\in C_0^\infty(\Omega\to \C)$
\begin{align*}
\int_a^b \ph \overline{\D^1 f}\ph\dmu = 
\int_a^b \ph \overline{\sign(x)}\ph\dmu 
\end{align*}
also $\D^1 f = \sign(x)$ in $L^p(\Omega)$.

$\sign(x)$ ist jedoch in $L^p(\Omega)$ nicht schwach differenzierbar (vgl.
Distributionentheorie).\bsphere
\end{bsp}


\begin{defn}
\label{defn:7.28}
Sei $1\le p < \infty$ und $m\in\N$. Dann heißt
\begin{align*}
W^{m,p}(\Omega) := \setdef{u\in L^p(\Omega)}{\D^\alpha u \in L^p(\Omega),\quad
\forall \abs{\alpha}\le m}
\end{align*}
versehen mit der Norm
\begin{align*}
\norm{u}_{m,p} = \left(\sum_{\abs{\alpha}\le m} \norm{\D^\alpha
u}_p^p\right)^{1/p}
\end{align*}
\emph{Sobolevraum}\index{Sobolevraum} der Ordnung $m$.\fishhere
\end{defn}

\begin{prop}
\label{prop:7.29}
$W^{m,p}(\Omega)$ ist ein Banachraum. $W^{m,2}(\Omega)$ ein
Hilbertraum.\fishhere
\end{prop}
\begin{proof}
\begin{proofenum}
\item Sei $(u_n)$ Cauchyfolge in $W^{m,p}(\Omega)$, so ist für $\abs{\alpha}\le
m$ auch $(\D^\alpha u)$ Cauchyfolge in $L^p(\Omega)$ und damit konvergent,
$\D^\alpha u_n \to g_\alpha$.

Wir zeigen nun $\D^\alpha g_0 = g_\alpha$ und $\norm{u_n-g_0}_{m,p} \to 0$.

Für $\ph\in C_0^\infty(\Omega)$ gilt
\begin{align*}
F_{g_\alpha}(\ph) &= \lim\limits_{n\to\infty} F_{\D^\alpha u_n}(\ph)
= \lim\limits_{n\to\infty}(-1)^{\abs{\alpha}}F_{u_n}(\nabla^\alpha \ph)\\
&= (-1)^{\abs{\alpha}} F_{g_0}(\ph)(\nabla^\alpha \ph) = F_{\D^\alpha g_0}(\ph).
\end{align*}
Also ist $g_\alpha = \D^\alpha g_0$ und daher
\begin{align*}
\norm{u_n-g_0}_{m,p}^p
= \sum_{\abs{\alpha}\le m}\norm{\D^\alpha u_n - \D^\alpha g_0}_{p}^p
=
\sum_{\abs{\alpha}\le m}\norm{\D^\alpha u_n - g_\alpha}_{p}^p
\to 0.
\end{align*}
\item $W^{2,m}$ ist Hilbertraum mit dem Skalarprodukt
\begin{align*}
\lin{u,v} = \sum_{\abs{\alpha}\le m} \lin{\D^\alpha u,\D^\alpha v}.\qedhere
\end{align*}
\end{proofenum}

\end{proof}

\begin{prop}
\label{prop:7.30}
$W^{m,p}(\Omega)$ ist separabel und reflexiv.\fishhere
\end{prop}
\begin{proof}
Sei $\alpha_1,\ldots,\alpha_k$ eine Nummerierung aller $\alpha$ mit
$\abs{\alpha}\le m$. Setze
\begin{align*}
P : W^{m,p}(\Omega)\to L^p(\Omega)^k,\quad
u\mapsto (\D^{\alpha_1}u,\ldots,\D^{\alpha_k}u),  
\end{align*}
so ist $P$ linear und isometrisch, denn
\begin{align*}
\norm{u} = 
\left(\sum_{j=1}^k \norm{\D^{\alpha_j}u}_p^p \right)^{1/p}
 = \norm{(\D^{\alpha_1}u,\ldots,\D^{\alpha_k}u)}_{L^p(\Omega)^k}.
\end{align*}
Diese Norm ist äquivalent zur Standard-Norm des $L^p(\Omega)^k$ und daher ist
$L^p(\Omega)^k$ auch bezüglich dieser Norm separabel und reflexiv.

$P$ ist Isometrie, also ist $W^{m,p}(\Omega)\cong \im P \subseteq L^p(\Omega)^k$
ein Banachraum und damit abgeschlossen. Da $L^p(\Omega)^k$ separabel, ist
somit auch $W^{m,p}(\Omega)$ separabel. Mit \ref{prop:4.23} folgt außerdem, dass
$W^{m,p}(\Omega)$ reflexiv.\qedhere
\end{proof}

Wir wollen nun die Produktregel auf die schwache Ableitung verallgemeinern. Für
$u,v\in L^p(\Omega)$ ist jedoch nicht zwingend $u\cdot v\in L^p(\Omega)$. Somit
ist klar, dass die Produktregel nur unter zusätzlichen Voraussetzungen an $u$
und $v$ gelten kann.

\begin{prop}[Produktregel]
\index{schwache!Produktregel}
\label{prop:7.31}
Sei $m\in\N$, $1\le p < \infty$, $u\in W^{m,p}(\Omega)$ und $f\in
C_0^\infty(\Omega\to \C)$. Dann gelten
\begin{propenum}
\item $f\cdot u\in W^{m,p}$.
\item $\forall \abs{\alpha}\le m : \D^\alpha (f\cdot u) = \sum_{\beta\le
\alpha} \binom{\alpha}{\beta} (\nabla^\beta f) (\D^{\alpha-\beta}u)$.\fishhere
\end{propenum}
\end{prop}

\begin{proof}
Sei $m=1$ und $\alpha=e_j$, so gilt für $\ph\in C_0^\infty(\Omega\to\C)$,
\begin{align*}
-\lin{f\cdot u,\nabla^{e_j}\ph} &= - \int_\Omega f\cdot u
\overline{\partial_{x_j}\ph}\dmu\\
&= - \int_\Omega u
\partial_{x_j}(f\overline{\ph})\dmu
+ \int_\Omega u
(\partial_{x_j}f)\overline{\ph}\dmu\\
&=
\int_\Omega \D^{e_j} u\,
f\,\overline{\ph}\dmu
+ \int_\Omega u\,
\partial_{x_j}f\,\overline{\ph}\dmu\\
&= \lin{f D^{e_j}u + \partial_{x_j}f u,\ph}
\end{align*}
Somit folgt, $\D^{e_j} (f\cdot u) = f\D^{e_j} u + \partial_{x_j}f\, u$ und
insbesondere, $\D^{e_j}(f\cdot u)\in L^p(\Omega)$, also $f\cdot u\in W^{1,p}$.

Der Rest folgt mit Induktion über $m$.\qedhere
\end{proof}

\begin{defn}
\label{defn:7.32}
Sei die partielle Differentialgleichung
\begin{align*}
\sum_{\abs{\alpha}\le m} a_\alpha\, \nabla^\alpha u = f,\qquad
a_\alpha\in C_0^\infty(\Omega\to\C),\quad f\in L^p(\Omega)
\end{align*}
gegeben. Falls $u\in L^p(\Omega)$ und für alle $\ph\in C_0^\infty(\Omega\to\C)$
gilt,
\begin{align*}
\sum_{\abs{\alpha}\le m} (-1)^{\abs{\alpha}}
\lin{u,\nabla^\alpha(\overline{a_\alpha}\ph)} = \lin{f,\ph},
\end{align*}
so heißt $u$ \emph{schwache Lösung}\index{schwache!Lösung} der
Differentialgleichung.\fishhere
\end{defn}

In obiger Definition wird nicht $u\in W^{m,p}(\Omega)$ vorausgesetzt, denn
falls zusätzlich $u\in W^{m,p}(\Omega)$ und $u$ schwache Lösung, gilt sogar
\begin{align*}
\sum_{\abs{\alpha}\le m} a_\alpha\, \D^\alpha u = f.
\end{align*}

\newcommand{\WO}{W_0}
\begin{defn}
\label{defn:7.33}
$\WO^{m,p}(\Omega) :=
\overline{C_0^\infty(\Omega\to\C)}^{W^{m,p}(\Omega)}$.\fishhere
\end{defn}

$\WO^{m,p}(\Omega)$ ist als abgeschlossene Teilmenge des Banachraums
$W^{m,p}(\Omega)$ selbst ein Banachraum und $\WO^{m,2}(\Omega)$ ein Hilbertraum.


\begin{prop}[Veranschaulichung]
\label{prop:7.34}
Sei $1<p<\infty$ und $u\in W_0^{1,p}((0,1))\cap C([0,1])$. Dann gilt $u(0) =
u(1)=0$. Allgemeiner: Ist $u\in W_0^{m,p}((0,1))\cap C^{m-1}([0,1])$, so gilt
\begin{align*}
\forall j=0,\ldots,m-1 : u^{(j)}(0) = 0 = u^{(j)}(1).\fishhere
\end{align*}
\end{prop}
\begin{proof}
$u$ ist stetig auf $[0,1]$, d.h. insbesondere ist
\begin{align*}
u(0) =\lim\limits_{h\to0}\frac{1}{h} \int_0^h u(x)\dx
\end{align*}
und da $u\in W_0^{1,p}((0,1))$, existiert eine Folge $(u_n)$ in
$C_0^\infty((0,1)\to\C)$ mit $\norm{u_n-u}_{1,p}\to 0$, wobei
\begin{align*}
u_n(x) = \int_0^x u_n'(s)\ds \le
\left(\int_0^x 1\ds\right)^{1/q}
\left(\int_0^x \abs{u_n'(s)}^p\ds\right)^{1/p}
\le C\cdot x^{1/q},
\end{align*}
mit $C=\sup\limits_{n\ge 1} \norm{u_n}_p < \infty$. Sei $A\subset[0,1]$
messbar, so gilt $\norm{(u_n-u)\chi_A}_1 \le
\underbrace{\norm{\chi_A}_q}_{\const}\underbrace{\norm{u_n-u}_p}_{\to 0}$. Somit
gilt
\begin{align*}
\abs{\int_0^h u(x)\dx} = \lim\limits_{n\to\infty}
\abs{\int_0^h u_n(x)\dx} \le C \int_0^h x^{1/q}\dx = C\frac{h^{1+1/q}}{1+1/q}.
\end{align*}
also
\begin{align*}
\abs{u(0)} = \lim\limits_{h\to 0}\frac{1}{h}
\abs{\int_0^h u(x)\dx} \le
\lim\limits_{h\to 0}\frac{1}{h}
C\frac{h^{1+1/q}}{1+1/q} = 0.
\end{align*}
$u(1)=0$ folgt analog.\qedhere
\end{proof}

\begin{prop}
\label{prop:7.35}
Das Dirichlet Problem
\begin{align*}
-\Delta u + u = f\text{ in }\Omega,\qquad u\big|_{\delta \Omega} = 0
\end{align*}
besitzt für $f\in L^2(\Omega)$ in $L^2(\Omega)$ eine eindeutig bestimmte
schwache Lösung $u$ mit $u\in W_0^{1,2}(\Omega)$, so dass für alle $\ph\in
C_0^\infty(\Omega\to\C)$ gilt
\begin{align*}
-\lin{u,\Delta \ph} + \lin{u,\ph} = \lin{f,\ph}.\fishhere
\end{align*}
\end{prop}
\begin{proof}
\textit{Eindeutigkeit}.
Seien $u,v$ Lösungen der inhomogenen Gleichung, so ist $u-v$ Lösung der
homogenen Gleichung. Es genügt also sich auf den homogenen Fall zu beschränken.

Sei $u\in W_0^{1,2}(\Omega)$ Lösung von $-\Delta u + u = 0$ und $(u_n)$
Folge in $C_0^\infty(\Omega\to\C)$ mit $\norm{u_n-u}_{1,2}\to 0$, so gilt
\begin{align*}
0 &= -\lin{u,\Delta u_n} + \lin{u,u_n} = \sum_{j=1}^n
\lin{\D^{e_j}u,\partial_{x_j}u_n} + \lin{u,u_n}\\
&\to \lin{\D^{e_j}u,\D^{e_j}u} + \lin{u,u} = \norm{u}_{1,2}^2.
\end{align*}
Folglich ist $u=0$ und damit die Lösung eindeutig.

\textit{Existenz}. Sei $F: W_0^{1,2}(\Omega)\to \C$ mit
\begin{align*}
F(v) = \lin{v,f},
\end{align*}
so ist $F\in  W_0^{1,2}(\Omega)'$, denn
\begin{align*}
\abs{F(v)} \le \norm{v}_{0,2}\norm{f}_2 \le \norm{f}_2 \norm{v}_{1,2}.
\end{align*}
Das Lemma von Riesz besagt nun, dass es genau ein $u\in W_0^{1,2}(\Omega)$
existiert so dass für alle $v\in W_0^{1,2}(\Omega)$ gilt
\begin{align*}
\lin{v,f}_2 = F(v) = \lin{v,u}_{1,2}
= \sum_{j=1}^n \lin{\D^{e_j}v,\D^{e_j}u} + \lin{v,u}.
\end{align*}
Für $v=\ph\in C_0^\infty(\Omega\to\C)$ folgt
\begin{align*}
\lin{f,\ph} = \sum_{j=1}^n \lin{\D^{e_j}u,\nabla^{e_j}\ph} + \lin{u,\ph}
= -\sum_{j=1}^n \lin{u,\Delta\ph} + \lin{u,\ph}.\qedhere
\end{align*}
\end{proof}

\begin{bem}
\label{bem:7.36}
Offensichtlich gilt
\begin{align*}
\sum_{j=1}^n \D^{2e_j}u = f-u \in L^2(\Omega).
\end{align*}
Die elliptische Regularitätstheorie zeigt, dass sogar $u\in W^{2,2}(\Omega)$,
falls $\Omega$ ``genügend gut''.\maphere
\end{bem}

\begin{prop}
\label{prop:7.37}
Sei $1<p<\infty$, $u\in L^p(\Omega)$ mit $\D^\alpha u=0$ für $\abs{\alpha}=1$.
Dann folgt $u=c$. Falls $\mu(\Omega)=\infty$ oder $u\in W_0^{1p}(\Omega)$ folgt
sogar $u=0$.\fishhere
\end{prop}
\begin{proof}
\begin{proofenum}
\item Sei $n=1$, $\Omega=(a,b)$ mit $-\infty\le a < b \le \infty$. Sei $\psi\in
C_0^\infty((a,b)\to \C)$ mit
\begin{align*}
\int_a^b \psi(x)\dx = 1,
\end{align*}
so lässt sich jedes $\ph\in C_0^\infty((a,b)\to\C)$ schreiben als
\begin{align*}
\ph = \tilde{\ph} + \left(\int_a^b \ph(t)\dt\right)\psi.
\end{align*}
Insbesondere ist $\int_a^b \tilde{\ph}\dmu = 0$ und $\tilde{\ph}\in C_0^\infty
((a,b)\to\C)$. Setze nun
\begin{align*}
\Phi(x) := \int_a^x \tilde{\ph}(t)\dt,
\end{align*}
so ist $\Phi\in C_0^\infty((a,b)\to\C)$, denn
\begin{align*}
\Phi(x) = 0
\begin{cases}
\text{für } a<x \le \min\supp\tilde{\ph},\\
\text{für } \max\supp\tilde{\ph}\le x < b.
\end{cases}
\end{align*}
Damit können wir $\ph$ darstellen durch,
\begin{align*}
\ph = \Phi' + \left(\int_a^b \ph(t)\dt\right)\psi. 
\end{align*}

Sei nun $u$ wie vorausgesetzt, so gilt
\begin{align*}
F_u(\ph) &= \lin{u,\ph} = \lin{u,\Phi'}
+ \int_a^b \overline{\ph}(t)\dt \lin{u,\psi}\\
&= -\underbrace{\lin{D^1u,\Phi}}_{=0} + \int_a^b \overline{\ph}(t)\dt
\underbrace{\lin{u,\psi}}_{=c} = \lin{c,\ph}.
\end{align*}
Und da $\ph$ beliebig war, ist $u=c$.
\item Wir behandlen exemplarisch den Fall $n=2$, die übrigen Fälle ergeben sich
dann automatisch.
\begin{enumerate}[label=\alph{*}),leftmargin=0pt]
  \item Sei $(x_1,x_2)\in\Omega$ und $W_\ep :=
  (x_1-\ep,x_1+\ep)\times(x_2-\ep,x_2+\ep)\subseteq \Omega$. Wähle $\psi_j\in
  C_0^\infty((x_j-\ep,x_j+\ep)\to\C)$ mit
\begin{align*}
\int_{x_j-\ep}^{x_j+\ep} \psi_j(x) = 1,\qquad j=1,2.
\end{align*}
Wir zeigen nun, dass $u=0$ auf $W_\ep$.

Für festes $y_2\in (x_2-\ep,x_2+\ep)$ besitzt $\ph\in C_0^\infty(W_\ep\to\C)$
analog zum Vorangegangenen eine Darstellung,
\begin{align*}
\ph(y_1,y_2) = F_{y_2}'(y_1) + \underbrace{\int_{x_1-\ep}^{x_1+\ep}
\ph(t,y_2)\dt}_{:=g(y_2)} \psi_1(y_1)
\end{align*}
mit $F_{y_2}\in C_0^\infty((x_1-\ep,x_1+\ep)\to\C)$. Aus dem ersten Teil folgt
ebenfalls, dass ein $G\in C_0^\infty((x_2-\ep,x_2+\ep)\to\C)$ existiert, so dass
\begin{align*}
g(y_2) = G'(y_2) + \int_{x_2-\ep}^{x_2+\ep} g(s)\ds \psi_2(y_2).
\end{align*}
Wir können somit $\ph$ schreiben als
\begin{align*}
\ph(y_1,y_2) =
\underbrace{F_{y_2}'(y_1)}_{\partial_1 \Phi_1(y_1,y_2)}
% \partial_1 \Phi_1(y_1,y_2)
+ \underbrace{G'(y_2)\psi_1(y_1)}_{\partial_2 \Phi_2(y_1,y_2)}
+ \int_{W_\ep} \ph\dmu\, \psi_1(y_1)\psi_2(y_2).
\end{align*}
Somit gilt (wobei $\lin{\cdot,\cdot}_\ep$ das Integral über $W_\ep$ beschreibt),
\begin{align*}
\lin{u,\ph}_\ep &= \lin{u,\partial_1 \Phi_1}_\ep
+ \lin{u,\partial_2 \Phi_2}_\ep +
\left(\int_{W_\ep}
\overline{\ph}\dmu\right)\underbrace{\lin{u,\psi_1\psi_2}_\ep}_{:=c}\\
&=\lin{u,\partial_1 \Phi_1}_\ep
+ \lin{u,\partial_2 \Phi_2}_\ep + \lin{c,\ph}_\ep
\end{align*}
Setzen wir $\Phi_1,\Phi_2$ durch Null auf ganz  $\Omega$ fort, so gilt
\begin{align*}
\lin{u,\ph}_\ep &= \lin{u,\partial_1 \Phi_1}
+ \lin{u,\partial_2 \Phi_2} +  \lin{c,\ph}_\ep
\\ &=
\underbrace{\lin{\D^{e_1}u, \Phi_1}
+ \lin{\D^{e_2}u,\Phi_2}}_{=0} + 
  \lin{c,\ph}_\ep.
\end{align*}
Also ist $u=c$ auf $W_\ep$.
\item Seien $x,y\in\Omega$. Da $\Omega$ wegzusammenhängend, existiert ein
Polygonzug $\Gamma$ von $x$ nach $y$, der ganz in $\Omega$ verläuft. Nun ist
$\Gamma$ kompakt, also kann ganz $\Gamma$ mit endlich vielen $W_\ep^{(j)}$
überdeckt werden, $j=1,\ldots,J$. Auf jedem $W_\ep^{(j)}$ ist $u$ konstant.

Da $\Gamma$ alle $W_\ep^{(j)}$ durchläuft und aneinandergrenzende $W_\ep^{(j)}$
nichtleeren Schnitt haben, ist $u=c$ mit einer einzigen Konstanten für alle
$W_\ep^{(j)}$. Somit ist $u=c$ auf $\Omega$.
\end{enumerate}
\item Sei $u=c$ und $\mu(\Omega)=\infty$, so gilt
\begin{align*}
\int_\Omega \abs{u}^p \dmu = \abs{c}^p\mu(\Omega)< \infty,
\end{align*}
also ist $c=0$.
\item Sei $u\in W_0^{1,p}(\Omega)$ und $\mu(\Omega) < \infty$. Eine leichte
Übung zeigt
\begin{align*}
\lin{\D^{e_j}v,u} = -\lin{v,\D^{e_j}u},
\end{align*}
für $v\in W^{1,p}(\Omega)$.  Setzen wir $v(x) = \arctan(x_1)$, so ist
\begin{align*}
\partial_1 v(x) = \frac{1}{1+x_1^2},\qquad \partial_j v(x) = 0,\quad
j=2,\ldots,n.
\end{align*}
Somit ist $v\in C^1(\Omega\to\R)$ und $v,\nabla v$ sind beschränkt, also ist
$v\in W^{1,p}(\Omega)$, da $\mu(\Omega)<\infty$.

Nach Voraussetzung ist $\lin{v,\D^{e_j}u} = 0$, also gilt
\begin{align*}
0 = -\lin{v,\D^{e_1}u} = \lin{D^{e_1}v,u} = \lin{\partial_1 v, c}
= \overline{c} \underbrace{\int_\Omega \frac{1}{1+x_1^2}\dx}_{>\mu(\Omega)\neq
0}
\end{align*}
und folglich ist $c=0$.\qedhere
\end{proofenum}
\end{proof}

\section{Verallgemeinerung der Ableitung II}

Für alles Weitere sei wieder $\Omega\subseteq\R^n$ ein Gebiet.

\begin{defn}
\label{defn:7.38}
\begin{defnenum}
\item Seien $1\le p<\infty$, $u\in L^p(\Omega)$ und
$\alpha\in\N_0^n$. Falls eine Folge $(u_n)$ in $C^\infty(\Omega\to\C)$
existiert mit $u_n\to u$ und $\nabla^\alpha u_n \to v$ in $L^p(\Omega)$,
so heißt $v$ \emph{starke Ableitung}\index{starke Ableitung} von $u$ der
Ordnung $\alpha$. Schreibe $\D^\alpha_s u := v$.
\item $H^{m,p}(\Omega) := \overline{\setdef{\ph\in
C^\infty(\Omega\to\C)}{\norm{\ph}_{m,p}<\infty}}$ heißt \emph{$m$-ter
Sobolevraum}\index{Sobolevraum}.

$H_0^{m,p}(\Omega) := W_0^{m,p}(\Omega)$.\fishhere
\end{defnenum}
\end{defn}

\begin{prop}
\label{prop:7.39}
\begin{propenum}
\item Sei $u\in L^p(\Omega)$ mit $\D_s^\alpha u\in L^p(\Omega)$ für ein
$\alpha\in\N_0^n$, so stimmen schwache und starke Ableitung überein,
\begin{align*}
\D_s^\alpha u = \D^\alpha u.
\end{align*}
\item $H^{m,p}(\Omega) \subseteq W^{m,p}(\Omega)$ und $H^{m,p}(\Omega)$ ist ein
Banachraum.\fishhere
\end{propenum}
\end{prop}
\begin{proof}
\begin{proofenum}
\item Sei $\ph\in C_0^\infty(\Omega\to\C)$ und $u$ wie vorausgesetz, dann
\begin{align*}
\lin{\D^\alpha u,\ph} &= (-1)^{\abs{\alpha}}\lin{u,\nabla^\alpha \ph}
= \lim\limits_{n\to\infty}
(-1)^{\abs{\alpha}}\lin{u_n,\nabla^\alpha \ph}\\
&= \lim\limits_{n\to\infty}
(-1)^{\abs{\alpha}}\int_\Omega u_n \overline{\nabla^\alpha \ph}\dmu
= \lim\limits_{n\to\infty}
\int_\Omega \nabla^\alpha u_n \overline{\ph}\dmu\\
&=\lin{\D_s^\alpha u,\ph}.
\end{align*}
Da die rechte Seite existiert, existiert auch die linke und somit ist
$\D^\alpha u = \D^\alpha_s u$.
\item Sei $\ph\in C^\infty(\Omega\to\C)$ und $\norm{\ph}_{m,p}< \infty$, so
ist $\ph\in W^{m,p}(\Omega)$. Also ist $H^{m,p}(\Omega)$ sinnvoll definiert und
$\subseteq W^{m,p}(\Omega)$. Da $H^{m,p}(\Omega)$ abgeschlossener Teilraum von
$W^{m,p}(\Omega)$ ist $H^{m,p}(\Omega)$ Banachraum.\qedhere
\end{proofenum}
\end{proof}

\begin{prop}[Satz von Meyers und Serrin (1964)]
\index{Satz!Meyers und Serrin}
Für $1\le p < \infty$ und $m\in\N$ ist
\begin{align*}
W^{m,p}(\Omega)=H^{m,p}(\Omega).\fishhere
\end{align*}
\end{prop}
\begin{proof}
Zum Beweis siehe \cite{Adams75}.\qedhere
\end{proof}

\begin{defn}
\label{defn:7.41}
Gilt $u\in H^{m,p}(\Omega)$ und
\begin{align*}
\sum_{\abs{\alpha}\le m} a_\alpha \D^\alpha u = f,\qquad
\qquad
a_\alpha\in C_b(\Omega\to\C),\quad f\in L^p(\Omega).
\end{align*}
Dann heißt $u$ \emph{starke Lösung}\index{starke Lösung} der
Differentialgleichung.

Gilt sogar $u\in C^m(\Omega\to\C)$, so heißt $u$ \emph{klassische
Lösung}\index{klassische Lösung}.\fishhere
\end{defn}

\section{Approximation}

Für alles Weitere sei $\Omega\subset O$ offen.

\begin{defn}
\label{defn:7.42}
\begin{defnenum}
\item Seien $M$ und $K$ Mengen mit $K\subset M$ und $K$ kompakt, so schreiben
wir $K\Subset M$.\index{kompakt enthalten}
\item Sei $A\subset\R^n$ messbar, dann ist
\begin{align*}
L^1_\loc(A) := \setdef{u: A\to\C}{\forall K\Subset A : u\big|_K \in L^1(K)}.
\end{align*}
\item Für $u: O\to\C$ sei die Nullfortsetzung\index{Nullfortsetzung} von
$u$ bezeichnet mit
\begin{align*}
\tilde{u}(x) := \begin{cases}
                u(x), & x\in \Omega,\\
                0, & \text{sonst}.
                \end{cases}
\end{align*}  
\item Sei $j\in C_0^\infty(\R^n\to\R)$ mit
\begin{equivenum}
\item $j(x)\ge 0$ auf $\R^n$,
\item $j(x) = 0$ für $\abs{x}\ge 1$,
\item $\int_{\R^n} j\dmu = 1$.
\end{equivenum}
Zu $\ep > 0$ sei
\begin{align*}
j_\ep(x) := \frac{1}{\ep^n}j\left(\frac{x}{\ep}\right),
\end{align*}
so ist
\begin{equivenum}
\item $j_\ep(x) = 0$ für $\abs{x}\ge \ep$,
\item $\int_{\R^n} j_\ep \dmu = 1$.
\end{equivenum}
Für $u\in L^1_\loc(\overline{O})$ sei
\begin{align*}
J_\ep u(x) := j_\ep * u(x) := \int_{\R^n} j_\ep(x-y)u(y)\dy. 
\end{align*}
$J_\ep$ heißt \emph{Glättungsoperator
(Mollifier)}\index{Glättungsoperator}\index{Mollifier}.\fishhere
\end{defnenum}
\end{defn}

Man rechnet leicht nach, dass eine Funktion $j$ mit den obigen Eigenschaften
gegeben ist durch
\begin{align*}
j(x) = \frac{1}{\int_{\R^n} f(x)\dx} f(x),\qquad f(x) = e^{-(1+\abs{x}^2)^{-1}}.
\end{align*}

\begin{prop}
\label{prop:7.43}
Sei $u\in L^1_\loc(\overline{O})$, so ist $J_\ep u \in
C^\infty(\R^n\to\C)$ und
\begin{align*}
\nabla^\alpha J_\ep u(x) = \int_{\R^n} \nabla^\alpha
j_\ep(x-y)\tilde{u}(y)\dy.\fishhere
\end{align*}
\end{prop}
\begin{proof}
\begin{proofenum}
\item $j_\ep$ ist stetig differenzierbar und hat kompakten Träger, ist also
global Lipschitz-stetig mit Konstante $L$.

Sei nun $x\in\R^n$ und $K_\ep(x)$ Umgebung von $x$.
Da $u\in L^1_\loc(\overline{O})$, wird das Integral über
$\overline{K_{2\ep}(x)\cap\Omega}$ durch eine
Konstante $C$ majorisiert. Also gilt für $x'\in K_\ep(x)$,
\begin{align*}
\abs{J_\ep u(x)-J_\ep u(x')} &\le 
\int_{\R^n} \abs{j_\ep(x-y)-j_\ep(x'-y)}\abs{\tilde{u}(y)}\dy\\
&=\int_{K_{2\ep}(x)} \abs{j_\ep(x-y)-j_\ep(x'-y)}\abs{\tilde{u}(y)}\dy\\
&\le  L\abs{x-x'} \int_{\overline{K_{2\ep}(x)\cap \Omega}}
\abs{\tilde{u}(y)}\dy
\le L\, C\, \abs{x-x'}.
\end{align*}
Somit ist $J_\ep u$ sogar lokal lipschitz stetig.
\item Mit dem verallgemeinerten Mittelwertsatz und dem Satz von der
majorisierten Konvergenz zeigt man analog,
\begin{align*}
\nabla^\alpha J_\ep u(x) = \int_{\R^n} \nabla^\alpha
j_\ep(x-y)\tilde{u}(y)\dy.\qedhere
\end{align*}
\end{proofenum}
\end{proof}

\begin{prop}[Abschneidefunktion]
\index{Abschneidefunktion}
\label{prop:7.44}
Sei $M\subset\R^n$ und für $\ep > 0$
\begin{align*}
M_\ep := \setdef{x\in\R^n}{d(x,M) < \ep} = \bigcup_{x\in M} K_\ep(x),
\end{align*}
so ist $M_\ep$ offen und damit messbar. Setze
\begin{align*}
\psi = J_\ep \chi_{M_\ep} = \int_{M_\ep} j_\ep(\cdot-y)\dy,
\end{align*}
so gelten
\begin{equivenum}
\item $0\le \psi(x)\le 1$ auf $\R^n$,
\item $\psi\in C^\infty(\R^n\to\R)$,
\item $\psi(x) = 0$,\quad $d(x,M_\ep) > \ep$,
\item $\psi(x) = 1$ auf $M$.\fishhere
\end{equivenum}
\end{prop}

\begin{figure}[!htpb]
\centering
\begin{pspicture}(-0.1,-1.2770555)(6.2682223,1.2511667)
\psline{<-}(0.38,1.2370555)(0.38,-1.0970556)
\psline{->}(0,-0.83705556)(6.2541113,-0.83705556)
\psline(1.7543283,-0.7)(1.7543283,-1)
\psline(4.3656735,-0.7)(4.3656735,-1)
\psline[linecolor=darkblue](1.374111,0.58294445)(4.7741113,0.58294445)
\psline[linecolor=purple](1.8141111,0.5629445)(4.374111,0.5629445)
\psline[linecolor=purple](5.334111,-0.80705553)(6.034111,-0.80705553)
\psline[linecolor=purple](0.8741111,-0.80705553)(0.37411112,-0.80705553)
\psbezier[linecolor=purple](4.374111,0.5629445)(5.054111,0.5629445)(4.7741113,-0.80705553)(5.354111,-0.80705553)
\psbezier[linecolor=purple](1.8141111,0.5629445)(1.1341112,0.5629445)(1.4741111,-0.80705553)(0.8741111,-0.80705553)

\rput(3.254111,0.7479445){\color{darkblue}$\chi_{M_\ep}$}
\rput(5.1941113,0.067944445){\color{purple}$\psi$}
\rput(3.054111,-1.0920556){\color{darkblue}$M$}
\rput(0.01,0.58205557){\color{gdarkgray}1}

\psline(0.2,0.59705555)(0.52,0.59705555)
\psline[linestyle=dotted,linecolor=darkblue](1.4,0.58)(1.4,-0.83705556)
\psline[linestyle=dotted,linecolor=darkblue](4.75,0.58)(4.75,-0.83705556)

\end{pspicture} 
\caption{Zur Abschneidefunktion.}
\end{figure}

\begin{prop}
\label{prop:7.45}
\begin{propenum}
\item\label{prop:7.45:1} Sei $u\in L^1_\loc(O)$ und $\supp u \Subset
O$, dann
\begin{align*}
J_\ep u \in C_0^\infty(O\to\C),\qquad \text{für } \ep < d(\supp u,\partial
O).
\end{align*}
\item\label{prop:7.45:2} Für $1\le p <\infty$ und $u\in L^p(O)$ gelten,
\begin{defnenum}
\item $J_\ep u \in L^p(O)$,
\item $\norm{J_\ep u}_p \le \norm{u}_p$,
\item $\norm{J_\ep u - u}_p \to 0$,\quad für $\ep \downarrow 0$.
\end{defnenum}
\item\label{prop:7.45:3} Ist $u\in L^\infty(O)$, so gilt $\abs{J_\ep
u(x)}\le \norm{u}_\infty$.
\item\label{prop:7.45:4} Ist $u\in C(O\to\C)$ und $K\Subset O$, dann $J_\ep u
\unito u$ auf $K$ für $\ep\downarrow 0$.\fishhere
\end{propenum}
\end{prop}
\begin{proof}
\begin{proofenum}
``\ref{prop:7.45:1}'': Siehe Skizze, die Details sind eine leichte Übung. 

\begin{figure}[!htpb]
\centering
\begin{pspicture}(0,-1.8093171)(6.1396832,1.7894094)
\psbezier(0.31968328,-0.5693171)(0.0,0.6307752)(1.7110358,0.120429)(2.4996834,0.21068287)(3.2883308,0.30093673)(4.669233,1.7694094)(4.819683,0.09068287)(4.9701333,-1.5880437)(0.63936657,-1.7694094)(0.31968328,-0.5693171)
\psbezier(4.239683,-0.32931712)(4.259683,-1.2093171)(3.6469872,-0.65047437)(3.3796833,-0.6293171)(3.1123793,-0.6081599)(2.5596833,-1.5093172)(2.3396833,-0.6093171)(2.1196833,0.29068288)(4.219683,0.55068284)(4.239683,-0.32931712)
\psbezier[linecolor=darkblue](4.159683,-0.32931712)(4.219683,-1.0693171)(3.6269872,-0.55047435)(3.3196833,-0.5493171)(3.0123794,-0.5481599)(2.5796833,-1.3493171)(2.4196832,-0.5693171)(2.2596834,0.21068287)(4.0996833,0.41068286)(4.159683,-0.32931712)

\rput(0.7096833,-0.34431714){\color{gdarkgray}$O$}
\psbezier(3.8796833,-0.24931712)(4.3596835,0.05068287)(4.759683,-0.70931715)(5.0596833,-0.12931713)

\rput(5.3796835,0.03568287){\color{darkblue}$\supp u$}
\psbezier(2.5196834,-0.88931715)(1.9414225,-0.94931716)(2.3196833,-1.5893171)(1.8196833,-1.6293172)

\rput(1.1096833,-1.5843171){\color{gdarkgray}$\supp J_\ep u$}
\psline[linecolor=purple]{<->}(4.0596833,-0.6893171)(4.219683,-0.9293171)

\rput(5.1396832,-1.1){\color{purple}$d(\supp u,\partialO)$}
\end{pspicture} 
\caption{Zum Beweis von Satz \ref{prop:7.45}.}
\end{figure}


``\ref{prop:7.45:3}'': $\abs{J_\ep u}(x) = \abs{\int_{\R^n}
j_\ep(x-y)\tilde{u}(y)\dy} \le \norm{u}_\infty\underbrace{\norm{j_\ep}_1}_{=1} = \norm{u}_\infty$.

``\ref{prop:7.45:4}'': Für $x\in K$ gilt
\begin{align*}
\abs{J_\ep u(x)-u(x)} &= 
\abs{\int_{\R^n} j_\ep(x-y)(u(y)-u(x))\dy} \\ &
\le \sup_{\abs{x-y}<\ep}
\abs{u(y)-u(x)}.
\end{align*}
Setzen wir $K_\ep := \overline{\setdef{z\in\R^n}{d(z,K)<\ep}}$, so ist $K_\ep$
kompakt und Teilmenge von $O$ für $\ep<d(K,\partial O)$. Somit
ist $u$ dort gleichmäßig stetig, d.h.
\begin{align*}
\sup_{\atop{\abs{x-y}<\ep}{x,y\in K_\ep}}
\abs{u(y)-u(x)} \to 0,\qquad \ep\downarrow0
\end{align*}
unabhängig vom speziell gewählten $x\in K$.

``\ref{prop:7.45:2}'':
Zu $p\neq 1$ sei $\frac{1}{p}+\frac{1}{q}=1$, so gilt
\begin{align*}
\abs{J_\ep u(x)} &=
\abs{\int_{\R^n}j_\ep(x-y)^{\frac{1}{p}+\frac{1}{q}}\tilde{u}(y)\dy}\\
&\overset{\text{Hölder}}{\le}
\underbrace{\left(\int_{\R^n} j_\ep(x-y)\dy \right)^{1/q}}_{=1}
\left(\int_{\R^n}j_\ep(x-y)\abs{\tilde{u}(y)}^p\dy \right)^{1/p}.
\end{align*}
Für $p=1$ ist diese Ungleichung trivialerweise ebenfalls erfüllt.

Für $1\le p < \infty$ gilt folglich,
\begin{align*}
\int_{O} \abs{J_\ep u(x)}^p \dx
&\le 
\int_{O}\int_{\R^n} j_\ep(x-y)\abs{\tilde{u(y)}}^p\dy \dx\\
&\overset{\text{Fubini}}{=}
\int_{\R^n} \underbrace{\int_{O}j_\ep(x-y) \dx}_{=1} \abs{\tilde{u(y)}}^p
\dy = \norm{u}_p^p.
\end{align*}
Somit wären (a) und (b) bewiesen. Zum Beweis von (c) benötigen wir noch etwas
Vorbereitung.\qedhere
\end{proofenum}
\end{proof}

\begin{bem}
\label{bem:7.46}
Es ist nicht einfach zu zeigen, dass für $O\subset\R^n$ offen und beschränkt
auch
\begin{align*}
\norm{J_\ep \chi_O - \chi_O}_p \to 0,
\end{align*}
denn es gibt solche $O$ mit $\mu(\partial O) > 0$.

Sei z.B. $n=1$ und $(q_j)$ eine Abzählung von $\Q\cap[-2,2]$. Setzen wir
\begin{align*}
O := \bigcup_{j\in\N} \left(q_j-\frac{1}{2^j},q_j+\frac{1}{2^j}\right),
\end{align*}
so ist $O$ offen und
\begin{align*}
\mu(O) \le \sum_{j\in\N} \frac{1}{2^{j-1}} = 2.
\end{align*}
Da $\setd{q_j}\subset O$, folgt $[-2,2]\subset O$, d.h. $\mu(\overline{O}) \ge
4$ also ist
\begin{align*}
\mu(\partial O) = \mu(\overline{O}\setminus O) = \mu(\overline{O})-\mu(O) =
2.\maphere
\end{align*}
\end{bem}

\begin{prop}[Satz von Lusin]
\index{Satz!Lusin}
\label{prop:7.45}
Sei $A\subset\R^n$ messbar, $\mu(A) < \infty$ und $f:\R^n\to\R$ messbar,
$f(x)=0$ für $x\in\R^n\setminus A$. Dann gilt
\begin{align*}
\forall \ep > 0 \exists f_\ep \in C_0(A\to\R) : \norm{f_\ep}_\infty
\le \norm{f}_\infty \land
\mu\setdef{x\in\R^n}{f_\ep(x)\neq f(x)}< \ep.\fishhere
\end{align*}
\end{prop}
\begin{proof}
Siehe \cite{Alt99} Anhang 4.7.\qedhere
\end{proof}

\begin{prop}
\label{prop:7.48}
Sei $1\le p <\infty$.
\begin{propenum}
\item\label{prop:7.48:1} Die Menge $\setdef{s_n : O\to\C}{s_n \text{ einfach und
} \supp s_n \Subset O}$ ist dicht in $L^p(\Omega)$.
\item\label{prop:7.48:2} $C_0(O\to\C)$ ist dicht in $L^p(\Omega)$.\fishhere
\end{propenum}
\end{prop}
\begin{proof}
Sei $f\in L^p(O)$ gegeben. Ohne Einschränkung ist $f$ reellwertig und positiv
ansonsten zerlege $f$ in $(\Re f)_\pm$, $(\Im f)_\pm$.

\begin{proofenum}
\item
Sei $O_k := \setdef{x\in O}{\abs{x} < k \land d(x,\partial O) > \frac{1}{k}}$.
\begin{figure}[!htpb]
\centering
\begin{pspicture}(0,-1.08)(5.3,1.08)
\psbezier[linecolor=darkblue](3.88,0.88)(3.3,0.82)(2.8971407,0.41556767)(2.34,0.32)(1.7828593,0.22443233)(1.32,1.06)(0.66,0.7)(0.0,0.34)(0.48,-0.7)(1.2,-0.82)(1.92,-0.94)(2.76,-0.62)(3.24,-0.72)(3.72,-0.82)(3.78,-1.06)(4.44,-0.68)
\psline[linecolor=darkblue,linestyle=dotted](3.88,0.88)(4.88,0.98)
\psline[linecolor=darkblue,linestyle=dotted](4.44,-0.68)(5.28,-0.22)
\psbezier[linecolor=purple](3.78,0.74)(3.2781081,0.69169813)(2.8569865,0.32068136)(2.36,0.22)(1.8630135,0.11931863)(1.3232433,0.9)(0.72,0.6)(0.11675676,0.3)(0.5902703,-0.62754714)(1.2356757,-0.7316981)(1.8810811,-0.83584905)(2.634054,-0.5181132)(3.08,-0.58)(3.525946,-0.6418868)(3.7083783,-0.96)(4.28,-0.66)
\psline[linecolor=purple](3.76,0.76)(4.28,-0.68)
\rput(4.91,-0.035){\color{darkblue}$O$}
\rput(1.13,-0.275){\color{purple}$O_k$}
\end{pspicture}
\caption{Zur Konstruktion von $O_k$.}
\end{figure}

Setze $f_k = \chi_{O_k}f$, dann ist $f_k$ messbar, $\supp f_k \subset
\overline{O}_k \Subset O$ und auf $O$ gilt
\begin{align*}
0\le f_k(x) \le f(x),\qquad f_k(x)\to f(x).
\end{align*}
Mit dem Satz von der majorisierten Konvergenz erhalten wir somit,
\begin{align*}
\norm{f_k-f}^p_p = \int_\Omega \abs{f_k-f}^p \dmu \to 0.
\end{align*}
Aus der Maßtheorie wissen wir weiterhin dass eine Folge einfacher Funktionen
$(s_n)$ existiert, die auf $O$ monoton geen $f_k$ konvergiert,
\begin{align*}
0\le s_n(x)\le f_k(x),\qquad s_n(x)\to f_k(x).
\end{align*}
Ebenfalls mit majorisierter Konvergenz folgt, dass $\norm{s_n-f_k}_p\to 0$.

Somit folgt \ref{prop:7.48:1}.
\item Sei nun $s=\sum_{j=1}^N a_j \chi_{A_j}$ mit $\norm{f-s}_p < \ep$,
$\supp s\Subset O$ und $A_j\subset \supp s$.
Für jedes der endlich vielen $j\in\setd{1,\ldots,N}$ existiert nach dem Satz von
Lusin ein $g_j\in C_0(A_j\to\C)$ mit $\abs{g_j(x)}\le 1$ auf $A_j$ und
\begin{align*}
\mu\setdef{x\in\R^n}{g_j(x)\neq \chi_{A_j}(x)} < \ep.
\end{align*}
Somit folgt
\begin{align*}
\norm{\chi_{A_j}-g_j}_p^p = \int_{\R^n} \underbrace{\abs{\chi_{A_j}-g_j}}_{\le
2}^p
\dmu
\le 2^p
\mu\setdef{x\in\R^n}{g_j(x)\neq \chi_{A_j}(x)}
< 2^p\ep.
\end{align*}
Da $g_j$ stetig und $\supp g_j \Subset O$ erhalten wir
analog für $s$,
\begin{align*}
\norm{s-\sum_{j=1}^N a_j g_j}_p^p \le N2^p\ep.\qedhere
\end{align*}
\end{proofenum}
\end{proof}

\begin{proof}[Beweis von \ref{prop:7.45} \ref{prop:7.45:2} (c).]
Zu $u\in L^p(\Omega)$ wähle $g\in C_0(\Omega\to\C)$ mit $\norm{u-g}_p < \ep$.
Aus \ref{prop:7.45} \ref{prop:7.45:1} wissen wir
\begin{align*}
J_\ep g\in C_0^\infty(\Omega\to\C),\qquad \text{für } \ep < \ep_0.
\end{align*} 
Für $\ep < \ep_0$ gilt außerdem $J_\ep g \subset \setdef{x\in\Omega}{d(x,\supp
g)<\ep}\Subset \Omega$. Setzen wir daher
\begin{align*}
K:= \setdef{x\in\Omega}{d(x,\supp g)\le \ep_0},
\end{align*}
so folgt mit \ref{prop:7.45} \ref{prop:7.45:4}
\begin{align*}
J_\ep g(x) - \tilde{g}(x) =
\begin{cases}
0, & x\in\R^n\setminus K\\
\to 0,& \text{gleichmäßig auf }K.
\end{cases}
\end{align*}
Nun gilt
\begin{align*}
\norm{J_\ep g - g}_{p}^p = 
\int_K \abs{J_\ep g - g}^p \dmu
\le \sup_{x\in K} \abs{J_\ep g(x)-g(x)}^p \mu(K)
\to 0,\qquad \ep\downarrow 0. 
\end{align*}
Zusammenfassend gilt also
\begin{align*}
\norm{u-J_\ep u}_p \le \norm{u-g}_p + \norm{g-J_\ep g}_p + \norm{J_\ep(g-u)}_p
< \delta,
\end{align*}
für $\ep$ hinreichend klein.\qedhere
\end{proof}

\begin{cor}
\label{cor:7.49}
Sei $1\le p < \infty$. Dann ist $C_0^\infty(\Omega\to\C)$
dicht in $L^p(O)$.\fishhere
\end{cor}
\begin{proof}
Aus dem Beweis von \ref{prop:7.45} verwenden wir
\begin{align*}
\norm{u-J_\ep g}_p \le \norm{u-g}_p + \norm{g-J_\ep g}_p  < 2\delta,
\end{align*}
für $\ep$ hinreichend klein wobei $J_\ep g\in C_0^\infty(O\to\C)$ für $\ep$
hinreichend.\qedhere
\end{proof}

\begin{prop}
\label{prop:7.50}
Sei $1<p<\infty$, $m\in\N$ und $u\in W^{m,p}(\R^n)$. Dann gilt
\begin{align*}
\forall \abs{\alpha} \le m : J_\ep\D^\alpha u = \nabla^\alpha (J_\ep
u).\fishhere
\end{align*}
\end{prop}
\begin{proof}
Sei $\ph\in C_0^\infty(\R^n\to\C)$. Dann gilt
\begin{align*}
\lin{J_\ep \D^\alpha u,\ph}
&= \int_{\R^n}\int_{\R^n} j_\ep(x-y)\D^\alpha u(y) \dy 
\overline{\ph}\dx\\
&\overset{\text{Fubini}}
{=} \int_{\R^n}\D^\alpha u(y)\overline{\int_{\R^n} j_\ep(x-y)\ph(x)\dx} \dy,
\end{align*}
da $j_\ep$ reellwertig. Nun ist $j_\ep(x-\cdot)\ph(\cdot)\in
C_0^\infty(\R^n\to\C)$ wir können somit die Definition der schwachen Ableitung
anwenden und erhalten,
\begin{align*}
\ldots &=
(-1)^{\abs{\alpha}} \int_{\R^n}u(y) \overline{\int_{\R^n}
\nabla^\alpha_y j_\ep(x-y)\ph(x)\dx}
\dy \\ &=
\int_{\R^n}u(y) \overline{\int_{\R^n}
(\nabla^\alpha j_\ep)(x-y)\ph(x)\dx}
\dy\\
&=
\int_{\R^n} \int_{\R^n}
u(y)
(\nabla^\alpha j_\ep)(x-y)\dy\overline{\ph}(x)
\dx\\
&=
\lin{\nabla^\alpha J_\ep u,\ph}
\end{align*}
und da $\ph$ beliebig war, folgt die Behauptung.\qedhere
\end{proof}

\begin{prop}
\label{prop:7.51}
Sei $\Omega=\R^n$, $m\in\N$. Dann gilt
\begin{align*}
W_0^{m,p}(\R^n) = W^{m,p}(\R^n).\fishhere
\end{align*}
\end{prop}
\begin{proof}
``$\subset$'': $W_0^{m,p}(\R^n)\subset W^{m,p}(\R^n)$ gilt nach Definition.\\
``$\supset$'': Sei $u\in W^{m,p}(\R^n)$. Zeige
\begin{align*}
\forall \ep > 0 \exists \ph\in C_0^\infty(\R^n\to\C) : \norm{\ph-u}_{m,p} < \ep.
\end{align*}
\textit{Abschneiden von $u$}. Sei $\psi\in C_0^\infty(\R^n\to\R)$ mit $0\le
\psi\le 1$ und
\begin{align*}
\psi(x) = 
\begin{cases}
1, & \abs{x}< 1,\\
0, & \abs{x} \ge 2.
\end{cases}
\end{align*}
Setzen wir $u_k(x) = \psi(k^{-1}x)u(x)$, dann gilt für $\abs{\alpha}\le m$:
\begin{align*}
\norm{\D^\alpha u-\psi(k^{-1}\cdot)\D^\alpha u}_p^p
&=
\int_{\R^n} \abs{\D^\alpha u(x)}^p\abs{1-\psi(k^{-1}x)}^p\dmu\\
&\to 0
\end{align*}
nach majorisierter Konvergenz. Nun ist
\begin{align*}
\norm{\D^\alpha u - \D^\alpha u_k}_p
\le \norm{\D^\alpha u - \psi(k^{-1}x)\D^\alpha u}_p
+ \sum_{\atop{\beta\le \alpha}{\beta\neq 0}}
\binom{\alpha}{\beta}
\norm{\nabla_x^\beta \psi(k^{-1}x)\D^{\alpha-\beta}u}_p
\end{align*}
und $\abs{\nabla_x^\beta \psi(k^{-1}x)} \le
\frac{1}{k^{\abs{\beta}}}\abs{(\nabla^\beta\psi)(k^{-1}x)}$ also
\begin{align*}
\ldots \le
\norm{\D^\alpha u - \psi(k^{-1}x)\D^\alpha u}_p
+ \frac{1}{k}c \sum_{\beta\le\alpha} \binom{\alpha}{\beta}
\norm{\D^{\alpha-\beta}u}_p
\to \infty.
\end{align*}
Somit existiert ein $k\in\N$, so dass
\begin{align*}
\norm{u-u_k}_{m,p} < \frac{\ep}{2}.
\end{align*}

Sei $\ph=J_\delta u_k$ mit $\delta$ noch zu bestimmen. Nun ist $\supp
u_k\subset K_{2k}(0)$, da $\psi$ außerhalb verschwindet. Somit ist $J_\delta
u_k\in C_0^\infty(\R^n\to\C)$. Für $\abs{\alpha}\le m$ gilt also
\begin{align*}
\norm{\D^\alpha(u_k-J_\delta u_k)}_p \overset{\ref{prop:7.50}}{=}
\norm{\D^\alpha u_k - J_\delta D^\alpha u_k}_p \overset{\ref{prop:7.45}}{\to}0.
\end{align*}
Somit existiert ein $\delta_0> 0$, so dass
\begin{align*}
\norm{u_k-J_{\delta}u_k}_{m,p}<\frac{\ep}{2},\qquad \delta \ge \delta _0.
\end{align*}
Zusammenfassend also
\begin{align*}
\norm{u-J_{\delta_0}u_k}_{m,p} < \ep.\qedhere
\end{align*}
\end{proof}

\begin{cor}
\label{prop:7.52}
Für $m\in\N$ gilt $C_0^m(O\to\C)\subset W_0^{m,p}(O)$.\fishhere
\end{cor}
\begin{proof}
Sei $u\in C_0^m(O\to \C)$. Setze $\ph_\ep:=J_\ep u$, so ist nach
\ref{prop:7.45} $\ph_\ep\in C_0^\infty(O\to\C)$ für $\ep < d(\supp u,\partial
O)$. Wie im Beweis von Satz \ref{prop:7.51} sieht man sofort, dass
\begin{align*}
\norm{\D^\alpha \ph_\ep - \D^\alpha u}_p = \norm{J_\ep \nabla^\alpha u -
\nabla^\alpha u}_0 \to 0.
\end{align*}
Somit $\norm{\ph_\ep - u}_{m,p}\to 0$.\qedhere
\end{proof}

\begin{prop}[Produktregel]
\index{schwache!Produktregel}
Sei $m\in \N$, $u\in W^{m,p}(O)$ und $f\in C^m(O\to\C)$ mit
$\norm{\nabla^\alpha f}_\infty \le c$ für $\abs{\alpha}\le m$. Dann gilt
\begin{align*}
f\cdot u \in W^{m,p}(O)
\end{align*}
und
\begin{align*}
\D^\alpha (f\cdot u) = \sum_{\beta\le \alpha} \nabla^\beta f \cdot
\D^{\alpha-\beta} u.\fishhere
\end{align*}
\end{prop}
\begin{proof}
Wir beweisen lediglich den Fall $m=1$, die übrigen Fälle ergeben sich analog.
Sei $\ph\in C_0^\infty(O\to\C)$, so ist
\begin{align*}
\lin{\D^{e_j}(f\cdot u),\ph} &= - \lin{f\cdot u,\nabla^{e_j}\ph}
= -\lin{u,\overline{f}\nabla^{e_j}\ph}\\ & =
-\lin{u,\nabla^{e_j}(\overline{f}\cdot u)} + \lin{u,(\nabla^{e_j}f)\ph}.
\end{align*}
In Übung 14.1 haben wir bewiesen $-\lin{u,\nabla^{e_j} (\overline{f}\cdot\ph)}
= \lin{(\D^{e_j}u)f,\ph}$ also gilt
\begin{align*}
\lin{\D^{e_j}(f\cdot u),\ph}
&= \lin{f(\D^{e_j}u),\ph} + \lin{(\nabla^{e_j}f)u,\ph}\\
&=
\lin{f(\D^{e_j}u)+(\nabla^{e_j}f)u,\ph}.\qedhere
\end{align*} 
\end{proof}

\clearpage
\chapter{Unbeschränkte Operatoren in Hilberträumen}

\begin{bemn}[Erinnerung.]
Ein linearer Operator $A: H\to H$ auf dem Hilbertraum $H$ heißt
\emph{symmetrisch}, falls
\begin{align*}
\lin{Ax,y} = \lin{x,Ay},\qquad x,y\in H.\maphere
\end{align*}
\end{bemn}

Der folgende Satz zeigt, dass auf ganz $H$ definierte symmetrische Operatoren
nichts neues ergeben.

\begin{prop}[Satz von Hellinger-Toeplitz]
\label{prop:8.1}
\index{Satz!Hellinger-Toeplitz}
Jeder symmetrische Operator $A:H\to H$ ist beschränkt.\fishhere
\end{prop}
\begin{proof}
Übungsaufgabe 8.4\qedhere
\end{proof}

\begin{bem}[Vereinbarung.]
\label{bem:8.2}
Im Folgenden sei $H$ stets Hilbertraum über $\C$ und $A: D(A)\to H$ mit
$D(A)\subset H$ linearer Operator.\maphere 
\end{bem}

\begin{defn}
\label{defn:8.3}
\index{Operator!symmetrisch}
$A$ heißt symmetrisch, wenn für $x,y\in D(A)$ gilt
\begin{align*}
\lin{Ax,y} = \lin{x,Ay}.\fishhere
\end{align*}
\end{defn}

\begin{bsp}
\label{bsp:8.4}
Sei $H=L^2(O)$, $O\subset\R^n$ offen und
\begin{align*}
Au = -\Delta u := -\sum_{j=1}^n \D^{2e_j}u
\end{align*}
für $u\in D(A) = W^{2,2}(O)\cap W_0^{1,2}(O)$. $A$ ist symmetrisch, denn
seien $u,v\in D(A)$, so ist
\begin{align*}
\lin{Au,v} &= - \sum_{j=1}^n \lin{\D^{2e_j}u,v}
= \sum_{j=1}^n \lin{\D^{e_j}u,\D^{e_j}v}
= -\sum_{j=1}^n \lin{u,\D^{2e_j}v}\\
&= \lin{u,\Delta v}.\bsphere
\end{align*}
\end{bsp}

\begin{defn}
\label{defn:8.5}
\index{Operator!adjungiert}
Sei $D(A)$ dicht in $H$. Der Operator
\begin{align*}
&\D(A^*) := \setdef{y\in H}{\exists y^*\in H \forall x\in D(A) : \lin{Ax,y} =
\lin{x,y^*}},\\
&A^* y = y^*
\end{align*}
heißt zu $A$ \emph{adjungierter Operator}.\fishhere
\end{defn}

\begin{prop}
\label{prop:8.6}
\begin{propenum}
\item $y^*$ ist durch $y$ und die Bedingung
\begin{align*}
\forall x\in D(A) : \lin{Ax,y} = \lin{x,y^*}
\end{align*}
eindeutig bestimmt.
\item $A^*$ ist linear.\fishhere
\end{propenum}
\end{prop}
\begin{proof}
\begin{proofenum}
\item Sei $\tilde{y}\in H$ und erfülle ebenfalls die Bedingung, dann folgt für
alle $x\in D(A)$,
\begin{align*}
\lin{x,y^*-\tilde{y}} = \lin{x,y^*}-\lin{x,\tilde{y}} = 0
\end{align*}
und da $D(A)\subset H$ dicht folgt auch
\begin{align*}
\lin{y^*-\tilde{y},y^*-\tilde{y}} = 0.
\end{align*}
\item Seien $y_1,y_2\in D(A^*)$ und $\alpha,\beta\in\C$ so gilt
\begin{align*}
\lin{Ax,\alpha y_1 + \beta y_2} = \overline{\alpha}\lin{x,y_1^*} +
\overline{\beta}\lin{x,y_2^*} = \lin{x,\alpha y_1^* + \beta y_2^*}
\end{align*}
also ist $\alpha y_1 + \beta y_2\in D(A^*)$ und
\begin{align*}
A^*(\alpha y_1 + \beta y_2)= \alpha y_1^* + \beta y_2^*.\qedhere
\end{align*}
\end{proofenum}
\end{proof}

\begin{prop}
\label{prop:8.7}
Sei $D(A)\subset H$ dicht und $A$ symmetrisch. Dann ist $A^*$ eine Fortsetzung
von $A$, d.h.
\begin{align*}
D(A)\subset D(A^*),\qquad A^*x = Ax\text{ für }x\in D(A).
\end{align*}
Schreibe $A\subset A^*$.\fishhere
\end{prop}
\begin{proof}
Sei $y\in D(A)$, so gilt
\begin{align*}
\lin{Ax,y} = \lin{x,Ay} = \lin{x,A^*y}.
\end{align*}
Also ist $y\in D(A^*)$ und $A^*y = Ay$.\qedhere
\end{proof}

\begin{defn}
\label{defn:8.8}
\index{Operator!selbstadjungiert}
Sei $D(A)$ dicht. $A$ heißt \emph{selbstadjungiert}, falls
\begin{align*}
A = A^*.\fishhere
\end{align*}
\end{defn}

\begin{prop}
\label{prop:8.9}
Jeder selbstadjungierte Operator ist symmetrisch.\fishhere
\end{prop}
\begin{proof}
Seien $x,y\in D(A)$. Dann gilt
\begin{align*}
\lin{Ax,y} = \lin{x,A^*y} =\lin{x,Ay}.\qedhere
\end{align*}
\end{proof}

\begin{bsp}
\label{bsp:8.10}
Sei $H=L^2((-1,1))$ und $D(A) = C_0^\infty((-1,1)\to\C)$ mit $A\ph = -\ph''$.
\begin{bspenum}
\item \text{$A$ ist symmetrisch}. Spezialisierung von Beispiel \ref{bsp:8.4}.
\item $D(A^*) = W^{2,2}((-1,1))$ und $A^* u = \D^2 u$.

``$\supset$'': Seien $u\in W^{2,2}((-1,1))$ und $\ph\in D(A)$ so gilt
\begin{align*}
\lin{A\ph,u} = -\lin{\ph, \D^2 u}.
\end{align*}
Da $u\in W^{2,2}$, ist $\D^2u\in L^2$ also $u^* = -\D^2 u$. Somit ist $u\in
D(A^*)$ und $A^* u = -\D^2 u$.

``$\subset$'': Sei $u\in D(A^*)$, so gilt für $\ph\in D(A)$
\begin{align*}
\lin{\ph,A^*u} = \lin{A\ph,u} =  - \lin{\ph'',u} = -\lin{\ph,\D^u}.
\end{align*}
Da $\ph$ beliebig gilt $\D^2u = -A^* u \in L^2(\Omega)$. Setzen wir
\begin{align*}
v(x)= \int_{-1}^x \D^2u\dmu, 
\end{align*}
so ist $v\in L^2((-1,1))$ und $\D^1 v = \D^2 u$. (Übungsaufgabe 13.1)

Da $\D^1(v-\D^1 u) = 0$ folgt $v-\D^1 u = \const$ und somit ist $\D^1 u =
v-\const\in L^2((-1,1))$ und somit $u\in W^{2,2}((-1,1))$ und $A^* u = -\D^2
u$.
\item $A^*$ ist nicht symmetrisch.

Seien $u\equiv 1$ und $v(x)=x^2$, so sind $u,v\in W^{2,2}((-1,1))$ aber
\begin{align*}
\lin{Au,v} = 0,\qquad \lin{Av,u} = -\lin{v'',u} = -\lin{2,1} = -4\neq 0.
\end{align*}
$A$ ist also insbesondere \textit{nicht} selbstadjungiert.
\item Wir vergrößern nun $D(A)$ in der Hoffnung, dass $A$ dadurch
selbstadjungiert wird.
\begin{align*}
&D(A_1) := \overline{D(A)}^{W^{2,2}} = W_0^{2,2}((-1,1)),\\
&A_1 u = \D^2 u.
\end{align*}
$A_1$ ist symmetrisch aber $A_1^*=A^*$. Somit ist der Versuch fehlgeschlagen.
\item Wir vergrößern $D(A)$ noch weiter.
\begin{align*}
&D(A_2) := W^{2,2}((-1,1))\cap W_0^{1,2}((-1,1)),\\
&A_2u = -\D^2 u.
\end{align*}
Dieser Operator ist selbstadjungiert (später).

\begin{figure}[!htpb]
\centering
\begin{pspicture}(0,-1.81)(4.8,1.77)
\psellipse[linecolor=darkblue](2.21,0.19)(2.21,1.58)
\psellipse[linecolor=purple](1.72,0.19)(1.52,0.86)
\pscircle[linecolor=yellow](1.52,0.19){0.78}
\psellipse(1.5,0.2)(0.7,0.31)

\rput(3.63,-1.585){\small\color{darkblue}$D(A_1^*)=D(A^*)$}
\rput(2.15,1.235){\small\color{purple}$D(A_2)=D(A_2^*)$}
\rput(2.75,0.195){\small\color{yellow}$D(A_1)$}
\rput(1.46,0.195){\small\color{gdarkgray}$D(A)$}
\end{pspicture} 
\caption{Zur Wahl von $A$, $A_1$ und $A_2$.}
\end{figure}

Man sagt $A_2$ ist eine selbstadjungierte Erweiterung von $A$.
Selbstadjungierte Erweiterungen sind im Allgemeinen nicht eindeutig. Operatoren
bei denen das der Fall ist heißen wesentlich selbstadjungiert.\bsphere
\end{bspenum}
\end{bsp}

\begin{defn}
\label{defn:8.11}
$A$ heißt \emph{abgeschlossen}\index{Operator!abgeschlossen}, falls der Graph
$G(A):= \setdef{(x,Ax)}{x\in D(A)}$ in $H\times H$ abgeschlossen ist.\fishhere
\end{defn}

\begin{prop}
\label{prop:8.12}
Jeder selbstadjungierte Operator ist abgeschlossen.\fishhere
\end{prop}
\begin{proof}
Sei $(x_n)$ Folge in $D(A)$ mit $(x_n,Ax_n)\to (x,y)\in H\times H$.

Für $z\in D(A)$ gilt nun
\begin{align*}
\lin{Az,x}  =\lim\limits_{n\to\infty}\lin{Az,x_n}
  =\lim\limits_{n\to\infty}\lin{z,Ax_n} = \lin{z,y}.
\end{align*}
Somit ist $x\in D(A*)$ und $A^* x = y$. Insbesondere ist $x\in D(A)$ und $Ax=y$
also $(x,y) = (x,Ax)\in G(A)$.\qedhere
\end{proof}

\begin{defn}
\label{defn:8.13}
\begin{defnenum}
\item Die Resolventenmenge\index{Resolventen!-Menge} von $A$ ist gegeben durch
\begin{align*}
\rho(A) := \setdef{\lambda\in\C}{\atop{A-\lambda\Id \text{ ist injektiv},\;
\overline{\im(A-\lambda\Id)}=H}{\text{ und }(A-\lambda\Id)^{-1}\in\LL(H)}}
\end{align*}
\item $\sigma(A) =\C\setminus \rho(A)$ heißt \emph{Spektrum}\index{Spektrum}
von $A$.
\begin{align*}
\sigma_p(A) &:= \setdef{\lambda\in \sigma(A)}{A-\lambda\Id\text{ ist
nicht injektiv}}\\ &= \setdef{\lambda\in\sigma(A)}{\lambda\text{ ist Eigenwert
von } A}
\end{align*}
heißt \emph{Punktspektrum}\index{Spektrum!Punkt-}.
\begin{align*}
\sigma_c(A) := \setdef{\lambda\in\sigma(A)}{\atop{A-\lambda\Id\text{ ist
injektiv},\;
\overline{\im(A-\lambda\Id)}=H}{\text{aber }(A-\lambda\Id)^{-1}\notin\LL(H)}}
\end{align*}
heißt \emph{stetiges (kontinuierliches)
Spektrum}\index{Spektrum!stetiges}\index{Spektrum!kontinuierliches} von $A$.
\begin{align*}
\sigma_r(A) := \setdef{\lambda\in\sigma(A)}{A-\lambda\Id\text{ ist injektiv
aber }\overline{\im(A-\lambda\Id)}\subsetneq H}
\end{align*}
heißt \emph{residuales Spektrum} von $A$\index{Spektrum!residuales}.\fishhere
\end{defnenum}
\end{defn}

\begin{prop}
\label{prop:8.14}
Falls $A$ abgeschlossen, ist
\begin{align*}
\rho(A) &:= \setdef{\lambda\in\C}{A-\lambda\Id\text{ ist injektiv und
}\im(A-\lambda\Id)=H}\\
&= \setdef{\lambda\in\C}{A-\lambda\Id\text{ bijektiv}}.\fishhere
\end{align*}
\end{prop}
\begin{proof}
\begin{proofenum}
``$\subset$'': Sei nun $\lambda\in\rho(A)$. Zeige $\im(A-\lambda\Id)=H$. Zu
$y\in H$ sei $(x_n)$ in $D(A)$ mit $(A-\lambda\Id)x_n\to y$. Insbesondere ist dann
$((A-\lambda\Id)x_n)$ Cauchyfolge und da $(A-\lambda\Id)^{-1}$ beschränkt auch
$(x_n)$, d.h. $x_n\to x$ in $H$.
\begin{align*}
Ax_n = \underbrace{(A-\lambda\Id)x_n}_{\to y} + \underbrace{\lambda x_n}_{\to
\lambda x} \to y +\lambda x.
\end{align*}
Somit $(x_n,Ax_n)\to (x,y+\lambda)\in G(A)$ und insbesondere $x\in D(A)$ und
$Ax = y+\lambda x$ also $y= Ax-\lambda x\in \im (A-\lambda\Id)$. D.h.
$\im(A+\lambda\Id)=H$.

``$\supset$'': Sei $A-\lambda\Id: D(A)\to H$ bijektiv. Zeige
$(A-\lambda\Id)^{-1}$ beschränkt. Ist $A$ abgeschlossen, so ist auch
$A-\lambda\Id$ abgeschlossen und daher $(A-\lambda\Id)^{-1}$ abgeschlossen, da
$G(A-\lambda\Id)=G((A-\lambda\Id)^{-1})$.

Aus dem closed Graph theorem folgt, dass $(A-\lambda\Id)^{-1}$ beschränkt
ist.\qedhere
\end{proofenum}
\end{proof}

\begin{lem}
\label{prop:8.15}
\begin{propenum}
\item Ist $A$ selbstadjungiert, so ist $\sigma_p(A)\subset\R$.
\item Ist $D(A)$ dicht, so gilt für jedes $\lambda\in\C:\im(A-\lambda\Id)^\bot = \ker(A^*-\overline{\lambda}\Id)$.\fishhere
\end{propenum}
\end{lem}
\begin{proof}
Der Beweis ist eine leichte Übung.\qedhere
\end{proof}

\begin{prop}
\label{prop:8.15}
Sei $A$ selbstadjungiert. Dann gilt
\begin{propenum}
\item $\sigma(A)\subset\R$.
\item $\sigma_r(A)=\varnothing$.\fishhere
\end{propenum} 
\end{prop}
\begin{proof}
\begin{proofenum}
\item Sei $\lambda\in\C\setminus\R$. Zeige $\lambda\in\rho(A)$. Für $x\in D(A)$
rechnet man ohne Umwege nach,
\begin{align*}
\norm{(A-\lambda\Id)x}^2 = \norm{(A-(2\Re \lambda)\Id)x}^2 +
(\Im\lambda)^2\norm{x}^2
\ge (\Im\lambda)^2\norm{x}^2.
\end{align*}
Da $\Im \lambda\neq 0$ ist $\ker(A-\lambda\Id)= (0)$ und daher $\lambda$
kein Eigenwert. Weiterhin ist für $y=(A-\lambda\Id)x$,
\begin{align*}
\norm{(A-\lambda\Id)^{-1}y} \le \frac{1}{\abs{\Im \lambda}}\norm{y},
\end{align*}
also ist $(A-\lambda\Id)^{-1}$ beschränkt.
Nach Lemma \ref{prop:8.15} ist
\begin{align*}
\im(A-\lambda\Id)^\bot = \ker(A^*-\overline{\lambda}\Id)
=\ker(A-\overline{\lambda}\Id) = (0)
\end{align*}
und daher $\overline{\im (A-\lambda\Id)}=H$.
\item Sei $\lambda\in \sigma(A)$, so ist $\lambda\in\R$. Sei $A-\lambda\Id$
injektiv, so gilt
\begin{align*}
\im(A-\lambda\Id)^\bot = \ker(A^*-\overline{\lambda}\Id)
\overset{\atop{A^*=A}{\lambda=\overline{\lambda}}}{=} \ker(A-\lambda\Id)=(0)      
\end{align*}
und daher $\overline{\im (A-\lambda\Id)} = H$, d.h. $\lambda\notin \sigma_r(A)$.\qedhere
\end{proofenum}
\end{proof}

\begin{prop}
\label{prop:8.17}
Sei $D(A)$ dicht und $A$ symmetrisch. Dann sind äquivalent
\begin{equivenum}
\item\label{prop:8.17:1} $A$ ist selbstadjungiert.
\item\label{prop:8.17:2} $\exists \lambda\in \C : \im(A-\lambda\Id) = H =
\im(A-\overline{\lambda}\Id)$.\fishhere
\end{equivenum}
\end{prop}
\begin{proof}
``\ref{prop:8.17:1}$\Rightarrow$\ref{prop:8.17:2}'': Klar, denn ist $A$
selbstadjungiert, so gilt $(A-\lambda\Id)=H$ für jedes $\lambda\in\rho(A)$.

``\ref{prop:8.17:2}$\Rightarrow$\ref{prop:8.17:1}'': Sei $y\in D(A^*)$. Zeige
$y\in D(A)$. Nach Voraussetzung existiert ein $x\in D(A)$, so dass
\begin{align*}
(A-\lambda\Id)x = (A^*-\lambda\Id)y.
\end{align*}
Da $A\subset A^*$ ist
\begin{align*}
(A^*-\lambda\Id)(x-y) = 0
\end{align*}
und daher gilt für alle $z\in D(A)$,
\begin{align*}
\lin{(A-\overline{\lambda}\Id)z,x-y} = 
\lin{z,(A^*-\lambda\Id)(x-y)} = 0 
\end{align*}
und da $\im(A-\overline{\lambda}\Id)=H$ ist $x-y=0$, d.h. $y=x\in\D(A)$.\qedhere
\end{proof}

\begin{bsp}
\label{bsp:8.18}
Wir setzen hier Beispiel \ref{bsp:8.10} fort. $A_2$ ist symmetrisch und
$W^{2,2}\cap W^{1,2}_0$ dicht in $H=L^2$. Wir zeigen nun für $\lambda=-1$, dass
\begin{align*}
\im(A_2-\lambda\Id) = \im(A_2+\Id) = H.
\end{align*}
Sei also $f\in L^2$ und $\phi: W_0^{1,2}\to\C,\; u\mapsto \lin{u,f}_2$, so ist
$\phi\in {W_0^{1,2}}'$ und daher existiert nach dem Lemma von Riesz ein $v\in
W_0^{1,2}$ mit
\begin{align*}
\phi = \lin{\cdot,v}_{1,2} = \lin{\cdot,f}_2.
\end{align*}
Für $u\in W_0^{1,2}$ gilt somit
\begin{align*}
\lin{u,f}_2 = \lin{u,v}_2 + \lin{\D^1 u,\D^1 v}_2
= \lin{u,v}_2 - \lin{u,\D^2 v}_2= \lin{u,v-\D^2 v}.
\end{align*}
Also ist $v-\D^2 v = f$ und 
$\D^2 v = v-f\in L^2$. Somit ist $v\in W_0^{1,2}\cap W^{2,2}$ und $(A_2+\Id)v =
f$.

$A_2$ ist daher nach Satz \ref{prop:8.17} selbstadjungiert.\bsphere
\end{bsp}


\printindex

\begin{thebibliography}{50}
\bibitem[Adams, R.A]{Adams75} Adams, R.A.: Sobolev-spaces. Academic Press 1975
\bibitem[Alt]{Alt99} Alt, Hans Wilhelm: Lineare Funktionalanalysis: eine
anwendungsorientierte Einführung. Springer 1999
\bibitem[Heuser]{Heuser92} Heuser, Harro: Funktionalanalysis: Theorie und
Anwendung. Teubner 1992
\bibitem[Hirzebruch]{Hirzebruch85} Hirzebruch, F. und Scharlau, W.: Einführung
in die Funktionalanalysis. BI-Wiss.-Verl. 1985
\bibitem[Reed, Simon]{SimonReed95} Michael Reed and Barry Simon: Methods of
modern mathematical physics, Band 1. Academic Press 1995
\bibitem[Werner]{Werner07} Werner, Dirk: Funktionalanalysis. 6. Auflage,
Springer 2007
\bibitem[Kosaku Yosida]{Kosaku80} Kosaku Yosida: Functional analysis. Springer
1980
\end{thebibliography}

\end{document}